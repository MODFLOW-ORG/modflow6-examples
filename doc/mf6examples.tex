\documentclass[11pt,twoside,onecolumn]{article}

\RequirePackage[letterpaper,
                left=1.in,
                right=.767in,
                top=1in,
                bottom=1in,
                headheight=14bp,
                headsep=9bp,
                columnsep=0.24in,
                footskip=14bp,
                heightrounded]{geometry}

\usepackage[mmddyyyy]{datetime}

\usepackage{amsmath}
\usepackage{algorithm}
\usepackage{algpseudocode}
\usepackage{bm}
\usepackage{calc}
\usepackage{natbib}
\usepackage{graphicx}
\usepackage{longtable}
\usepackage{dcolumn}
\usepackage{ltablex}
\usepackage{tabularx, multirow}
\usepackage{booktabs}
\usepackage[tight-spacing=true]{siunitx}
\usepackage{color, colortbl}
\usepackage{tocloft}
\usepackage[nottoc,notlot,notlof]{tocbibind}
\usepackage{lipsum}
\usepackage{siunitx}
\usepackage{catchfile}
\usepackage[width=.95\textwidth]{caption}

\usepackage{wallpaper}

%Do not allow a page break to result in a line appearing by itself 
% https://tex.stackexchange.com/questions/4152/how-do-i-prevent-widow-orphan-lines 
\usepackage[all]{nowidow}

\makeindex
\usepackage{setspace}
% uncomment to make double space 
%\doublespacing
\usepackage{etoolbox}
\usepackage{xspace}
\usepackage{verbatim}

% set up the listings package for highlighting block definitions and input files
\usepackage{listings}
\usepackage{xcolor}

% Python style for highlighting
\newcommand\pythonstyle{\lstset{
language=Python,
basicstyle=\ttm,
tabsize=4,
morekeywords={self},              % Add keywords here
basicstyle=\small \ttfamily,
numberstyle=\color{gray},
stringstyle=\color[HTML]{933797},
showstringspaces=false,
commentstyle=\color[HTML]{228B22}\sffamily,
emph={[2]from,import,pass,return}, emphstyle={[2]\color[HTML]{DD52F0}},
emph={[3]range}, emphstyle={[3]\color[HTML]{D17032}},
emph={[4]for,in,def}, emphstyle={[4]\color{blue}},
breaklines=false,
breakatwhitespace=true,
literate={\_}{}{0\discretionary{\_}{}{\_}},
showstringspaces=false
xleftmargin=.1in
}}

% Python environment
\lstnewenvironment{python}[1][]
{
\pythonstyle
\lstset{#1}
}
{}

\usepackage[hidelinks]{hyperref}
\hypersetup{
    pdftitle={MODFLOW 6 Example Problems},
    pdfauthor={MODFLOW 6 Development Team},
    pdfsubject={MODFLOW 6 Example Problems},
    pdfkeywords={MODFLOW, groundwater model, simulation},
    bookmarksnumbered=true,     
    bookmarksopen=true,         
    bookmarksopenlevel=1,       
    colorlinks=true,
    allcolors={blue},          
    pdfstartview=Fit,           
    pdfpagemode=UseOutlines,
    pdfpagelayout=TwoPageRight
}

\makeatletter
\newcommand{\customlabel}[2]{%
   \protected@write \@auxout {}{\string \newlabel {#1}{{#2}{\thepage}{#2}{#1}{}} }%
   \hypertarget{#1}{}
}

\usepackage{array}
\newcolumntype{L}[1]{>{\raggedright\let\newline\\\arraybackslash\hspace{0pt}}m{#1}}
\newcolumntype{C}[1]{>{\centering\let\newline\\\arraybackslash\hspace{0pt}}m{#1}}
\newcolumntype{R}[1]{>{\raggedleft\let\newline\\\arraybackslash\hspace{0pt}}m{#1}}


\newcommand{\modflowdate}{\today}
\newcommand{\currentmodflowversion}{\modflowdate}


% redefine equations, figures, and tables in body
\newcommand{\insection}{%
\renewcommand{\theequation}{\thesection--\arabic{equation}}%
\setcounter{equation}{0}%
\renewcommand{\thefigure}{\thesection--\arabic{figure}}%
\setcounter{figure}{0}%
\renewcommand{\thetable}{\thesection--\arabic{table}}%
\setcounter{table}{0}%
\renewcommand{\thepage}{\thesection--\arabic{page}}%
\setcounter{page}{1}%
}


% prepend appendix equations, figures, and tables with R 
% establish custom counter
\newcounter{appendixno}
\setcounter{appendixno}{0}

% redefine equations, figures, and tables in appendix
\newcommand{\inappendix}{%
\addtocounter{appendixno}{1}%
\renewcommand{\theequation}{\Alph{appendixno}--\arabic{equation}}%
\setcounter{equation}{0}%
\renewcommand{\thefigure}{\Alph{appendixno}--\arabic{figure}}%
\setcounter{figure}{0}%
\renewcommand{\thetable}{\Alph{appendixno}--\arabic{table}}%
\setcounter{table}{0}%
\renewcommand{\thepage}{\Alph{appendixno}--\arabic{page}}%
\setcounter{page}{1}%
}

% redefine equations, figures, and tables in references
\newcommand{\inreferences}{%
\renewcommand{\thepage}{R--\arabic{page}}%
\setcounter{page}{1}%
}


\def\tsc#1#2{\csdef{#1}{#2\xspace}}
\tsc{MF}{MODFLOW}
\tsc{mf}{MODFLOW~6}
\tsc{mff}{MODFLOW-2005}
\tsc{mfn}{MODFLOW-NWT}
\tsc{mfu}{MODFLOW-USG}

%siunitx english units
\DeclareSIUnit\feet{ft}

%generic command for creating matrix and vector notation \matr{}
% http://tex.stackexchange.com/questions/199789/which-bold-style-is-recommended-for-matrix-notation
\newcommand{\matr}[1]{\mathbf{#1}}
\newcommand{\mli}[1]{\mathit{#1}}

% table variables
\definecolor{Gray}{gray}{0.9}


% table environments
\newenvironment{ScenarioTable}[3]{% #1 = caption #2 = label #3 = external file
    \small
    \begin{longtable}[!htbp]{
                                          p{.1\linewidth-2\arraycolsep}
                                          p{.25\linewidth-2\arraycolsep}
                                          p{.3\linewidth-2\arraycolsep}
                                          p{.15\linewidth-2\arraycolsep}
                                          }
    \caption{#1} \label{#2} \\

    \hline \hline
    \rowcolor{Gray}
    \textbf{Scenario} & \textbf{Scenario Name} & \textbf{Parameter} & \textbf{Value}  \\
    \hline
    \endfirsthead

    \hline \hline 
    \rowcolor{Gray}
    \textbf{Scenario} & \textbf{Scenario Name} & \textbf{Parameter} & \textbf{Value}  \\
    \hline
    \endhead    

    \input{#3}
}{%
    \hline \hline
    \end{longtable}
    \normalsize
}

\newenvironment{StandardTable}[3]{% #1 = caption #2 = label #3 = external file
    \small
    \begin{longtable}[!htbp]{
                                          p{.5\linewidth-2\arraycolsep}
                                          p{.3\linewidth-2\arraycolsep}
                                          }
    \caption{#1} \label{#2} \\

    \hline \hline
    \rowcolor{Gray}
    \textbf{Parameter} & \textbf{Value}  \\
    \hline
    \endfirsthead

    \hline \hline 
    \rowcolor{Gray}
    \textbf{Parameter} & \textbf{Value} \\
    \hline
    \endhead    

    \input{#3}
}{%
    \hline \hline
    \end{longtable}
    \normalsize
}

\newenvironment{StandardFigure}[3]{% #1 = caption #2 = label #3 = figure file
\begin{figure}[!htbp]
	\begin{center}
		\includegraphics{#3}
		\caption{#1}
		\label{#2}
}{%
	\end{center}
\end{figure}
}



\title{\color{white} MODFLOW 6 -- Example problems}
\author{\color{white} MODFLOW 6 Development Team}
\date{\color{white} \currentmodflowversion}

\urlstyle{rm}

\makeatletter
\patchcmd{\@verbatim}
  {\verbatim@font}
  {\verbatim@font\footnotesize}
  {}{}
\makeatother

\renewcommand\bibname{References Cited}

\begin{document}

\raggedright
\onecolumn
\hbadness=10000
\setlength{\parindent}{1.5pc}

\ThisTileWallPaper{\paperwidth}{1.25\paperheight}{../images/coverimage}
\begingroup
\fontfamily{phv}\fontshape{sl}\selectfont
\maketitle
\endgroup

\newpage
\tableofcontents

\newpage
\listoffigures

\newpage
\listoftables

% -------------------------------------------------
\clearpage
\inintroduction
\section*{Introduction}
\addcontentsline{toc}{section}{Introduction}

This document describes \mf example scenarios. The examples demonstrate the capabilities of select components of \mf. Examples have been included for the \mf components summarized in table~\ref{tab:ex-table}.

\input{../tables/ex-table.tex}


% GWF Model examples
\clearpage
\insection
\section{TWRI Example}

% Describe source of problem
This example is a modified version of the original MODFLOW example (TWRI) described in \cite{modflow88} and duplicated in \cite{harbaugh1996user}. This problem is also is distributed with \mff~\citep{modflow2005} and \mf~\citep{modflow6software}. The problem has been modified from a quasi-3D problem, where confining beds are not explicitly simulated, to an equivalent three-dimensional problem.

\subsection{Example description}
% spatial discretization  and temporal discretization
There are three simulated aquifers, which are separated from each other by confining layers (fig.~\ref{fig:ex-gwf-twri-01}). The confining beds are 50 $ft$ thick and are explicitly simulated as model layers 2 and 4, respectively. Each layer is a square 75,000 $ft$ on a side and is divided into a grid with 15 rows and 15 column, which forms squares 5,000 $ft$ on a side. A single steady-stress period with a total length of 86,400 seconds (1 day) is simulated.

\begin{StandardFigure}{
                                     Illustration of the system simulated in the TWRI example problem (from \cite{modflow88}).
                                     }{fig:ex-gwf-twri-01}{../images/twri-system.png}
\end{StandardFigure}                                 

% material properties
The transmissivity of the middle and lower aquifers (fig.~\ref{fig:ex-gwf-twri-01}) was converted to a horizontal hydraulic conductivity using the layer thickness (table~\ref{tab:ex-gwf-twri-01}). The vertical hydraulic conductivity in the aquifers was set equal to the horizontal hydraulic conductivity. The vertical hydraulic conductivity of the confining units was calculated from the vertical conductance of the confining beds defined in the original problem and the confining unit thickness (table~\ref{tab:ex-gwf-twri-01}); the horizontal hydraulic conductivity of the confining bed was set to the vertical hydraulic conductivity and results in vertical flow in the confining unit.

% add static parameter table(s)
\input{../tables/ex-gwf-twri-01}


% initial conditions
An initial head of zero $ft$ was specified in all model layers. Any initial head exceeding the bottom of model layer 1 (-150 $ft$) could be specified since the model is steady-state.

% boundary conditions
Flow into the system is from infiltration from precipitation and was represented using the recharge (RCH) package. A constant recharge rate of $3 \times 10^{-7}$ $ft/s$ was specified for every cell in model layer 1. Flow out of the model is from buried drain tubes represented by drain (DRN) package cells in model layer 1, discharging wells represented by well (WEL) package cells in all three aquifers, and a lake represented by constant head (CHD) packages cells in the unconfined and middle aquifers (fig.~\ref{fig:ex-gwf-twri-01}).

% for examples without scenarios
\subsection{Example Results}

Simulated results in the unconfined, middle, and lower aquifers are shown in figure~\ref{fig:ex-gwf-twri-02}. Simulated results for a quasi-3D \mff simulation are also shown in figure~\ref{fig:ex-gwf-twri-02}. \mf and \mff results differ by less that 0.05 $ft$ in any aquifer unit.

% a figure
\begin{StandardFigure}{
                                     Simulated water levels and normalized specific discharge vectors in 
                                     the unconfined, upper, and lower aquifers. 
                                     \textit{A}. \mf unconfined aquifer results.
                                     \textit{B}. \mf middle aquifer results.
                                     \textit{C}. \mf lower aquifer results.
                                     \textit{D}. \mff unconfined aquifer results.
                                     \textit{E}. \mff middle aquifer results.
                                     \textit{F}. \mff lower aquifer results.
                                     }{fig:ex-gwf-twri-02}{../figures/ex-gwf-twri01.png}
\end{StandardFigure}                                 



\clearpage
\insection
\section{Simulation of a Two-Layer Aquifer System that Converts between Wet and Dry (BCFSS2)}

% Describe source of problem
This example is a version of the original MODFLOW rewetting example (BCF2SS) described in \cite{mcdonaldetal1991wetdry}. This problem is also is distributed with \mff~\citep{modflow2005} and \mf~\citep{modflow6software}. The problem has been modified to use a vertical hydraulic conductivity that is equivalent to the original quasi-3D vertical conductance (VCONT) value used in the original problem.

In an aquifer system where two aquifers are separated by a confining bed, large pumpage withdrawals from the bottom aquifer can desaturate parts of the upper aquifer. If pumpage is discontinued, resaturation of the upper aquifer can occur. This problem demonstrates the capability of the Node Property Flow (NPF) Package to successfully simulate this common hydrologic situation which is difficult or impossible to simulate without use of the REWET option and the Standard Formulation or the Newton-Raphson Formulation.

\subsection{Conceptual Model}

The hypothetical aquifer system consists of two aquifers separated by a confining unit. No-flow boundaries surround the system on all sides, except that the lower aquifer discharges to a stream along the right side of the area (fig.~\ref{fig:ex-gwf-bcf2ss-grid}). Recharge from precipitation is applied evenly over the entire area. The stream penetrates the lower aquifer; in the region above the stream, the upper aquifer and confining unit are missing  \cite[see][figure~1]{mcdonaldetal1991wetdry}. Under natural conditions, recharge flows through the system to the stream. Under stressed conditions, two wells withdraw water from the lower aquifer. If enough water is pumped, cells in the upper aquifer will desaturate. Removal of the stresses will then cause the desaturated areas to resaturate.


\begin{StandardFigure}{
                                     Diagram showing the model domain. \textit{A}, plan view, and \textit{B}, cross-section 
                                     view. The locations of river cells, and wells are also shown. The steady-state water-table
                                     is shown in the cross-section view.
                                     }{fig:ex-gwf-bcf2ss-grid}{../figures/ex-gwf-bcf2ss-grid.png}
\end{StandardFigure}                                 

\subsection{Example Description}
% spatial discretization  and temporal discretization
The model consists of two layers--one for each aquifer. A uniform horizontal grid of 10 rows and 15 columns is used (fig.~\ref{fig:ex-gwf-bcf2ss-grid}). Two steady-state solutions were obtained to simulate natural conditions and pumping conditions. 

% material properties
Because horizontal flow in the confining bed is small compared to horizontal flow in the aquifers and storage is not a factor in steady-state simulations, the confining bed is not treated as a separate layer. A horizontal hydraulic conductivity of 10 and 5 $ft/day$ was specified for model layer 1 and 2, respectively (table~\ref{tab:ex-gwf-twri-01}); the horizontal conductivity of model layer 2 was calculated based on the transmissivity (500 $ft^2/day$) and 100 $ft$ layer thickness used in the original problem \citep{mcdonaldetal1991wetdry}. The vertical hydraulic conductivity of the confining units was calculated from the vertical conductance of the confining beds defined in the original problem ($0.9999999 \times 10^{-3}$ per day)  (table~\ref{tab:ex-gwf-twri-01}).

% add static parameter table(s)
\input{../tables/ex-gwf-bcf2ss-01}

% initial conditions and boundary conditions
An initial head of zero $ft$ was specified in all model layers. Flow into the system is from infiltration from precipitation and was represented using the recharge (RCH) package. A constant recharge rate of $4 \times 10^{-3}$ $ft/day$ was specified for every cell in model layer 1. Flow out of the model is from a stream represented by river (RIV) package cells in model layer 2, discharging wells represented by well (WEL) package cells in the lower aquifer (fig.~\ref{fig:ex-gwf-bcf2ss-grid}).

% wetdry properties

In the process of simulating this problem, several trial simulations were made using different values for the WETDRY parameter. The absolute value of the WETDRY parameter is the wetting threshold, and the sign of the WETDRY parameter indicates which neighboring cells can cause a cell to become wet. Determination of the WETDRY parameter often requires considerable effort. The user may have to make multiple test runs trying different values in different areas of the model. On the right side of the model, the
WETDRY parameter (table 1) is negative in order to cause a cell to become wet only when head in the layer below exceeds the wetting threshold. This was done to avoid incorrectly converting dry cells to wet because of the large head differences between adjacent horizontal cells. For example, the simulation of natural conditions (Stress Period 1) shows cells in column 14 of layer 1 being dry, which is reasonable based on the head below these cells. That is, the head in column 14 of layer 2 is over 20 feet below the bottom of layer 1. However, the head in column 13 of layer 1 is 21 feet above the bottom of the aquifer, which means that, if head in adjacent horizontal cells is allowed to wet cells, column 14 would convert to wet. Thus, it is not readily apparent whether column 14 should be wet or dry. The trial simulations showed that, when horizontal wetting is allowed, column 14 repeatedly oscillates between wet and dry, indicating that column 14 should be dry. If horizontal wetting is used, oscillation between wet and dry can be prevented by raising the wetting threshold, but this also can prevent some cells that should be partly saturated from converting to wet.

On the left side of the model, horizontal head changes between adjacent cells generally are small, so head in the neighboring horizontal cells is a good indicator of whether or not a dry cell should become wet. Therefore, positive WETDRY parameters are used in most of this area to allow wetting to occur either from the cell below or from horizontally adjacent cells. Near the well, the horizontal head gradients under pumping conditions also are relatively large; consequently, a negative WETDRY parameter was used at the cells above the well. This prevents these cells from incorrectly becoming wet. It is also possible to use a larger positive wetting threshold to prevent these cells from incorrectly becoming wet. The WETDRY threshold used in model layer 1 is shown in figure~\ref{fig:ex-gwf-bcf2ss-01}.

\begin{StandardFigure}{
                                     WETDRY threshold values used in the upper aquifer.
                                     }{fig:ex-gwf-bcf2ss-01}{../figures/ex-gwf-bcf2ss-01.png}
\end{StandardFigure}                                 


% for examples without scenarios
\subsection{Example Results}

Two steady-state solutions were obtained to simulate natural conditions and pumping conditions. The two solutions are designed to demonstrate the ability of the BCF2 Package to handle a broad range of possibilities for cells converting between wet and dry in the top aquifer. When solving for natural conditions, the top aquifer initially is specified as being entirely dry and many cells must convert to wet. When solving for pumping conditions, the top aquifer is initially specified to be under natural conditions and many cells must convert to dry.

The steady-state solutions were obtained through a single simulation consisting of two stress periods. The first stress period simulates natural conditions and the second period simulates the addition of pumping wells. The simulation is declared to be steady state, so no storage values are specified and each stress period requires only a single time step to produce a steady-state result. The PCG2 Package is used to solve the flow equations for the simulations. The complete output from the model simulation is provided in Table 1.

\begin{StandardFigure}{
                                     Simulated water levels and normalized specific discharge vectors in 
                                     the upper and lower aquifers under natural and pumping conditions. 
                                     \textit{A}. Upper aquifer results under natural conditions.
                                     \textit{B}. Lower aquifer results under natural conditions
                                     \textit{C}. Upper aquifer results under pumping conditions.
                                     \textit{D}. Lower aquifer results under pumping conditions
                                     }{fig:ex-gwf-bcf2ss-02}{../figures/ex-gwf-bcf2ss-02.png}
\end{StandardFigure}                                 



\clearpage
\insection
\section{Toth Problem}

% Describe source of problem
A classic conceptual model of groundwater flow in small drainage basins is described by \cite{toth1963}.  Using a mathematical model of cross-sectional groundwater flow in response to an imposed sinusoidally varying water table, Toth postulated the presence of local, intermediate, and regional groundwater flow systems.  In his classic paper, Toth showed the different types of flow patterns resulting from different water table configurations, domain sizes, and aquifer properties.  This MODFLOW 6 example is intended to approximate the Toth flow system shown in Figure 2i of \cite{toth1963}.

\subsection{Example Description}
% spatial discretization  and temporal discretization
Model parameters for the example are summarized in table~\ref{tab:ex-gwf-toth-01}. Confined groundwater flow is simulated within a cross section that is 20,000 ft in the x direction and 10,000 ft in the z direction.  This model domain is discretized into 200 columns, 1 row, and 100 layers. The discretization is 200 ft in the row direction and 50 ft in the layer direction for all cells. A single steady-stress period with a total length of 1 day is simulated.

% add static parameter table(s)
\input{../tables/ex-gwf-toth-01}

A constant and isotropic hydraulic conductivity of 1.0 $ft/d$ was specified in all cells. An initial head equal to the domain top of 10,000 $ft$ was specified in all model cells. Constant head boundary cells were specified for all cells in model layer 1. The head values assigned for the constant-head cells were calculated from a linearly increasing sine function as reported by \cite{toth1963}.

% for examples without scenarios
\subsection{Example Results}

Simulated results for this variant of the Toth problem are shown in figure~\ref{fig:ex-gwf-toth}.  Head contours are shown as blue lines.  Contours of the stream function, calculated by accumulating lateral flows in each vertical column of cells, are shown as colored contours.  The head value prescribed to the top model layer is shown as a black sinusoidally varying line varying between elevations of 10,000 and 11,000 ft.

\begin{StandardFigure}{
                                     Simulation results for the classic Toth problem.
                                     Simulated contours of head are shown in blue.
                                     Contours of the stream function are shown as colored contours.
                                     The black line ranging in elevation between 10,000 and 11,000 ft 
                                     represents the prescribed head value assigned to the top of the model.
                                     }{fig:ex-gwf-toth}{../figures/ex-gwf-toth.png}
\end{StandardFigure}                                 




\clearpage
\insection
\section{Tidal Model}

% Describe source of problem
This example demonstrates use of \mf observations and time series and the capability to use multiple stress packages of the same type in a single groundwater flow model.  This is a synthetic example problem that has not been documented elsewhere.

\subsection{Example Description}

% spatial discretization  and temporal discretization
The problem consists of two aquifers, which are separated from each other by a confining layer. The upper aquifer is unconfined and the lower aquifer is confined.  The confining layer is 15 $m$ thick and is explicitly simulated as model layer 2. The grid consists of 15 rows and 10 columns.  Each cell is 500 $m$ on a side.  A single steady-stress period of 1 day is followed by three 10-day transient stress periods, each with 120 time steps.  Model parameters are listed in table~\ref{tab:ex-gwf-advtidal-01}. 

% add static parameter table(s)
\input{../tables/ex-gwf-advtidal-01}

% initial conditions
An initial head of 50 $m$ was specified in all model layers. Any initial head exceeding the bottom of model layer 1 (5 $m$) could be specified since the model is steady-state.

% boundary conditions
The model demonstrates use of the GHB, WEL, RIV RCH, and EVT stress packages. Locations for these boundaries are shown in figure~\ref{fig:ex-gwf-advtidal-grid}. The GHB is used to apply a tidally varying boundary condition to the right side of the model in layers 2 and 3.  The GHB Package uses the time series capability to represent  tidally varying stage for these stress boundaries.  Stage values are linearly interpolated to each time step from a time series of tidal fluctuations.  The WEL Package also uses time series to change pumping rates by well according to a time series of pumping records.  The pumping rates are interpolated using the stepwise option, which indicates that rates are held constant at the specified value until a new value is specified.  This is an alternative to the linear interpolation method.  The WEL Package contains pumping rates specified with time series and pumping rates specified with a constant value for the stress period.  A simple RIV Package is also used with time series to change the river stage by time step.  

The RCH and EVT Packages are used to assign and calculate recharge and evapotranspiration, respectively.  Three separate RCH Packages are used in this example to assign a different recharge rate to each of the three zones shown in figure~\ref{fig:ex-gwf-advtidal-grid}D.

\begin{StandardFigure}{
                                     Model grid and boundary conditions used for the Tidal example problem: (A) river and well boundaries in layer 1;  (B) general-head boundaries in layer 2;  (C) general-head and well boundaries in layer 3; and (D) zones used to assign recharge to layer 1.
                                     }{fig:ex-gwf-advtidal-grid}{../figures/ex-gwf-advtidal-grid.png}
\end{StandardFigure}                                 

% for examples without scenarios
\subsection{Example Results}

The observation capability in \mf was used to extract time series of simulated heads and flows.  Time series of model results are shown in figures~\ref{fig:ex-gwf-advtidal-obs-head}, \ref{fig:ex-gwf-advtidal-obs-flow}, and \ref{fig:ex-gwf-advtidal-ghb-obs}.  

% a figure
\begin{StandardFigure}{
                                     Simulated groundwater head in model cell (1, 13, 8).
                                     }{fig:ex-gwf-advtidal-obs-head}{../figures/ex-gwf-advtidal-obs-head.png}
\end{StandardFigure}                                 

\begin{StandardFigure}{
                                     Simulated groundwater flow for model cell (1, 5, 6) and its connection with model cell (1, 6, 6).  Positive values indicate flow into model cell (1, 5, 6).
                                     }{fig:ex-gwf-advtidal-obs-flow}{../figures/ex-gwf-advtidal-obs-flow}
\end{StandardFigure}                                 

\begin{StandardFigure}{
                                     Simulated flow between general-head boundary cells and the groundwater flow model.  ESTUARY2 is the combined flow for all general-head cells in layer 2.  ESTUARY3 is the combined flow for all general-head cells in layer 3.  A positive value represents from from the general-head boundary into the groundwater model.
                                     }{fig:ex-gwf-advtidal-ghb-obs}{../figures/ex-gwf-advtidal-ghb-obs.png}
\end{StandardFigure}                                 


\clearpage
\insection
\section{Flow and Head Boundary (FHB) Package Replication}

% Describe source of problem
This example shows how the time series capability in \mf can be combined with the constant-head (CHD) and Well (WEL) Packages to replicate the capabilities of the Flow and Head Boundary (FHB) Package in previous versions of MODFLOW.  This synthetic example problem has been released with previous MODFLOW versions, such as MODFLOW-2005 \citep{modflow2005} and was first described by \cite{leake1997documentation}.

\subsection{Example Description}

% spatial discretization  and temporal discretization
The problem consists of a very simple single-layer model representing a confined aquifer.  The grid consists of 3 rows and 10 columns.  Each cell is 1000 $m$ on a side.  There are three transient stress periods with lengths of 400, 200, and 400 days.   There are 10, 4, and 6 time steps per stress period.  Model parameters are listed in table~\ref{tab:ex-gwf-fhb-01}. 

% add static parameter table(s)
\input{../tables/ex-gwf-fhb-01}

% initial conditions
An initial head of 0 $m$ was specified for the model.  The value is important as the model begins with a transient stress period.

% boundary conditions
The model demonstrates use of the CHD and WEL packages.  Locations for these boundaries are shown in figure~\ref{fig:ex-gwf-fhb-grid}. Both of these packages use time varying values for the constant head and the well flow rate.   

\begin{StandardFigure}{
                                     Model grid and boundary conditions used for the FHB example problem.
                                     }{fig:ex-gwf-fhb-grid}{../figures/ex-gwf-fhb-grid.png}
\end{StandardFigure}                                 

% for examples without scenarios
\subsection{Example Results}

The observation capability in \mf was used to extract time series of simulated heads and flows.  Time series of model results are shown in figures~\ref{fig:ex-gwf-fhb-obs-head} and \ref{fig:ex-gwf-fhb-obs-flow}.  

% a figure
\begin{StandardFigure}{
                                     Simulated groundwater head in model cells (1, 2, 1) and (1, 2, 10).
                                     }{fig:ex-gwf-fhb-obs-head}{../figures/ex-gwf-fhb-obs-head.png}
\end{StandardFigure}                                 

\begin{StandardFigure}{
                                     Simulated groundwater flow for model cell (1, 2, 2) and its connection with model cell (1, 2, 1).  Positive values indicate flow into model cell (1, 2, 2).
                                     }{fig:ex-gwf-fhb-obs-flow}{../figures/ex-gwf-fhb-obs-flow}
\end{StandardFigure}                                 



\clearpage
\insection
\section{Nested Grid Problem}

% Describe source of problem
This example reproduces the nested grid problem described in the MODFLOW-USG documentation \citep{modflowusg}.  The problem is recreated here using the Discretization by Vertices (DISV) input and is run with the standard groundwater flow formulation and the XT3D  formulation \citep{modflow6xt3d}, which provides more accurate results for nested grids.  

\subsection{Example Description}
% spatial discretization  and temporal discretization
The problem consists of a nested grid as shown in figure~\ref{fig:ex-gwf-u1disv-grid}.  The outer grid has cells that are 100 $m$ on a side; the nested grid has cells with sides that are 1/3 this length.  Cells and vertices are numbered in figure~\ref{fig:ex-gwf-u1disv-grid} and can be compared to the vertices and cell information listed in the input file for the DISV Package.  The top of the model is uniform and set to zero (table~\ref{tab:ex-gwf-u1disv-01}).  The bottom of the model is also uniform, and is set to -100 $m$.  The simulation consists of a single steady-state stress period with a length of one day.  The simulation starts with an initial head of zero. 

Constant-heads are assigned a value of 1.0 $m$ on the left side of the model and 0.0 $m$ on the right side of the model.  


\begin{StandardFigure}{
                                     Model grid used for the nested grid problem.  Constant-head cells are marked in blue.  Cell numbers are shown inside each model cell.  Vertices are also numbered and are shown in red.  
                                     }{fig:ex-gwf-u1disv-grid}{../figures/ex-gwf-u1disv-grid.png}
\end{StandardFigure}                                 


% add static parameter table(s)
\input{../tables/ex-gwf-u1disv-01.tex}

% for examples without scenarios
\subsection{Scenario Results}

The nested grid problem was run with the standard groundwater flow formulation and the XT3D formulation \citep{modflow6xt3d} (table~\ref{tab:ex-gwf-u1disv-scenario}). 

% scenario table
\input{../tables/ex-gwf-u1disv-scenario.tex}

\subsubsection{Standard Groundwater Flow Formulation}

Model results for the simulation with the standard groundwater flow formulation are shown in figure~\ref{fig:ex-gwf-u1disv-head}.  Flow is from left to right and should be perfectly one dimensional.  The head surface should represent a flat plane with a value of 1.0 on the left and zero on the right.  Because the configuration of a nested grid violates the control-volume finite-difference assumptions, there are errors in the simulated heads as shown in figure~\ref{fig:ex-gwf-u1disv-head}B.  Head errors are larger than solution tolerances.

\begin{StandardFigure}{
                                     Simulated head (A) and errors in simulated head (B) for the nested grid problem simulated with the standard groundwater flow formulation.
                                     }{fig:ex-gwf-u1disv-head}{../figures/ex-gwf-u1disv-head.png}
\end{StandardFigure}                                 

\subsubsection{XT3D}

Model results for the simulation with the XT3D groundwater flow formulation \citep{modflow6xt3d} are shown in figure~\ref{fig:ex-gwf-u1disv-x-head}.  In this simulation, the XT3D formulation gives a much better solution than the standard groundwater flow formulation.  Errors in simulated head (figure~\ref{fig:ex-gwf-u1disv-x-head}B) are smaller than the tolerances used for the problem.  

% a figure
\begin{StandardFigure}{
                                     Simulated head (A) and errors in simulated head (B) for the nested grid problem simulated with the XT3D groundwater flow formulation \citep{modflow6xt3d}.
                                     }{fig:ex-gwf-u1disv-x-head}{../figures/ex-gwf-u1disv-x-head.png}
\end{StandardFigure} 

\clearpage
\insection
\section{Nested Grid Problem, Two Domains}

% Describe source of problem
This example reproduces the nested grid problem described in the MODFLOW-USG documentation \citep{modflowusg}. A single model setup with a grid based on the Discretization by Vertices (DISV) input is presented elsewhere in these examples. This problem is set up using two individual GWF models with a regular grid (DIS) that are coupled through a GWF Exchange. A plan view of the combined grid for the two models is shown in figure~\ref{fig:ex-gwf-u1gwfgwf-s1-grid}. The XT3D option in the NPF package can be applied to avoid inaccuracies at the cell refinement interface \citep{modflow6xt3d}, which is the model boundary in this example. However, for this coupled system it is not sufficient to enable XT3D for the models independently: the correct flow calculation around the model interface relies on information from both models.

\begin{StandardFigure}{A top view of the grids of the outer and inner model used in this example. The dashed red line indicates the interface between the two. The blue shaded areas are the cells with a constant head boundary condition.}{fig:ex-gwf-u1gwfgwf-s1-grid}{../figures/ex-gwf-u1gwfgwf-s1-grid.png}
\end{StandardFigure}

\begin{StandardFigure}{Flow calculation stencils for XT3D for the coupled model system. Details in the text.}{fig:ex-gwf-u1gwfgwf-s1-stencils}{../figures/ex-gwf-u1gwfgwf-s1-stencils.png}
\end{StandardFigure} 

This is illustrated in more detail in figure~\ref{fig:ex-gwf-u1gwfgwf-s1-stencils}. The red dots show (examples of) cell connections where the flow relies on data from both models. The cells which are involved in the flow calculation are colored blue. In the left panel this is the case for flows that cross the model boundary. On the right it is shown how interior connections can still be dependent on cell data from the neighboring model. With the release of the generalized coupling framework in \mf (as of version 6.3.0) it is now possible to activate XT3D not just for the internal model connections, but also for connections between models. Additionally, it will correctly calculate the XT3D fluxes near the model boundary using data from both models (c.f. the right panel in figure~\ref{fig:ex-gwf-u1gwfgwf-s1-stencils}). In this example problem we study how these options affect the accuracy of the simulation results.


\subsection{Example Description}
% spatial discretization  and temporal discretization
This is a typical Local Grid Refinement (LGR) example with the coarse outer cells ($100 m \times 100 m$) being part of one model and the refined inner cells part of the other. Some essential and uniform parameters are given in table~\ref{tab:ex-gwf-u1gwfgwf-01}. The two models are connected by a GWF Exchange which enables groundwater flow through the grid faces marked by the dashed red square. The blue cells indicate where a constant head boundary condition is imposed. The condition is set to 1$m$ for the cells on the left and to 0$m$ for the right. As a result, the analytical solution of the problem is trivial and given by the expression
\begin{equation}
	h = 1.0 - \frac{x - 50.0}{600.0} m	
\label{ex-gwf-u1gwfgwf-eq1}
\end{equation}
for the head $h$ and $x \in (50.0,650.0)$. This formula will be used to test the accuracy of the simulated results presented below.
                               

% add static parameter table(s)
\input{../tables/ex-gwf-u1gwfgwf-01.tex}

% for examples without scenarios
\subsection{Scenario Results}

The nested grid problem was run for 4 different scenarios using the parameter configuration listed in table~\ref{tab:ex-gwf-u1gwfgwf-scenario}. 

% scenario table
\input{../tables/ex-gwf-u1gwfgwf-scenario.tex}

Model results for the simulation of these scenarios are shown in figures~\ref{fig:ex-gwf-u1gwfgwf-s1-head}, \ref{fig:ex-gwf-u1gwfgwf-s2-head}, \ref{fig:ex-gwf-u1gwfgwf-s3-head}, \ref{fig:ex-gwf-u1gwfgwf-s4-head}. Flow is from left to right and should be perfectly one-dimensional. The head surface should represent a flat plane with a value of 1.0 on the left and zero on the right, following the analytical expression given in equation~\ref{ex-gwf-u1gwfgwf-eq1}. 

Because the configuration of a nested grid with a refinement violates the control-volume finite-difference assumptions, there are errors in the simulated heads for scenario 1 with the standard NPF flow formulation, as shown in figure~\ref{fig:ex-gwf-u1gwfgwf-s1-head}B. Enabling the XT3D method in both models as done in scenario 2, is insufficient to get rid of these accuracies as can be seen from figure~\ref{fig:ex-gwf-u1gwfgwf-s2-head}B. Scenario 3 applies the advanced XT3D calculation globally on both models \emph{and} at the interface. This setup is now equivalent to what was presented in the “Nested Grid Problem” elsewhere in these examples for the case of a single DISV-based model with XT3D enabled. And as expected, figure~\ref{fig:ex-gwf-u1gwfgwf-s3-head}B illustrates how in this case the deviation from the analytical result is well within the solver tolerance ($h_\textrm{close}=1 \times 10^{-9}$). Scenario 4 is a new capability of the generalized coupling and allows to apply XT3D where it is needed: in the exchange region between the models (c.f. figure~\ref{fig:ex-gwf-u1gwfgwf-s1-stencils}B). Because the XT3D calculation is quite costly, this is an efficient alternative to the setup in scenario 3 which leads to the same level of accuracy. The latter is clearly demonstrated by the error plot in figure~\ref{fig:ex-gwf-u1gwfgwf-s4-head}B.

\begin{StandardFigure}{
                                     Results for scenario 1: simulated head (A) and differences in head with respect to the analytical result (B)  for a system of coupled models without XT3D.
                                     }{fig:ex-gwf-u1gwfgwf-s1-head}{../figures/ex-gwf-u1gwfgwf-s1-head.png}
\end{StandardFigure}  

\begin{StandardFigure}{
                                     Results for scenario 2: simulated head (A) and differences in head with respect to the analytical result (B)  for a system of coupled models with XT3D active in both models but not at the exchange between them.
                                     }{fig:ex-gwf-u1gwfgwf-s2-head}{../figures/ex-gwf-u1gwfgwf-s2-head.png}
\end{StandardFigure}

\begin{StandardFigure}{
                                     Results for scenario 3: simulated head (A) and differences in head with respect to the analytical result (B)  for a system of coupled models with XT3D active in both models and at the exchange between them.
                                     }{fig:ex-gwf-u1gwfgwf-s3-head}{../figures/ex-gwf-u1gwfgwf-s3-head.png}
\end{StandardFigure}

\begin{StandardFigure}{
                                     Results for scenario 4: simulated head (A) and differences in head with respect to the analytical result (B)  for a system of coupled models without XT3D in the models but with XT3D enabled at the exchange between them.
                                     }{fig:ex-gwf-u1gwfgwf-s4-head}{../figures/ex-gwf-u1gwfgwf-s4-head.png}
\end{StandardFigure}


\clearpage
\insection
\section{\mfn Problem 2}

% Describe source of problem
This example is based on problem 2 in \cite{modflownwt} which used the Newton-Raphson formulation to simulate dry cells under a recharge pond. This problem is also described in \cite{mcdonaldetal1991wetdry} and used the \MF rewetting option to rewet dry cells.

\subsection{Example Description}

The simulation represents a rectangular, unconfined aquifer with a deep water table. The model uses symmetry to simplify the problem by simulating one-quarter of the pond and the downgradient model domain (fig.~\ref{fig:ex-gwf-nwt-p02}). Model parameters for the example are summarized in table~\ref{tab:ex-gwf-nwt-p02-01} 

\begin{StandardFigure}{
                                     Hydrogeology, model grid, and model boundary conditions
                                     \cite[from][]{mcdonaldetal1991wetdry}.
                                     }{fig:ex-gwf-nwt-p02}{../images/ex-gwf-nwt-p02.png}
\end{StandardFigure}                                 


% spatial discretization  and temporal discretization
The model consists of a grid of 40 columns, 40 rows, and 14 layers. The model domain is  5,000 $ft$ in the x- and y-directions. The discretization is 125 $ft$ in the row and column direction for all cells. The upper model layer is 15 $ft$ thick and the remaining model layers (layers 2 through 14) are 5 $ft$ thick. Four stress periods are simulated. The first three stress periods are transient and the last stress period is steady state. The stress periods are 190, 518, 1921, and 1 days in length and are broken up into 10, 2, 17, and 1 time steps of equal length. The total simulation time at the end of the four stress periods are 190, 708, 2,630, and 2,631 days, respectively.

% add static parameter table(s)
\input{../tables/ex-gwf-nwt-p02-01}

% material properties
The horizontal hydraulic conductivity is 5 $ft/day$ and vertical hydraulic conductivity is 0.25 $ft/day$. The upper ten model layers are convertible and the lower four model layers are confined. The specific yield is 0.2 (unitless) and the specific storage is 0.0002 $1/day$. Unconfined and confined storage change is simulated in the upper ten model layers; confined storage change is simulated the lower four model layers.

% initial conditions and boundary conditions
A initial head of 25 $ft$ was specified in all model cells, which results in the upper 9 model layers being dry at the start of the simulation. Constant heads boundary condition cells with a specified value of 25 $ft$ were specified on the right and lower edges of the model in model layer 10 through 14. The pond area above the aquifer is approximately 6 acres and recharge is added to four cells in the upper left corner of the model (fig.~\ref{fig:ex-gwf-nwt-p02}). A constant recharge rate of 0.05 $ft/day$ is applied to the pond area and results in a total pond leakage rate equal to 12,500 $ft^3/day$ for the full model domain.


\subsection{Scenario Results}

Example model results are evaluated using the Newton-Raphson Formulation and the Standard Conductance Formulation with rewetting (table~\ref{tab:ex-gwf-nwt-p02-scenario}). Complex and simple complexity Iterative Model Solver options were used for the simulation using the Newton-Raphson formulation and the Standard Conductance Formulation with rewetting scenarios, respectively. Rewetting was only activated in the upper 9 layers. The pseudo-transient continuation option \citep{modflow6framework} was disabled in the Newton-Raphson Formulation scenario.

Water-table elevations were compared for four simulation times: 190 days; 708 days; 2,630 days; and at steady state (2,631 days). Water-table elevation in row 1 were very similar for the two solutions (fig.~\ref{fig:ex-gwf-nwt-p02-01}), with a maximum difference in head of 2.5 $ft$ directly under the pond (row 2, column 2). The mean absolute water-table error for the model domain ranged from 0.061 to 0.012 $ft$ (fig.~\ref{fig:ex-gwf-nwt-p02-01}). A portion of the difference between the two scenarios is likely a result of the upstream horizontal conductance weighting used with the Newton-Raphson formulation. 

% scenario table
\input{../tables/ex-gwf-nwt-p02-scenario.tex}

\begin{StandardFigure}{
                                      Comparison of water-table elevations simulated using the Newton-Raphson
                                      Formulation and Standard Conductance Formulation with rewetting.
                                      Water-table altitudes in row 1 are shown for \textit{A}, 190 days, \textit{B}, 
                                      708 days, \textit{C}, 2,630 days, and \textit{D}, at steady state.
                                      The location of the uppermost constant head boundary cell is also shown.
                                     }{fig:ex-gwf-nwt-p02-01}{../figures/ex-gwf-nwt-p02-01.png}
\end{StandardFigure} 



\clearpage
\insection
\section{\mfn Problem 3}

% Describe source of problem
This example is based on problem 3 in \cite{modflownwt} which used the Newton-Raphson formulation to simulate water levels in a rectangular, unconfined aquifer with a complex bottom elevation and receiving areally distributed recharge. This problem provides a good example of the utility of Newton-Raphson Formulation for solving problems with wetting and drying of cells.

\subsection{Example Description}

Model parameters for the example are summarized in table~\ref{tab:ex-gwf-nwt-p03-01}. 

% add static parameter table(s)
\input{../tables/ex-gwf-nwt-p03-01}

% spatial discretization  and temporal discretization
The model consists of a grid of 40 columns, 40 rows, and 1 layer. The model domain is  8,000 $m$ in the x- and y-directions. The discretization is 100 $m$ in the row and column direction for all cells. The top of the model is specified to be 200 $m$ and the bottom of the model ranges from about 4 to 80 $m$ (fig.~\ref{fig:ex-gwf-nwt-p03-grid}). A single steady-state stress period, 365 days in length, with a single time step is simulated.

\begin{StandardFigure}{
                                     Distribution of layer-bottom elevations. The location of constant
                                     head boundary cells is also shown.
                                     }{fig:ex-gwf-nwt-p03-grid}{../figures/ex-gwf-nwt-p03-grid.png}
\end{StandardFigure}                                 

% material properties and initial conditions
The horizontal hydraulic conductivity is 1 $m/day$ and each cell is convertible. A initial head 20 $m$ above the cell bottom was specified in all model cells.

% initial conditions and boundary conditions
Constant heads boundary condition cells with a specified value of 24 $m$ are specified in column 80 for rows 46 through 48 (fig.~\ref{fig:ex-gwf-nwt-p03-grid}). Recharge that is a function of the bottom elevation is specified for each active cells; two different recharge distributions are specified and are discussed further below.

% solution 
Newton under-relaxation to maintain water-levels above the aquifer bottom \citep{modflow6framework}. The simple complexity Iterative Model Solver option and preconditioned bi-conjugate gradient stabilized linear accelerator was used for both scenarios.

\subsection{Scenario Results}

The model was evaluated using high and low recharge rates (fig.~\ref{fig:ex-gwf-nwt-p03-01}). Low recharge rates are 3 orders of magnitude less than the high recharge rates and simulate more arid conditions. Simulation results with large recharge rates are shown in figure~\ref{fig:ex-gwf-nwt-p03-02}. The entire model domain is saturated and simulated saturated thickness ranges between zero and 25 $m$ (fig.~\ref{fig:ex-gwf-nwt-p03-02}\textit{B}). Simulation results with low recharge rates are shown in figure~\ref{fig:ex-gwf-nwt-p03-03}. Only a small portion of the model domain has heads that are significantly above the cell bottom. However, all cells in the simulation have a non-zero saturated thickness (greater than 2 $mm$), which allows the applied recharge to flow horizontally toward the constant-head boundaries at the outlet of the aquifer.

\begin{StandardFigure}{
                                      Distribution of groundwater recharge applied in the model scenarios.  
                                      \textit{A}, higher recharge rates, and \textit{B}, lower recharge rates. 
                                      The location of constant head boundary cells is also shown.
                                     }{fig:ex-gwf-nwt-p03-01}{../figures/ex-gwf-nwt-p03-01.png}
\end{StandardFigure} 

\begin{StandardFigure}{
                                      Distribution of heads and saturated thicknesses for the high recharge
                                      scenario. \textit{A}, heads, and \textit{B}, saturated thicknesses.
                                      The location of constant head boundary cells is also shown.
                                     }{fig:ex-gwf-nwt-p03-02}{../figures/ex-gwf-nwt-p03-02.png}
\end{StandardFigure} 



\begin{StandardFigure}{
                                      Distribution of heads and saturated thicknesses for the low recharge
                                      scenario. \textit{A}, heads, and \textit{B}, saturated thicknesses.
                                      The white regions indicate cells that have saturated thicknesses less
                                      than 1 $mm$. The location of constant head boundary cells is also shown.
                                     }{fig:ex-gwf-nwt-p03-03}{../figures/ex-gwf-nwt-p03-03.png}
\end{StandardFigure} 



\clearpage
\insection
\section{Zaidel Problem}

% Describe source of problem
One of the most challenging numerical cases for \MF arises from drying-rewetting problems often associated with abrupt changes in the elevations of impervious base of a thin unconfined aquifer. This problem simulates a discontinuous water table configuration over a stairway impervious base and flow between constant-head boundaries at the left and right sides of the model domain. This problem is based on the problems that compared the analytical solution of \cite{zaidel2013discontinuous}~to \mfn~\cite[see][figure~6]{zaidel2013discontinuous}.

\subsection{Example Description}
% spatial discretization  and temporal discretization
Model parameters for the example are summarized in table~\ref{tab:ex-gwf-zaidel-01}. The model consists of a grid of 200 columns, 1 row, and 1 layer and a bottom altitude of ranging from 20 to 0 m (fig.~\ref{fig:ex-gwf-zaidel-01}). The discretization is 5 m in the row direction and 1 m in the column direction for all cells. A single steady-stress period with a total length of 1 day is simulated.

% add static parameter table(s)
\input{../tables/ex-gwf-zaidel-01}

A constant horizontal hydraulic conductivity of 0.0001 $m/d$ was specified in all cells. An initial head of 23 $m$ was specified in all model cells. Constant head boundary cells were specified in column 1 and 200. The constant head value in column 1 is 23 $m$ and was used in all simulations. A constant head value of 1 and 10 $m$ was specified in column 200 based on the values evaluated by \cite{zaidel2013discontinuous}.


\begin{StandardFigure}{
                                     Discontinuous water table configuration over a multistep impervious base.
                                     Simulated results for the case where the constant head in column 200 is equal
                                     to 1 meter is shown. The impervious model base and the location of constant
                                     head boundary cells is also shown.
                                     }{fig:ex-gwf-zaidel-01}{../figures/ex-gwf-zaidel-01.png}
\end{StandardFigure}                                 


% for examples without scenarios
\subsection{Example Results}

Simulated results for the case with the constant head in column 200 equal to 1 $m$ and 10 $m$ are shown in figures~\ref{fig:ex-gwf-zaidel-01} and~\ref{fig:ex-gwf-zaidel-02}, respectively. Simulated results compare well with the results in \cite{zaidel2013discontinuous}.

% a figure
\begin{StandardFigure}{
                                     Discontinuous water table configuration over a multistep impervious base.
                                     Simulated results for the case where the constant head in column 200 is equal
                                     to 10 meters is shown. The impervious model base and the location of constant
                                     head boundary cells is also shown.
                                     }{fig:ex-gwf-zaidel-02}{../figures/ex-gwf-zaidel-02.png}
\end{StandardFigure}                                 



\clearpage
\insection
\section{Streamflow Routing Package Problem 1}

% Describe source of problem
This example is a modified version of the Streamflow Routing (SFR) Package described in \cite{modflowsfr1pack}. The problem has been modified by converting all of the SFR reaches to use rectangular channels.                               

\subsection{Conceptual Model}

The example represents a hypothetical problem of stream-aquifer interaction for an alluvial basin in a semiarid region in which recharge to the aquifer is primarily leakage from streams that enter the basin from mountains on the northwest, northeast, and southeast (fig.~\ref{fig:ex-gwf-sfr-p01-grid}).

\begin{StandardFigure}{
                                     Land surface and aquifer bottom elevations. 
                                     \textit{A}. Land surface elevation. The location of inactive cells 
                                     and cells with streamflow routing reaches are also shown.
                                     \textit{B}. Aquifer bottom elevations. The location of cells with 
                                     general-head and well boundaries are also shown.
                                     }{fig:ex-gwf-sfr-p01-grid}{../figures/ex-gwf-sfr-p01-grid.png}
\end{StandardFigure}   

The principal aquifer is unconsolidated deposits of mostly sand and gravel. The mountains consist of bedrock that is many times less permeable than the unconsolidated deposits. Upland areas adjacent to the basin contribute some recharge to the aquifer either as underflow through the perimeter bedrock or from intermittent channels that have small drainage areas. The southern stream is perennial across the valley. Groundwater flow trends in the same direction as the streams.


\subsection{Example Description}
% spatial discretization  and temporal discretization
Model parameters for the example are summarized in table~\ref{tab:ex-gwf-sfr-p01-01}.  The model consists of a grid of 10 columns, 15 rows, and 1 layer. The model domain is  50,000 $ft$ and 80,000 $ft$ in the x- and y-directions, respectively. The discretization is 5,000 $ft$ in the row and column direction for all cells. The top of the model ranges from about 1,000 to 1,100 $ft$ (fig.~\ref{fig:ex-gwf-sfr-p01-grid}\textit{A}) and the bottom of the model ranges from about 500 to 1,000 $ft$ (fig.~\ref{fig:ex-gwf-sfr-p01-grid}\textit{B}).

Three stress periods are simulated. The first stress period is steady state and the remaining stress periods are transient. The stress periods are 0, 50, and 50 years in length and are broken up into 1, 50, and 50 time steps. A time step multiplier of 1, 1.1, and 1.1 are used in stress periods 1 through 3. respectively.

% add static parameter table(s)
\input{../tables/ex-gwf-sfr-p01-01}

% material properties
The basin fill thickens toward the center of the valley and hydraulic conductivity of the basin fill is highest in the region of the stream channels. Hydraulic conductivity is 173 $ft/day$ ($2 \times 10^{-4}$ $ft/s$) in the vicinity of the stream channels and 35 $ft/day$ ($4 \times 10^{-4}$ $ft/s$) elsewhere in the alluvial basin. A constant specific storage value of $1 \times 10^{-6}$ ($1/day$) was specified throughout the alluvial basin. Specific yield is 0.2 (unitless) in the vicinity of the stream channels and 0.1 (unitless) elsewhere in the alluvial basin.

% initial conditions
An initial head of 1,050 $ft$ is specified in all model layers. Any initial head exceeding the bottom of each cell could be specified since the model is steady-state.

% boundary conditions
Flow into the system is from infiltration from precipitation and was represented using the recharge (RCH) package. Recharge rates applied to each cell ranged $2.5 \times 10^{-10}$ to $2 \times 10^{-9}$ $ft/s$, with lower rates in the vicinity of the stream channels and higher rates elsewhere in the alluvial basin. Flow out of the model is from groundwater evapotranspiration represented by evapotranspiration (EVT) package cells and discharging wells represented by well (WEL) package cells. Groundwater evapotranspiration occurs where depth to water is within 15 $ft$ of land surface, has a maximum rate of 3 $ft/yr$ at land surface, and is coincident with the valley lowland through which several streams flow. Wells are only active in the second stress period and were located in ten cells (rows 6 through 10 and columns 4 and 5) along the west side of the valley (fig.~\ref{fig:ex-gwf-sfr-p01-grid}\textit{B}). Each well extracted 10 $ft^{3}/s$ of groundwater for a total withdrawal rate of 100 $ft^{3}/s$ (about twice the steady-state ground-water inflow). Two general-head boundary cells were added in (row 13, column 1) and (row 14, column 8) with a specified head equal to 988 and 1,045 $ft$, respectively, and a constant conductance of 0.038 $ft^{2}/s$.

The streams in the model domain were represented using a total of 36 reaches. External inflows of 25, 10, and 100 $ft^{3}/s$ were specified for reach 1, 16, and 28, respectively. Reach 1 is located in (row 1, column 1), reach 16 is in (row 5, column 10), and reach 28 is in (row 14, column 9). Streamflow discharges from the model at the downstream end of reach 36 in (row 13, column 1). Reach widths were specified to be 12, 0, 5, 12, 55, and 40 $ft$ for reaches 1--9, 10--18, 19--22, 23--27, 28--30, and 31--36, respectively. The remaining streambed properties and stream dimensions used for each stream reach are the same as those used in 
 \cite{modflowsfr1pack} \cite[see][Table~1]{modflowsfr1pack}. Constant stage reaches were used to define the ditch represented by reaches 10--15 and ranged from approximately 1,075.5--1061.6 $ft$. A diversion from reach 4 to 10 was specified to represent managed inflows to the ditch. Ditch inflows were specified to be 10 $ft^{3}/s$ except if the downstream flow in reach 4 is less than the specified diversion rate; in cases where the downstream flow in reach 4 is less than the specified diversion rate all of the downstream flow in reach 4 is diverted to the ditch and the inflow to reach

% solution 
The model uses the Newton-Raphson Formulation. The simple complexity Iterative Model Solver option and preconditioned bi-conjugate gradient stabilized linear accelerator is also used.

% for examples without scenarios
\subsection{Example Results}

Simulated results for the initial steady-state stress period and at the end of the stress period with groundwater pumping (stress period 2) are shown in figure~\ref{fig:ex-gwf-sfr-p01-01}. Reach stage and downstream discharge were also evaluated for reach 4, 14, 27, and 36.

% a figure
\begin{StandardFigure}{
                                     Simulated water levels and normalized specific discharge vectors  
                                     under steady state and pumping conditions. 
                                     \textit{A}. steady-state results.
                                     \textit{B}. results after 50 years of pumping.
                                     }{fig:ex-gwf-sfr-p01-01}{../figures/ex-gwf-sfr-p01-01.png}
\end{StandardFigure}                                 

Simulated stage and flow for reach 4 in (row 3, column 4) is shown in figure~\ref{fig:ex-gwf-sfr-p01-02}\textit{A} and \textit{B}. Flow out decreased rapidly when pumping began, but the decrease slowed after only 3 years. The marked change in flow and stream depth was caused by a decline in ground-water levels relative to the head in the stream for all cells corresponding to all the upstream reaches. After 3 years, cells upstream of the reach 4 began to decline below the streambed causing the slope of the decline in flow to decrease. Flow in the stream no longer changed after about 9 years of pumping because the groundwater level in cells corresponding to the reaches upstream of reach 4 had declined below the streambed and the leakage rates had become constant. Once withdrawals ceased, flow began to increase in the last reach of segment 5 after about 21 yrs after pumping had ceased and after about 19 yrs in reach 4. The increase of flow in reach 4 during the recovery period was slower than the decrease in flow during the pumping period and largely was controlled by the gradual recovery of ground-water levels in areas distant from the pumping wells.

The last reach (reach 15) along the ditch (reaches 10--15) was used to illustrate how the option of specifying stream stage works when flow in a channel ceases and when flow commences again (fig.~\ref{fig:ex-gwf-sfr-p01-02}\textit{C} and \textit{D}). The stream stage and related depth is constant in reaches 10--15 as long as there is flow in the reach. Once flow in the reach ceases, the streambed elevation (depth = 0) is used for comparing head differences between the stream and groundwater. The slight lag between when flow out of the reach went to zero and when stream depth went to zero during the rapid decline was the result of all inflow into the reach leaking through the streambed. Inflow into the reach ceases during the following time step and consequently the entire reach became dry and stream depth became zero. The same lag occurred during the recovery period and was again caused by all inflow into the reach leaking through the streambed. Outflow from the last reach did not occur until inflow into the reach exceeded that which could leak through the streambed.

Flow in reach 27 (fig.~\ref{fig:ex-gwf-sfr-p01-02}\textit{E} and \textit{F}) is used to illustrate what happens when there is no inflow from upstream reaches, but the groundwater level in the corresponding model cell is higher than the elevation of the streambed. Flow
and stream depth decreased rapidly in the pumping (second stress) period and depth became zero after only 2 years, even though there was still a small quantity of outflow from the reach. The reason for stream depth being zero with minor outflow is that the only source of water to the reach is from ground-water leakage. When this occurs, stream depth at the midpoint of the reach
is set to zero and the stream head at the midpoint is equal to the streambed elevation.

In the steady-state stress period (stress period 1), reach 36 is generally gaining, but became losing during the pumping period (stress period 2) . Streamflow out of the reach and the depth decreased during the pumping period (fig.~\ref{fig:ex-gwf-sfr-p01-02}\textit{G} and \textit{H}) because inflow from upstream reaches declined and because leakage in the reach switched from flow out of the aquifer into the stream to flow from the stream into the aquifer. Although the water table declined in all cells corresponding to reaches 28--36, the water table did not decline below the streambed and consequently, streambed leakage did not become constant. Once pumping ceases, flow in the reach increases in a manner similar to how it decreased. The slight increase in flow after 70 years was due to an increase in flow from reaches upstream of reach 27. The response in reach 36 after pumping ceases in stress period 3 was much different than that for reach 4 and 27 (compare fig.~\ref{fig:ex-gwf-sfr-p01-02}\textit{H} with fig.~\ref{fig:ex-gwf-sfr-p01-02}\textit{B}) because groundwater levels did not decline below the bottom of the streambed beneath reaches 31--36 whereas they did beneath reaches 1--4. 

\begin{StandardFigure}{
                                     Simulated reach depth and downstream discharge at select reaches. 
                                     \textit{A}. Stage in reach 4.
                                     \textit{B}. Downstream discharge in reach 4.
                                     \textit{C}. Stage in reach 15.
                                     \textit{D}. Downstream discharge in reach 15.
                                     \textit{E}. Stage in reach 27.
                                     \textit{F}. Downstream discharge in reach 27.
                                     \textit{G}. Stage in reach 36.
                                     \textit{H}. Downstream discharge in reach 36.
                                     }{fig:ex-gwf-sfr-p01-02}{../figures/ex-gwf-sfr-p01-02.png}
\end{StandardFigure}                                 



\clearpage
\insection
\section{Modified Pinder-Sauer surface water/groundwater exchange problem}

% Describe source of problem
The modified Pinder-Sauer is a surface-water/groundwater exchange problem that is based on the problem of \cite{pinder1971numerical}, which has been used as a benchmark test for coupled surface-water/groundwater models \citep[\textit{e.g.,}][]{hughes2015modflow,swain1996coupled}. \cite{lal2001modification} modified the problem of \cite{pinder1971numerical} to make it easier to set up and to include a sinusoidal inflow hydrograph boundary condition at the upstream end of the surface-water system.

The analytical solution of \cite{lal2001modification} for the discharge at a distance $x$ from the upstream boundary is

\begin{equation} \label{lalanaleqn}
Q_a = 509.70 + 141.58 \exp \left( \frac{\hat{\lambda}_{1} x}{\Lambda} \right) \sin \left( f_{r} t + \frac{\hat{\lambda}_{2} x}{\Lambda} \right),
\end{equation}

\noindent where $f_r$ is the characteristic frequency of the system, $\hat{\lambda}_{1}$ is the amplitude decay constant, $\Lambda$ is the characteristic length related to the wave number of the water-level disturbance, and $\hat{\lambda}_{2}$ is a dimensionless wave number. The terms in \textbf{equation~\ref{lalanaleqn}} are calculated using model parameters such as  friction slope, reach sediment hydraulic conductivity, reach width, \textit{etc.} and are defined in \cite{lal2001modification}. For the case without aquifer exchange, $\hat{\lambda}_{1} = -4.779 \times 10^{-2}$, and $\hat{\lambda}_{2} = -0.3608$. With aquifer exchange, the appropriate values for the variables in \textbf{equation~\ref{lalanaleqn}} are $\hat{\lambda}_{1} = -0.1785$, $\Lambda = 4894.3$ m, $f_{r} = 3.49 \times 10^{-4}$ sec$^{-1}$, $\hat{\lambda}_{2} = -0.3409$, and all other variables are the same as the case without aquifer exchange. 

The model domain represents a flood plane that is 39,624 m long, 427 m across the valley, and underlain by an unconfined aquifer. The flood plane and underlying aquifer are surrounded by impermeable boundaries on all sides. The base of the aquifer is horizontal and specified to be at an elevation of 0.0 m. The flow direction in the model domain is along the long axis of the model from top to bottom.

A total of 65 rows, 15 columns, and 1 layer were used to discretize the model domain. A constant grid spacing of 609.61 m was used for each row. A grid spacing of 28.30 m was used for all columns except the center column (column 8); the grid spacing of column 8 was 30.48 m. The total simulation length was 24 hours and a constant time-step length of 5 minutes was used.

The hydraulic conductivity of the aquifer is 3.048$\times$10$^{-3}$ m/s. The specific storage and specific yield are 0.25, and 1$\times$10$^{-7}$ 1/s.
 
% add static parameter table(s)
\input{../tables/ex-gwf-sfr-pindersauer-01}

A single river channel is located at the center of the aquifer (column 8) parallel to the long axis of the model domain and is simulated using the kinematic-wave approximation option available in the Streamflow Routing (SFR) package. The channel has a bed slope of 0.001, a width of 30.48 m, and a Manning roughness coefficient of 0.03858 s/m$^{1/3}$. For the case with aquifer exchange, the leakage coefficient is $1.402 \times 10^{-4}$ sec$^{-1}$ and seepage is assumed to occur only from the bottom.

Initially, the saturated thickness of the aquifer is 67.05 and 27.43 m at the upstream and downstream ends of the aquifer, respectively. The initial reach stage in each reach was calculated using \textbf{equation~\ref{lalanaleqn}}, the bed elevation, and Mannings equation using reach parameters.


% add scenario table
\input{../tables/ex-gwf-sfr-pindersauer-scenario}

The sinusoidal flood hydrograph introduced in the upstream SFR reach is,
\nolinebreak
\begin{equation} \label{lalhydrographeqn}
Q = 509.70 + 141.58 \sin \left( \frac{2 \pi t}{T_p} \right),
\end{equation}
\noindent
where $T_p$ is the period of disturbance (sec.) and $t$ is the simulation time (sec.). A $T_p$ value of 5 hours (18,000 sec.) was used.

% for examples without scenarios
\subsection{Example Results}

Simulated relative stage and discharge results 15,240 m downstream of the top end of the model domain are shown in \textbf{figure~\ref{fig:ex-gwf-sfr-ps-obs}}. Relative stage and discharge results were calculated using the initial stage and discharge of 52.12 m and 509.70 m$^3$/s, respectively. For comparison, simulated relative stage and discharge for a simulation without aquifer exchange (leakage coefficient $= 0.00$ sec$^{-1}$) are also shown in  \textbf{figure~\ref{fig:ex-gwf-sfr-ps-obs}}. Analytical results calculated using \textbf{equation~\ref{lalanaleqn}} 15,240 m downstream of the top end of the model domain are also shown in \textbf{figure~\ref{fig:ex-gwf-sfr-ps-obs}}; SFR results are comparable to the analytical solution.


% a figure
\begin{StandardFigure}
	{(A) Simulated relative stage change 15,240 m downstream of the top end of the model domain with and without aquifer exchange (leakage) for the modified Pinder-Sauer problem. (B) Comparison of relative discharge change 15,240 m downstream of the top end of the model domain and the analytical solution of \cite{lal2001modification}  with and without aquifer exchange (leakage).}
	{fig:ex-gwf-sfr-ps-obs}{../figures/ex-gwf-sfr-pindersauer-observations.png}
\end{StandardFigure}





\clearpage
\insection
\section{Advanced Packages with MVR}

% Describe source of problem
The Advanced \mf Packages example problem is designed to demonstrate the combined use of the UZF, SFR and LAK Packages in \mf.  It originally appeared in the ``Getting Started'' pdf document in the \mf -docs.git repository, and is included here for completeness of the examples. 

Flows exchanged between the advanced packages (i.e., UZF, SFR, and LAK) and the WEL package are made with the MVR Package. The problem was adapted from a previously developed test problem documented in \cite{modflowsfr1pack} and \cite{modflowsfr2pack} and represents a developed basinfill aquifer in the northern Great Basin. Streamflow enters the valley from three perennial streams. Because the valley is much lower in altitude than the surrounding uplands, recharge in the valley floor is relatively low as compared to recharge from streams and lakes that receive flow from precipitation falling in the mountains. In the lowland areas, native plants use the ground water. Two lakes were added to the original test problem to illustrate the capabilities of the Lake Package (figure~\ref{fig:ex-gwf-sfr-p01b}).

% a figure
\begin{StandardFigure}
	{Plan view of lakes and stream reaches in the \mf model that demonstrates usage of the MVR package with the advanced packages (i.e., UZF, SFR, LAK) and includes a linkage with the WEL package.}
	{fig:ex-gwf-sfr-p01b}{../images/ex-gwf-sfr-p01b.png}
\end{StandardFigure}

The topography, and the locations of streams and lakes within the basin are shown in figure~\ref{fig:ex-gwf-sfr-p01b}. Note that unlike the streamflow routing package for MODFLOW-2005, the SFR Package for \mf does not support designation of stream segments. Stream networks now consist solely of reaches that are numbered consecutively from 1 to the total number of reaches, and all stream property information is input on a reach basis. The simulation also includes and general head boundary conditions (GHB Package) and 10 agricultural wells. Aquifer properties are provided in table~\ref{tab:ex-gwf-sfr-p01b-01}.

% add scenario table
\input{../tables/ex-gwf-sfr-p01b-01}

% for examples without scenarios
\subsection{Example Results}

Unlike the original UZF1 Package \citep{UZF} for MODFLOW-2005, the UZF Package for MODFLOW 6 can simulate unsaturated flow separately for each layer. Thus, rather than simulating homogenous unsaturated flow between the water table and land surface, the unsaturated zone in cells in different MODFLOW layers can be specified with different unsaturated hydraulic properties to represent vertical heterogeneity. Results from the UZF Package are shown below for the cell at row 5, column 2, layers 1 and 2 (the infiltration is shown for the cell in layer 1 while the recharge occurs in layer 2.  The moisture content is for the cell located in layer 1).  The unsaturated-zone package is ideal for simulating the delay between when water infiltrates at land-surface and recharges the saturated zone.  

% a figure
\begin{StandardFigure}
	{Simulated infiltration and recharge to and from the unsaturated-zone (left y-axis), respectively, for the cells located at row 5 column 2, including the moisture content of the cell in layer 1 (right y-axis).}
	{fig:ex-gwf-sfr-p01b-uz}{../figures/ex-gwf-sfr-p01b-uz.png}
\end{StandardFigure}

Additionally, after calculating any rejected infiltration based on the user-specified infiltration rate, the unsaturated-zone package will partition the net infiltration into evapotranspiration (ET), changes in soil moisture storage, and recharge as mentioned above.  For this problem, the MVR package is used to transfer pumped groundwater (WEL package) to land surface for simulating irrigation with groundwater.  Figure~\ref{fig:ex-gwf-sfr-p01b-mvr} shows the monthly totals of pumped water used for irrigation for each month of the simulation.  Figure~\ref{fig:ex-gwf-sfr-p01b-mvr} also shows the total runoff, primarily groundwater discharge to land surface (spring flow) that is transferred to the stream network (or lakes) using the MVR package.

% a figure
\begin{StandardFigure}
	{Monthly totals of pumped water from all 10 simulated wells transferred to UZF cells via the MVR package.  Total monthly runoff amounts transferred to the surface water network are also shown.}
	{fig:ex-gwf-sfr-p01b-mvr}{../figures/ex-gwf-sfr-p01b-mvr.png}
\end{StandardFigure}



\clearpage
\insection
\section{Lake Package Problem 1}

% Describe source of problem
This example is based on problem 1 in the Lake (LAK) Package for \mftk described in \cite{modflowlak3pack}.                                

\subsection{Conceptual Model}

The example represents a hypothetical problem of stream-aquifer interaction for an alluvial basin in a semiarid region in which recharge to the aquifer is primarily leakage from streams that enter the basin from mountains on the northwest, northeast, and southeast (fig.~\ref{fig:ex-gwf-lak-p01-grid}).

\begin{StandardFigure}{
                                     Land surface and aquifer bottom elevations. 
                                     \textit{A}. Land surface elevation. The location of inactive cells 
                                     and cells with streamflow routing reaches are also shown.
                                     \textit{B}. Aquifer bottom elevations. The location of cells with 
                                     general-head and well boundaries are also shown.
                                     }{fig:ex-gwf-lak-p01-grid}{../figures/ex-gwf-lak-p01-grid.png}
\end{StandardFigure}   

The principal aquifer is unconsolidated deposits of mostly sand and gravel. The mountains consist of bedrock that is many times less permeable than the unconsolidated deposits. Upland areas adjacent to the basin contribute some recharge to the aquifer either as underflow through the perimeter bedrock or from intermittent channels that have small drainage areas. The southern stream is perennial across the valley. Groundwater flow trends in the same direction as the streams.


\subsection{Example Description}
% spatial discretization  and temporal discretization
Model parameters for the example are summarized in table~\ref{tab:ex-gwf-lak-p01-01}.  The model consists of a grid of 10 columns, 15 rows, and 1 layer. The model domain is  50,000 $ft$ and 80,000 $ft$ in the x- and y-directions, respectively. The discretization is 5,000 $ft$ in the row and column direction for all cells. The top of the model is specified to be ranges from about 1,000 to 1,100 $ft$ (fig.~\ref{fig:ex-gwf-lak-p01-grid}\textit{A}) and the bottom of the model ranges from about 500 to 1,000 $ft$ (fig.~\ref{fig:ex-gwf-lak-p01-grid}\textit{B}).

Three stress periods are simulated. The first stress period is steady state and the remaining stress periods are transient. The stress periods are 0, 50, and 50 years in length and are broken up into 1, 50, and 50 time steps. A time step multiplier of 1, 1.1, and 1.1 are used in stress periods 1 through 3. respectively.

% add static parameter table(s)
\input{../tables/ex-gwf-lak-p01-01}

% material properties
The basin fill thickens toward the center of the valley and hydraulic conductivity of the basin fill is highest in the region of the stream channels. Hydraulic conductivity is 173 $ft/day$ ($2 \times 10^{-4}$ $ft/s$) in the vicinity of the stream channels and 35 $ft/day$ ($4 \times 10^{-4}$ $ft/s$) elsewhere in the alluvial basin. A constant specific storage value of $1 \times 10^{-6}$ ($1/day$) was specified throughout the alluvial basin. Specific yield is 0.2 (unitless) in the vicinity of the stream channels and 0.1 (unitless) elsewhere in the alluvial basin.

% initial conditions
An initial head of 1,050 $ft$ was specified in all model layers. Any initial head exceeding the bottom of each cell could be specified since the model is steady-state.

% boundary conditions
Flow into the system is from infiltration from precipitation and was represented using the recharge (RCH) package. Recharge rates applied to each cell ranged $2.5 \times 10^{-10}$ to $2 \times 10^{-9}$ from  of $3 \times 10^{-7}$ $ft/s$, with lower rates in the vicinity of the stream channels and higher rates elsewhere in the alluvial basin. Flow out of the model is from groundwater evapotranspiration represented by evapotranspiration (EVT) package cells and discharging wells represented by well (WEL) package cells. Groundwater evapotranspiration occurs where depth to water is within 15 $ft$ of land surface, has a maximum rate of 3 $ft/yr$ at land surface, and is coincident with the valley lowland through which several streams flow. Wells are only active in the second stress period and were located in ten cells (rows 6 through 10 and columns 4 and 5) along the west side of the valley (fig.~\ref{fig:ex-gwf-lak-p01-grid}\textit{B}). Each well extracted 10 $ft^{3}/s$ of groundwater for a total withdrawal rate of 100 $ft^{3}/s$ (about twice the steady-state ground-water inflow). Two general-head boundary cells were added in (row 13, column 1) and (row 14, column 8) with a specified head equal to 988 and 1,045 $ft$, respectively, and a constant conductance of 0.038 $ft^{2}/s$.

The streams in the model domain were represented using a total of 36 reaches. External inflows of 25, 10, and 100 $ft^{3}/s$ were specified for reach 1, 16, and 28, respectively. Reach 1 is located in (row 1, column 1), reach 16 is in (row 5, column 10), and reach 28 is in (row 14, column 9). Streamflow discharges from the model at the downstream end of reach 36 in (row 13, column 1). Reach widths were specified to be 12, 0, 5, 12, 55, and 40 $ft$ for reaches 1--9, 10--18, 19--22, 23--27, 28--30, and 31--36, respectively. The remaining streambed properties and stream dimensions used for each stream reach are the same as those used in 
 \cite{modflowsfr1pack} \cite[see][Table~1]{modflowsfr1pack}. Constant stage reaches were used to define the ditch represented by reaches 10--15 and ranged from approximately 1,075.5--1061.6 $ft$. A diversion from reach 4 to 10 was specified to represent managed inflows to the ditch. Ditch inflows were specified to be 10 $ft^{3}/s$ except if the downstream flow in reach 4 is less than the specified diversion rate; in cases where the downstream flow in reach 4 is less than the specified diversion rate all of the downstream flow in reach 4 is diverted to the ditch and the inflow to reach

% solution 
The model uses the Newton-Raphson Formulation. The simple complexity Iterative Model Solver option and preconditioned bi-conjugate gradient stabilized linear accelerator is also used.

% for examples without scenarios
\subsection{Example Results}

Simulated results for the initial steady-state stress period and at the end of the stress period with groundwater pumping (stress period 2) are shown in figure~\ref{fig:ex-gwf-lak-p01-01}. Reach stage and downstream discharge were also evaluated for reach 4, 14, 27, and 36.

% a figure
\begin{StandardFigure}{
                                     Simulated water levels and normalized specific discharge vectors  
                                     under steady state and pumping conditions. 
                                     \textit{A}. steady-state results.
                                     \textit{B}. results after 50 years of pumping.
                                     }{fig:ex-gwf-lak-p01-01}{../figures/ex-gwf-lak-p01-01.png}
\end{StandardFigure}                                 

Simulated stage and flow for reach 4 in (row 3, column 4) is shown in figure~\ref{fig:ex-gwf-lak-p01-02}\textit{A} and \textit{B}. Flow out decreased rapidly when pumping began, but the decrease slowed after only 3 years. The marked change in flow and stream depth was caused by a decline in ground-water levels relative to the head in the stream for all cells corresponding to all the upstream reaches. After 3 years, cells upstream of the reach 4 began to decline below the streambed causing the slope of the decline in flow to decrease. Flow in the stream no longer changed after about 9 years of pumping because the groundwater level in cells corresponding to the reaches upstream of reach 4 had declined below the streambed and the leakage rates had become constant. Once withdrawals ceased, flow began to increase in the last reach of segment 5 after about 21 yrs after pumping had ceased and after about 19 yrs in reach 4. The increase of flow in reach 4 during the recovery period was slower than the decrease in flow during the pumping period and largely was controlled by the gradual recovery of ground-water levels in areas distant from the pumping wells.

The last reach (reach 15) along the ditch (reaches 10--15) was used to illustrate how the option of specifying stream stage works when flow in a channel ceases and when flow commences again (fig.~\ref{fig:ex-gwf-lak-p01-02}\textit{C} and \textit{D}). The stream stage and related depth is constant in reaches 10--15 as long as there is flow in the reach. Once flow in the reach ceases, the streambed elevation (depth = 0) is used for comparing head differences between the stream and groundwater. The slight lag between when flow out of the reach went to zero and when stream depth went to zero during the rapid decline was the result of all inflow into the reach leaking through the streambed. Inflow into the reach ceases during the following time step and consequently the entire reach became dry and stream depth became zero. The same lag occurred during the recovery period and was again caused by all inflow into the reach leaking through the streambed. Outflow from the last reach did not occur until inflow into the reach exceeded that which could leak through the streambed.

Flow in reach 27 (fig.~\ref{fig:ex-gwf-lak-p01-02}\textit{E} and \textit{F}) is used to illustrate what happens when there is no inflow from upstream reaches, but the groundwater level in the corresponding model cell is higher than the elevation of the streambed. Flow
and stream depth decreased rapidly in the pumping (second stress) period and depth became zero after only 2 years, even though there was still a small quantity of outflow from the reach. The reason for stream depth being zero with minor outflow is that the only source of water to the reach is from ground-water leakage. When this occurs, stream depth at the midpoint of the reach
is set to zero and the stream head at the midpoint is equal to the streambed elevation.

In the steady-state stress period (stress period 1), reach 36 is generally gaining, but became losing during the pumping period (stress period 2) . Streamflow out of the reach and the depth decreased during the pumping period (fig.~\ref{fig:ex-gwf-lak-p01-02}\textit{G} and \textit{H}) because inflow from upstream reaches declined and because leakage in the reach switched from flow out of the aquifer into the stream to flow from the stream into the aquifer. Although the water table declined in all cells corresponding to reaches 28--36, the water table did not decline below the streambed and consequently, streambed leakage did not become constant. Once pumping ceases, flow in the reach increases in a manner similar to how it decreased. The slight increase in flow after 70 years was due to an increase in flow from reaches upstream of reach 27. The response in reach 36 after pumping ceases in stress period 3 was much different than that for reach 4 and 27 (compare fig.~\ref{fig:ex-gwf-lak-p01-02}\textit{H} with fig.~\ref{fig:ex-gwf-lak-p01-02}\textit{B}) because groundwater levels did not decline below the bottom of the streambed beneath reaches 31--36 whereas they did beneath reaches 1--4. 


\clearpage
\insection
\section{Lake Package Problem 2}

% Describe source of problem
This example is based on problem 2 in the Lake (LAK) Package for \mftk described in \cite{modflowlak3pack}. The example represents a two lakes connected by a stream surrounded by a surficial aquifer (fig.~\ref{fig:ex-gwf-lak-p02-grid}).                                

\begin{StandardFigure}{
                                     Lateral grid discretization. The location of lake and constant head boundary
                                     cells are also shown. Simulated heads and lake stage at the end of stress 
                                     period 1, normalized specific discharge vectors, and the location of three 
                                     observation locations are also shown.
                                     }{fig:ex-gwf-lak-p02-grid}{../figures/ex-gwf-lak-p02-grid.png}
\end{StandardFigure}   


\subsection{Example Description}
% spatial discretization  and temporal discretization
Model parameters for the example are summarized in table~\ref{tab:ex-gwf-lak-p02-01}.  The model consists of a grid of 17 columns, 27 rows, and 5 layers. The model domain is 13,000 and 20,500 $ft$ in the x- and y-directions, respectively. The discretization is in the row and column directions ranges from 250 to 1,000 $ft$ and is 500 $ft$ where the lakes are located (fig.~\ref{fig:ex-gwf-lak-p02-grid}). The top of the model is specified to be 200 $ft$ and the bottom of each layer is specified to be 102, 97, 87, 77, and 67 $ft$. Groundwater flow was inactivated in the location of the lakes in model layers 1 and 2 by specifying an \texttt{IDOMAIN} value of zero in these cells.

One transient stress period 1,500 days in length is simulated. The stress period has 200 stress periods and uses a time step multiplier equal to 1.005 that results in time step lengths that range 4.40 to 11.82 days.

% add static parameter table(s)
\input{../tables/ex-gwf-lak-p02-01}

% material properties and initial conditions
The horizontal and vertical hydraulic conductivity is 30 $ft/d$. A constant specific storage value of $3 \times 10^{-4}$ ($1/day$) and specific yield of 0.2 (unitless) were specified. All model layers were specified to convert between confined and unconfined conditions. An initial head of 115 $ft$ was specified in all model layers.

% boundary conditions
Flow into the system is from infiltration from precipitation and was represented using the recharge (RCH) package and a constant recharge rate of 0.0116 $ft/d$. Flow out of the model is from groundwater evapotranspiration represented by evapotranspiration (EVT) package cells. Groundwater evapotranspiration occurs where depth to water is within 15 $ft$ of land surface, has a maximum rate of 0.0141 $ft/d$ at land surface. The evapotranspiration surface, the elevation in the aquifer below which evapotranspiration is assumed to decline linearly, is represented as linearly sloping from 160 $ft$ on the left side of the model to 140 $ft$ on the right side of the model, except in the lake area, where the elevation is specified as 3 $ft$ below the lakebed, which is equal to 2 $ft$ below the layer bottom elevation. This means that in the possible case of lake drying, land-surface evapotranspiration from the water table under the dry lakebed is at the maximum rate if the water table is not deeper than 3 $ft$ below the lakebed. Away from the lake, the evapotranspiration surface is an implicit representation of land surface, since the latter is normally equal to or slightly above the evapotranspiration surface. Constant head boundary cells were added in column 1 and column 17 in all rows and layers; constant heads are specified to be 160 and 140 $ft$ on the left and right sides of the model, respectively.

The lakes are located in the center of the finest resolution model cells (500 $ft$) in model layers 1 and 2 and has an initial stage of 130 $ft$. The lakes is connected horizontally to the aquifer in model layers 1 and 2 and vertically to cells in model layer 2 and 3 that directly underly the lake. A lakebed leakance value of 0.1 $1/d$ was specified for all lake connections to the aquifer. The connection length for horizontal lake connections were calculated from grid dimensions and are 500 $ft$ in layer 1 and 250 $ft$ in layer 2; the connection width for horizontal connections was 500 $ft$. Rainfall and evaporation rates equal to 0.0116 and 0.0103 $ft/d$ are specified for the lake, respectively.



% solution 
The model uses the Newton-Raphson Formulation. The simple complexity Iterative Model Solver option and preconditioned bi-conjugate gradient stabilized linear accelerator is also used.

% for examples without scenarios
\subsection{Example Results}

Simulated results at the end of the stress period are shown in figure~\ref{fig:ex-gwf-lak-p02-grid}. Transient results for the lake stage and groundwater heads at two aquifer locations are shown in figure~\ref{fig:ex-gwf-lak-p02-01}. Both water-table elevations and the lake stage converge asymptotically to equilibrium values. The water-table elevations approximately converge in about 1,000 days and the lake stage in about 2,000 days of simulation time.

% a figure
\begin{StandardFigure}{
                                     Selected lake stages and heads in the aquifer. 
                                     \textit{A}. Lake stages. \textit{B}. Heads in the aquifer. 
                                     The location of points A, B, and C are shown in 
                                     figure~\ref{fig:ex-gwf-lak-p02-grid}.
                                     }{fig:ex-gwf-lak-p02-01}{../figures/ex-gwf-lak-p02-01.png}
\end{StandardFigure}                                 


\clearpage
\insection
\section{Multi-Aquifer Well Package Problem 1}

% Describe source of problem
This is the multi-aquifer well simulation described in \cite{nevilletonkin2004}. The example simulates an upper and lower aquifer separated by an impermeable confining unit but connected by a well that is open across both aquifers.                                

\begin{StandardFigure}{
                                     Location of inactive cells and the multi-aquifer well. 
                                     }{fig:ex-gwf-maw-p01-grid}{../figures/ex-gwf-maw-p01-grid.png}
\end{StandardFigure}   


\subsection{Example Description}
% spatial discretization  and temporal discretization
Model parameters for the example are summarized in table~\ref{tab:ex-gwf-maw-p01-01}.  The model consists of a grid of 101 columns, 101 rows, and 2 layers. The model domain is 14,342 $m$ in the x- and y-directions (fig.~\ref{fig:ex-gwf-maw-p01-grid}). The discretization is in the row and column directions is 142 $m$. The top of the model is specified to be -50 $m$ and the bottom of each layer is specified to be 142.9 and -514.5 $m$. Groundwater flow was inactivated beyond a distance of 7,163 $m$ from the center cell (row 51, column 51) in model layers 1 and 2 by specifying an \texttt{IDOMAIN} value of zero in these cells (fig.~\ref{fig:ex-gwf-maw-p01-grid}).

One transient stress period 2.1314815 days in length is simulated. The stress period has 50 time steps and uses a time step multiplier equal to 1.2, which results in time step lengths that range $0.51 \times 10^{-4}$ to $0.39$ days. A short simulation time is specified to prevent the effect of the well propagating to the model boundary.

% add static parameter table(s)
\input{../tables/ex-gwf-maw-p01-01}

% material properties and initial conditions
The horizontal and vertical hydraulic conductivity is 1 and $1 \times 10^{-16}$ $m/d$. The transmissivity of of the upper and lower aquifer is 92.9 and 371.6 $m^2/d$. A constant specific storage value of $1 \times 10^{-4}$ ($1/d$) is specified. All model layers are specified to be confined. An initial head of 3.05 and 9.14 $m$ are specified in the upper and lower aquifer, respectively. 

% boundary conditions
The multi-aquifer well was the only boundary condition specified in the model. The well is located in the center of the model domain (fig.~\ref{fig:ex-gwf-maw-p01-grid}), fully penetrates both aquifers, and has a well radius of 0.15 $m$. The Thiem conductance equation was used to calculate the well conductance in each aquifer. The initial head in the well was set equal to the initial head in the lower aquifer (9.14 $m$) and well storage was not simulated.

% for examples without scenarios
\subsection{Example Results}

The model was run for the case where the well was not pumping and a case where the well is pumping 1,767 $m^{3}/d$. Transient results for non-pumping and pumping case are shown in figure~\ref{fig:ex-gwf-maw-p01-01}. For the non-pumping case, the flow from the lower aquifer is balanced by flow to the upper aquifer (fig.~\ref{fig:ex-gwf-maw-p01-01}\textit{A}). The water level in the multi-aquifer well under non-pumping can be calculated using the Sokol solution \citep{sokol1963position}. The Sokol solution is

\begin{equation}
	\label{eq:Sokol}
	h_w = \frac{\sum\limits_{m=1}^{N} T_m h_m}{\sum\limits_{m=1}^{N} T_m}
\end{equation}

\noindent where $h_w$ is the water-level in the well (L), $T_m$ is the aquifer transmissivity (L$^{2}$/T), and $h_m$ is the aquifer head at the outer-constant head boundary (L). For the non-pumping case, the water-level in the well calculated using equation~\ref{eq:Sokol} is 7.922 $m$, which is identical the water-level in the multi-aquifer well.


% a figure
\begin{StandardFigure}{
                                     Simulated aquifer discharges to the multi-aquifer well. Discharge rates 
                                     are relative to the multi-aquifer well; positive and negative discharge rates 
                                     represent inflow to and outflow from the multi-aquifer, respectively.                                    
                                     \textit{A}. Non-pumping case.
                                     \textit{B}. Pumping case.
                                     }{fig:ex-gwf-maw-p01-01}{../figures/ex-gwf-maw-p01-01.png}
\end{StandardFigure}                                 

For the pumping case, the flow from the upper aquifer is actually initially negative, indicating that at early time water flows up the wellbore and into the upper aquifer, rather than discharging from it (fig.~\ref{fig:ex-gwf-maw-p01-01}\textit{B}).


\clearpage
\insection
\section{Multi-Aquifer Well Package Problem 2}

% Describe source of problem
This is a modified version of the multi-aquifer well simulation described in \cite{nevilletonkin2004}. The example simulates an upper and lower aquifer separated by an impermeable confining unit but connected by a well that is open across both aquifers. The multi-aquifer well uses the flowing well Multi-Aquifer Well (MAW) Package option to simulate discharge of water from the well at land surface.                               

\begin{StandardFigure}{
                                     Location of inactive cells and the multi-aquifer well. 
                                     }{fig:ex-gwf-maw-p02-grid}{../figures/ex-gwf-maw-p02-grid.png}
\end{StandardFigure}   


\subsection{Example Description}
% spatial discretization  and temporal discretization
Model parameters for the example are summarized in table~\ref{tab:ex-gwf-maw-p02-01}.  The model consists of a grid of 101 columns, 101 rows, and 2 layers. The model domain is 14,342 $m$ in the x- and y-directions (fig.~\ref{fig:ex-gwf-maw-p02-grid}). The discretization is in the row and column directions is 142 $m$. The top of the model is specified to be -50 $m$ and the bottom of each layer is specified to be 142.9 and -514.5 $m$. Groundwater flow was inactivated beyond a distance of 7,163 $m$ from the center cell (row 51, column 51) in model layers 1 and 2 by specifying an \texttt{IDOMAIN} value of zero in these cells (fig.~\ref{fig:ex-gwf-maw-p02-grid}).

One transient stress period 2.1314815 days in length is simulated. The stress period has 50 time steps and uses a time step multiplier equal to 1.2, which results in time step lengths that range $0.51 \times 10^{-4}$ to $0.39$ days. A short simulation time is specified to prevent the effect of the well propagating to the model boundary.

% add static parameter table(s)
\input{../tables/ex-gwf-maw-p02-01}

% material properties and initial conditions
The horizontal and vertical hydraulic conductivity is 1 and $1 \times 10^{-16}$ $m/d$. The transmissivity of of the upper and lower aquifer is 92.9 and 371.6 $m^2/d$. A constant specific storage value of $1 \times 10^{-4}$ ($1/d$) is specified. All model layers are specified to be confined. An initial head of 3.05 and 9.14 $m$ are specified in the upper and lower aquifer, respectively. 

% boundary conditions
The multi-aquifer well was the only boundary condition specified in the model. The well is located in the center of the model domain (fig.~\ref{fig:ex-gwf-maw-p02-grid}), fully penetrates both aquifers, has a well radius of 0.15 $m$, and is not pumping. The well conductance is specified to be 111.3763 and 445.9849 $m^{2}/d$ in the upper and lower aquifer, respectively. The initial head in the well was set equal to the initial head in the lower aquifer (9.14 $m$) and well storage is simulated. The flowing well discharge elevation and conductance are specified to be 0.0 $m$ and $m^{2}/d$.

% for examples without scenarios
\subsection{Example Results}

Transient results for non-pumping and pumping case are shown in figure~\ref{fig:ex-gwf-maw-p02-01}. Inflow to the well from the upper and lower aquifers is equal to the flowing well discharge. Initially the water level in the well and aquifers is above the flowing well discharge elevation. As a result, flowing well discharge begins immediately and continues throughout the simulation. Flowing well discharge decreases during the simulation as water-levels in the well and in the aquifer adjacent to the well decrease during the simulation.


% a figure
\begin{StandardFigure}{
                                     Simulated aquifer discharges to the multi-aquifer well and flowing well discharge. 
                                     Discharge rates are relative to the multi-aquifer well; positive and negative discharge rates 
                                     represent inflow to and outflow from the multi-aquifer, respectively.
                                     \textit{A}. Aquifer discharge to the multi-aquifer well.
                                     \textit{B}. Flowing well discharge. 
                                     }{fig:ex-gwf-maw-p02-01}{../figures/ex-gwf-maw-p02-01.png}
\end{StandardFigure}                                 



\clearpage
\insection
\section{Multi-Aquifer Well Package Problem 3}

% Describe source of problem
This is a modified version of the multi-aquifer well simulation described in \cite{reilly1989bias}. The example simulates an upper and lower aquifer separated by an impermeable confining unit but connected by a well that is open across both aquifers. The multi-aquifer well uses the flowing well Multi-Aquifer Well (MAW) Package option to simulate discharge of water from the well at land surface.                               

\begin{StandardFigure}{
                                     Location of inactive cells and the multi-aquifer well. 
                                     }{fig:ex-gwf-maw-p03-regional-grid}{../figures/ex-gwf-maw-p03-regional-grid.png}
\end{StandardFigure}   

\begin{StandardFigure}{
                                     Location of inactive cells and the multi-aquifer well. 
                                     }{fig:ex-gwf-maw-p03-local-grid}{../figures/ex-gwf-maw-p03-local-grid.png}
\end{StandardFigure}   


\subsection{Example Description}
% spatial discretization  and temporal discretization
Model parameters for the example are summarized in table~\ref{tab:ex-gwf-maw-p03-01}.  The model consists of a grid of 101 columns, 101 rows, and 2 layers. The model domain is 14,342 $m$ in the x- and y-directions (fig.~\ref{fig:ex-gwf-maw-p03-grid}). The discretization is in the row and column directions is 142 $m$. The top of the model is specified to be -50 $m$ and the bottom of each layer is specified to be 142.9 and -514.5 $m$. Groundwater flow was inactivated beyond a distance of 7,163 $m$ from the center cell (row 51, column 51) in model layers 1 and 2 by specifying an \texttt{IDOMAIN} value of zero in these cells (fig.~\ref{fig:ex-gwf-maw-p03-grid}).

One transient stress period 2.1314815 days in length is simulated. The stress period has 50 time steps and uses a time step multiplier equal to 1.2, which results in time step lengths that range $0.51 \times 10^{-4}$ to $0.39$ days. A short simulation time is specified to prevent the effect of the well propagating to the model boundary.

% add static parameter table(s)
\input{../tables/ex-gwf-maw-p03-01}

% material properties and initial conditions
The horizontal and vertical hydraulic conductivity is 1 and $1 \times 10^{-16}$ $m/d$. The transmissivity of of the upper and lower aquifer is 92.9 and 371.6 $m^2/d$. A constant specific storage value of $1 \times 10^{-4}$ ($1/d$) is specified. All model layers are specified to be confined. An initial head of 3.05 and 9.14 $m$ are specified in the upper and lower aquifer, respectively. 

% boundary conditions
The multi-aquifer well was the only boundary condition specified in the model. The well is located in the center of the model domain (fig.~\ref{fig:ex-gwf-maw-p03-grid}), fully penetrates both aquifers, has a well radius of 0.15 $m$, and is not pumping. The well conductance is specified to be 111.3763 and 445.9849 $m^{2}/d$ in the upper and lower aquifer, respectively. The initial head in the well was set equal to the initial head in the lower aquifer (9.14 $m$) and well storage is simulated. The flowing well discharge elevation and conductance are specified to be 0.0 $m$ and $m^{2}/d$.

% for examples without scenarios
\subsection{Example Results}

Transient results for non-pumping and pumping case are shown in figure~\ref{fig:ex-gwf-maw-p03-01}. Inflow to the well from the upper and lower aquifers is equal to the flowing well discharge. Initially the water level in the well and aquifers is above the flowing well discharge elevation. As a result, flowing well discharge begins immediately and continues throughout the simulation. Flowing well discharge decreases during the simulation as water-levels in the well and in the aquifer adjacent to the well decrease during the simulation.


% a figure
\begin{StandardFigure}{
                                     Simulated aquifer discharges to the multi-aquifer well and flowing well discharge. 
                                     Discharge rates are relative to the multi-aquifer well; positive and negative discharge rates 
                                     represent inflow to and outflow from the multi-aquifer, respectively.
                                     \textit{A}. Aquifer discharge to the multi-aquifer well.
                                     \textit{B}. Flowing well discharge. 
                                     }{fig:ex-gwf-maw-p03-01}{../figures/ex-gwf-maw-p03-01.png}
\end{StandardFigure}                                 



\clearpage
\insection
\section{Flow Diversion}

% Describe source of problem
This problem simulates unconfined groundwater flow in an aquifer with a high bottom elevation in the center of the aquifer and groundwater flow around a high bottom elevation.

\subsection{Example Description}
% spatial discretization  and temporal discretization
Model parameters for the example are summarized in table~\ref{tab:ex-gwf-bump-01}. The model consists of a grid of 51 columns, 51 rows, and 1 layer. The discretization is 1.96 $m$ in the row and column direction for all cells. The top of the aquifer is 25 $m$ and the bottom elevation of the aquifer ranges from 0 $m$ at the edges of the domain to 10 $m$ in the center of the domain (fig.~\ref{fig:ex-gwf-bump-grid}). A single steady-stress period with a total length of 1 day is simulated.


\begin{StandardFigure}{
                                     Bottom elevation of the flow diversion problem in meters. Bottom elevation are lowest at 
                                     the edges of the domain and highest in the center of the domain. The location of constant
                                     head boundary cells is also shown.
                                     }{fig:ex-gwf-bump-grid}{../figures/ex-gwf-bump-grid.png}
\end{StandardFigure}                                 


% add static parameter table(s)
\input{../tables/ex-gwf-bump-01.tex}

A constant horizontal hydraulic conductivity of 1 $m/day$ was specified in all cells. An initial head of 7.5 $m$ was specified in all model cells. Constant head boundary cells were specified for all rows in column 1 and 51. A constant head value of 7.5 and 2.5 $m$ is specified in column 1 and 51, respectively.


% for examples without scenarios
\subsection{Scenario Results}

The flow diversion model was evaluated using the Newton-Raphson Formulation and the Standard Conductance Formulation with rewetting (table~\ref{tab:ex-gwf-bump-scenario}). A scenario that used a cylindrical obstruction and the Newton-Raphson Formulation was also evaluated.

% scenario table
\input{../tables/ex-gwf-bump-scenario.tex}

\subsubsection{Newton-Raphson Formulation}

Simulated results using the Newton-Raphson Formulation is shown in figure~\ref{fig:ex-gwf-bump-01}. Newton under-relaxation is used with the Newton-Raphson Formulation to maintain water-levels above the aquifer bottom \citep{modflow6framework}. Cells with bottom elevation ranging from approximately 7.5 to 10 $m$ are dry and normalized specific discharge vectors show that groundwater is flowing around the dry cells. Note the relatively constant heads of approximately 7.5 $m$ adjacent to the dry cells. A total of 10.94 $m^3/day$ is discharged to the constant head cells in column 51.


\begin{StandardFigure}{
                                     Simulated heads and normalized specific discharge vectors for the scenario using 
                                     the Newton-Raphson Formulation. Bottom elevation contours are shown in areas 
                                     with dry cells. The location of constant head boundary cells is also shown.
                                     }{fig:ex-gwf-bump-01}{../figures/ex-gwf-bump-01.png}
\end{StandardFigure}                                 

\subsubsection{Standard Conductance Formulation with rewetting}

Rewetting parameters specified in the Node Property Flow Package for this scenario are summarized in table~\ref{tab:ex-gwf-bump-scenario}. A positive WETDRY parameter value of 2 is used to allow wetting to occur from horizontally adjacent cells. Simulated results using the Standard Conductance Formulation with rewetting is shown in figure~\ref{fig:ex-gwf-bump-02}. Although the rewetting allows the model to converge the area of dry cells is significantly larger than the area of dry cells in the Newton-Raphson Formulation scenario. Furthermore, the increase in the area of dry cells is not symmetric and extends further in areas where there is a significant component of downstream flow. Simulated heads for both scenarios are comparable in areas that are not dry in both scenarios. A total of 10.50 $m^3/day$ is discharged to the constant head cells in column 51. The slight flow reduction for this scenario is a result of upstream horizontal conductance weighting used with the Newton-Raphson formulation and the slightly reduced total simulated horizontal conductance for the rewetting scenario. 

% a figure
\begin{StandardFigure}{
                                     Simulated heads and normalized specific discharge vectors for the scenario using 
                                     the Standard Conductance Formulation with rewetting. Bottom elevation contours 
                                     are shown in areas with dry cells. The location of constant head boundary cells is 
                                     also shown.
                                     }{fig:ex-gwf-bump-02}{../figures/ex-gwf-bump-02.png}
\end{StandardFigure}                                 

\subsubsection{Newton-Raphson Formulation with a cylindrical obstruction}

The model setup for this scenario is identical to the Newton-Raphson Formulation scenario except for the aquifer bottom elevations, which are used to represent the cylindrical obstruction. Aquifer bottom elevations are derived from aquifer bottom elevations shown in figure~\ref{fig:ex-gwf-bump-grid} and are equal to 0 and 20 $m$ where elevations are less than 7.5 $m$ and greater than or equal to 7.5 $m$, respectively. Simulated results using the Newton-Raphson Formulation and a cylindrical obstruction is shown in figure~\ref{fig:ex-gwf-bump-03}. Dry cells are limited to the location of the cylindrical obstruction. Simulated heads for this scenario show the typical characteristics of flow around a cylinder including increased heads on the upstream side of the cylinder and decreased heads on the downstream side of the cylinder. A total of 23.10 $m^3/day$ is discharged to the constant head cells in column 51, which is more than double to discharge rate in the other scenarios as a result of an increase in total simulated horizontal conductance resulting from bottom elevations equal to 0 $m$ except in the location of the cylindrical obstruction.


\begin{StandardFigure}{
                                     Simulated heads and normalized specific discharge vectors for the scenario using 
                                     the Newton-Raphson Formulation with a cylindrical obstruction. The location of 
                                     constant head boundary cells is also shown.
                                     }{fig:ex-gwf-bump-03}{../figures/ex-gwf-bump-03.png}
\end{StandardFigure}                                 


\clearpage
\insection
\section{Circular Island with Triangular Mesh}

% Describe source of problem
This example shows how a triangular mesh can be used to simulate a circular island.  This is a synthetic example problem that has not been documented elsewhere.

\subsection{Example Description}

% spatial discretization  and temporal discretization
The problem consists of a simple two-layer model representing groundwater flow in a circular island.  The island has a radius of 1025 $m$.  The extent of the grid extends out to 1500 $m$ from the island center.  The grid consists of 5240 triangular cells and 2778 vertices per layer.  The grid was created using an external mesh generator.  Vertices and cell definitions are provided to \mf using the Discretization by Vertices (DISV) Package.  There is a single steady-state stress period with a length of 1 day.   Model parameters are listed in table~\ref{tab:ex-gwf-disvmesh-01}. 

% add static parameter table(s)
\input{../tables/ex-gwf-disvmesh-01}

% initial conditions
An initial head of 0 $m$ was specified for the model.  The value is not important as the model is steady state.

% boundary conditions
Recharge is assigned to cells in layer one that are within the 1025 $m$ radius of the island.  Outside the island, general-head boundaries (GHB) are assigned with a stage of zero.  GHB locations are shown in cyan in figure~\ref{fig:ex-gwf-disvmesh-grid}. Cells not colored in figure~\ref{fig:ex-gwf-disvmesh-grid} are assigned recharge.   

\begin{StandardFigure}{
                                     Model grid and boundary conditions used for the circular island problem.  General-head boundaries are shown in blue, and represent an off-shore boundary condition.  Cell not colored represent the island and are assigned groundwater recharge.
                                     }{fig:ex-gwf-disvmesh-grid}{../figures/ex-gwf-disvmesh-grid.png}
\end{StandardFigure}                                 

% for examples without scenarios
\subsection{Example Results}

Simulated heads for model layers 1 and 2 are shown in figure~\ref{fig:ex-gwf-disvmesh-head}.  

% a figure
\begin{StandardFigure}{
                                     Simulated groundwater head in (A) model layer  1 and (B) model layer 2.
                                     }{fig:ex-gwf-disvmesh-head}{../figures/ex-gwf-disvmesh-head.png}
\end{StandardFigure}                                 


\clearpage
\insection
\section{Hani Problem}

% Describe source of problem
This problem simulates groundwater flow to a pumping well under horizontally anisotropic groundwater flow conditions.  This is a synthetic example problem that has not been documented elsewhere. 

\subsection{Example Description}

% spatial discretization  and temporal discretization
Model parameters for the example are summarized in table~\ref{tab:ex-gwf-hani-01}. The model consists of a grid of 51 columns, 51 rows, and 1 layer. The discretization is 10 $m$ in the row and column direction for all cells (fig.~\ref{fig:ex-gwf-hanir-grid}). The top of the aquifer is zero and the bottom elevation of the aquifer is -10 $m$. A single steady-stress period with a total length of 1 day is simulated.

% add static parameter table(s)
\input{../tables/ex-gwf-hani-01.tex}

\begin{StandardFigure}{
                                     Model grid and boundary conditions used for the horizontal anisotropy problem.  Blue cells are constant-head cells and the pumping well is located in the cell shown in red.
                                     }{fig:ex-gwf-hanir-grid}{../figures/ex-gwf-hanir-grid.png}
\end{StandardFigure}                                 

For this problem, hydraulic conductivity is anisotropic with K11 specified as 100 times larger than K22.  For the first scenario the hydraulic conductivity ellipse is not rotated.  For the second scenario, the ellipse is rotated 25 degrees counter clockwise in the horizontal plane.  Because the ellipse axes do not align with the model grid, the XT3D option \citep{modflow6xt3d} is required to simulate this scenario.  For the third scenario, the ellipse is rotated 90 degrees, so that groundwater flows more easily in the column direction.  

% scenario table
\input{../tables/ex-gwf-hani-scenario.tex}

An initial head of zero was specified in all model cells. Constant head boundary cells with a value of zero were specified for all perimeter model cells.  A pumping well with a rate of -1.0 $m^3/d$ is located in the center of the model domain.

% for examples without scenarios
\subsection{Scenario Results}

Simulated drawdown for the three scenarios are shown in figures~\ref{fig:ex-gwf-hanir-head},  \ref{fig:ex-gwf-hanix-head}, and \ref{fig:ex-gwf-hanic-head}.

\begin{StandardFigure}{
                                     Simulated drawdown for anisotropic groundwater flow to a pumping well for scenario 1.  The dominant hydraulic conductivity ellipse is aligned with the x-axis.
                                     }{fig:ex-gwf-hanir-head}{../figures/ex-gwf-hanir-head.png}
\end{StandardFigure}                                 

\begin{StandardFigure}{
                                     Simulated drawdown for anisotropic groundwater flow to a pumping well for scenario 2.  The dominant hydraulic conductivity ellipse axis is rotated 25 degrees counter clockwise from the x-axis.  The XT3D option is required for this scenario because the ellipse axes do not align with the row and column directions.
                                     }{fig:ex-gwf-hanix-head}{../figures/ex-gwf-hanix-head.png}
\end{StandardFigure}                                 

\begin{StandardFigure}{
                                     Simulated drawdown for anisotropic groundwater flow to a pumping well for scenario 3.  The dominant hydraulic conductivity ellipse axis is rotated 90 degrees counter clockwise from the x-axis so that the ellipse axes align with the column and row directions.
                                     }{fig:ex-gwf-hanic-head}{../figures/ex-gwf-hanic-head.png}
\end{StandardFigure}                                 


                


\clearpage
\insection
\section{Groundwater Whirls}

% Describe source of problem
This problem is based on the first whirl problem described by \cite{modflow6xt3d}.  Using steady-state groundwater flow simulations, \citet{hemker2004ground} have shown that ``spiraling flow lines occur in layered aquifers that have different anisotropic horizontal hydraulic conductivities in adjacent layers.'' They refer to such spiraling flow lines as ``groundwater whirls.'' This example demonstrates the use of the XT3D option to implement anisotropy that induces groundwater whirls in a highly idealized two-aquifer system. 

\subsection{Example Description}

% spatial discretization  and temporal discretization
The model domain (fig.~\ref{fig:whirl}) is a box that is 5,100 m by 1,000 m horizontally and 1,000 m thick, discretized into a regular grid of 10 rows, 51 columns, and 10 layers. The top five model layers form the top aquifer, and the bottom five model layers form the bottom aquifer. All cells are confined. Aquifer properties are homogeneous within each aquifer but differ between the aquifers. Groundwater recharges one end of the box at a total rate of 1 m$^{3}$/d, distributed equally among 100 cells, and is removed from the opposite end of the box at the same rate, also distributed equally among 100 cells.  There is a single steady-state stress period with a length of 1 day.   Model parameters are listed in table~\ref{tab:ex-gwf-whirl-01}. An initial head of 0 $m$ was specified for the model.  The value is not important as the model is steady state.

The simulation reported, which corresponds to the first whirl problem presented by \cite{modflow6xt3d} here uses 10:1 horizontal anisotropy ($K11 = K33 =$ 1 m/d, $K22 =$ 0.1 m/d) rotated 45 degrees counterclockwise from the $x$ axis in the top aquifer and 45 degrees clockwise from the $x$ axis in the bottom aquifer.

\begin{StandardFigure}{
                                     Diagram showing the model domain for the whirl problem, which consists of a three-dimensional box with two hydrogeologic units (light and dark shading). Groundwater is injected into one end of the box (column 1, pink shading) at a rate of 0.01 cubic meters per day ($m^3/d$) into each cell and is removed from the opposite end of the box (column 51, pink shading) at a rate of 0.01 $m^3/d$ from each cell.  From \cite{modflow6xt3d}.
                                     }{fig:whirl}{../images/whirl.png}
\end{StandardFigure}                                 

% add static parameter table(s)
\input{../tables/ex-gwf-whirl-01}

% for examples without scenarios
\subsection{Example Results}

The whirl flow pattern is shown by plotting specific discharge vectors in cross section for column 1 (fig.~\ref{fig:ex-gwf-whirl-spdis}).  

% a figure
\begin{StandardFigure}{
                                     Cross section showing vectors of specific discharge for column 1 of the groundwater whirl problem.  Vectors show the swirling flow pattern caused by the contrast in the hydraulic conductivity ellipsoid which occurs and the interface between model layer 5 and model layer 6.
                                     }{fig:ex-gwf-whirl-spdis}{../figures/ex-gwf-whirl-spdis.png}
\end{StandardFigure}                                 


\clearpage
\insection
\section{Multi-model Simulation with Local Grid Refinement (Example 3 in \cite{mehl2013}}


In this example, two models are arranged in a nested parent-child grid setup, where the parent model uses a coarse grid that surrounds the finely-gridded child model (Figure~\ref{fig:mvr_lgr_planview}).  Parent-child model setups limit grid refinement to localized regions of interest within large model domains to keep model runtimes to a minimum.  Owing to the relative ease of setup for this type of model arrangement in \mf, particularly when developing a model (i.e., simulation) with a support utility like flopy \citep{bakker2016}, the use of multi-model arrangements necessitates a generalized supporting package like MVR to transfer water among features within a simulation.  To demonstrate, this example uses MVR to cascade streamflow from an upstream SFR reach in the parent model to a downstream SFR reach in the child model.  Further downstream MVR again cascades streamflow from the last child model SFR reach to the appropriate SFR reach in the parent model.  

\begin{StandardFigure}
	{Plan view of a three-dimensional aquifer system used to test the local grid refinement method in \mf. A 15$\times$15 horizontal grid discretization is shown for the parent grid and the locally refined grid (15$\times$18) spacing is equivalent to a 45$\times$45 discretization over the whole domain.  Two between-model MVR connections are invoked for the SFR packages (parent and child) where the river crosses between domains}
	{fig:mvr_lgr_planview}{../images/mvr_lgr_planview}
\end{StandardFigure}

\subsection{Example description}

The parent model is 15 rows by 15 columns by 3 layers with the child model occupying a 5 row by 6 column by 2 layer portion of the parent grid as shown in figure~\ref{fig:mvr_lgr_planview}.  The child model applies a 3:1 refinement to each parent grid cell, including in the vertical direction, resulting in a 15 row by 18 column by 6 layer child model.  In this way, each parent grid cell is replaced by 27 ($3^3$) child model grid cells.  Aquifer properties are uniform and equal throughout both model domains.  Interested readers are referred to \cite{mehl2013} for additional model details.  

% add static parameter value table
\begin{StandardTable}
	{Hydraulic model parameter values used in the local-grid refinement example problem.}
	{tab:ex-gwf-lgr01.tex}
	{../tables/ex-gwf-lgr01.tex}
\end{StandardTable}

Figure~\ref{fig:ex-gwf-lgr} demonstrates that both MVR connections transfer streamflow to the appropriate stream reach in the receiving downstream model.  This is evidenced by the continuous and smooth downward trend in streamflow at the first MVR connection as well as by the uninterrupted upward trend in streamflow at the second MVR connection.  

\begin{StandardFigure}
	{Simulated streamflow and groundwater surface-water exchange for each stream reach in figure~\ref{fig:mvr_lgr_planview}.  The transfer of water between adjacent stream reaches in the parent versus child models occurs at locations identified by the vertical black lines on the plot. Bar widths are indicative of the relative stream reach lengths within the host grid cell (figure \ref{fig:mvr_lgr_planview}).}
	{fig:ex-gwf-lgr}{../figures/ex-gwf-lgr}
\end{StandardFigure}

\clearpage
\insection
\section{Buried Valley Aquifer with LGR}

% Describe source of problem
This example reproduces several models described by \cite{vilhelmsen2012}, which simulate groundwater flow in a buried valley aquifer at different grid resolution.  The domain is simulated using a globally refined (GR) grid, a globally coarsened (GC) grid, and a coarse outer grid with a locally refined inset grid (LGR).  

\subsection{Example Description}

% spatial discretization  and temporal discretization
Groundwater flow is simulated using three different simulations corresponding to three different grid configurations: GR, GC, and LGR (tab.~\ref{tab:ex-gwf-lgrv-scenario}).  The GR grid consists of 25 layers, 183 rows and 147 columns.  The grid spacing along rows is 35 $m$ and the grid spacing along columns is 25 $m$.  The top of the model is variable, based on topography in the area (fig.~\ref{fig:ex-gwf-lgrv-gr-grid}).  Layer bottoms are flat.  The bottom for layer 1 is set to 30 $m$.  Bottom elevations for layers 2 through 25 are calculated using a uniform layer thickness of 5 $m$.  A cross section for the GR grid is shown in figure~\ref{fig:ex-gwf-lgrv-gr-xsect}

The GC grid fits evenly into the GF grid, and has 1/3 the number of cells in the row and column directions.  Instead of 25 layers, the GC grid has only 9 layers.  Layer 1 corresponds to layer 1 in the GF model; however, each underlying layer in the GC grid corresponds to three layers in the GF model.  The GC grid is shown in map view in figure~\ref{fig:ex-gwf-lgrv-gc-grid} and as a cross section in figure~\ref{fig:ex-gwf-lgrv-gc-xsect}.

The LGR grid is a combination of the GF and GC grids.  The course outer parent grid is comprised of the GC grid, whereas the inset child grid consists of the GF grid (fig~\ref{fig:ex-gwf-lgrv-gc-grid}).  The LGR simulation consists of two separate model input files, one for the parent model and one for the child model.  The two models are connected in \mf using the GWF-GWF Exchange, which is used to connect cells from the parent model to cells in the child model.

There are three different hydrogeologic units.  The upper layer represents overburden material.  The buried valley is filled with valley sediments, and the bottom material consists of the deposits into which the valley is incised.  These three units are assigned a single value for hydraulic conductivity.

Recharge is uniformly applied to the water table at a rate of 1.1098e-9 $m/s$.  A river is represented in model layer 1 using the River (RIV) Package.  The models are run as steady state using a single time step with a duration of 1 second. 

Describe discretization (\ref{fig:ex-gwf-lgrv-grid}).  Model parameters are listed in table~\ref{tab:ex-gwf-lgrv-01}. 

% add static parameter table(s)
\input{../tables/ex-gwf-lgrv-01}

% add static parameter table(s)
\input{../tables/ex-gwf-lgrv-scenario}

% a figure
\begin{StandardFigure}{
                                     Globally refined model grid showing top elevation and river cells.  Area of interest is shown as a dashed line.
                                     }{fig:ex-gwf-lgrv-gr-grid}{../figures/ex-gwf-lgrv-gr-grid.png}
\end{StandardFigure}                                 

\begin{StandardFigure}{
                                     Cross section for y = 3000 $m$ showing globally refined model grid.  Color flood of hydraulic conductivity shows overburden material, valley fill, and deposits into which the valley is incised.
                                     }{fig:ex-gwf-lgrv-gr-xsect}{../figures/ex-gwf-lgrv-gr-xsect.png}
\end{StandardFigure}                                 

% a figure
\begin{StandardFigure}{
                                     Globally coarsened model grid showing top elevation and river cells.  Area of interest is shown as a dashed line.
                                     }{fig:ex-gwf-lgrv-gc-grid}{../figures/ex-gwf-lgrv-gc-grid.png}
\end{StandardFigure}                                 

\begin{StandardFigure}{
                                     Cross section for y = 3000 $m$ showing globally coarsened model grid.  Color flood of hydraulic conductivity shows overburden material, valley fill, and deposits into which the valley is incised.
                                     }{fig:ex-gwf-lgrv-gc-xsect}{../figures/ex-gwf-lgrv-gc-xsect.png}
\end{StandardFigure}                                 


% a figure
\begin{StandardFigure}{
                                     Local grid refinement model showing the outer coarse grid and the inner refined model grid.  Top elevation and river cells are shown for both the outer and inner grids.  Refinement area is shown as a dashed line.
                                     }{fig:ex-gwf-lgrv-grid}{../figures/ex-gwf-lgrv-grid.png}
\end{StandardFigure}                                 

% for examples without scenarios
\subsection{Example Results}

Model results for the three different simulations are shown in figures~\ref{fig:ex-gwf-lgrv-gf-head}, \ref{fig:ex-gwf-lgrv-gc-head}, and \ref{fig:ex-gwf-lgrv-head}.  Simulated results from the three simulations are in good agreement and demonstrate the different levels of detail that can be achieved with the three grids.  Testing of the three simulations indicates that the LGR model is about 25 times slower than the GC model; however the GF model is about 100 times slower than the GC.  These numbers indicate that LGR can be used effectively in \mf to include refined inset models within coarser regional models.

% a figure
\begin{StandardFigure}{
                                     Simulated head in layer 1 for the GR model.
                                     }{fig:ex-gwf-lgrv-gr-head}{../figures/ex-gwf-lgrv-gr-head.png}
\end{StandardFigure}                                 

% a figure
\begin{StandardFigure}{
                                     Simulated head in layer 1 for the GC model.
                                     }{fig:ex-gwf-lgrv-gc-head}{../figures/ex-gwf-lgrv-gc-head.png}
\end{StandardFigure}                                 

% a figure
\begin{StandardFigure}{
                                     Simulated head in layer 1 for the LGR model.
                                     }{fig:ex-gwf-lgrv-head}{../figures/ex-gwf-lgrv-head.png}
\end{StandardFigure}                                 


\clearpage
\insection
\section{Laattoe Periodic Boundary Condition}

% Describe source of problem
This example shows how the exchange capability in \mf can be used to simulate spatial periodic boundary conditions (SPBC), such as the one described by \cite{laattoe2014spatial}.  A SPBC can be used to represent spatially repeating groundwater flow conditions, such as those that might form beneath repeating bedforms on the sea floor.  The example simulated here is equivalent to the first MODFLOW simulation reported by \cite{laattoe2014spatial}.  

\subsection{Example Description}

% spatial discretization  and temporal discretization
The problem consists of a two-dimensional cross-section model consisting of 190 layers and 100 columns.  Each cell is 0.06 $m$ wide and each layer has a width of 0.03 $m$.  Model parameters are listed in table~\ref{tab:ex-gwf-spbc-01}. 

% add static parameter table(s)
\input{../tables/ex-gwf-spbc-01}

% initial conditions
An initial head of 0 $m$ was specified for the model; however this model is not important as the model represents steady-state conditions.

% boundary conditions
The top of model has a constant-head condition assigned to layer 1.  A different constant-head value is assigned to each cell based on a sine wave with an amplitude of 1.0 $m$ and a wavelength of 6 $m$, which is the length of the model in the x direction.  The GWF-GWF Exchange is used to connect the cells on the left side of the model with the cells on the right side of the model.  The first cell in each model cell is hydraulically connected to the last cell in each model layer.  For example, the cell in (1, 1, 1) is hydraulically connected to the cell in (1, 1, 100).  In \mf, these cells are connected at the matrix solution level, rather than through outer iterations as was done by \cite{laattoe2014spatial}.  

% for examples without scenarios
\subsection{Example Results}

Model results are shown in figure~\ref{fig:ex-gwf-spbc-grid}.  Groundwater flowing into cells on the left side of the model is instantaneously applied to cells on the right side of the model.   Because the first column of cells is hydraulically connected to the last column of cell through the GWF-GWF Exchange, flow exiting the model through the left face automatically flows back into the model through the right face.

% a figure
\begin{StandardFigure}{
                                     Cross section showing simulated head and vectors of specific discharge for the spatial periodic boundary condition example problem.  Vectors of specific discharge are shown for every fifth cell in the layer and column directions.
                                     }{fig:ex-gwf-spbc-grid}{../figures/ex-gwf-spbc-grid.png}
\end{StandardFigure}                                 


\clearpage
\insection
\section{Elastic aquifer loading}
This problem simulates elastic compaction of aquifer materials in response to the loading of an aquifer by a passing train. Water-level responses were simulated for an eastbound train leaving the Smithtown Station in Long Island, New York at 13:04 on April 23, 1937 \citep{jacob1939fluctuations}. 

\subsection{Example Description}

The problem is simulated as a two-dimensional half-cell cross-section model. The model grid for this problem consists of three layers, 1 row, and 35 columns (fig.~\ref{fig:ex-gwf-csub-p01-grid}). The model layers were defined based on hydrostratigraphic information in \cite{jacob1939fluctuations}. The upper and lower layer represent an unconfined upper aquifer and confined lower aquifer separated by a confining unit (fig.~\ref{fig:ex-gwf-csub-p01-grid}\textit{B}). The upper and lower aquifers are composed of sand and gravel, respectively, and the confining unit is composed of clay. 

%jacob grid
\begin{StandardFigure}{
                                     Diagram showing the model domain for the elastic aquifer 
                                     loading problem. \textit{A}, plan view, and \textit{B}, cross-section view
                                     }{fig:ex-gwf-csub-p01-grid}{../figures/ex-gwf-csub-p01-grid.png}
\end{StandardFigure}                                 


The model has a top elevation of 0 meters and layer bottom elevations of -12.2, -21.3, and -30.5 meters, for layers 1, 2, and 3 respectively.  DELR increases from 0.5 to 98.9 meters  in columns 1 to 30, using a multiplier of 1.2; a DELR value of 100 meters is specified in columns 31 to 35. DELC is specified with a constant value of 100.6 meters and is based on the an estimate of the total length of the train (table~\ref{tab:ex-gwf-csub-p01-train}). The simulation consists of two stress periods. The first stress period is steady-state with a single time step and is 0.5 seconds in length. The second stress period is transient, 58.5 seconds in length, and is divided into 117 equally sized time steps. 

\small
\begin{longtable}[!htbp]{
                                      p{.1\linewidth-2\arraycolsep}
                                      p{.1\linewidth-2\arraycolsep}
                                      p{.3\linewidth-2\arraycolsep}
                                      p{.3\linewidth-2\arraycolsep}
                                      }
	\caption{Assumed train length and weight} \label{tab:ex-gwf-csub-p01-train} \\

	\hline \hline
	\rowcolor{Gray}
	\textbf{Type} & \textbf{Number} & \textbf{Length, in meters} & \textbf{Weight, in kilograms}  \\
	\hline
	\endhead
	
	Engine & 2 &  21.3 &   108,862.08 \\
	Car      & 4 &    79.3 &   199,580.48 \\ 
	\hline
	Total     & -- & 100.6 &   308,442.56 \\ 
	\hline \hline
\end{longtable}
\normalsize

Initial hydraulic properties were based on aquifer material data in \cite{freeze1979groundwater} and are summarized in table~\ref{tab:ex-gwf-csub-p01-01}. Hydraulic conductivity was assumed to be isotropic in the horizontal and vertical directions in each layer. Hydraulic conductivity and specific storage values were modified from initial values during model calibration, which is described below. The specific storage was defined to be 0 for all layers in the storage (STO) package. All model layers were defined to be convertible for hydraulic conductivity and storage properties. Default flow property (NPF) and storage (STO) package settings were used. An initial head of -10.7 meters was defined for each layer.

\input{../tables/ex-gwf-csub-p01-01.tex}

The effective stress formulation of the CSUB package was used to simulate one-dimensional compaction of aquifer materials. A specific gravity of 1.7 and 2.0 was defined for moist and saturated sediments, respectively. Water compressibility was simulated using a specific gravity of water of 9,806.65 Newtons per cubic meters and water compressibility of $4.6512 \times 10^{-10}$ per Pascal. The thickness of compressible materials and total porosity were updated during the simulation in response compaction.

\cite{jacob1939fluctuations} measured water-level fluctuations in well S-201 (fig.~\ref{fig:ex-gwf-csub-p01-grid}\textit{A}). S-201 is located 16.5 meters north of the tracks (column 12) and has a total depth of 27.1 meters (model layer 3). A limited amount of data on the position of the train relative to well S-201 was provided by \cite{jacob1939fluctuations}. As a result, it was assumed that the original water-level fluctuation data is a proxy for train loading. The maximum water-level fluctuation value was assumed to correspond to loading by the full weight of the train (table~\ref{tab:ex-gwf-csub-p01-train}) and a zero water-level fluctuation corresponded to complete unloading. The estimated loading of the aquifer (fig.~\ref{fig:ex-gwf-csub-p01-01}\textit{A}) was converted to an equivalent height of water over the first cell of the model using the cell area, one-half the total train weight (because the problem is simulated as a half-cell problem), and the density of water (1,000 kilograms per cubic meters). Because well S-201 is located 16.5 meters north of the tracks, the estimated loading was translated in time by -1.5 seconds to account for the time for loading to cause water-level fluctuations at the well; the -1.5 second adjustment was determined through trial and error. Train loading was applied in column 1 using a time series file. Flow was not allowed to leave the model domain and no sources/sinks were applied to the model. The left and right side of the model domain are represented as a free-slip (roller) boundaries.

%jacob figure 4 results
\begin{StandardFigure}{
                                     Graphs showing the applied loading and water-level fluctuations for the elastic aquifer loading 
                                     problem. \textit{A}, shows the loading applied to the top of the first column in layer 1, and \textit{B}, 
                                     shows simulated and offset water-level fluctuations at well S-201
                                     }{fig:ex-gwf-csub-p01-01}{../figures/ex-gwf-csub-p01-01.png}
\end{StandardFigure}                                 

The water-level fluctuation data used to calibrate the hydraulic parameters was offset so that the initial water-level fluctuation reported (after rewinding the pen-carriage cable) corresponded to a zero value. The adjusted water-level at the end of the simulation period is less than zero since loading of the aquifer by the train was already occurring at the beginning of the data presented in \cite{jacob1939fluctuations} (fig~\ref{fig:ex-gwf-csub-p01-01}\textit{B}). The water-level fluctuation data was offset rather than extended because of uncertainties about the train velocity and acceleration prior to the simulation period.

PEST++ \citep{welter2015approaches} was used to calibrate the horizontal hydraulic conductivity in layers 1 and 3, the vertical hydraulic conductivity in layer 2, and specific storage values in all model layers. The water-level fluctuation observations were weighted by $\text{max} ( 0.01, h_i / \text{max(h)} )$ to force PEST++ to favor the peak water-level fluctuations. The water-level fluctuations were only sensitive to the hydraulic conductivity and specific storage of model layer 3; as a result, PEST++ only modified the hydraulic properties of layer 3. Final hydraulic properties used in the model are shown in table~\ref{tab:ex-gwf-csub-p01-01}. 

\subsection{Example Results}

A comparison of simulated and observed water-level fluctuations is shown in figure~\ref{fig:ex-gwf-csub-p01-01}\textit{B}. For this problem, the \mf solution does not show perfect agreement with the offset water-level fluctuations. The primary source of model error is likely due primarily to inaccuracies in the loading being applied in the model. Use of a two-dimensional cross-section model instead of a three-dimensional model may also be responsible for a portion of the model error shown in figure~\ref{fig:ex-gwf-csub-p01-01}\textit{B}. Horizontal strain has also been found to be significant in close proximity pumping wells (the source of strain) and may also contribute to the model error \citep{burbey2001storage}.


\clearpage
\insection
\section{Delay interbed drainage}
This problem simulates the drainage of a thick interbed caused by a step decrease in hydraulic head in the aquifer and is based on sample problem 1 in \cite{hoffmann2003modflow}. 


\subsection{Theory}

The equilibration of hydraulic heads in thick interbeds imbedded in an aquifer system typically lags head changes in the surrounding aquifer as a result of the characteristically low vertical hydraulic conductivity of fine-grained silts and clays that constitute the interbeds. Similarly, the hydraulic gradient within the interbeds can be treated as vertical if the horizontal extents of the interbeds are much greater than their thicknesses, the delayed dissipation of unequilibrated heads within the interbeds can be described by the one-dimensional diffusion equation,

\begin{equation}
	\frac{\partial ^2 h}{\partial z^2} = \frac{S^{\prime}_{S}}{K^{\prime}_{v}} \frac{\partial h}{\partial t},
	\label{eq:diffusion}
\end{equation}

\noindent where $z$ is the vertical spatial coordinate (L), $S^{\prime}_{S}$ is the specific storage of the interbed (unitless), $K^{\prime}_{v}$ is the vertical hydraulic conductivity of the interbed (L/T), and $t$ is time (T). The solution of this diffusion problem is identical to heat diffusion. \cite{carslaw1959conduction} developed an analytical solution for heat diffusion from a slab with the ends at a constant temperature that can be recast to solve equation~\ref{eq:diffusion} for delayed flow from a thick interbed. If the initial head at $t = 0$ is $h_0$ throughout the thickness of the interbed ($b_0$), and the head in the surrounding aquifer is $\Delta h$ above $h_0$ for $t > 0$, the head distribution $[h(z, t)]$ for the interbed can be written as the infinite series

\begin{equation}
	h(z, t) - h_0 = \Delta h - \frac{4 \Delta h}{\pi} \sum^{\infty}_{k = 0} \frac{-1^k}{2k + 1} e^{-\frac{\pi^2}{4} \frac{t}{\tau_k}} \cos \left( \frac{(2k + 1) \pi z}{b_0} \right),
	\label{eq:headdist}
\end{equation}

\noindent where the time constant, $\tau_k$, is defined as 

\begin{equation}
	\tau_k = \frac{ \left( \frac{b_0}{2} \right)^2 S^{\prime}_{S} }{\left( 2k + 1 \right)^2 K^{\prime}_{v}}.
	\label{eq:tauk}
\end{equation}

In equation~\ref{eq:headdist}, $z = 0$ is assumed to be at the midplane of the interbed, with the boundaries at $\pm \frac{b_0}{2}$. Note that both the coefficients in the sum and the $\tau_k$ decrease as $k$ increases. Thus, the true head distribution can be adequately described by a finite number of addends ($k$), particularly for later times. In the context of interbed compaction and land subsidence, the time delay caused by slow dissipation of transient overpressures is often given in terms of the time constant

\begin{equation}
	\tau_0 = \frac{ \left( \frac{b_0}{2} \right)^2 S^{\prime}_{S} }{K^{\prime}_{v}},
	\label{eq:tau0}
\end{equation}

\noindent which is the time during which about 93 percent of the ultimate compaction for a given decrease in head occurs \citep{riley1969analysis}. Because $\tau_0$ is proportional to $S^{\prime}_{S} $, which generally is much larger for inelastically deforming interbeds than for elastically deforming interbeds, deformation in elastically deforming interbeds is often assumed to occur instantaneously. The same is true for very thin inelastically deforming interbeds. Thus, equation~\ref{eq:tau0} can be used to determine in which interbeds the time constant exceeds the model time step, necessitating consideration of use of delay-interbeds, which account for delayed drainage processes, instead of no-delay interbeds.

Under constant geostatic stress conditions, compaction in the interbed can be directly related head changes using

\begin{equation}
	\Delta b = S^{\prime}_{S} \Delta h,
	\label{eq:compaction}
\end{equation}

\noindent where $\Delta b$ is the change in thickness of the interbed (L).

\subsection{Example Description}

Static model parameters are summarized in table~\ref{tab:ex-gwf-csub-p02-01}. The model grid for this problem consists of 1 layer, 1 row, and 3 columns (fig.~\ref{fig:ex-gwf-csub-p02-grid}). The model has a top elevation of 0 meters and bottom elevation of -1,000 meters.  DELR and DELC are equal to 1 meter. The simulation consists of one transient stress period 1,000 days in length, and is divided into 100 variable length time steps calculated using a time step multiplier equal to 1.05.

\input{../tables/ex-gwf-csub-p02-01.tex}

\begin{StandardFigure}{
                                     Model domain and setup for the delay interbed drainage problem. Interbed 
                                     drainage is the result of step decrease in head in the aquifer
                                     }{fig:ex-gwf-csub-p02-grid}{../figures/ex-gwf-csub-p02-grid.png}
\end{StandardFigure}                                 

The hydraulic conductivity in the aquifer was set to a very large value (\num{1e6} meters per day), so that the head in the aquifer in the center cell remains constant. The specific yield and specific storage in the STO package were set to 0. Default flow property (NPF) and storage (STO) package settings were used. Initial heads were specified to be 0 meters.

\subsubsection{Head-Based Formulation}
 
Initially, the head-based formulation of the CSUB package was used to simulate compaction of the delay interbed and compare to analytical results calculated using equations~\ref{eq:headdist} and~\ref{eq:compaction} (table~\ref{tab:ex-gwf-csub-p02-scenario}). Ten finite-difference nodes represent the half-thickness of the interbed. The time constant, $\tau_0$ (eq.~\ref{eq:tau0}), was chosen to be 1,000 with vertical hydraulic conductivity set to \num{2.5e-6} meters per day, interbed thickness set to 1 meters, and elastic skeletal specific storage set to \num{1e-5} per meter and inelastic skeletal specific storage set to 0.01 per meter. Meters and days units have been used in this problem but any consistent set of length and time units results in the same solution. The specific storage of coarse-grained aquifer material were specified to be \num{0e-6} per meter. Water compressibility was not simulated in this problem and the thickness of compressible materials and total porosity were not updated during the simulation. 

\input{../tables/ex-gwf-csub-p02-scenario}
 
Constant-head cells, with a value of 0 meters, bound the delay interbed in column 2. The water released from the interbed during the simulation can leave the system through these constant-head cells. The starting head and the preconsolidation head in the delay interbed were specified to be 1 meter higher than the initial head in the surrounding aquifer. 

The resulting compaction of the interbed is compared to the analytical solution (derived using equations~\ref{eq:headdist},~\ref{eq:tau0}, and~\ref{eq:compaction}) in figure~\ref{fig:ex-gwf-csub-p02-grid}. The CSUB-computed values closely match the analytical values. The small differences, particularly at early times, may be at least partly due to the fact that the aquifer head in the simulation does not remain exactly constant as a result of water entering the aquifer from the interbed. Because of the finite transmissivity of the aquifer, the head in the aquifer briefly rises to about 2 percent of the starting head in the interbed during the first time step.

\begin{StandardFigure}{
                                     Graphs showing comparisons of simulated compaction with the 
                                     head-based formulation and the analytical solution for the delay 
                                     interbed drainage problem. \textit{A}, comparison of the compaction 
                                     history simulated with the analytical solution to the problem, and 
                                     \textit{B}, difference between the analytical solution and simulated 
                                     compaction
                                     }{fig:ex-gwf-csub-p02a-01}{../figures/ex-gwf-csub-p02a-01.png}
\end{StandardFigure}         

\subsubsection{Effective Stress-Based Formulation}

To evaluate differences between the head- and effective stress-based formulations, the problem shown in figure~\ref{fig:ex-gwf-csub-p02-grid} was modified to use the effective stress-formulation (table~\ref{tab:ex-gwf-csub-p02-scenario}). A total of 19 finite-difference nodes were used in the effective stress-based formulation so that results could be directly compared to the head-based formulation that used 10 finite-difference cells to represent the half-thickness of the interbed. A specific gravity of 1.7 and 2.0 was defined for moist and saturated sediments, respectively. The initial preconsolidation stress was set to be 1 meter less than the initial effective stress of 1,000 meters and is based on the initial preconsolidation head, which was defined to be 1 meter above the initial head in the head-based formulation.

The resulting effective stress-based compaction of the interbed is compared to the head-based solution in figure~\ref{fig:ex-gwf-csub-p02b-01}. The effective stress-based values closely match the head-based values. The small differences ($<$~0.1\%) are partly due to the fact that calculated specific storage values are not constant in the effective stress-formulation. Furthermore, the inelastic and elastic compression indices (41.8 and \num{4.18e-2} (unitless)), respectively), which are internally calculated from the initial effective stress and the user-provided inelastic and elastic specific storage values, results in a slightly smaller initial inelastic storativity value (\num{9.5e-3} versus \num{1.0e-2}) that increase to values slightly larger than the user-provided inelastic storativity in subsequent time steps.

\begin{StandardFigure}{
                                     Graphs showing comparisons of simulated compaction for head- and 
                                     effective stress-based formulations for the delay interbed drainage 
                                     problem. \textit{A}, comparison of the compaction history simulated 
                                     with the head- and effective stress-based formulation solution to 
                                     the problem, and \textit{B}, difference between the simulated head- 
                                     and effective stress-based formulation compaction
                                     }{fig:ex-gwf-csub-p02b-01}{../figures/ex-gwf-csub-p02b-01.png}
\end{StandardFigure}         

Another reason the difference between the head- and effective stress-based compaction shown in figure~\ref{fig:ex-gwf-csub-p02b-01} is small is the interbed thickness is small (1 meter) and as a result the difference between the effective stress at the top and bottom of the interbed is also small. 

\subsubsection{Effect of Interbed Thickness on the Effective Stress-Based Formulation}

To evaluate the effect of the interbed thickness affects compaction, head- and effective stress-based models were run with interbed thicknesses ranging from 1 to 100 m (table~\ref{tab:ex-gwf-csub-p02-scenario}).  A time constant ($\tau_0$) of 1,000 was used with elastic and inelastic skeletal specific storage values of \num{1e-5} and 0.01 per meter, respectively, were used for each interbed thickness evaluated. The vertical hydraulic conductivity for each interbed thickness evaluated was calculated using equation~\ref{eq:tauk} and the specified $\tau_0$ and specific storage values. The calculated vertical hydraulic conductivity ranged from \num{2.5e-6} to 0.025 meters per day for interbed thickness ranging from 1 to 100 meters, respectively. A total of 1,001 finite-difference nodes were used to simulate the interbed for the head- and effective stress-based formulation simulations to provide additional spatial resolution for simulated interbed heads; the head-based simulations were simulated using a full-cell formulation. All other model parameters for the simulations that evaluated different interbed thicknesses were unchanged from the original values.

The difference between the analytical and simulated compaction and drainage rates at the top and bottom of the interbed relative to analytical drainage rates are shown in figure~\ref{fig:ex-gwf-csub-p02c-01}. The difference between head- and effective stress-based compaction for a 1 meter interbed thickness shown in figure~\ref{fig:ex-gwf-csub-p02c-01}A are identical to the results shown in figure~\ref{fig:ex-gwf-csub-p02b-01}B. In general, the differences between the simulated results and the analytical solution are comparable for interbed thickness less than 20 meters. Coincident with compaction differences, the average difference in drainage from the top and bottom of the interbed to the aquifer is greater than 0.7\%  (fig.~\ref{fig:ex-gwf-csub-p02c-01}B) for interbed thicknesses greater than 10 meters as a result of larger differences in the effective stress at the top and bottom of the interbed. The average difference between the effective stress at the bottom and top of the interbed is 2.02\%, 5.13\%, and 10.5\% of the average interbed effective stress for the simulations with 20, 50, and 100 meters interbed thicknesses, respectively.

Figure~\ref{fig:ex-gwf-csub-p02c-02} shows the vertical distribution of the difference in head- and effective stress-based formulation interbed heads relative to head-based interbed heads for each of the interbed thicknesses evaluated. Head-based interbeds heads are symmetric about the center line of the interbed, with lower heads at the top and bottom of the interbed and the highest heads at the center of the interbed. As a result, negative and positive differences shown in figure~\ref{fig:ex-gwf-csub-p02c-02} represent higher and lower interbed heads in the effective stress-based formulation than the head-based formulation, respectively. Generally, effective stress-based interbed heads are higher and lower in the top and bottom halves of the interbed, respectively, and differences are greatest for interbed thicknesses greater than 10 meters. The spatial distribution of interbed head differences is controlled by the decrease in the inelastic specific storage value resulting from the increase in effective stress with depth and the reduction in the water released from storage with depth in the interbed, which results in increased head changes with depth with the effective stress-formulation. As the simulation progresses, differences propagate from the top and bottom of interbed into the interbed as the maximum difference decreases.

\begin{StandardFigure}{
                                     Graphs showing the difference between head- and effective 
                                     stress-based compaction for different interbed thicknesses for the 
                                     delay interbed drainage problem. \textit{A}, difference between the 
                                     head- and effective stress-based formulation compaction, and 
                                     \textit{B}, difference between drainage at the top and bottom of the 
                                     interbed relative to the head-based formulation interbed drainage
                                     }{fig:ex-gwf-csub-p02c-01}{../figures/ex-gwf-csub-p02c-01.png}
\end{StandardFigure}         

\begin{StandardFigure}{
                                     Graphs showing differences between head- and effective stress-based 
                                     formulation interbed heads relative to head-based interbed heads for variable 
                                     interbed thicknesses at select fractions of the time constant ($\tau_0$) for the 
                                     delay interbed drainage problem. \textit{A}, 1 percent of $\tau_0$, \textit{B}, 
                                     5 percent of $\tau_0$, \textit{C}, 10 percent of $\tau_0$, \textit{D}, 50 percent 
                                     of $\tau_0$,  and \textit{E}, 100 percent of $\tau_0$
                                     }{fig:ex-gwf-csub-p02c-02}{../figures/ex-gwf-csub-p02c-02.png}
\end{StandardFigure}         





\clearpage
\insection
\section{One-Dimensional Compaction}

A one-dimensional \mf model was developed by \cite{sneed2008} to simulate aquitard drainage, compaction and, land subsidence at the Holly site, located at the Edwards Air Force base, in response to effective stress changes caused by groundwater pumpage in the Antelope Valley in southern California (fig.~\ref{fig:ex-gwf-csub-p03-location}). Land subsidence resulting from groundwater level declines, has long been recognized as a problem in Antelope Valley, California. The original one-dimension compaction model was calibrated to extensometer data from the USGS Holly site (station name \href{https://waterdata.usgs.gov/ca/nwis/dv/?site_no=344835117531305}{008N010W01Q005S}) for the period from 1990 to 2006, and used a head based-formulation to represent compaction. The model of \cite{sneed2008} has been modified to use the effective stress formulation available in the CSUB package for \mf.

\begin{StandardFigure}{
                                     Maps showing the location of the study area for the one-dimensional 
                                     compaction problem in southern California. \textit{A}, location of the 
                                     study area in southern California and \textit{B}, location of the Holly site and
                                     Edwards Air Force Base in the Antelope Valley. Cross-blended hypsometric 
                                     tints with relief, water, drains and ocean bottom background image from 
                                     Natural Earth and is available at 
                                     \url{https://www.naturalearthdata.com/downloads/50m-raster-data/50m-cross-blend-hypso/}, 
                                     accessed on July 9, 2019 
                                     }{fig:ex-gwf-csub-p03-location}{../images/ex-gwf-csub-p03-location.png}
\end{StandardFigure}                                 

\subsection{Site Description}

The subsurface geology at the Holly site comprises Quaternary alluvial and lacustrine deposits from land surface to about 260 meters below land surface, consolidated late Tertiary and early Quaternary sedimentary continental deposits from about 260 to 330 meters below land surface, and decomposed basement complex. Lithologic and geophysical logs of the Holly site indicate the presence of relatively thin interbedded aquitards, ranging from 0.3 to 5 meters thick, and two thicker aquitards 20 meters (37--57 meters below land surface) and 19 meters (92--111 meters below land surface) thick. The upper aquitard is interpreted as a regionally extensive, confining unit. The groundwater system at the Holly site is comprises two aquifer systems--an unconfined system and a confined system, which are separated by lacustrine blue-clay deposits that constitute the confining unit. The upper aquifer is unconfined, about 37 meters thick and the water table is about 20 meters below land surface. The confined-aquifer system at the site extends about 275 meters below the confining unit, where it is underlain by weathered bedrock. The middle aquifer is the source of most of the groundwater pumped from the well field closest to the Holly site. Additional information on the hydrogeology of the Holly site and the Antelope Valley can be found in \cite{sneed2000aquifer} and  \cite{sneed2008}.

Compaction at the Holly site for the period from 1990 through 2006 was measured using a counterweighted pipe extensometer designed to measure compaction in the interval from 4.6 to about 260 meters below land surface. The principal mode of compaction at the Holly site is a seasonally dependent step response. Larger rates of compaction are associated with summer water-level drawdowns and despite groundwater level recoveries of more than 3 meters during the winter, compaction continues, at a reduced rate. The absence of aquifer-system expansion during seasonal water-level recovery is consistent with the delayed drainage and resultant delayed, or residual, compaction of thick aquitards.

\subsection{Example Description}

The model grid for this problem consists of 14 layers, 1 row, and 1 column (fig.~\ref{fig:ex-gwf-csub-p03-grid}). The model layers are based on the model of \cite{sneed2008}, with the exception of the top of model layer 1 which was modified from the original value of -27.74 meters to 0 meters to allow the model to account for unsaturated conditions above the water table when calculating geostatic and effective-stresses. Model layer thicknesses are summarized in table~\ref{tab:ex-gwf-csub-p03-01}. DELR and DELC are equal to 1 meter. The model consists of 353 stress periods covering the period from May 8, 1908 to September 4, 2006. The duration of model stress periods and time steps for the period from May 8, 1908 to May 9, 1990 (“early time”) were annual and monthly--365.25 and 30.4375 days, respectively, and were 22 and 1 days, respectively, for the period from May 9, 1990 to September 4, 2006 (“late time”). The nearly century duration of the simulations allows for comparisons of aquifer-system compaction owing to sustained groundwater pumpage and water-level declines through the period of groundwater development and seasonal groundwater level cycling since 1990.

\begin{StandardFigure}{
                                      Diagram showing the model domain and setup for the one-dimensional 
                                      compaction problem. \textit{A}, hydrostratigraphy, \textit{B}, interbed types 
                                      used in aquifer and confining units, and \textit{C}, location of constant-head 
                                      boundary conditions
                                     }{fig:ex-gwf-csub-p03-grid}{../figures/ex-gwf-csub-p03-grid.png}
\end{StandardFigure}        

\input{../tables/ex-gwf-csub-p03-01.tex} 

Hydraulic properties used in the model are shown in table~\ref{tab:ex-gwf-csub-p03-01}.  \cite{sneed2008} specified different horizontal and vertical hydraulic conductivity values for each layer; however, vertical hydraulic conductivity values from \cite{sneed2008} were assigned as the horizontal hydraulic conductivity values in each layer since there is no horizontal flow in this one-dimensional problem. The specific yield and specific storage in the STO package were defined to be 0 for all layers.  All model layers are defined to be non-convertible for hydraulic conductivity and storage properties. Default NPF and STO package settings were used. Initial heads in the model range from 0. to 6.77 meters (table~\ref{tab:ex-gwf-csub-p03-01}).

The effective stress formulation of the CSUB package was used to simulate compaction of aquifer materials.  Fine-grained materials defined as delay interbeds were were discretized using 39 cells and assigned a uniform vertical hydraulic conductivity of $4.57 \times 10^{-6}$ meters per day. A specific gravity of 1.7 and 2.0 was defined for moist and saturated sediments, respectively. Water compressibility was simulated using a specific gravity of water of 9,806.65 Newtons per cubic meters and water compressibility of $4.6512 \times 10^{-10}$ per Pascal. The thickness of compressible materials and total porosity were updated during the simulation in response compaction.

Interbedded aquitards ranged from 0.3 to 5.5 meters in thickness. The Holly model simulates interbedded aquitards less than 1.5 meters thick using no-delay interbeds (ultimate compaction occurs within a model time step), and simulates interbedded aquitards 1.5 meters thick or greater using delay-interbeds (ultimate compaction does not occur within a model time step). A total of 18 interbedded aquitards ranging from approximately 0.3 to 1.2 meters thick, with a total aggregate thickness of approximately 12 meters, were modeled as no-delay interbeds and 10 interbedded aquitards ranging from approximately 1.5 to 5.5 meters thick, with a total aggregate thickness of approximately 27 meters, were modeled as delay interbeds (table~\ref{tab:ex-gwf-csub-p03-02}). The confining unit (approximately 20 meters thick) and the thick aquitard (approximately 19 meters thick), were modeled as no-delay interbeds (table~\ref{tab:ex-gwf-csub-p03-02}). Simulation of delayed drainage and residual compaction in each of these units was simulated implicitly using 3 model layers as recommended in \cite{hoffmann2003modflow}. Compaction was not simulated for the upper aquifer because the upper aquifer is relatively coarse grained and heads are changing very slowly and are hydraulically isolated from seasonal groundwater fluctuations in the production zones of the aquifer system. A constant porosity of 0.30 was used for the coarse- and fine-grained materials in the model.

\input{../tables/ex-gwf-csub-p03-02.tex} 

Initial preconsolidation stresses were calculated from initial preconsolidation heads developed by \cite{sneed2008} and effective stresses calculated using initial heads (table~\ref{tab:ex-gwf-csub-p03-01}). Initial preconsolidation stresses for no-delay and delay interbeds are summarized in table~\ref{tab:ex-gwf-csub-p03-02}, respectively. \cite{sneed2008} estimated initial preconsolidation heads from the time series for paired bench marks near the Holly site and middle aquifer water levels (fig.~\ref{fig:ex-gwf-csub-p03-01}). Delay beds in the middle aquifer (model layers 5 and 12) and the lower aquifer (model layer 12) were specified to be 6.77 and 5.55 meters above land surface, respectively, which are the same as initial heads in these layers. 

Boundary conditions in the one-dimensional compaction model of the Holly site consist of constant (time-variant) heads for those parts of the coarse-grained aquifer that represent measured (or estimated) hydraulic head (fig.~\ref{fig:ex-gwf-csub-p03-grid}\textit{C}). The upper, middle, and lower aquifers at the Holly site are represented in the model by specifying heads in each aquifer using data from \cite{sneed2008} and are shown in figure~\ref{fig:ex-gwf-csub-p03-01}. The upper model boundary is a time-variant, constant-head boundary that represents measured or estimated heads in the upper aquifer (model layer 1) at the Holly site and is about 28 meters below land surface (fig.~\ref{fig:ex-gwf-csub-p03-grid}\textit{C}). Three boundaries within the model domain consist of time-variant, specified heads that represent measured or estimated heads in the middle (model layers 6 and 11) and lower aquifers (model layer 13) at the Holly site (fig.~\ref{fig:ex-gwf-csub-p03-grid}\textit{C}). Time-varying heads for constant-head cells were defined using a time series file. Although compaction is not thought to be important in the upper aquifer it should be noted that compaction and related release/storage of water are not simulated in cells with constant-head boundaries.

\begin{StandardFigure}{
                                      Graph showing depth to water values used in the one-dimensional 
                                      compaction problem for the upper, middle, and lower aquifers at the 
                                      Holly site, Edwards Air Force Base, Antelope Valley, California. Modified from \cite{sneed2008}
                                     }{fig:ex-gwf-csub-p03-01}{../figures/ex-gwf-csub-p03-01.png}
\end{StandardFigure}        

The compaction at the end of the simulation for the confining unit (0.28 meters), the thick aquitard (0.66 meters), no-delay interbeds contained in aquifer units (0.06 meters), delay interbeds contained in aquifer units (0.35 meters), and coarse grained materials  (0.06 meters) from a modified \mf version of the model of \cite{sneed2008} were used to calibrate initial specific storage values used in the model. The specific storage values used by \cite{sneed2008} were uniformly scaled by a factor of $9.9408 \times 10^{-1}$ to better match the observed compaction at the Holly site for the period from October 1, 1992 to May 9, 2006. The total compaction at the end of the simulation period (1.42 meters) and the total compaction from October 1, 1992 to May 9, 2006 (0.19 meters) were also used to calibrate initial specific storage values used in the model. 

PEST++ \citep{welter2015approaches} was used to calibrate 1) the elastic specific storage value for coarse grained materials in model layers 5, 7, 12, and 14; 2) inelastic and elastic specific storage values for the confining unit (model layers 2--4); 3) inelastic and elastic specific storage values for the thick aquitard (model layers 8--10); and 4) inelastic and elastic specific storage value were calibrated for no-delay and delay interbeds contained in aquifer units (model layers 5, 6, 12, and 14). Compaction values simulated using the head-based formulation for the confinining unit, the thick aquitard, no-delay interbeds contained in aquifer units, delay interbeds contained in aquifer units, and coarse grained materials were given a weight of 1; total compaction and the total compaction from October 1, 1992 to May 9, 2006 were given an increased weight of 5 to favor fitting the total compaction and observed change in total compaction over material based compaction values.  

\subsection{Example Results}

Comparison of head- and effective stress-based compaction results are shown in figure~\ref{fig:ex-gwf-csub-p03-02}. The compaction at the end of the simulation in the model using head- and effective stress-based formulation were essentially identical and mean errors calculated for the entire simulation ranged from -0.0045 meters (confining unit compaction) to 0.0042 meters (compaction in delay interbeds contained in aquifer units), with the largest differences occurring between approximately 1947 and 1977.

\begin{StandardFigure}{
                                      Graphs showing simulated compaction in different material types using head- 
                                      and effective stress based formulations for the one-dimensional compaction 
                                      problem. \textit{A}, Total compaction, \textit{B}, compaction in interbeds in the 
                                      thick aquitard, \textit{C}, compaction in delay interbeds contained in aquifers, 
                                      \textit{D}, compaction in interbeds in the confining unit, \textit{E}, compaction 
                                      in no-delay interbeds contained in aquifers, and \textit{F}, compaction in 
                                      coarse-grained materials
                                     }{fig:ex-gwf-csub-p03-02}{../figures/ex-gwf-csub-p03-02.png}
\end{StandardFigure}        

The thickness of compressible materials and calibrated specific storage values for coarse-grained materials, fine-grained materials represented as no-delay interbeds, and fine-grained materials represented as delay interbeds are summarized in tables~\ref{tab:ex-gwf-csub-p03-03} and \ref{tab:ex-gwf-csub-p03-04}, respectively. Calibrated specific storage values are larger than values used with the head-based formulation. Percent differences relative to values used by \cite{sneed2008} were 27.0\% for the elastic specific storage values of coarse-grained materials, averaged 79.2 and 210.2\% for the inelastic and elastic specific storage values of no-delay interbeds, and averaged 20.6 and 52.2\% for the inelastic and elastic specific storage values of delay interbeds. Larger specific storage values are expected for the effective stress-based formulation since effective-stress values increase during the simulation, as a result of groundwater pumpage induced water-level declines, and result in reduced specific storage values and compaction with time relative to the head-based formulation model using the uniformly scaled specific storage values from \cite{sneed2008}.

\input{../tables/ex-gwf-csub-p03-03.tex} 

\input{../tables/ex-gwf-csub-p03-04.tex} 

The simulations for the period 1908--2006 provide information about how the aquifer-system components, aquifers and aquitards, contributed to overall compaction because of the continual lowering of water levels throughout the 1900s and because of seasonal water-level cycling since 1990. Simulated compaction totaled 1.42 meters for the period 1908--2006. Of the total simulated compaction, the confining unit (thickness = 20.12 meters) accounted for 20.0\% of the total; the thick aquitard (thickness = 19.20 meters) accounted for 46.7\% of the total; delay interbeds in aquifers (aggregate thickness = 18.14 meters) accounted for 24.6\%; coarse-grained materials (aggregate thickness = 225.39 meters) accounted for 4.5\% of the total; and no-delay interbeds in aquifers (aggregate thickness = 11.89 meters) accounted for 4.4\% of the total (fig.~\ref{fig:ex-gwf-csub-p03-03}\textit{A}). During 1990--2006, a total of 0.23 meters of compaction was simulated; the confining unit accounted for 31.2\% of the total; the thick aquitard accounted for 66.3\% of the total; delay interbeds in aquifers accounted for 1.7\%; coarse-grained materials accounted for -0.1\% (representing expansion of coarse grained materials); and no-delay interbeds in aquifers accounted for 0.9\% of the total. For these relatively quickly equilibrating thin aquitards, the fairly stable stresses since the mid-1970s and cyclic stresses during the late time were often in the elastic range of stress. In fact, beginning in about 1976, the delay and no-delay interbeds in aquifers had significantly reduced compaction rates, contributing only 0.01 meters (2.4\%) and 0.004 meters (0.1\%) of compaction, respectively, during the last 30 years of the simulation. These thin aquitards deformed mostly elastically during the late time (fig.~\ref{fig:ex-gwf-csub-p03-03}).

\begin{StandardFigure}{
                                      Graphs showing history matches for simulated and measured aquifer-system 
                                      compaction in the one-dimensional compaction problem. \textit{A}, compaction 
                                      in the different material types for the full simulation period (for 1908--2006 ), 
                                      \textit{B}, comparison of simulated and observed compaction at the Holly site 
                                      for the period from 1990 to 2007, and \textit{C}, simulated and observed 
                                      stress/displacement for the period from October 1, 1992 to September 4, 2006. 
                                      Elastic and inelastic specific storage values were calibrated using observed 
                                      Holly site compaction data for the period from October 1, 1992 to September 4, 
                                      2006 and simulated compaction from the model based on \cite{sneed2008}, 
                                      which used a head-based formulation to simulate compaction
                                     }{fig:ex-gwf-csub-p03-03}{../figures/ex-gwf-csub-p03-03.png}
\end{StandardFigure}        

The simulated stress/displacement trajectory also compares well in magnitude and timing with the measured stress/displacement trajectory between October 1, 1992 and September 4, 2006 (fig.~\ref{fig:ex-gwf-csub-p03-03}\textit{C}). The effective stress at the base of the lower aquifer was estimated using water-level data for the upper and lower aquifers (fig.~\ref{fig:ex-gwf-csub-p03-01}). The estimated stress/displacement trajectory fit is poorest from 1993 to 1996 when seasonal compaction changes lag behind observed changes. After April 1997, simulated compaction is generally consistent with observed compaction.

Vertical distributions of hydraulic head in the aquitards can be used as a direct measure of residual compaction \citep{riley1969analysis, riley1998mechanics}. A linear profile showing deviations in the simulated 1908 to 2006 head distributions for the two thick clay sequences--the confining unit and the thick aquitard--indicate large residual excess pore pressures exist at the end of the simulation (fig.~\ref{fig:ex-gwf-csub-p03-04}). Residual excess pore pressures in these thick aquitards began accumulating in about 1950, when water levels in the aquifers began declining at rate faster (fig.~\ref{fig:ex-gwf-csub-p03-01}) than these aquitards could dissipate excess pore pressures. The simulations indicate that about 98\% of the compaction during late time is residual compaction occurring in these two thick clay sequences (fig.~\ref{fig:ex-gwf-csub-p03-03}\textit{A}).

\begin{StandardFigure}{
                                      Graph showing simulated vertical head profiles and approximately 
                                      decadal head changes in the one-dimensional compaction problem. 
                                      Vertical head profiles are shown for 1908, 1916, 1926, 1936, 1946, 
                                      1956, 1966, 1976, 1986, 1996, and 2006. The colored area between 
                                      plotted years represents the simulated head change over approximately 
                                      decadal period of time between simulated vertical head profiles. The 
                                      vertical location of model layer tops and bottom and layers with 
                                      constant-head boundaries are also shown.
                                     }{fig:ex-gwf-csub-p03-04}{../figures/ex-gwf-csub-p03-04.png}
\end{StandardFigure}   



\clearpage
\insection
\section{One-Dimensional Compaction in a Three-Dimensional Flow Field}

This problem is based on the problem presented in the SUB-WT report \citep{leake2007modflow} and represent groundwater development in a hypothetical aquifer that includes some features typical of basin-fill aquifers in an arid or semi-arid environment. The problem of \cite{leake2007modflow} was modified to include compaction of coarse-grained aquifer materials and water compressibility. Specific stress packages were also modified but net inflows to the model domain are identical.

\subsection{Example Description}

The model grid for this problem consists of four layers, 20 rows, and 15 columns (fig.~\ref{fig:ex-gwf-csub-p04-grid}).  The model has a top elevation of 150 meters and layer bottom elevations of 50, -100, -150, and -350 meters for layers 1, 2, 3, and 4, respectively.  DELR and DELC are specified with a constant value of 2,000 meters. The simulation consists of three stress periods. The first is an initial steady-state period for the purpose of computing the head distribution, which is used with other quantities to compute the initial hydrostatic, effective, geostatic, and preconsolidation stresses. The second stress period is used to simulate 60 years of pumping by the two wells at locations shown in figure~\ref{fig:ex-gwf-csub-p04-grid}. The third stress period is used to simulate 60 years of recovery following cessation of pumping. The second and third stress periods are divided into 60, 1-year time steps.

\begin{StandardFigure}{
                                     Diagram showing the model domain for the one-dimensional compaction in a 
                                     three-dimensional flow field problem. \textit{A}, plan view, and \textit{B}, cross-section 
                                     view. The locations of active and inactive areas of the model domain, steady-state 
                                     heads in model layer 1, and locations of recharge cells, constant-head cells, 
                                     and wells are also shown
                                     }{fig:ex-gwf-csub-p04-grid}{../figures/ex-gwf-csub-p04-grid.png}
\end{StandardFigure}                                 

The aquifer system consists of an unconfined upper aquifer, an extensive confining unit, and a confined lower aquifer (fig.~\ref{fig:ex-gwf-csub-p04-grid}\textit{B}). The model uses two layers to represent the water-table aquifer and one layer each to represent the confining unit and the lower aquifer. Hydraulic conductivity was assumed to be isotropic in the horizontal and vertical directions in each layer. Hydraulic properties for coarse-grained materials are listed in table~\ref{tab:ex-gwf-csub-p04-01}. The specific storage was defined to be 0 for all layers in the STO package. Model layer 1 and layers 2--4 were defined to be convertible and non-convertible, respectively, for hydraulic conductivity and storage properties. Default NPF and STO package settings were used. An initial head of 100 meters was defined for each layer.

\input{../tables/ex-gwf-csub-p04-01.tex}

The effective stress formulation of the CSUB package was used to simulate compaction of aquifer materials. Storage properties for coarse- and fine-grained materials were specified using compression indices. No-delay interbeds were specified in each active model cell. Hydraulic properties for fine-grained materials represented as no-delay interbeds are listed in table~\ref{tab:ex-gwf-csub-p04-01}; no-delay interbeds in model 3 comprise the full thickness of the confining unit. 

The specific gravity of fully- and partially-saturated materials for each layer was calculated using 

\begin{equation}
	G = (1 - \bar{\theta}) G_{\text{solid}} + k_r \bar{\theta} G_{\text{water}},
	\label{eq:p4sg}
\end{equation}


\noindent where $G$ is the specific gravity of a control volume that includes solids and water (unitless), $\bar{\theta}$ is the thickness weighted porosity of coarse- and fine-grained materials in a control volume (unitless), $G_{\text{solid}}$ is the specific gravity of the solids in a control volume (unitless), $k_r$ is a scaling factor used to scale the specific gravity of water if a control volume is not fully saturated (unitless), and $G_{\text{water}}$ is the specific gravity of water (unitless). $k_r$ is 1 for saturated materials or the ratio of the volume of water to the total volume for materials that are not fully saturated.

The specific gravity of saturated materials for each layer was calculated using the thickness weighted porosity listed in table~\ref{tab:ex-gwf-csub-p04-01}, a $G_{\text{solid}}$ value of 2.7, and a  $G_{\text{water}}$ value of 1.0. The specific gravity of moist materials was calculated using a $k_r$ value of 0.25 and the other values used to calculated the specific gravity of saturated materials. The specific gravity of saturated and moist materials used for the one-dimensional compaction in a three-dimensional flow field problem are listed in table~\ref{tab:ex-gwf-csub-p04-01}

Water compressibility was simulated and default specific gravity of water ($\gamma_{\text{water}}$ = \num{4.6512e-10} per Pascal) and the compressibility of water ($\beta$ = 9806.65 Newtons per cubic meter) values were used. Initial specific storage values related to the compressibility of water ($S_{s_{\text{water}}} = \theta \beta \gamma_{\text{water}}$) were \num{1.46e-6} and \num{2.05e-6} per meter for coarse- and fine-grained materials, respectively. The porosity and thickness of coarse- and fine-grained materials were adjusted during the simulation in response to compaction. The initial preconsolidation stress for the no-delay interbeds was specified to be 15 meters greater than the steady-state effective stress calculated in each cell at the end of stress period 1. 

Inflow to the flow system is simulated using the recharge package at 18 recharge locations in layer 1 shown on figure~\ref{fig:ex-gwf-csub-p04-grid}\textit{A}. Recharge at each of these locations is specified at a rate of \num{5.5e-4} meters per day throughout the entire simulation, resulting in a total recharge rate of 39,600 cubic meters per day. Under steady-state conditions, all of the flow leaves the system through eight constant-head cells, two of which are in each layer at the horizontal locations shown on figure~\ref{fig:ex-gwf-csub-p04-grid}\textit{A}. Head at the eight constant-head cells is specified to be 100 meters. During stress period 2, each of the two wells shown on figure~\ref{fig:ex-gwf-csub-p04-grid}\textit{A} withdraw water from the upper and lower aquifer at a rate of 72,000 cubic meters per day.

\subsection{Example Results}

The initial steady-state stress period results in a maximum head of 143.0 meters in row 1, column 8. Steady-state hydraulic gradients slope down valley and toward the center of the valley to the constant-head cells in row 20, layers 1–4 (fig.~\ref{fig:ex-gwf-csub-p04-grid}). The steady-state head distribution is used to compute initial effective stress, preconsolidation stress, and geostatic stress for the transient part of the simulation. For the location of the well in row 9, column 10, layer 2, these stress values are 273.3, 288.3, and 509.5 meters, respectively.

Equivalent skeletal specific storage values were calculated from recompression and compression indices and the initial effective stress distribution. Computed values of elastic skeletal specific storage at row 9, column 10, were \num{2.03e-5}, \num{8.58e-6}, \num{7.33e-6}, and \num{4.41e-6} per meter for layers 1--4, respectively. Values of inelastic (virgin) skeletal specific storage for the same location were \num{2.03e-4}, \num{8.58e-5}, \num{7.33e-5}, and \num{4.41e-5} per meter for layers 1--4, respectively. These values are not used explicitly in further calculations by the CSUB package, but are provided to illustrate how effective stress influences the spatial distribution of specific storage.

\subsubsection{Simulated Stresses}

Values of effective stress, preconsolidation stress, and geostatic stress at the bottom of the cell for the 60-year pumping and 60-year recovery periods are shown in figure~\ref{fig:ex-gwf-csub-p04-01} for row 9, column 10, layers 1 and 2. In both layers, preconsolidation stress is exceeded early in the simulation and tracks with increases in effective stress until recovery of water levels after the start of stress period 3 (year 60). Head change at this location is similar in layers 1 and 2, and the shapes and magnitudes of change-of-effective- and preconsolidation-stress curves are nearly the same (figs.~\ref{fig:ex-gwf-csub-p04-01}\textit{A, C}). The curves, however, are at different head elevations because of the increase in magnitude of stress with depth. The shapes of curves representing geostatic stress in layers 1 and 2 (figs.~\ref{fig:ex-gwf-csub-p04-01}\textit{B, D}) are identical because all of the change results from movement of the water table in layer 1 (fig.~\ref{fig:ex-gwf-csub-p04-grid}\textit{B}). Note how simulated effective stress and geostatic stress does not return to initial values 60 years after cessation of pumpage; approximately 2,000 years without pumping is required for simulated effective stress and geostatic stress at row 9, column 10, layers 1 and 2 to return initial values.

\begin{StandardFigure}{
                                     Graphs showing computed stresses for row 9, column 10 for the 
                                     one-dimensional compaction in a three-dimensional flow field problem. 
                                     \textit{A}, Effective and preconsolidation stress in layer 1, \textit{B}, 
                                     geostatic stress in layer 1, \textit{C}, effective and preconsolidation 
                                     stress in layer 2, and \textit{D}, geostatic stress in layer 2
                                     }{fig:ex-gwf-csub-p04-01}{../figures/ex-gwf-csub-p04-01.png}
\end{StandardFigure}                                 

\subsubsection{Simulated Compaction}

The computed vertical displacements for the tops of layers 1--4 resulting from fine- (interbed) and coarse-grained material compaction at the locations of the two pumping wells are shown in figure~\ref{fig:ex-gwf-csub-p04-02}. The total compaction at the top of layer 1 represent the time series of land subsidence for the two locations (fig.~\ref{fig:ex-gwf-csub-p04-02}\textit{E, F}). Similarly, differences between displacement curves for adjacent layers are the time distributions of compaction in the each layer. At both locations, compaction is greatest in the layer in which pumping takes place. Coarse-grained compaction is small relative to inelastic compaction of fine-grained interbeds. Similar to stress results (fig.~\ref{fig:ex-gwf-csub-p04-01}), elastic compaction of coarse-grained materials does not fully recover 60 years after cessation of pumpage.

\begin{StandardFigure}{
                                     Graphs showing computed  downward displacement for the tops of model 
                                     layers 1--4 for the one-dimensional compaction in a three-dimensional flow 
                                     field problem. \textit{A}, interbed compaction at row 9, column 10, \textit{B}, 
                                     interbed compaction at row 12, column 7, \textit{C}, coarse-grained material 
                                     compaction at row 9, column 10, \textit{D}, coarse-grained material 
                                     compaction at row 12, column 7, \textit{E}, total compaction at row 9, 
                                     column 10, and \textit{F}, total compaction at row 12, column 7
                                     }{fig:ex-gwf-csub-p04-02}{../figures/ex-gwf-csub-p04-02.png}
\end{StandardFigure}                                 

\subsubsection{Simulated Storage Changes}

At the end of the second stress period a total of 3,155,760,128 cubic meters of water was pumped from the water-table and confined aquifers. Water released from specific yield, elastic coarse-grained materials, elastic fine-grained materials, inelastic fine-grained materials, and water compressibility accounted for 96.31\% (3,039,304,920 cubic meters) of groundwater pumpage; the remainder of groundwater pumpage came from reduction in the discharge to the eight constant-head cells. Individually specific yield, elastic coarse-grained materials, elastic fine-grained materials, inelastic fine-grained materials, and water compressibility accounted for 94.80\%, 0.40\%, 0.63\%, 0.21\%, and 0.27\%, respectively, of total groundwater pumpage. For the cell containing well 1 (layer 2, row 9, column 10) elastic coarse-grained materials, elastic fine-grained materials, inelastic fine-grained materials, and water compressibility accounted for 5.18\%, 3.81\%, 87.68\%, and 3.33\%, respectively, of the total water released from storage (3,628.59 cubic meters) in response to groundwater pumpage. For the cell containing well 2 (layer 4, row 12, column 7) elastic coarse-grained materials, elastic fine-grained materials, inelastic fine-grained materials, and water compressibility accounted for 4.63\%, 2.62\%, 87.57\%, and 5.18\%, respectively, of the total water released from storage (3,147.01 cubic meters) in response to groundwater pumpage.
    





\clearpage
\insection
\section{Drain Package Drainage Depth Option Problem 1}

% Describe source of problem
This example is a modified version of the Unsaturated Zone Flow (UZF) Package problem 2 described in \cite{UZF}. UZF Package problem 2 is based on the Green Valley problem (Streamflow Routing (SFR) Package problem 1) described in \cite{modflowsfr1pack}. The problem has been modified by converting all of the SFR reaches to use rectangular channels and to use the drain package drainage option to simulated groundwater discharge to the land surface.                               

\subsection{Example Description}
% spatial discretization  and temporal discretization
Model parameters for the example are summarized in table~\ref{tab:ex-gwf-drn-p01-01}.  The model consists of a grid of 10 columns, 15 rows, and 1 layer. The model domain is  50,000 $ft$ and 80,000 $ft$ in the x- and y-directions, respectively. The discretization is 5,000 $ft$ in the row and column direction for all cells. The top of the model ranges from about 1,000 to 1,100 $ft$ and the bottom of the model ranges from about 500 to 1,000 $ft$.

Twelve stress periods are simulated. The first stress period is steady state and the remaining stress periods are transient. The stress periods are $2.628 \times 10^{6}$ seconds (30.42 days) in length. The first stress period is broken into one time step. Stress periods 2 through 12 are each broken up into 15 time steps and use a time step multiplier of 1.1.

% add static parameter table(s)
\input{../tables/ex-gwf-drn-p01-01}

% material properties
The basin fill thickens toward the center of the valley and hydraulic conductivity of the basin fill is highest in the region of the stream channels. Hydraulic conductivity is 173 $ft/day$ ($2 \times 10^{-4}$ $ft/s$) in the vicinity of the stream channels and 35 $ft/day$ ($4 \times 10^{-4}$ $ft/s$) elsewhere in the alluvial basin. A constant specific storage value of $1 \times 10^{-6}$ ($1/day$) was specified throughout the alluvial basin. Specific yield is 0.2 (unitless) in the vicinity of the stream channels and 0.1 (unitless) elsewhere in the alluvial basin.

% initial conditions
An initial head of 1,050 $ft$ was specified in all model layers. Any initial head exceeding the bottom of each cell could be specified since the model is steady-state.

% boundary conditions
Flow into the system is from infiltration from precipitation and was represented using the UZF package. Recharge rates applied to each cell ranged $2.5 \times 10^{-10}$ to $2 \times 10^{-9}$ from  of $3 \times 10^{-7}$ $ft/s$, with lower rates in the vicinity of the stream channels and higher rates elsewhere in the alluvial basin. Flow out of the model is from groundwater evapotranspiration represented by evapotranspiration (EVT) package cells and discharging wells represented by well (WEL) package cells. Groundwater evapotranspiration occurs where depth to water is within 15 $ft$ of land surface, has a maximum rate of 3 $ft/yr$ at land surface, and is coincident with the valley lowland through which several streams flow. Wells are  located in ten cells (rows 6 through 10 and columns 4 and 5) along the west side of the valley (fig.~\ref{fig:ex-gwf-drn-p01-grid}\textit{B}). Each well extracted 10 $ft^{3}/s$ of groundwater for a total withdrawal rate of 100 $ft^{3}/s$ (about twice the steady-state ground-water inflow). Two general-head boundary cells were added in (row 13, column 1) and (row 14, column 8) with a specified head equal to 988 and 1,045 $ft$, respectively, and a constant conductance of 0.038 $ft^{2}/s$.

\input{../tables/ex-gwf-drn-p01-02}

The streams in the model domain were represented using a total of 36 reaches. External inflows of 25, 10, and 100 $ft^{3}/s$ were specified for reach 1, 16, and 28, respectively. Reach 1 is located in (row 1, column 1), reach 16 is in (row 5, column 10), and reach 28 is in (row 14, column 9). Streamflow discharges from the model at the downstream end of reach 36 in (row 13, column 1). Reach widths were specified to be 12, 0, 5, 12, 55, and 40 $ft$ for reaches 1--9, 10--18, 19--22, 23--27, 28--30, and 31--36, respectively. The remaining streambed properties and stream dimensions used for each stream reach are the same as those used in 
 \cite{modflowsfr1pack} \cite[see][Table~1]{modflowsfr1pack}. Constant stage reaches were used to define the ditch represented by reaches 10--15 and ranged from approximately 1,075.5--1061.6 $ft$. A diversion from reach 4 to 10 was specified to represent managed inflows to the ditch. Ditch inflows were specified to be 10 $ft^{3}/s$ except if the downstream flow in reach 4 is less than the specified diversion rate; in cases where the downstream flow in reach 4 is less than the specified diversion rate all of the downstream flow in reach 4 is diverted to the ditch and the inflow to reach

% solution 
The model uses the Newton-Raphson Formulation. The simple complexity Iterative Model Solver option and preconditioned bi-conjugate gradient stabilized linear accelerator is also used.

% for examples without scenarios
\subsection{Example Results}

Simulated results for the initial steady-state stress period and at the end of the stress period with groundwater pumping (stress period 2) are shown in figure~\ref{fig:ex-gwf-drn-p01-01}. Reach stage and downstream discharge were also evaluated for reach 4, 14, 27, and 36.

% a figure
\begin{StandardFigure}{
                                     Simulated water levels and normalized specific discharge vectors  
                                     under steady state and pumping conditions. 
                                     \textit{A}. steady-state results.
                                     \textit{B}. results after 50 years of pumping.
                                     }{fig:ex-gwf-drn-p01-01}{../figures/ex-gwf-drn-p01-01.png}
\end{StandardFigure}                                 




\clearpage
\insection
\section{Sagehen---UZF1 Package Problem 1}

The first test problem in \cite{UZF} presents a conceptual model of the Sagehen watershed that showcased the first version of the unsaturated-zone flow (UZF1) package. The Sagehen watershed is located on the eastern side of the northern Sierra Nevada, north of Lake Tahoe and the city of Truckee (see inset in figure~\ref{fig:sagehen}). The simulation spans the upper 27 $mi^2$ of the watershed.  The lowest land surface altitude of 1,928 $m$ above mean sea level occurs where Sagehen Creek exits the model.  The highest altitude of 2,649 $m$ above mean sea level occurs on the western-most crest of the watershed, resulting in roughly 720 $m$ of topographic relief across the watershed.  Figure~\ref{fig:sagehen} depicts the setting and topographic relief of the watershed. The geologic setting is primarily comprised of volcanic rocks overlain with a veneer of alluvium. Annual precipitation averages approximately 970 $mm$ which falls primarily in the form of snow.  

\begin{StandardFigure}
	{3-Dimensional perspective view of the Sagehen watershed looking westward.  Colors depict zones of varying hydraulic conductivity within the active grid domain.}
	{fig:sagehen}{../images/sagehen.png}
\end{StandardFigure}

\subsection{Example description}

For this \mf demonstration of the Sagehen model, a single model layer ranging between 53 $m$ and 899 $m$ thick is used to represent both the saturated and unsaturated zones.  Spatially, 73 rows and 81 columns with uniform 90 $\times$ 90 $m$ grid cells are used to represent the watershed.  The simulated period uses daily stress periods with one time-step per stress period, starting on December 1st and ending on November 30th.  The first stress period is steady state, while all others are transient.  Because the perimeter of the active model domain follows a topographic watershed divide, a no-flow boundary is used around the perimeter and bottom of the groundwater flow model.  However, the cell hosting Sagehen Creek at its exit point as well as the two adjecent cells were specified constant head cells.  The specified vertical hydraulic conductivity of the unsaturated zone was set equal to the vertical hydraulic conductivity used in the node-property flow (NPF) package. The stream network is represented with 213 inter-connected stream reaches, all with a constant width of 3 $m$ and range from 30 to 114 $m$ in length.  

A set of infiltration factors shown in figure~\ref{fig:ex-gwf-sagehen-finfFact} tie precipitation rates to land-surface altitude to account for the orographically-driven variations in precipitation.  Infiltration factors are three times greater along the western crest of the watershed compared to the lower valley elevations near the outlet.  The infiltration factors were multiplied by infiltration rates that varied daily as shown by the ``infiltration'' time series shown in figure~\ref{fig:ex-gwf-sagehen-uzFlow}.  

\begin{StandardFigure}
	{Plot of static infiltration factors used to spatially vary infiltration rates within the watershed. Infiltration factors were multiplied by a time series of infiltration rates shown in figure~\ref{fig:ex-gwf-sagehen-uzFlow}}
	{fig:ex-gwf-sagehen-finfFact}{../figures/ex-gwf-sagehen-finfFact.png}
\end{StandardFigure}

The water content in the unsaturated zone is calculated based on the flux through, and properties of, the unsaturated-zone during the initial steady-state stress period when the recharge rate equals the specified infiltration rate at land surface.  A uniform extinction depth of 2.5 $m$ is applied across the entire model.  Below 2.5 $m$, Evapotranspiration (ET) ceases.  Using the same approach implemented in UZF1 \citep{UZF}, water is first removed from the unsaturated zone in fulfillment of the ET demand.  Any remaining residual ET demand is then  satisfied using water in the saturated zone when the water table is within 2.5 $m$ of land surface.  An extinction water content of 0.10 was specified for the entire model

A summary of model parameter values are listed in table~\ref{tab:ex-gwf-sagehen-01}. Interested readers are referred to \cite{UZF} for a discussion of additional details, including how the distribution of hydraulic conductivity was derived and aspects of the model calibration, among others.

% add static parameter value table
\input{../tables/ex-gwf-sagehen-01.tex}

\subsection{Example results}

Figure~\ref{fig:ex-gwf-sagehen-gwDepth} shows the calculated depth to groundwater during the steady-state stress period.  Depths to groundwater varied significantly over the model domain, ranging from more than 25 $m$ below land surface to at, or slightly above, land surface near streams.

\begin{StandardFigure}
	{Calculated depth to groundwater during the steady-state stress period at the beginning of the simulation.}
	{fig:ex-gwf-sagehen-gwDepth}{../figures/ex-gwf-sagehen-gwDepth.png}
\end{StandardFigure}

The UZF2 package partitions infiltration into runoff, resulting from saturation excess and/or rejected infiltration; recharge; ET; and changes in unsatruated zone storage \cite{modflow6software}.  Although changes in unsaturated zone storage are not shown, figure~\ref{fig:ex-gwf-sagehen-uzFlow} shows how the infiltration was partitioned among recharge and ET through time.  Runoff processes are depicted in figure~\ref{fig:ex-gwf-sagehen-swFlow}, and together with groundwater discharge directly to streams equal the total streamflow generated by the watershed.  

A noteworthy enhancement between this version of the model and that first documented in \citep{UZF} is that the drain (DRN) package has been added to the model to capture groundwater discharge to land-surface.  In this way, groundwater discharge is kept separate from rejected infiltration, enabling easier viewing of the relative contribution of each of these processes since they are now separated in the calculated model budget.  The UZF2 package could be used to simulate both processes; however, if this approach is used the relative contribution of groundwater discharge to land surface and rejected infiltration to stream flow are lumped.  In flow-only simulations, this approach may suffice.  In a groundwater transport simulation, where these two contributing sources to stream flow may have very different concentrations, the approach taken here will enable improved tracking of solute contributions to surface water.  In addition to the DRN package, the mover (MVR) package has been invoked to deliver groundwater discharge and rejected infiltration to the nearest downgradient (i.e., downhill) stream reach.  Previously, the IRUNBND array in UZF1 was used to route groundwater discharge and rejected infiltration to the stream network.

\begin{StandardFigure}
	{Volumetric rates of infiltration, recharge, and evapotranspiration from the unsaturated and saturated zones summed over the model domain.}
	{fig:ex-gwf-sagehen-uzFlow}{../figures/ex-gwf-sagehen-uzFlow.png}
\end{StandardFigure}

\begin{StandardFigure}
	{Volumetric flow rates of surface-water generation and runoff.}
	{fig:ex-gwf-sagehen-swFlow}{../figures/ex-gwf-sagehen-swFlow.png}
\end{StandardFigure}


%MODFLOW API GWF examples
\clearpage
\insection
\section{Capture Fraction Analysis}

All groundwater pumped is balanced by removal of water somewhere, initially from storage in the aquifer and later from capture in the form of increase in recharge and decrease in discharge \citep{leake2010new}. Capture that results in a loss of water in streams, rivers, and wetlands now is a concern in many parts of the United States. Hydrologists commonly use analytical and numerical approaches to study temporal variations in sources of water to wells for select points of interest. Much can be learned about coupled surface/groundwater systems, however, by looking at the spatial distribution of theoretical capture for select times of interest. Development of maps of capture requires (1) a reasonably well-constructed transient or steady state model of an aquifer with head-dependent flow boundaries representing surface water features or evapotranspiration and (2) an automated procedure to run the model repeatedly and extract results, each time with a well in a different cell to evaluate the effect of a flux in a cell on an external boundary condition.

% Describe source of problem
This example presents a streamflow capture analysis of the hypothetical aquifer system of \cite{freyberg1988exercise} and based on the model datasets presented in \cite{hunt2020revisiting}. The original problem of \cite{freyberg1988exercise} documented an exercise where graduate students calibrated a groundwater model and then used it to make forecasts.


\subsection{Example Description}

% spatial discretization  and temporal discretization
A single-layer was used to simulate a shallow, water-table aquifer. The aquifer is surrounded by no-flow boundaries on the bottom and north-east-west sides (fig.~\ref{fig:ex-gwf-capture-01}). There is an outcrop area within the grid, where the water-table aquifer is missing. The aquifer is discretized using 40 rows and 20 columns (250 m on a side). A single steady-state stress period, with a single time step, was simulated. An arbitrary simulation period of 1 day was simulated. Model parameters for the example are summarized in table~\ref{tab:ex-gwf-capture-01}. 

The top of the model was set at an arbitrary elevation of 200 $m$ (table~\ref{tab:ex-gwf-capture-01}). The bottom elevations were not uniform; the aquifer was relatively flat on the east side and sloped gently to the south and west sides \citep[see][fig.~1b]{hunt2020revisiting}. In the western area, and southeastern and southwestern corners of the aquifer, the impermeable bottom elevation was higher making for no-flow outcrop areas within the grid. 


\begin{StandardFigure}{
                                     Diagram showing the model domain and the simulated streamflow capture fraction. The location of river cells, constant head cells, and wells are also shown.
                                     }{fig:ex-gwf-capture-01}{../figures/ex-gwf-capture-01.png}
\end{StandardFigure}                                 


% add static parameter table(s)
\input{../tables/ex-gwf-capture-01}


% material properties
Hydraulic conductivity (K) consisted of six zones  \citep[see][fig.~1c]{hunt2020revisiting} with relatively small changes among them; areas of higher values of K were along the north, east and west boundaries, and adjacent to the western outcrop; K was lower in the south.

% initial conditions and boundary conditions
An initial head of 45 $m$ is specified in each model cell. Flow into the system is from infiltration from precipitation and was represented using the recharge (RCH) package. A constant recharge rate of $1 \times 10^{-9}$ $m/s$ was specified for every active cell in the aquifer. A stream represented by river (RIV) package cells in column 15 in every row in the model and have river stages ranging from 20.1 $m$ in row 1 to 11.25 $m$ in row 40, a conductance of 0.05 $m^2/s$, and bottom elevations ranging from 20 $m$ in row 1 to 10.25 $m$ in row 40 (fig.~\ref{fig:ex-gwf-capture-01}). River cells discharge groundwater from the model in every cells except river cells in row 9 and 10, which are a source of water for the model. Additional discharge of groundwater out of the model is from discharging wells represented by well (WEL) package cells and specified head cells. There are six pumping wells (fig.~\ref{fig:ex-gwf-capture-01}) with a total withdrawal rate of 22.05 $m^3/s$. Heads are specified to be constant on the southern boundary in all active cells in row 40 and range from 16.9 $m$ in column 6 to 11.4 $m$ in column 15.

The Newton-Raphson formulation and Newton-Raphson under-relaxation were used to improve model convergence.


% for examples without scenarios
\subsection{Example Results}

The capture fraction analysis was performed using the \MF Application Program Interface (API). The \mfapi is used because the capture fraction for each cell can be calculated without regenerating the input files and running the model for each cell perturbed with an additional flux term. The \mfapi also allows the capture fraction analysis to be performed without ever running or the model to completion (finalizing the time step), which writes output to the listing file. The capture fraction perturbation flux is added to the model using the API package, which adds a specified flux directly to the right-hand side of the system of equations.

The python function that adds the specified flux to each active cell in the model is listed below.

\begin{python}
def calculate_capture_fraction(mobj, inode, q):
  # update the api package with the well
  update_well(mobj, inode, q=q)

  # solve with the updated well
  solve_current(mobj)

  # calculate the total streamflow
  rivcf = get_streamflow(mobj)

  # process the results
  cf = (rivcf - qbase) / abs(cfq)
	
  return cf
\end{python}

\noindent The blah 

\begin{python}
def solve_current(mobj):
    max_iter = mobj.get_value(mobj.get_var_address("MXITER", "SLN_1"))

    # convergence loop
    kiter = 0
    mobj.prepare_solve()

    while kiter < max_iter:
        has_converged = mobj.solve()
        kiter += 1
        if has_converged:
            break

    mobj.finalize_solve()
\end{python}

blah 

\begin{python}
\end{python}


blah 

\begin{python}
def update_well(mobj, node, q=-1e-3):
    tag = mobj.get_var_address("NBOUND", sim_name, "CF-1")
    nbound = mobj.get_value(tag)
    if nbound[0] < 1:
        nbound[0] = 1
        mobj.set_value(tag, nbound)
    tag = mobj.get_var_address("NODELIST", sim_name, "CF-1")
    nodelist = mobj.get_value(tag)
    nodelist[0] = node + 1 # convert from zero-based to one-based node number
    mobj.set_value(tag, nodelist)
    tag = mobj.get_var_address("RHS", sim_name, "CF-1")
    rhs = mobj.get_value(tag)
    rhs[:] = -q
    mobj.set_value(tag, rhs)
    return
\end{python}

blah

\begin{python}
def get_streamflow(mobj):
    tag = mobj.get_var_address("SIMVALS", sim_name, "RIV-1")
    return mobj.get_value(tag).sum()
\end{python}

The simulated streamflow capture fraction results are shown in figure~\ref{fig:ex-gwf-capture-01}. The stream captures all of the inflow to the model except west of the river in the vicinity close to the constant head boundaries. Cells close to the western-most constant head boundary cells do not contribute any groundwater to the stream.




% GWT Model examples
\clearpage
\insection
\section{Keating Problem}

% Describe source of problem
This example is based on an unsaturated flow and transport problem described in \cite{keating2009stable} under the section ``Extension to Mixed Vadose/ Saturated Zone Simulations.''  The problem consists of a two-dimensional cross-section model with a perched aquifer overlying a water table aquifer.  This problem offers a difficult test for the Newton flow formulation in \mf as well as for the transport model, which must transmit solute through dry cells.  This problem was also used by \cite{mt3dusgs} as a test for their solute routing approach implemented in MT3D-USGS for flow models solved with MODFLOW-NWT \citep{modflownwt}.

In addition to the transport model, this example also includes a particle tracking model.  The model demonstrates the default approach taken by \mf to track particles through dry cells under the Newton formulation, which is to drop the particles instantaneously to the top-most active cell in the vertical column and resume tracking as usual.

\subsection{Example description}

The parameters used for this problem are listed in table~\ref{tab:ex-gwt-keating-01}.  The model grid consists of 1 row, 400 columns, and 80 layers.  The flow problem consists of a perched aquifer overlying an unconfined water table aquifer.  A perched aquifer forms due the presence of a thin, discontinuous low permeability lens located near the center of the model domain.  Flow conditions are simulated as steady state.  The solute transport simulation represents transient conditions, which begin with an initial concentration specified as zero everywhere within the model domain.  For the first 730 days, recharge enters at a concentration of one.  For the remainder of the simulation, recharge has a solute concentration of zero.

Constant-head conditions are prescribed on the left and right sides of the model with head values of 800 $m$ and 100 $m$, respectively.  These constant-head conditions are only assigned if the cell bottom elevation is below the prescribed head value.  Water entering the model from the constant head cells is assigned a concentration of zero.  Water leaving the constant head cells on the right side of the model leaves at the simulated concentration in the cell.  

The Newton formulation is used to simulate flow through the domain.  Recharge is assigned to the top of the model, and although upper model cells are dry, this recharge water is instantaneously transmitted down to the perched aquifer.  The perched aquifer flows to the left and right and spills over the edges of the confining unit.  Water flowing over the edges instantaneously recharges the underlying water table aquifer.  A negligible amount of water flows through the low permeability lens and then down into the water table aquifer.  The flow problem is challenging to solve and converges best with backtracking, under relaxation, and solver parameters designed for complex problems.

For the transport model, the simulation period is divided into 3000, 10-day time steps.  Advection and dispersion are simulated.  Because the longitudinal and transverse dispersivities are equal, the computationally demanding XT3D approach is not needed to represent dispersion.  The simpler method for calculating dispersion is used instead, and gives comparable results to those obtained with XT3D.  

% add static parameter table(s)
\input{../tables/ex-gwt-keating-01}

% for examples without scenarios
\subsection{Example Results}

Simulated heads from \mf are shown in figure~\ref{fig:ex-gwt-keating-head}.  Cells with a calculated head beneath the cell bottom are considered ``dry'' and are not shown with a color.   The zone of recharge is shown in red on the top of the plot.  

% a figure
\begin{StandardFigure}{
                                     Color shaded plot of heads simulated by \mf for the \cite{keating2009stable} problem involving groundwater flow and transport through an unsaturated zone.  Recharge is applied to the narrow strip shown in red on the top of the model domain.
                                     }{fig:ex-gwt-keating-head}{../figures/ex-gwt-keating-head.png}
\end{StandardFigure}                                 

Simulated concentrations from \mf are shown in figure~\ref{fig:ex-gwt-keating-conc} for three different times.  These plots show the development of the solute plume in the perched aquifer, and then shows flushing of the perched aquifer when the recharge concentration becomes zero.  In the underlying water table aquifer, two solute plumes are formed as groundwater flows over the edges of the low permeability lens.  These plumes then flow toward the right, and eventually exit through the constant head cells.  Plots of concentration versus time for the two yellow points shown in figure~\ref{fig:ex-gwt-keating-conc} are shown in figure~\ref{fig:ex-gwt-keating-cvt}.  Results from \mf are shown with the results presented by \cite{keating2009stable} and are in good agreement considering the complexity of the problem.  Similar plots for this problem are also presented by \cite{mt3dusgs}.

Simulated particle tracks from \mf are shown in figure~\ref{fig:ex-gwt-keating-tracks}.  A particle is released at the center of every cell in the top layer of the grid.  Particle pathlines are colored such that adjacent pathlines may be easily distinguished.

% a figure
\begin{StandardFigure}{
                                     Color shaded plots of concentrations simulated by \mf for the \cite{keating2009stable} problem involving groundwater flow and transport through an unsaturated zone.  This plot can be compared to figure 11 in \cite{mt3dusgs}, which shows similar plots for MT3D-USGS results.  Plots of concentration versus time are shown in figure~\ref{fig:ex-gwt-keating-cvt} for the two points shown in yellow.
                                     }{fig:ex-gwt-keating-conc}{../figures/ex-gwt-keating-conc.png}
\end{StandardFigure}                                  

\begin{StandardFigure}{
                                    Particle pathlines simulated by \mf for the \cite{keating2009stable} problem involving groundwater flow and transport through an unsaturated zone.  A particle is released at the center of every cell in the top layer of the grid.
                                    }{fig:ex-gwt-keating-tracks}{../figures/ex-gwt-keating-tracks.png}
\end{StandardFigure}      

\begin{StandardFigure}{
                                     Concentrations versus time for two observation points as simulated by \mf and by \cite{keating2009stable} for a problem involving groundwater flow and transport through an unsaturated zone.  This plot can be compared to figure 12 in \cite{mt3dusgs}, which shows a similar plot for MT3D-USGS results.
                                     }{fig:ex-gwt-keating-cvt}{../figures/ex-gwt-keating-cvt.png}
\end{StandardFigure}                                  


\clearpage
\insection
\section{MOC3D Problem 1}

% Describe source of problem
This problem corresponds to the first problem presented in the MOC3D report \cite{konikow1996three}, which involves the transport of a dissolved constituent in a steady, one-dimensional flow field.  An analytical solution for this problem is given by \cite{wexler1992}.  This example is simulated with the GWT Model in \mf, which receives flow information from a separate simulation with the GWF Model in \mf.  Results from the GWT Model are compared with the results from the \cite{wexler1992} analytical solution.  

\subsection{Example description}

The parameters used for this problem are listed in table~\ref{tab:ex-gwt-moc3d-p01-01}.  The model grid for this problem consists of one layer, 120 rows, and 1 columns.  The top for each cell is assigned a value of 1.0 $cm$ and the bottom is assigned a value of zero.  DELR is set to 1.0 $cm$ and DELC is specified with a constant value of 0.1 $cm$.  The simulation consists of one stress period that is 120 $s$ in length, and the stress period is divided into 240 equally sized time steps.  By using a uniform porosity value of 0.1, a velocity value of 0.1 $cm/s$ results from the injection of water at a rate of 0.001 $cm^3/s$ into the first cell.  The last cell is assigned a constant head with a value of zero, though this value is not important as the cells are marked as being confined.  The concentration of the injected water is assigned a value of 1.0, and any water that leaves through the constant-head cell leaves with the simulated concentration of the water in that last cell.   Advection is solved using the TVD scheme to reduce numerical dispersion.

% add static parameter table(s)
\input{../tables/ex-gwt-moc3d-p01-01}

% for examples with scenarios
\subsection{Example Scenarios}

This example problem consists of several different scenarios, as listed in table~\ref{tab:ex-gwt-moc3d-p01-scenario}.  Two different levels of dispersion were simulated, and these simulations are referred to as the low dispersion case and the high dispersion case.  The low dispersion case has a dispersion coefficient of 0.01 $cm^2/s$, which, for the specified velocity, corresponds to a dispersivity value of 0.1 $cm$.  The high-dispersion case has a dispersion coefficient of 0.1 $cm^2/s$, which corresponds to a dispersivity value of 1.0 $cm$.

% add scenario table
\input{../tables/ex-gwt-moc3d-p01-scenario}

\subsubsection{Scenario Results}

%Scenario 1
For the first scenario with a relatively small dispersivity value (0.1 $cm$), plots of concentration versus time and concentration versus distance are shown in figures~\ref{fig:ex-gwt-moc3d-p01a-ct} and ~\ref{fig:ex-gwt-moc3d-p01a-cd}, respectively.  Figure~\ref{fig:ex-gwt-moc3d-p01a-ct} can be compared to figure 18 in \cite{konikow1996three}. The three separate concentration versus time curves represent the three different distances (0.05, 4.05, and 11.05 cm).  For this low-dispersion case, the MODFLOW 6 solution is in relatively good agreement with the analytical solution, but some slight differences are observed.  These differences are due primarily to limitations with the second-order TVD scheme implemented in MODFLOW 6, which can suffer from numerical dispersion.

% a figure
\begin{StandardFigure}{
                                     Concentrations simulated by the \mf GWT Model and calculated by the analytical solution for one-dimensional flow with transport for the low dispersion case.  Circles are for the GWT Model results; the lines represent the analytical solution by \cite{wexler1992}.  Results are shown for three different distances (0.05, 4.05, and 11.05 $cm$ from the end of the first cell).  Every fifth time step is shown for the MODFLOW 6 results.
                                     }{fig:ex-gwt-moc3d-p01a-ct}{../figures/ex-gwt-moc3d-p01a-ct.png}
\end{StandardFigure}            

% a figure
\begin{StandardFigure}{
                                     Concentrations simulated by the \mf GWT Model and calculated by the analytical solution for one-dimensional flow with transport for the low dispersion case.  Circles are for the GWT Model results; the lines represent the analytical solution by \cite{wexler1992}.  Results are shown for three different times (6, 60, and 120 $s$).  Every fifth cell is shown for the MODFLOW 6 results.
                                     }{fig:ex-gwt-moc3d-p01a-cd}{../figures/ex-gwt-moc3d-p01a-cd.png}
\end{StandardFigure}            

%Scenario 2
For the second scenario, which has a relatively large dispersivity value (1.0 $cm$), plots of concentration versus time and concentration versus distance are shown in figures~\ref{fig:ex-gwt-moc3d-p01b-ct} and ~\ref{fig:ex-gwt-moc3d-p01b-cd}, respectively.  For the high-dispersion case, the results from the MODFLOW 6 simulation are in better agreement with the analytical solution than for the low dispersion case.

% a figure
\begin{StandardFigure}{
                                     Concentrations simulated by the \mf GWT Model and calculated by the analytical solution for one-dimensional flow with transport for the high dispersion case.  Circles are for the GWT Model results; the lines represent the analytical solution by \cite{wexler1992}.  Results are shown for three different distances (0.05, 4.05, and 11.05 $cm$ from the end of the first cell).  Every fifth time step is shown for the MODFLOW 6 results.
                                     }{fig:ex-gwt-moc3d-p01b-ct}{../figures/ex-gwt-moc3d-p01b-ct.png}
\end{StandardFigure}            

% a figure
\begin{StandardFigure}{
                                     Concentrations simulated by the \mf GWT Model and calculated by the analytical solution for one-dimensional flow with transport for the high dispersion case.  Circles are for the GWT Model results; the lines represent the analytical solution by \cite{wexler1992}.  Results are shown for three different times (6, 60, and 120 $s$).  Every fifth cell is shown for the MODFLOW 6 results.
                                     }{fig:ex-gwt-moc3d-p01b-cd}{../figures/ex-gwt-moc3d-p01b-cd.png}
\end{StandardFigure}            


%Scenarios  3 and 4
For the remaining scenarios, the results from MODFLOW 6 are compared with the \cite{wexler1992} analytical solution for two variations of the high-dispersion case.  The effects sorption are included in scenario 3 and the effects of decay are included in scenario 4.  Plots of concentration versus time and concentration versus distance for a simulation with a retardation factor of 2.0 are shown in figures~\ref{fig:ex-gwt-moc3d-p01c-ct} and ~\ref{fig:ex-gwt-moc3d-p01c-cd}, respectively.  Plots of concentration versus time and concentration versus distance for a simulation with a decay rate of 0.01 $s^{-1}$ are shown in figures~\ref{fig:ex-gwt-moc3d-p01d-ct} and ~\ref{fig:ex-gwt-moc3d-p01d-cd}, respectively.

% a figure
\begin{StandardFigure}{
                                     Concentrations simulated by the \mf GWT Model and calculated by the analytical solution for one-dimensional flow with transport for the high dispersion case and a retardation factor of 2.  Circles are for the GWT Model results; the lines represent the analytical solution by \cite{wexler1992}.  Results are shown for three different distances (0.05, 4.05, and 11.05 $cm$ from the end of the first cell).  Every fifth time step is shown for the MODFLOW 6 results.
                                     }{fig:ex-gwt-moc3d-p01c-ct}{../figures/ex-gwt-moc3d-p01c-ct.png}
\end{StandardFigure}            

% a figure
\begin{StandardFigure}{
                                     Concentrations simulated by the \mf GWT Model and calculated by the analytical solution for one-dimensional flow with transport for the high dispersion case and a retardation factor of 2.  Circles are for the GWT Model results; the lines represent the analytical solution by \cite{wexler1992}.  Results are shown for three different times (6, 60, and 120 $s$).  Every fifth cell is shown for the MODFLOW 6 results.
                                     }{fig:ex-gwt-moc3d-p01c-cd}{../figures/ex-gwt-moc3d-p01c-cd.png}
\end{StandardFigure}            

% a figure
\begin{StandardFigure}{
                                     Concentrations simulated by the \mf GWT Model and calculated by the analytical solution for one-dimensional flow with transport for the high dispersion case and a decay rate of 0.01 $s^{-1}$.  Circles are for the GWT Model results; the lines represent the analytical solution by \cite{wexler1992}.  Results are shown for three different distances (0.05, 4.05, and 11.05 $cm$ from the end of the first cell).  Every fifth time step is shown for the MODFLOW 6 results.
                                     }{fig:ex-gwt-moc3d-p01d-ct}{../figures/ex-gwt-moc3d-p01d-ct.png}
\end{StandardFigure}            

% a figure
\begin{StandardFigure}{
                                     Concentrations simulated by the \mf GWT Model and calculated by the analytical solution for one-dimensional flow with transport for the high dispersion case and a decay rate of 0.01 $s^{-1}$.  Circles are for the GWT Model results; the lines represent the analytical solution by \cite{wexler1992}.  Results are shown for three different times (6, 60, and 120 $s$).  Every fifth cell is shown for the MODFLOW 6 results.
                                     }{fig:ex-gwt-moc3d-p01d-cd}{../figures/ex-gwt-moc3d-p01d-cd.png}
\end{StandardFigure}            



\clearpage
\insection
\section{Three-Dimensional Steady Flow with Transport (MOC3D Problem 2)}

% Describe source of problem
This problem corresponds to the second problem presented in the MOC3D report \cite{konikow1996three}, which involves the transport of a dissolved constituent in a steady, three-dimensional flow field.  An analytical solution for this problem is given by \cite{wexler1992}.  This example is simulated with the GWT Model in \mf, which receives flow information from a separate simulation with the GWF Model in \mf.  Results from the GWT Model are compared with the results from the \cite{wexler1992} analytical solution.

\subsection{Example description}
\cite{wexler1992} presents an analytical solution for three dimensional solute transport from a point source in a one-dimensional flow field.  As described by \cite{konikow1996three}, only one quadrant of the three-dimensional domain is represented by the numerical model.  Thus, the solute mass flux specified for the model is one quarter of the solute mass flux used in the analytical solution.  

The parameters used for this problem are listed in table~\ref{tab:ex-gwt-moc3d-p0201}.  The model grid for this problem consists of 40 layers, 12 rows, and 30 columns.  The top for layer 1 is set to zero, and flat bottoms are assigned to all layers based on a uniform layer thickness of 0.05 $m$.  DELR is set to 3.0 $m$ and DELC is specified with a constant value of 0.5 $m$.  The simulation consists of one stress period that is 400 $d$ in length, and the stress period is divided into 400 equally sized time steps.  Velocity is specified to be 0.1 $m/d$ in the x direction and zero in the y and z directions.  The uniform flow field is represented by specifying a constant inflow rate into all of the cells in column 1 and by specifying a constant head condition to all of the cells in column 30.  A specified solute flux of 10 grams per day is specified to the cell in layer 1, row 12, and column 8.  Any water that leaves through the constant-head cell leaves with the simulated concentration of the water in that last cell.   Advection is solved using the TVD scheme to reduce numerical dispersion.  In addition to the longitudinal dispersion, transverse dispersion is represented with a different value in the horizontal direction than in the vertical direction.  Because the velocity field is perfectly aligned with the model grid, there are no cross-dispersion terms and the problem can be simulated accurately without the need for XT3D.

% add static parameter table(s)
\begin{StandardTable}{Parameters used for the example of three-dimensional flow with transport (MOC3D Problem 2 model parameters).}{tab:ex-gwt-moc3d-p0201}{../tables/ex-gwt-moc3d-p0201}
\end{StandardTable}

% for examples without scenarios
\subsection{Example Results}

A comparison of the MODFLOW 6 results with the analytical solution of \cite{wexler1992} is shown for layer 1 in figure~\ref{fig:ex-gwt-moc3dp2-map}.

% a figure
\begin{StandardFigure}{
                                     Concentrations simulated by the \mf GWT Model and calculated by the analytical solution for three-dimensional flow with transport.  Results are for the end of the simulation (time=400 $d$) and for layer 1.  Black lines represent solute concentration contours from the analytical solution \citep{wexler1992}; blue lines represent solute concentration contours simulated by \mf.  An aspect ratio of 4.0 is specified to enhance the comparison.
                                     }{fig:ex-gwt-moc3dp2-map}{ex-gwt-moc3dp2-map}
\end{StandardFigure}            

                


\clearpage
\insection
\section{MOC3D Problem 2 with Triangular Grid}

% Describe source of problem
This problem corresponds to the second problem presented in the MOC3D report \cite{konikow1996three}, which involves the transport of a dissolved constituent in a steady, three-dimensional flow field.  An analytical solution for this problem is given by \cite{wexler1992}.  As for the previous example, this example is simulated with the GWT Model in \mf, which receives flow information from a separate simulation with the GWF Model in \mf.  In this example, however, a triangular grid is used for the flow and transport simulation.  Results from the GWT Model are compared with the results from the \cite{wexler1992} analytical solution.

% a figure
\begin{StandardFigure}{
                                     Triangular model grid used for the \mf simulation.  Model grid is shown using an aspect ratio of 4.
                                     }{fig:ex-gwt-moc3d-p02tg-grid}{../figures/ex-gwt-moc3d-p02tg-grid.png}
\end{StandardFigure}            


\subsection{Example description}

\cite{wexler1992} presents an analytical solution for three dimensional solute transport from a point source in a one-dimensional flow field.  As described by \cite{konikow1996three}, only one quadrant of the three-dimensional domain is represented by the numerical model.  Thus, the solute mass flux specified for the model is one quarter of the solute mass flux used in the analytical solution.  

The parameters used for this problem are listed in table~\ref{tab:ex-gwt-moc3d-p02tg-01}.  The model grid for this problem consists of 40 layers, 695 cells per layer, and 403 vertices in a layer.  The top for layer 1 is set to zero, and flat bottoms are assigned to all layers based on a uniform layer thickness of 0.05 $m$.  The remaining parameters are set similarly to the previous simulation with a regular grid, except for in this simulation, the XT3D method is used for flow and dispersive transport.

% add static parameter table(s)
\input{../tables/ex-gwt-moc3d-p02tg-01}

% for examples without scenarios
\subsection{Example Results}

A comparison of the MODFLOW 6 results with the analytical solution of \cite{wexler1992} is shown for layer 1 in figure~\ref{fig:ex-gwt-moc3d-p02tg-map}.

% a figure
\begin{StandardFigure}{
                                     Concentrations simulated by the \mf GWT Model and calculated by the analytical solution for three-dimensional flow with transport.  Results are for the end of the simulation (time=400 $d$) and for layer 1.  Black lines represent solute concentration contours from the analytical solution \citep{wexler1992}; blue lines represent solute concentration contours simulated by \mf.  An aspect ratio of 4.0 is specified to enhance the comparison.
                                     }{fig:ex-gwt-moc3d-p02tg-map}{../figures/ex-gwt-moc3d-p02tg-map.png}
\end{StandardFigure}            

                


\clearpage
\insection
\section{MT3DMS Problem 1}

% Describe source of problem
Section 7 of \cite{zheng1999mt3dms} details a number of test problems that verify the accuracy of MT3DMS.  The first problem presented, titled "one-dimensional transport in a uniform flow field," compared MT3DMS solutions to analytical solutions given by \cite{vanGenuchtenAlves1982}.  For verifying the accuracy of transport calculations within MODFLOW6, the transport solutions calculated by MT3DMS serve as the benchmark to which the \mf solution is compared.  The first 1-dimensional simulation solves for advection only.  The second model permutation uses both advection and dispersion to verify \mf results.  Next, the accuracy of \mf is verified when advection, dispersion, and some simple checmial reactions represented with sorption processes are used.  The fourth and final model permutation adds solute decay to the previous model setup. As of the first release of \mf with transport capabilities, a linear isotherm is the only option available for simulating sorption. Arbitrary values of bulk density and distribution coefficient are uniformly applied to the entire model domain to achieve the indicated retardation factor. The following table summarizes how the four simulations incrementally increase model complexity.

% add scenario table
\input{../tables/ex-gwt-mt3dms-p01-scenario}

\subsection{Example description}

All four model scenarios have 101 columns, 1 row, and 1 layer. The first and last columns use constant-head boundaries to simulate steady flow in confined conditions. Because the analytical solution assumes an infinite 1-dimensional flow field, the last column is set far enough from the source to avoid interfering with the final solution after 2,000 days. Initially, the model domain is devoid of solute; however, the first column uses a constant concentration boundary condition to ensure that water entering the simulation has a unit concentration of 1. Additional model parameters are shown in table~\ref{tab:ex-gwt-mt3dms-p01-01}.

% add 2nd static parameter value table
\input{../tables/ex-gwt-mt3dms-p01-01}

% for examples without scenarios
\subsection{Example Results}

Currently no options are available with \mf for simulating solute transport using particle tracking methods [referred to as Method of Characteristics (MOC) in the MT3DMS manual \cite{zheng1999mt3dms}.  Thus, the \mf solution is compared to an MT3DMS solution that uses the third-order total variation diminishing (TVD) option for solving the advection-only problem rather than invoking one of the MOC options available within MT3DMS.  Owing to different approaches between the two codes, namely TVD scheme of  MT3DMS and the second-order approach of \mf, differences between the two solutions and reflected in figure~\ref{fig:ex-gwt-mt3dms-p01a} are expected.  However, the differences are within acceptable tolerances.

The comparison of the MT3DMS and\mf solutions for problem 1a, an advection dominated problem, represents an end-member test as the migrating concentration front is sharp (i.e., discontinuous). In technical terms, the grid Peclet number is infinity for this problem ($P_e$ = $v\Delta x/D_{xx}$ = $\Delta x$/$\alpha_L$ = $\infty$).

% a figure
\begin{StandardFigure}
	{Comparison of the MT3DMS and \mf numerical solutions for a one-dimensional advection dominated test problem.  The analytical solution for this problem was originally given in \cite{vanGenuchtenAlves1982} and is not shown here}
	{fig:ex-gwt-mt3dms-p01a}{../figures/ex-gwt-mt3dms-p01a.png}
\end{StandardFigure}

A comparison of MT3DMS and \mf for scenario 2 in the MT3DMS manual represents a more common situation whereby dispersion acts to spread or smooth the advancing concentration front.  For this problem, the dispersion term $\alpha_L$ is set equal to the length of the grid cell in the direction of flow, 10 cm, resulting in a Peclet number equal to one ($P_e$ = $v\Delta x/D_{xx} = 10/10 = 1$).  Owing to the presense of dispersion, the finite-difference solutions employed by both MT3DMS and \mf for this problem are more accurate, and as a result are in closer agreement (figure \ref{fig:ex-gwt-mt3dms-p01b}).

% a figure
\begin{StandardFigure}
	{Comparison of the MT3DMS and MODFLOW 6 numerical solutions for a one-dimensional test problem with dispersion ($\alpha_L$ = 10 $cm$).  The analytical solution for this problem was originally given in \cite{vanGenuchtenAlves1982} and is not shown here}
	{fig:ex-gwt-mt3dms-p01b}{../figures/ex-gwt-mt3dms-p01b.png}
\end{StandardFigure}

The third comparison for the one-dimensional transport in a steady flow field includes uses the same dispersion specified for the second scenario, but adds retardation.  For this problem, retardation slows the advance of the migrating concentration front by simulating sorption of the dissolved solute onto the matrix material through which the fluid is moving. In this way, dissolved mass is transferred from the aqueous phase to the solid phase. Appropriate reaction package parameter values are determined for obtaining the specified retardation (5.0) within the code. Figure \ref{fig:ex-gwt-mt3dms-p01c} shows a close match between MT3DMS and \mf. In addition, figure~\ref{fig:ex-gwt-mt3dms-p01c} also shows that after 2,000 days, the concentration front did not advance as far as shown in figure~\ref{fig:ex-gwt-mt3dms-p01b}.

% a figure
\begin{StandardFigure}
	{Comparison of the MT3DMS and \mf numerical solutions for a one-dimensional test problem with dispersion ($\alpha_L$ = 10 $cm$) and retardation ($R$ = $5.0$). The analytical solution for this problem was originally given in \cite{vanGenuchtenAlves1982} and is not shown here}
	{fig:ex-gwt-mt3dms-p01c}{../figures/ex-gwt-mt3dms-p01c.png}
\end{StandardFigure}

The final comparison for the fourth scenario of problem 1 adds decay to the dispersion and retardation simulated in the third scenario. For this case, decay represents the irreversible loss of mass from both the aqueous and sorbed phases, further stunting the advance of the migrating concentration front \ref{fig:ex-gwt-mt3dms-p01d}.  

% a figure
\begin{StandardFigure}
	{Comparison of the MT3DMS and MODFLOW 6 numerical solutions for a one-dimensional test problem with dispersion ($\alpha_L$ = 10 $cm$), retardation ($R$ = $5.0$) and decay ($\lambda$ = 0.002 $d^{-1}$).  The analytical solution for this problem was originally given in \cite{vanGenuchtenAlves1982} and is not shown here}
	{fig:ex-gwt-mt3dms-p01d}{../figures/ex-gwt-mt3dms-p01d.png}
\end{StandardFigure}





\clearpage
\insection
\section{MT3DMS Problem 2}

% Describe source of problem
This is the second example problem presented in \cite{zheng1999mt3dms}, titled ``one-dimensional transport with nonlinear or nonequilibrium sorption''.  The purpose of this example is to demonstrate simulation of nonlinear and nonequilibrium sorption.   For the nonlinear sorption, \cite{zheng1999mt3dms} compared MT3DMS results to the results from an independent transport simulator called CXTFIT.  For the nonequilibrium sorption example \cite{zheng1999mt3dms} compared the results from MT3DMS to an analytical solution.  In this section the results from the \mf GWT Model are compared with the results from MT3DMS.

In MT3DMS, nonequilibrium sorption is represented using the following expression for the dissolved and sorbed solute mass,

\begin{equation}
	\rho_b \frac{\partial \overline{C}}{\partial t} = \beta \left ( C - \frac{\overline{C}}{K_d} \right ),
	\label{eq:nonequilbrium}
\end{equation}

\noindent where $\rho_b$ is the bulk density, $\overline{C}$ is the sorbed concentration, $C$ is the dissolved solute mass, $\beta$ is the first-order mass transfer rate between the dissolved and sorbed phases, $t$ is time, and $K_d$ is the linear distribution coefficient.  The current version of the \mf GWT Model does not have the capability to represent this type of nonequilibrium sorption.  

For the example presented by \cite{zheng1999mt3dms}, the distribution coefficient has a value close to one (0.933). Because this value is so close to one, we can take advantage of immobile domain capability to approximate nonequilibrium sorption.  If it is possible to neglect a separate sorption process within the immobile domain and decay or production within the immobile domain (as has been defined for this example), then transfer of dissolved solute mass between a mobile and immobile domain is expressed as

\begin{equation}
	\theta_{im} \frac{\partial C_{im}}{\partial t} = \zeta \left ( C - C_{im} \right ),
	\label{eq:nonequilbrium}
\end{equation}

\noindent where $\theta_{im}$ is the porosity of the immobile domain, $\zeta$ is the first-order mass transfer rate between the mobile and immobile domains, and $C_{im}$ is the concentration of the immobile domain.  From the similar form of these two equations, it is possible to approximate the nonequilibrium sorption capability in MT3DMS by assigning the immobile domain porosity as bulk density and the first-order mass transfer rate for the mobile and immobile domain as the first-order mass transfer rate for the dissolved and sorbed phases.

There are six scenarios described here (tab. \ref{tab:ex-gwt-mt3dms-p02-scenario}).  The first two represent nonlinear sorption with the Freundlich and Langmuir isotherms, respectively.  The remaining four scenarios represent nonequilibrium sorption with different values for the first-order mass transfer coefficient.  These six scenarios and corresponding figures correspond to the ones reported by \cite{zheng1999mt3dms}.

% add scenario table
\input{../tables/ex-gwt-mt3dms-p02-scenario}

\subsection{Example description}

All model scenarios have 101 columns, 1 row, and 1 layer. The first column has a specified inflow rate that results in a velocity of 0.1 $cm/s$. The last column is assigned as a constant-head boundary to allow water to exit.  For the first 160 $s$, a pulse of solute is introduced to the inflowing water.  For the remaining 1340 $s$, the concentration of inflowing water is zero.  Additional model parameters are shown in table~\ref{tab:ex-gwt-mt3dms-p02-01}.

% add 2nd static parameter value table
\input{../tables/ex-gwt-mt3dms-p02-01}

% for examples without scenarios
\subsection{Example Results}
Simulated results for nonlinear sorption are shown for the Freundlich isotherm in figure~\ref{fig:ex-gwt-mt3dms-p02a} and for the Langmuir isotherm in figure~\ref{fig:ex-gwt-mt3dms-p02b}.  Results for the four simulations with different values for beta are shown in figure~\ref{fig:ex-gwt-mt3dms-p02}.

% a figure
\begin{StandardFigure}
	{Comparison of the MT3DMS (solid lines) and \mf (circles) results for Freundlich isotherm.}
	{fig:ex-gwt-mt3dms-p02a}{../figures/ex-gwt-mt3dms-p02a-ct.png}
\end{StandardFigure}

% a figure
\begin{StandardFigure}
	{Comparison of the MT3DMS (solid lines) and \mf (circles) results for Langmuir isotherm.}
	{fig:ex-gwt-mt3dms-p02b}{../figures/ex-gwt-mt3dms-p02b-ct.png}
\end{StandardFigure}

% a figure
\begin{StandardFigure}
	{Comparison of the MT3DMS (solid lines) and \mf (circles) results for four different values of the nonequilibrium exchange coefficient (beta).}
	{fig:ex-gwt-mt3dms-p02}{../figures/ex-gwt-mt3dms-p02.png}
\end{StandardFigure}





\clearpage
\insection
\section{Two-Dimensional Transport in a Uniform Flow Field (MT3DMS Example Problem 3)}

The second example problem in \cite{zheng1999mt3dms} verifies accurate simulation of nonlinear and nonequilibrium sorption by comparing results to corresponding analytical solutions. As neither capability is included in the initial release of the transport process written for\mf, the next MT3DMS-\mf transport comparison is the third problem appearing in \cite{zheng1999mt3dms}, titled, "two-dimensional transport in a uniform flow field." In contrast to the first demonstrated test problem, transport is simulated in two dimensions with dispersion but no reactions. An analytical solution for this problem was originally published in \cite{wilson1978}. Two assumptions that make the analytical solution possible are that (1) the aquifer is areally infinite and relatively thin to support the assumption that instantaneous mixing occurs in the vertical direction, and (2) that compared to the ambient flow field, the injection rate is insignificant.  

\subsection{Example description}

Steady uniform flow enters the left edge of a numerical grid with 31 rows, 46 columns, and 1 layer through a constant head boundary and exits along the right edge. Constant heads are selected to ensure the hydraulic gradient matches with the analytical solution. The other boundaries are all no flow. Boundaries are sufficiently far away from the injection well where the contaminant is released so as not to interfere with the final solution after 365 days. Table~\ref{tab:ex-gwt-mt3dms-p03-01} summarizes model setup:

% add static parameter value table
\input{../tables/ex-gwt-mt3dms-p03-01}

After 365 days, the \mf solution aligns well with the MT3DMS solution \ref{fig:mt3dms_p03}. In addition to the good agreement that was seen in the first  MT3DMS test problem, the current comparison confirms that the lateral dispersion is accurately simulated within \mf.

% MT3DMS manual figure 35
%\begin{figure}
%	\centering
%	\includegraphics[width=0.8\textwidth]{../figures/p03mt3d-f1}
%	\caption{Comparison of the MT3DMS and MODFLOW 6 numerical solutions for a two-dimensional transport in a uniform flow field problem. The analytical solution for this problem was originally given in \cite{wilson1978} and is not shown here} 
%	\label{fig:mt3dms_p03}
%\end{figure}

\begin{StandardFigure}
	{Comparison of the MT3DMS and \mf numerical solutions for a two-dimensional advection-dispersion test problem.  The analytical solution for this problem was originally given in \cite{wilson1978} and is not shown here}
	{fig:ex-gwf-mt3d-p03}{../figures/ex-gwt-mt3dms-p03.png}
\end{StandardFigure}

\clearpage
\insection
\section{Two-Dimensional Transport in a Diagonal Flow Field (MT3DMS Example Problem 4)}

The third demonstrated MT3DMS-\mf transport comparson is for a two-dimensional transport in a diagonal flow field. This problem is similar to the preceding problem with two important changes. First, the flow direction is now oriented at a 45-degree angle relative to rows and columns of the numerical grid. Owing the use of MT3DMS for comparison, the \mf solution uses the traditional DIS package (not DISU or DISV).  The second notable change is that the number of rows and columns has been expanded in order to accomodate a longer simulation period of 1,000 days. Because of the orientation of the flow field relative to the model grid, and the sharpness of the migrating concentration front, this test problem presents a challenging set of conditions to simulate. Three scenarios test alternative advection formulations, as summarized in table~\ref{tab:ex-gwt-mt3dms-p04-scenario}

% add scenario table
\begin{ScenarioTable}{
       Variation in advection scheme for testing transport solutions. The $mixelm$ parameter is an integer flag used by the model to set the advection solution option.  A value of -1 uses a third-order TVD scheme \citep{zheng1999mt3dms}, 0 uses the standard finite-difference (FD) solution, and 1 invokes the method-of-characteristics (MOC) solution in MT3DMS.  Because MF6 does not have a method-of-characteristics solution for transport, a value of 1 is the same as specifying 0 (FD solution)}
       {tab:ex-gwt-mt3dms-p04-scenario}
       {../tables/ex-gwt-mt3dms-p04-scenario}
\end{ScenarioTable}

Model parameter values for this problem are provided in table~\ref{tab:ex-gwt-mt3dms-p0401}.

% add 2nd static parameter value table
\begin{StandardTable}
	{Hydraulic and transport model parameter values used in the two-dimensional transport in a diagonal flow field problem.}
	{tab:ex-gwt-mt3dms-p0401}
	{../tables/ex-gwt-mt3dms-p0401}
\end{StandardTable}

The same analytical solution used in the previous problem can be used for this problem after applying the necessary updates to select parameters - most noteably the dispersion and porosity terms and that an inter-model comparison is drawn after 1,000 days instead of 365 days. Figure 36 in \cite{zheng1999mt3dms} shows four different solutions for this problem: (1) analytical, (2) Method of Characteristics (MOC), (3) upstream finite difference (FD), and (4) Total Variation Dimishing (TVD) or "ULTIMATE" scheme.  Both the MOC and TVD solutions demonstrate a reasonable agreement with the analytical solution. However, the upstream finite difference solution reflects considerably more spread from simulation of too much dispersion - in this case numerical dispersion instead of hydrodynamic dispersion. 

The \mf transport solution is compared to all three numerical solutions (FD, TVD, and MOC) presented in \cite{zheng1999mt3dms}. The first comparison shows complete agreement between MT3DMS and the \mf transport solution when the finite difference approach is applied (figure~\ref{fig:ex-gwt-mt3d-p04a}). 

% MT3DMS manual figure 36a
\begin{StandardFigure}
	{Comparison of the MT3DMS and \mf numerical solutions for two-dimensional transport in a diagonal flow field. Both models are using their respective finite difference solutions without the TVD option} 
	{fig:ex-gwt-mt3d-p04a}
	{../figures/ex-gwt-mt3d-p04a}
\end{StandardFigure}

Figure~\ref{fig:ex-gwt-mt3d-p04b} shows a comparison between the MT3DMS and \mf solution with the respective TVD options for each model activated.  Owing to the fact that MT3DMS uses a third-order TVD scheme while \mf uses a second-order scheme, differences between the two solutions are expected. 

% MT3DMS manual figure 36b
\begin{StandardFigure}
	{Comparison of the MT3DMS and \mf numerical solutions for two-dimensional transport in a diagonal flow field. Both models are using their respective finite difference solutions with the use of the TVD option, which serves as the main difference with results displayed in figure~\ref{fig:ex-gwt-mt3d-p04b}} 
	{fig:ex-gwt-mt3d-p04b}
	{../figures/ex-gwt-mt3d-p04b}
\end{StandardFigure}

The third model comparison shows the largest difference between the two solutions (figure~\ref{fig:ex-gwt-mt3d-p04c}). Because the MOC solution is the closest facsimile of the analytical solution, comparison of \mf with the MT3DMS MOC solution is as close to a comparison with the analytical solution as will be shown for the current set of model runs. 

% MT3DMS manual figure 36c
\begin{StandardFigure}
	{Comparison of the MT3DMS and \mf numerical solutions for two-dimensional transport in a diagonal flow field. Here, MT3DMS is using a MOC technique to find a solution while \mf uses finite difference without TVD activated.} 
	{fig:ex-gwt-mt3d-p04c}
	{../figures/ex-gwt-mt3d-p04c}
\end{StandardFigure}





\clearpage
\insection
\section{MT3DMS Problem 5}

The next example problem tests two-dimensional transport in a radial flow field.  The radial flow field is established by injecting water in the center of the model domain (row 16, column 16) and allowing it to flow outward toward the model perimeter. No regional groundwater flow gradient exists as in some of the previous comparisons with MT3DMS. Constant head cells located around the perimeter of the model drain water and solute from the simulation domain. Solute enters the model domain through the injection well with a unit concentration.  The starting concentration is zero across the entire domain. Flow remains steady and confined throughout the 27 day simulation period. The aquifer is homogeneous, isotropic, and boundaries are sufficiently far from the injection well to avoid solute reaching the boundary during the simulation interval. Table~\ref{tab:ex-gwt-mt3dms-p05-01} summarizes many of the model inputs:

% add 2nd static parameter value table
\input{../tables/ex-gwt-mt3dms-p05-01}

An analytical solution for this problem was originally given in \cite{moench1981}. The MT3DMS solution with the TVD option activated most closely matched the analytical solution.  Therefore the TVD option is activated in both MT3DMS and \mf for verifying the transport solution. Figure \ref{fig:ex-gwt-mt3dms-p05-xsec} shows a slight under simulation of the outward spread of solute in the \mf solution compared to MT3DMS. Figure \ref{fig:ex-gwt-mt3dms-p05-planView} shows close agreement among the MT3DMS and \mf isoconcentration contours with the TVD advection scheme activated. 

% This figure doesn't exist in the MT3DMS manual
\begin{StandardFigure}
	{Comparison of the MT3DMS and \mf numerical solutions for a point source in a two-dimensional radial flow field simulation.  The thick black line in figure~\ref{fig:ex-gwt-mt3dms-p05-planView} shows the location of this profile view of concentrations.  The analytical solution for this problem was originally given in \citep{moench1981} and is not shown here} 
	{fig:ex-gwt-mt3dms-p05-xsec}
	{../figures/ex-gwt-mt3dms-p05-xsec.png}
\end{StandardFigure}

% MT3DMS manual figure 37
\begin{StandardFigure}
	{Comparison of the MT3DMS and \mf numerical solutions for two-dimensional transport in a radial flow field.  The thick black line shows the location of the concentration profile shown in figure~\ref{fig:ex-gwt-mt3dms-p05-xsec}. Both models are using their respective finite difference solutions with the use of the TVD option.} 
	{fig:ex-gwt-mt3dms-p05-planView}
	{../figures/ex-gwt-mt3dms-p05-planView.png}
\end{StandardFigure}


\clearpage
\insection
\section{Concentration at an Injection/Extraction Well (MT3DMS Example Problem 6)}

In this example problem, concentrations are compared between \mf and MT3D-USGS at an injection/extraction well. The well is fully penetrating in a confined aquifer and injects contaminated water for a period of 2.5 years at a rate of 1 $ft^3/sec$.  At the end of 2.5 years, the injection well is reversed and begins pumping (extracting) contaminated groundwater for a period of 7.5 years, also at the rate of 1 $ft^3/sec$.  \cite{elkadi1988} was the first to develop the test problem which was later used by \cite{zheng1993} to test ongoing method-of-characteristics (MOC) developments.  The model boundary is placed far enough away from the injection/extraction well to ensure no solute exits the model domain during the injection period.  Moreover, steady flow conditions are reached immediately during the injection and extraction stress periods.  Problem specifics are provided in table~\ref{tab:ex-gwt-mt3dms-p06-01}.

% add 2nd static parameter value table
\input{../tables/ex-gwt-mt3dms-p06-01}

An analytical solution for this problem was originally given in \cite{gelhar1971}.  Because this is an advection dominated problem, both numerical solutions invoke their TVD schemes.  The \mf solution shows a quicker rise in concentration at the well site than does the MT3D-USGS solution (figure~\ref{fig:ex-gwt-mt3dms-p06}). 

% MT3DMS manual figure 38
\begin{StandardFigure}
	{Comparison of the MT3D-USGS and \mf numerical solutions at an injection/extraction well. The analytical solution for this problem was originally given in \citep{gelhar1971} and is not shown here} 
	{fig:ex-gwt-mt3dms-p06}
	{../figures/ex-gwt-mt3dms-p06.png}
\end{StandardFigure}



\clearpage
\insection
\section{Three-Dimensional Transport in a Uniform Flow Field (MT3DMS Example Problem 7)}

In a previous problem titled ``two-dimensional transport in a uniform flow field'' concentrations were compared between \mf and MT3D-USGS for a relatively thin aquifer (10 $m$) wherein instantaneous vertical mixing was assumed.  In order to test transport simulation in a thicker aquifer, where all three spatial dimensions are required to adequately simulate the movement of solute, the current problem was devised. \cite{hunt1978} provides an analytical solution, which is used to verify MT3DMS in \cite{zheng1999mt3dms}, but not shown here.  Instead, only the \mf and MT3DMS solutions are compared here. 

Problem dimensions and aquifer properties are given in table~\ref{tab:ex-gwt-mt3dms-p0701}.  The point source is located in layer 7, row 8, and column 3

% add 2nd static parameter value table
\begin{StandardTable}
	{Hydraulic and transport properties used in the three-dimensional transport in a uniform flow field test problem.  From \cite{zheng1999mt3dms}}
	{tab:ex-gwt-mt3dms-p0701}
	{../tables/ex-gwt-mt3dms-p0701}
\end{StandardTable}

An analytical solution for this problem was originally given in \cite{hunt1978}.  Both numerical solutions invoke their respective TVD schemes.  Moreover, \mf is using the XT3D package for simulating dispersion.  The \mf solution shows great agreement with the MT3DMS calculated concentrations for the three layers displayed in figure~\ref{fig:ex-gwt-mt3dms-p06}. 

% MT3DMS manual figure 38
\begin{StandardFigure}
	{Comparison of the MT3D-USGS and \mf numerical solutions for three-dimensional transport in a uniform flow field. The analytical solution for this problem was originally given in \citep{hunt1978} and is not shown here} 
	{fig:ex-gwt-mt3dms-p07}
	{../figures/ex-gwt-mt3dms-p07}
\end{StandardFigure}



\clearpage
\insection
\section{MT3DMS Problem 8}

This example problem originally appeared \cite{sudicky1989} for finding the solution in a hypothetical field-scale example.  Among the MT3DMS suite of examples, this is the first one with a heterogeneous hydraulic conductivity field.  Defining characteristics that set this problem apart from earlier MT3DMS example problems include a highly irregular flow field and large contrast between longitudinal and transverse dispersivities.  \cite{vanderheijde1995} points out that this particular model test potentially troublesome parameter combinations while at the same time testing the ability of a flow and transport code to simulate real-world problems.

\subsection{Example description}

Using a ``deformed quadrilateral'' \citep{zheng1999mt3dms}, the model domain is 250 $m$ wide by 6.75 $m$ at at the left boundary and and 5.375 $m$ at the right boundary. The conceptual model is divided into 27 layers and 50 columns all contained within a single row.  For the duration of the simulation, steady flow is maintained by a constant recharge rate of 10 $cm/yr$, no-flow on the left and bottom boundaries and a constant head of 5.375 $m$ on the right boundary (figure~\ref{fig:mt3dms-p08}).  The simulation uses unconfined conditions to model the water table.  Heterogeneity within the aquifer is represented by a fine grained silty sand with a hydraulic conducivity of $5 \times 10^{-4} cm/sec$ that hosts two lenses of medium-grained sand with a hydraulic conductvity of $1 \times 10^{-2} cm/sec$ (table~\ref{tab:ex-gwt-mt3dms-p08-01}).  The vertical hydraulic conductivity is assumed to equal the horizontal conductivity in both materials.

% MT3DMS manual figure 40
\begin{StandardFigure}
	{MT3DMS test problem 8 configuration showing flow (upper figure) and transport (lower figure) boundary conditions (from \cite{sudicky1989}).} 
	{fig:mt3dms-p08}
	{../images/mt3dms-p08.png}
\end{StandardFigure}

% add 2nd static parameter value table
\input{../tables/ex-gwt-mt3dms-p08-01}

Boundary conditions within the transport model also are shown in figure~\ref{fig:mt3dms-p08}.  A relative concentration of 1.0 is assigned to the recahrge entering the model domain between 40 and 80 $m$ from the left boundary and 0.0 elsewhere.  Mass continues to enter the simulation at the given location for the first five years of the simulation, after which time the source is removed and the model continues to run for an additional 15 years.  An initial concentration of 0.0 is specified throughout the model domain.  A uniform porosity of 0.35 is assigned to the entire model domain.  Additional transport-related parameters are listed in table~\ref{tab:ex-gwt-mt3dms-p08-01}.

\subsection{Example results}

In order to achieve a close match between \mf and MT3D-USGS, the XT3D solver was activated within \mf.  Moreover, both solutions use their respective TVD advection schemes.  Results are shown after 8, 12, and 20 years (figures \ref{fig:ex-gwt-mt3dms-p08-8yrs} - \ref{fig:ex-gwt-mt3dms-p08-20yrs}).  In each of the selected years for which results are plotted, model results are similar with a couple of small differences.  For example, all plots show the leading edge (as defined by the 0.05 isoconcentration contour) of the MT3DMS-calculated plume ahead of the leading the edge of the \mf plume.  Also, the \mf plume is less dispersed compared to the MT3DMS plumes as demonstrated by the presence of higher concentrations (i.e., isoconcentration contours) located at the center of the plume, particularly in years 12 and 20.

% MT3DMS manual figure 41
\begin{StandardFigure}
	{Migrating contaminant plume as calculated by (A) MT3D-USGS and (B) \mf after 8 years.  The original problem was given in \citep{sudicky1989}} 
	{fig:ex-gwt-mt3dms-p08-8yrs}
	{../figures/ex-gwt-mt3dms-p08-8yrs.png}
\end{StandardFigure}

% MT3DMS manual figure 41
\begin{StandardFigure}
	{Migrating contaminant plume as calculated by (A) MT3D-USGS and (B) \mf after 12 years.  The original problem was given in \citep{sudicky1989}} 
	{fig:ex-gwt-mt3dms-p08-12yrs}
	{../figures/ex-gwt-mt3dms-p08-12yrs.png}
\end{StandardFigure}

% MT3DMS manual figure 41
\begin{StandardFigure}
	{Migrating contaminant plume as calculated by (A) MT3D-USGS and (B) \mf after 20 years.  The original problem was given in \citep{sudicky1989}} 
	{fig:ex-gwt-mt3dms-p08-20yrs}
	{../figures/ex-gwt-mt3dms-p08-20yrs.png}
\end{StandardFigure}


\clearpage
\insection
\section{MT3DMS Problem 9}

This example compares \mf and MT3DMS transport solutions in a two-dimensional plan-view setting with heterogeneity in the hydraulic conductivity field. The problem is purely hypothetical and was originally used for comparing different MT3DMS solutions (i.e., finite-difference, method-of-characteristics, and TVD advection schemes) to each other.  No analytical solution exists for this problem \citep{zheng1999mt3dms}.

\subsection{Example description}

A contaminant plume originates in an injection well located near the upper portion of the model (figure~\ref{fig:mt3dms-p09}).  After its injection, contaminant migrates toward a pumping well located several rows below a low hydraulic conductivity zone, itself located a couple rows below where contaminant enters the aquifer. 

% MT3DMS manual figure 43
\begin{StandardFigure}
	{MT3DMS test problem 9 configuration showing grid discretization, constant head boundaries, and the locations of injection and pumping wells (from \cite{zheng1999mt3dms}).} 
	{fig:mt3dms-p09}
	{../images/mt3dms-p09.png}
\end{StandardFigure}

The model domain is discretized by 18 rows, 14 columns, and a single confined layer.  Grid cell dimensions are uniformly set to 100 $\times$ 100 $m$.  The left, right and bottom model boundaries are noo-flow.  The upper (north) boundary applies a specified head of 250 $m$ that is held constant throughout the simulation period.  Similarly, a specified head is applied to the lower (south) boundary that varies from 20 $m$ on the left to 52.5 $m$ on the right, a gradient of 2.5/100 $m/m$.  Water injected into the aquifer enters at a concentration of 57.87 $ppm$.  Mass is removed from the aquifer by both the pumping well and with the flow that exits the domain at the lower (south) boundary of the model.  Mass that exits the lower boundary does so at the simulated concentration of the water that is removed by the constant head boundary condition.  Mass is not allowed to leave through the upper, bottom, left, or right boundaries in the respective transport simulations.  All flow boundary conditions are held constant for the duration of the simulation period, resulting in steady-flow conditions.  However, the concentration of the injected water is reduced to 0.0 $ppm$ during the second of two stress periods.  Both the injection and pumping wells are assumed to be fully penetrating.  Aquifer heterogeneity is depicted in figure~\ref{fig:mt3dms-p09}.  Model parameters are summaried in table~\ref{tab:ex-gwt-mt3dms-p09-01}.

% add 2nd static parameter value table
\input{../tables/ex-gwt-mt3dms-p09-01}

In \mf, the XT3D solver is not activated; however, both \mf and MT3DMS use their respective TVD advection schemes.  

\subsection{Example results}

Contrasting model results after one year, the leading edge of the plume advanced further within the \mf solution (figures~\ref{fig:ex-gwt-mt3dms-p09}). Confirmation that the same amount of mass entered both simultions (\mf and MT3DMS) was achieved by checking the mass budget information in the respective mode output listing files. In both simulations, the effect of the aquitard aquitard is clearly seen, though the isoconcentration lines appear to have infiltrated the aquitard more with MT3DMS compared to the \mf solution.  

% MT3DMS manual figure 44 (but in 2D instead of 3D)
\begin{StandardFigure}
	{Migrating contaminant plume as calculated by (A) MT3D-USGS and (B) \mf after one year of simulation time.  The original problem was given in \citep{zheng1999mt3dms}.  Cells shaded in blue show the location of contstant head boundary cells.  Light brown shaded cells show the location of the aquitard.} 
	{fig:ex-gwt-mt3dms-p09}
	{../figures/ex-gwt-mt3dms-p09.png}
\end{StandardFigure}


\clearpage
\insection
\section{MT3DMS Problem 10}

This example problem describes an actual field problem \citep{zheng1999mt3dms}.  While the original discussion of this problem compared various MT3MDS solution schemes and evaluated the effectiveness of MT3DMS for solving real-world problems, the current goal is much more modest: to compare the \mf transport solution against the MT3DMS solution.  

\subsection{Example description}

The geological setting for this final problem from \cite{zheng1999mt3dms} is roughly shown in figure~\ref{fig:mt3dms-p10}.  An unconfined aquifer beneath a spill site is approximately divided into an upper and lower zone with hydraulic conductivities of 60 and 520 $ft/day$, respectively.  Each zone is represented by two 25 $ft$ uniformlyh thick layers, for a total of 4 layers.  Each layer is comprised of 61 rows and 40 columns with variable row and column widths.  The highest level of grid refinement exists near the detailed study site containing the bulk of the contaminant plume.  In this area, row and column widths are 50 $\times$ 50 $ft$ (area denoted by ABCD in figure~\ref{fig:mt3dms-p10}). Outside this area, cell lengths and widths progressively increase toward the model boundary.  A no-flow no-flux boundary conditions is applied to the bottom boundary while specified head boundaries are used on all four sides.  Heads along the sides of the model domain estable a regional flow gradient of $5 \times 10^{-4}$ from east to west and $1 \times 10^{-3}$ from north to south.  Mass may leave the model domain with the regional flow gradient, but does so at the computed concentration of the groundwater. Water that enters the model domain via the regional flow gradient does so with a concentration of 0.0.  However, model boundaries are intentionally set far enough away from the area of concern so that their effect on the flow and transport simulation is minimized.  A constant recharge rate of 5 $in/yr$ is applied to the surface of the model.  A uniform porosity of 0.30 is applied to the entire model.  Owing to the presence of organic contaminants within the groundwater, including 1,2-dichloroethane (1,2-DCA), chemical reactions are simulated using an equilibrium-controlled linear sorption isotherm.  Reactive parameters are specified such that the retardation term equals 2.  The initial concentration of 1,2-DCA for layer 1 is shown in figure~\ref{fig:ex-gwt-mt3dms-p10}A. The initial concentration of 1,2-DCA in layer 2 is 20 percent of the layer 1 initial concentration.  Layers 3 and 4 start out with no contaminant.  The location of extraction wells are given by the black squares in figure~\ref{fig:ex-gwt-mt3dms-p10}. The extraction wells are used to pump-and-treat contaminated goundwater.  The total specified extraction rate among all wells is $4.25 \times 10^3 m^3/day$ which occurs from layer 3. The Additional model parameter values are listed in table~\ref{tab:ex-gwt-mt3dms-p10-01}.  

% MT3DMS manual figure 48
\begin{StandardFigure}
	{MT3DMS test problem 10 showing the numerical model configuration.  Image not to scale. (11,600 ft = 3.535.68 m; 20,450 ft = 6,233 m; 25 ft = 7.62 m)} 
	{fig:mt3dms-p10}
	{../images/mt3dms-p10.png}
\end{StandardFigure}

% add 2nd static parameter value table
\input{../tables/ex-gwt-mt3dms-p10-01}

\subsection{Example results}

The calculated concentration fields are shown for layer 3 at 500, 750, and 1,00 days in figures~\ref{fig:mt3dms-p10}B-D, respectively.  Both models use their finite-difference solutions and therefore do no invoke their respective TVD scheme.  Relatively good agreement between the two simulations is achieved in the lower (southerly) portion of the model for all 3 simulation times displayed.  In the upper (northerly) portion of the model, however, several cells separate the isoconcentration contours.

% MT3DMS manual figure 49
\begin{StandardFigure}
	{(A) Initial and simulated contaminant plume concentrations as calculated by MT3DMS and \mf after (B) 500 days, (C) 750 days, and (D) 1,000 days.} 
	{fig:ex-gwt-mt3dms-p10}
	{../figures/ex-gwt-mt3dms-p10.png}
\end{StandardFigure}


\clearpage
\insection
\section{Example}

% Describe source of problem

\subsection{Example description}

% spatial discretization 

% temporal discretization

% material properties

% initial conditions

% boundary conditions

% add static parameter table(s)
%\input{../tables/ex-gwf-twri01-01}

% for examples without scenarios
\subsection{Example Results}

Describe the scenarios

% for examples with scenarios
\subsection{Example Scenarios}

Describe the scenarios

% add scenario table
%\input{../tables/ex-gwf-twri-scenario}


\subsubsection{Scenario 1 Results}

Some results...

% a figure
%\begin{StandardFigure}{
%                                     A figure caption that describes the figure. 
%                                     \textit{A}. Plot shows some good stuff.
%                                     \textit{B}. Plot shows more good stuff.
%                                     }{fig:ex-gwf-01}{../figures/ex-gwf-twri-01.png}
%\end{StandardFigure}                                 

\subsubsection{Scenario 2 Results}


Some more results...

% a figure
%\begin{StandardFigure}{
%                                     A figure caption that describes the figure. 
%                                     \textit{A}. Plot shows some good stuff.
%                                     \textit{B}. Plot shows more good stuff.
%                                     }{fig:ex-gwf-02}{../figures/ex-gwf-twri-02.png}
%\end{StandardFigure}                 


\clearpage
\insection
\section{Zero-Order Growth in a Uniform Flow Field (MT3DMS Supplemental Guide Problem 6.3.1)}

% Describe source of problem
This example is for zero-order production in a uniform flow field.  It is based on example problem 6.3.1 described in \cite{zheng2010mt3dmsv5.3}.  The problem consists of a one-dimensional model grid with inflow into the first cell and outflow through the last cell.  This example is simulated with the GWT Model in \mf, which receives flow information from a separate simulation with the GWF Model in \mf.  Results from the GWT Model are compared with the results from a MT3DMS simulation \citep{zheng1990mt3d} that uses flows from a separate MODFLOW-2005 simulation \citep{modflow2005}.  

\subsection{Example description}

The parameters used for this problem are listed in table~\ref{tab:ex-gwt-mt3dsupp63101}.  The model grid consists of 101 columns, 1 row, and 1 layer.  The flow problem is confined and steady state with an initial head set to the model top.  The solute transport simulation represents transient conditions, which begin with an initial concentration specified as zero everywhere within the model domain.  A specified flow condition is assigned to the first model cell.  For the source pulse duration, the inflow concentration is specified as one.  Following the source pulse duration the inflowing water is assigned a concentration of zero.  A specified head condition is assigned to the last model cell.  Water exiting the model through the specified head cell leaves with the simulated concentration of that cell.

% add static parameter table(s)
\begin{StandardTable}{Parameters used for the example of zero-order growth in a uniform flow field (MT3DMS Supplemental Guide problem 6.3.1 model parameters).}{tab:ex-gwt-mt3dsupp63101}{../tables/ex-gwt-mt3dsupp63101}
\end{StandardTable}

% for examples without scenarios
\subsection{Example Results}

Simulated concentrations from the \mf GWT Model and MT3DMS are shown in figure~\ref{fig:ex-gwt-mt3dsupp631}.  The close agreement between the simulated concentrations demonstrate the zero-order-production capabilities implemented in the GWT Model.

% a figure
\begin{StandardFigure}{
                                     Concentrations simulated by the \mf GWT Model and MT3DMS for zero-order growth in a uniform flow field.
                                     }{fig:ex-gwt-mt3dsupp631}{ex-gwt-mt3dsupp631}
\end{StandardFigure}                                 



\clearpage
\insection
\section{Zero-Order Growth in a Dual-Domain System (MT3DMS Supplemental Guide Problem 6.3.2)}

% Describe source of problem
This example is for zero-order production in a dual-domain system.  It is based on example problem 6.3.2 described in \cite{zheng2010mt3dmsv5.3}.  The problem consists of a one-dimensional model grid with inflow into the first cell and outflow through the last cell.  This example is simulated with the GWT Model in \mf, which receives flow information from a separate simulation with the GWF Model in \mf.  This example is designed to test the capabilities of the GWT Model to simulate zero-order production in a dual-domain system with and without sorption.  Results from the GWT Model are compared with the results from a MT3DMS simulation \citep{zheng1990mt3d} that uses flows from a separate MODFLOW-2005 simulation \citep{modflow2005}.  This example was described by \cite{zheng2010mt3dmsv5.3} who showed that the results from MT3DMS were in good agreement with an analytical solution.

\subsection{Example description}

The parameters used for this problem are listed in table~\ref{tab:ex-gwt-mt3dsupp63201}.  The model grid consists of 401 columns, 1 row, and 1 layer.  The flow problem is confined and steady state with an initial head set to the model top.  The solute transport simulation represents transient conditions, which begin with an initial concentration specified as zero everywhere within the model domain.  A specified flow condition is assigned to the first model cell.  For the source pulse duration, a specified concentration with a value of one is assigned to the first model cell.  Following the source pulse duration the specified concentration in the first cell is zero.  A specified head condition is assigned to the last model cell.  Water exiting the model through the specified head cell leaves with the simulated concentration of that cell.

% add static parameter table(s)
\begin{StandardTable}{MT3DMS Supplemental Guide problem 6.3.2 model parameters}{tab:ex-gwt-mt3dsupp63201}{../tables/ex-gwt-mt3dsupp63201}
\end{StandardTable}

% for examples with scenarios
\subsection{Example Scenarios}

This example problem consists of several different scenarios, as listed in table~\ref{tab:ex-gwt-mt3dsupp632-scenario}.  The first two scenarios represent zero-order growth when sorbtion is active.  Sorbtion is not active in the last scenario.  For all three scenarios, there is mass transfer between the mobile domain and the immobile domain.

% add scenario table
\begin{ScenarioTable}{
                                   MT3DMS Supplemental Guide problem 6.3.2 scenarios
                                   }{tab:ex-gwt-mt3dsupp632-scenario}{../tables/ex-gwt-mt3dsupp632-scenario}
\end{ScenarioTable}


\subsubsection{Scenario Results}

Results from the three scenarios are shown in figure~\ref{fig:ex-gwt-mt3dsupp632}.  The close agreement between the simulated concentrations for the \mf GWT Model and MT3DMS demonstrate the zero-order growth and immobile-domain transfer capabilities for \mf.

% a figure
\begin{StandardFigure}{
                                     Concentrations simulated by the \mf GWT Model and MT3DMS for zero-order growth in a dual-domain system.  Circles are for the GWT Model results; the lines represent simulated concentrations for MT3DMS.
                                     }{fig:ex-gwt-mt3dsupp632}{ex-gwt-mt3dsupp632}
\end{StandardFigure}                                 



\clearpage
\insection
\section{Simulating the Effect of a Recirculating Well (MT3DMS Supplemental Guide Problem 8.2)}

% Describe source of problem
This example is for a recirculating well.  It is based on example problem 8.2 described in \cite{zheng2010mt3dmsv5.3}.  The problem consists of a two-dimensional, one-layer model with flow from left to right.  A solute is introduced into the flow field by an injection well.  Downgradient, an extraction well pumps at the same rate as the injection well.  This extracted water is then injected into two other injection wells.  This example is simulated with the GWT Model in \mf, which receives flow information from a separate simulation with the GWF Model in \mf.  Results from the GWT Model are compared with the results from a MT3DMS simulation \citep{zheng1990mt3d} that uses flows from a separate MODFLOW-2005 simulation \citep{modflow2005}.  

\subsection{Example description}

The parameters used for this problem are listed in table~\ref{tab:ex-gwt-mt3dsupp8201}.  The model grid consists of 31 rows, 46 columns, and 1 layer.  The flow problem is confined and steady state.  The solute transport simulation represents transient conditions, which begin with an initial concentration specified as zero everywhere within the model domain.

For the MT3DMS representation of this problem, the Well Package is used to inject water at a rate of 1 $m^3/d$ into model cell (1, 16, 16).  Water is extracted at a rate of -1 $m^3/d$ from model cell (1, 16, 21).  Two additional wells, located in cells (1, 5, 16) and (1, 27, 16), reinject water at the concentration of the extracted water from cell (1, 16, 21).  The injection rate for each of these reinjection wells is 0.5 $m^3/d$.

For the \mf representation of this problem, the Multi-Aquifer Well (MAW) Package is used for these injection and extraction wells, although because the model is only a single layer, the MAW Package behaves just like the Well Package.  The Water Mover (MVR) Package is used to send half of the extracted water into each of the reinjection wells.  For the \mf transport simulation, the Multi-Aquifer Transport (MWT) Package is used to calculate the concentration in each of the well bores.  The Mover Transport (MVT) Package is used to move the solute from the extraction well to the two reinjection wells based on the simulated flows.  Note that this approach used in \mf is slightly different than the approach used in MT3DMS, because \mf is calculating the concentrations in the well boreholes rather than using concentrations directly from the model cells.  By specifying a small well radius for the MAW Package, the approaches are similar.

% add static parameter table(s)
\begin{StandardTable}{Parameters used for the example of a recirculating well (MT3DMS Supplemental Guide problem 8.2 model parameters).}{tab:ex-gwt-mt3dsupp8201}{../tables/ex-gwt-mt3dsupp8201}
\end{StandardTable}

% for examples without scenarios
\subsection{Example Results}

Simulated concentrations from \mf and MT3DMS are shown in figure~\ref{fig:ex-gwt-mt3dsupp82-map}.  The close agreement between the simulated concentrations demonstrate the ability of \mf to simulate the transfer of water and solute using the mover package capability.

% a figure
\begin{StandardFigure}{
                                     Concentrations simulated by \mf and MT3DMS for a problem involving a recirculating well.
                                     }{fig:ex-gwt-mt3dsupp82-map}{ex-gwt-mt3dsupp82-map}
\end{StandardFigure}                                 



\clearpage
\insection
\section{Stream-Lake Interaction with Solute Transport (SFR1 Manual Test Problem 2)}

This problem is based on the stream-aquifer interaction problem described as test 2 by \cite{modflowsfr1pack}.  \cite{modflowsfr1pack} designed their test 2 problem by modifying a variant originally described by \cite{modflowlak3pack}.  The description in the text and the figures presented here are largely based on the text and figures presented by \cite{modflowsfr1pack}.  The purpose for including this problem here is to demonstrate the use of \mf to simulate solute transport through a coupled system consisting of an aquifer, streams, and lakes.  The example requires accurate simulation of transport within the streams and lakes and also between the surface water features and the underlying aquifer.  

% Describe source of problem
\subsection{Example description}

The example problem consists of two lakes and a stream network.  Figure \ref{fig:ex-gwt-prudic2004t2-bcmap} shows the configuration of the hypothetical (but realistic) problem that was used by \cite{modflowsfr1pack} to demonstrate integration of the SFR1 Package with the LAK3 Package and the MODFLOW-2000 GWT Process.  

% a figure
\begin{StandardFigure}{
                                     Model grid, boundary conditions and locations of lakes and streams used for the stream-lake interaction with solute transport problem.  From  \cite{modflowsfr1pack}.
                                     }{fig:ex-gwt-prudic2004t2-bcmap}{../images/ex-gwt-prudic2004t2-bcmap}
\end{StandardFigure}            

As described in \cite{modflowsfr1pack}, the aquifer is moderately permeable and has homogeneous properties and uniform thickness (table \ref{tab:ex-gwt-prudic2004t201}).  The aquifer was discretized into 8 layers (each 15 ft thick), 36 rows (at equal spacing of 405.7 ft), and 23 columns (at equal spacing of 403.7 ft).  The flow field is represented as steady state.  Uniform recharge was applied at a rate of 21 in/yr to layer 1 of the model.  Two lakes are located within the model domain.  Lake 1 has inflow from stream segment 1 and has outflow to stream segment 2.  Lake 2 has no stream inflows or outflows.  Both streams are in contact with the aquifer.  Lakes are in horizontal contact with the aquifer in layer 1 around their edges and in vertical contact with the aquifer in layer 2.

% add static parameter table(s)
\begin{StandardTable}{Parameters used for the stream-lake interaction with solute transport problem (SFR1 Manual Test Problem 2).}{tab:ex-gwt-prudic2004t201}{../tables/ex-gwt-prudic2004t201}
\end{StandardTable}

The boundary conditions are illustrated in figure \ref{fig:ex-gwt-prudic2004t2-bcmap} and were designed to produce flow that is generally from north to south. For the numerical model, constant-head conditions were specified along the northern and southern edges of the model domain, and no-flow boundaries were set along the east and west edges of the grid. In \mf lakes can either sit on top of the model grid, or they can be incised into the model grid as is done with previous MODFLOW versions.  For this example, aquifer cells in layer 1 (fig. \ref{fig:ex-gwt-prudic2004t2-bcmap}) that share the same space as the lake are made inactive in the model grid by setting their IDOMAIN values to zero. The constant-head boundaries were placed in all 8 layers at the map locations shown in figure \ref{fig:ex-gwt-prudic2004t2-bcmap}, with two exceptions.  The first exception is related to the last downstream reach of stream segment 4. The grid cell in which the reach is located and the cell underlying it in layer 2 are both specified as active aquifer cells rather than as constant-head cells because a constant head in that cell with the stream reach would have set the gradient across the streambed. The second exception is related to the contaminant source (from treated sewage effluent), which is only introduced into the upper two model layers (that is, to a total of 8 cells). The four cells in layer 3 that underlie the contaminant source are specified as active aquifer cells rather than as constant-head cells to simulate transport beneath the source. Constant-head elevations were specified as 50.0 ft along the north boundary, except at the 4 cells in each of the upper two model layers that represent inflow from the contaminant source, where the fixed heads were 50.15 ft in the two middle cells and 50.10 ft in the two outer cells. The constant-head elevations were set to 28.0 ft along the south boundary. 

The stream network consists of four segments and 38 reaches. (In \mf there is no concept of a segment, however the reaches are assigned names based on the segment number so that their combined flows can be compared with the results from MODFLOW-GWT.) The stream depths are calculated using Manning’s equation assuming a wide rectangular channel. For all stream reaches, the channel width was assumed constant at 5.0 ft and the roughness coefficient for the channel was 0.03.  Inflows to segments 1 and 3 were specified (86,400 and 8,640 ft3/d, respectively). The inflow to stream segment 2 was equal to the outflow from lake 1, and it was calculated using Manning’s equation assuming a wide rectangular channel with a depth based on the difference between the calculated lake stage and the elevation of the top of the streambed (see \cite{modflowlak3pack}, p. 11). The inflow to segment 4 is calculated as the sum of outflows from tributary segments 2 and 3.  In \mf the Water Mover Package was used to route the water from the end of segment 1 into lake 1, and from the southern outlet of lake 1 into stream segment 2.  

As reported by \cite{modflowsfr1pack} the test problem focuses on simulation of a boron plume, which results from sewage effluent.  Variables related to the transport simulation are listed in table \ref{tab:ex-gwt-prudic2004t201}.  For the purposes of this test, boron was assumed nonreactive. Molecular diffusion in the aquifer was assumed a negligible contributor to solute spreading at the scale of the field problem, so that hydrodynamic dispersion was related solely to mechanical dispersion, which was computed in \mf as a function of the specified dispersivity of the medium and the velocity of the flow field. The initial boron concentrations in the aquifer and in the lakes were assumed to be zero, and the sewage effluent concentration was assumed to be 500 micrograms/L. The source concentration in recharge was assumed to be zero and the concentration in specified inflow to stream segments 1 and 3 was also zero. The solute-transport model was run for a period of 25 years.


% for examples without scenarios
\subsection{Example Results}

The calculated steady-state head distributions in layers 1 and 2 are shown in figure \ref{fig:ex-gwt-prudic2004t2-head} for \mf.  This figure can be compared to figure 14 in \cite{modflowsfr1pack}. The heads in layers 3 through 8 are almost identical to the heads shown for layer 2 (fig. \ref{fig:ex-gwt-prudic2004t2-head}).  Flow is generally from north to south and predominantly horizontal. Because of recharge at the water table, however, there is a slight vertically downward flow in most areas. The lakes and streams exert a strong influence on the location and magnitude of vertical flow. In layer 2, the good hydraulic connection with the lakebed results in an almost flat horizontal hydraulic gradient in head beneath the lakes (fig. \ref{fig:ex-gwt-prudic2004t2-head}).  In general the simulated water table and aquifer heads simulated by \mf are in good agreement with those simulated by MODFLOW-2000.

% a simulated head figure. fig:ex-gwt-prudic2004t2-head
\begin{StandardFigure}{
                                     Contours of head simulated by the \mf GWF Model for the stream-lake interaction example.  Contours are for (A) layer 1, and (B) layer 2.  This figure can be compared with figure 14 in \cite{modflowsfr1pack}, which shows head contours as simulated by MODFLOW-2000.
                                     }{fig:ex-gwt-prudic2004t2-head}{ex-gwt-prudic2004t2-head}
\end{StandardFigure}            

The \mf model calculated steady-state stage in lake 1 was 45.07 ft (compared to 44.97 ft in MODFLOW-GWT) and in lake 2 was 37.15 ft (compared to 37.14 ft in MODFLOW-2000). Stream segment 1 was mostly a gaining stream (leakage across streambed was from ground water), and stream segment 2 was losing such that outflow from the last reach in segment 2 was only half of the inflow from lake 1. Stream segment 3 was mostly gaining and outflow from this segment was only from groundwater leakage. Lastly, stream segment 4 was a losing stream and leakage was from the stream to the aquifer in every reach.  Simulated flows between \mf and MODFLOW-2000 are in good qualitative agreement, however there are differences in individual flows, which can be attributed to slight differences in the way \mf and MODFLOW-2000 simulate lakes, streams, and groundwater flow.

For the \mf simulation, the 25-year period was divided into 300 time steps (compared with 1229 time steps used for the MODFLOW-GWT simulation).  Advective groundwater flow was solved using the second-order implicit Total Variation Diminishing (TVD) scheme.  Dispersion was solved using the XT3D approach, which was originally designed to represent full three-dimensional anisotropic groundwater flow \citep{modflow6xt3d}.  Transport through the surface water system was solved using the Lake Transport (LKT) Package, the Streamflow Transport (SFT) Package, and the Mover Transport (MVT) Package, which transfers solute between the lake and stream according to simulated flows.  Additional detail on the transport parameters for the MODFLOW-GWT simulation are described in \cite{modflowsfr1pack} and in \cite{modflowlak3pack}.  The calculated concentration in lake 1 and in the outflow from the last reach in stream segments 2, 3, and 4 during the 25-year simulation period are shown in figure \ref{fig:ex-gwt-prudic2004t2-cvt}.  Note that because stream segment 2 lost flow to ground water in all reaches and its only source was inflow from lake 1, solute concentration in the outflow from the last reach in stream segment 2 was equal to the concentration in the discharge from lake 1 (fig. \ref{fig:ex-gwt-prudic2004t2-cvt}).  The leading edge of the plume reaches the upstream edge of lake 1 after about 4 years, at which time the concentration in the lake begins to increase rapidly. After about 22 years, the part of the plume close to the source and near the lake has stabilized and the concentration in lake 1 reaches an equilibrium concentration of 37.2 micrograms/L as simulated by \mf and 37.4 micrograms/L as simulated by MODFLOW-GWT.  Although there are differences in the surface water concentrations simulated by \mf and MODFLOW-GWT, the general pattern and behavior is quite similar.  Differences between the two models are generally attributed to slight differences in simulated flows as well as slight differences in how solute transport is represented.  For example, MODFLOW-GWT uses the method-of-characteristics to simulate advective flow, whereas \mf uses an implicit TVD approach.  The methods-of-characteristics approach implemented in MODFLOW-GWT is exceptional for reducing numerical dispersion, whereas the second-order TVD approach implemented in \mf is relatively fast and efficient, but it has more numerical dispersion than MODFLOW-GWT.  

% a concentration versus time figure
\begin{StandardFigure}{
                                     Concentration versus time simulated by \mf for the stream-lake interaction example.  Lines represent the change in simulated boron concentration in lake 1 and at the end of stream segments 2, 3, and 4.  This figure can be compared with figure 16 in \cite{modflowsfr1pack}, which shows an equivalent plot using model results simulated by MODFLOW-GWT.
                                     }{fig:ex-gwt-prudic2004t2-cvt}{ex-gwt-prudic2004t2-cvt}
\end{StandardFigure}            

As the lake concentration increases, it in turn acts as a source of contamination to the aquifer in the areas where the lake is a source of water to the aquifer. Although the lake significantly dilutes the contaminants that enter it from the aquifer, the lake and the stream segments downstream from it in effect provide a short circuit for the relatively fast transmission of low levels of the contaminant. This is evident in figure \ref{fig:ex-gwt-prudic2004t2-conc}, which shows the computed solute distributions in layers 1, 3, 5, and 8 after 25 years. The low-concentration part of the plume emanating from the downgradient side of lake 1 has advanced farther, and is wider, than the main plume that emanated directly from the source at the north edge of the model. The influence of groundwater discharge to stream segment 3 is most apparent in the concentration pattern shown in layer 1 of figure \ref{fig:ex-gwt-prudic2004t2-conc} (that is, the 25 microgram/L contour). Comparison of concentration levels at different depths in the system indicates that in the southern part of the area, concentrations generally increase with depth. In contrast, in the northern part downgradient from the source, the highest concentrations occur in layer 3. These various patterns result from the dilution effect of recharge of uncontaminated water at the water table coupled with the consequent downward component of flow, which causes the solute to move slowly downward as it migrates to the south.  Solute concentrations simulated by \mf and shown in figure \ref{fig:ex-gwt-prudic2004t2-conc} are generally in good agreement with those simulated by MODFLOW-GWT (figure 17 in \cite{modflowsfr1pack}).  There are some differences, which can be attributed to the slightly different flow field and the differences in solute transport solution schemes.

% a simulated concentration figure
\begin{StandardFigure}{
                                     Contours of concentration simulated by the \mf GWT Model for the stream-lake interaction example.  Contours are shown for (A) layer 1, (B) layer 3, (C) layer 5, and (D) layer 8. This figure can be compared with figure 16 in \cite{modflowsfr1pack}, which shows an equivalent plot using model results simulated by MODFLOW-GWT.
                                     }{fig:ex-gwt-prudic2004t2-conc}{ex-gwt-prudic2004t2-conc}
\end{StandardFigure}            



\clearpage
\insection
\section{Two-Dimensional Test of Unsaturated-Saturated Transport}

This example first appeared as scenario 6 in \cite{morway2013}.  At that time, new capabilites programmed into MT3DMS \citep{zheng1999mt3dms} allowed for simulation of solute transport in the unsaturated-zone using flux terms calculated by the unsaturated-zone flow (UZF1; \cite{UZF}) package. \cite{morway2013} referred to the published MT3DMS variant as UZF-MT3DMS. Eventually, however, these capabilities were better documented and released with MT3D-USGS \citep{mt3dusgs}. For the purpose of testing unsaturated-zone transport using the UZF/UZT packages inside \mf, the \mf solution is compared against the MT3D-USGS solution. Moreover, we note that the results of the  MT3D-USGS simulation were compared to results calculated by VS2DT \citep{lappalaetal1987VS2D} for establishing the accuracy of the MT3D solution.  VS2DT solves Richards' equation and as such can simulate flow and solute fluxes across the unsaturated-saturated interface.  Therefore, this example problem tests the ability of \mf to accurately simulate the infiltration, unsaturated-zone transport, recharge, and subsequent saturated transport of dissolved solute.

Currently, the UZT package inside \mf does not simulate dispersion in the unsaturated zone.  As a result, two scenarios were setup in both MT3D-USGS and \mf.  The first maintains fidelity with \cite{morway2013} and simulates dispersion in the unsaturate zone.  In the second scenario, no dispersion is simulated in the unsaturated zone by setting the longitudinal, transverse, and vertical dispersion equal to zero in the upper 11 layers as summarized in the following table.  Brackets (``[ ]'') in the table indicate a list of values, one per layer, is used to define the value for the entire layer.  Where only one value appears inside the brackets, a constant value is used throughout the model domain.

% add scenario table
\input{../tables/ex-gwt-uzt-2d-scenario}

\subsection{Example description}

For this problem, a relatively small, two-dimensional profile that is 10 $m$ wide by 5 $m$ deep is used.  Layer thickness, as well as the column widths, are 0.25 $m$. A constant head boundary of 1.625 $m$ is set on the left and right sides of the active model domain.  An infiltration rate of 0.1 $m/day$ is specified all along the top boundary except for the left- and right-most columns where the constant head boundary condition exists.  A no-flow boundary is used along the bottom of the simulation domain.  The simulation domain starts out clean; however, solute enters with the infiltrating water in the middle 10 columns only.  Additional model parameter values are listed in table~\ref{tab:ex-gwt-uzt-2d-01}.  

% add 2nd static parameter value table
\input{../tables/ex-gwt-uzt-2d-01}

Given the relatively dry initial condition within the unsaturated zone, the infiltrating front reaches the water table on day 8 of the 60 day simulation period. Once the infiltrating wave reaches the saturated zone, the water table rises into the unsaturated zone and further tests the accuracy of the transport solution.

\subsection{Example results}

Because \mf does not (yet) simulate dispersion in the unsaturated-zone, there are some significant differences between the two solutions (figure~\ref{fig:ex-gwt-uzt-2d-a}).  Whereas longitudinal and transverse dispersive fluxes spread solute ahead and to the side of the downward migrating plume within the MT3D-USGS unsaturated zone solution, the UZF/UZT formulation within \mf simulates purely advective transport in the unsaturated zone.  However, once solute reaches the saturated zone, dispersion is simulated using the XT3D package (the default setting).  

% Morway et al. (2013) figure 6
\begin{StandardFigure}
	{A two-dimensional problem first published in \cite{morway2013}.  MT3D-USGS results closely match a benchmark solution calculated by VS2DT \citep{lappalaetal1987VS2D}.  The light blue shaded region shows the location of the saturated zone.}
	{fig:ex-gwt-uzt-2d-a}
	{../figures/ex-gwt-uzt-2d-a.png}
\end{StandardFigure}

A second scenario in which longitudinal, transverse, and vertical dispersion are set equal to zero in the unsaturated zone of the MT3D-USGS solution was run.  Because this setup more closely mimicks the \mf unsaturated zone solution, results between the respective models more closely match one another.  We note, however, that because many transport problems originate at land surface, the final transport solution within the satured zone may ultimately depend on accurately simulating unsaturated zone transport processes.  That is, the extent and severity of the saturated zone plume may be inextricably tied to the spread and delay of migrating solute in the unsaturated zone, as demonstrated by the two different solutions within the saturated zone when dispersion is and is not accounted for in the unsaturated zone.

% Morway et al. (2013) figure 6
\begin{StandardFigure}
	{A two-dimensional transport problem with no dispersive fluxes in either \mf or MT3D-USGS within the unsaturated-zone.  The light blue shaded region shows the location of the saturated zone.}
	{fig:ex-gwt-uzt-2d-b}
	{../figures/ex-gwt-uzt-2d-b.png}
\end{StandardFigure}


\clearpage
\insection
\section{Henry Problem}

% Describe source of problem
This problem simulates the classic Henry problem \citep{henry1964}  for variable-density groundwater flow and solute transport.  The \mf simulations presented here are based on the hydraulic-head formulation for variable-density flow as presented by \cite{langevin2020hydraulic}.  

\subsection{Example Description}

Variations on the Henry problem \citep{henry1964} are commonly used as benchmark test problems for variable-density flow and transport codes.  The model domain for the Henry problem is 2 $m$ long by 1 $m$ tall. In the original version of the problem, for which \cite{henry1964} presented a semianalytical solution, freshwater with a density of 1000 $kg/m^3$ flows into the domain through the left side at a rate of 5.7024 $m^3/d$. \cite{Simpson2004} reduced the rate of inflow to 2.851 $m^3/d$ in their numerical simulations, rendering the flow system less ``advective dominant'' and effecting ``an increase in the relative importance of density-driven processes'' \citep{Simpson2004}. This modified version of the problem, called the ``low-inflow'' version here, provides a better benchmark test of density-dependent flow behavior than Henry's original version. Results from both versions of the problem are presented.

\cite{henry1964} and \cite{Simpson2004} assign the same flow and transport boundary conditions at the right boundary. The boundary condition for flow is a hydrostatic condition based on seawater concentration, 35 $kg/m^3$, which corresponds to a density of 1024.5 $kg/m^3$. For transport, the concentration at the right boundary is fixed at seawater concentration. The simulations presented here use a variation in which the right boundary is assigned a mixed boundary condition: water that flows into the model from the right boundary enters at seawater concentration, and water that flows out of the model at the right boundary exits at the groundwater concentration computed for that boundary cell. Use of a mixed boundary condition in the Henry problem was introduced by \cite{Segol1975}, who imposed a Neumann-type condition for transport when water flows out of the model. The mixed boundary condition used here, in which outflow is at the prevailing groundwater concentration, is the condition used for the Henry problem by \cite{Voss1984sutra} and \cite{VossSouza1987}. This manner of representing the seawater boundary, which is often used in saltwater intrusion models, allows a freshwater outflow zone to form above the zone of recirculating saltwater. 

% spatial discretization  and temporal discretization
The freshwater hydraulic conductivity is set to 864 $m/d$, and the porosity to 0.35 (tab.~\ref{tab:ex-gwt-henry-01}).  Mechanical dispersion is not represented; all mixing occurs solely by molecular diffusion with a diffusion coefficient of 0.57024 $m^2/d$. The simulation begins with the model domain initially filled with seawater, although the problem can also be simulated with the domain initially filled with freshwater.  If the hydraulic head is fixed for the seawater boundary and the mixed boundary condition is used for transport, then it may be necessary to start the simulation with some saltwater in the domain or there may be no seawater inflow.  In the simulations presented here, the domain is divided into 40 layers and 80 columns of cells, and a simulation period of 0.5 $d$ is divided into 500 equally sized time steps of 0.001 $d$.

% add static parameter table(s)
\input{../tables/ex-gwt-henry-01.tex}

% for examples without scenarios
\subsection{Scenario Results}

The original and low-inflow versions of the Henry problem (tab.~\ref{tab:ex-gwt-henry-scenario}) were  simulated with a mixed boundary condition for concentration in cells along the right boundary. The mixed boundary condition is represented using the General-Head Boundary (GHB) Package, which allows the hydrostatic boundary condition to be effectively imposed at the right edge of the model domain by accounting for the conductance of aquifer material between the cell center and the right edge of the model domain. Conceptually, a seawater reservoir is attached to the edge of each model cell at the right boundary. Flow into the model domain enters at the concentration of seawater, and flow out of the model domain exists at the concentration computed in the corresponding boundary cell. Figure~\ref{fig:ex-gwt-henry-a-conc} shows contours of concentration (relative seawater concentrations of 0.01, 0.1, 0.5, 0.9, and 0.99) at the end of the 0.5 $d$ simulation period for the classic Henry problem.  Figure~\ref{fig:ex-gwt-henry-b-conc} shows results from the low-inflow version of the Henry problem.  These same simulations were reported by \cite{langevin2020hydraulic} and were shown to be in good agreement with results from SEAWAT simulations.

% scenario table
\input{../tables/ex-gwt-henry-scenario.tex}

\begin{StandardFigure}{
                                     Simulation results for the classic Henry problem.
                                     }{fig:ex-gwt-henry-a-conc}{../figures/ex-gwt-henry-a-conc.png}
\end{StandardFigure}                                 

% a figure
\begin{StandardFigure}{
                                    Simulation results for the low-inflow version of the Henry problem.
                                     }{fig:ex-gwt-henry-b-conc}{../figures/ex-gwt-henry-b-conc.png}
\end{StandardFigure}                                 

                      


\clearpage
\insection
\section{Salt Lake Problem}

% Describe source of problem
The salt lake problem was suggested by \cite{simmons1999} as a comprehensive benchmark test for variable-density groundwater flow models. The problem is based on dense salt fingers that descend from an evaporating salt lake. Although an analytical solution is not available for the salt lake problem, an equivalent Hele-Shaw analysis was performed in the laboratory to investigate the movement of dense salt fingers \citep{wooding1997a, wooding1997b}. In addition to the SUTRA simulation, this salt lake problem was simulated by \cite{langevin2003seawat} using the MODFLOW-based SEAWAT-2000 program.  The approach described by   \cite{langevin2003seawat} is followed here to reproduce the salt lake problem with \mf.

\subsection{Example description}

Model parameters used for the \mf simulation of the salt lake problem are shown in table~\ref{tab:ex-gwt-saltlake-01} The model grid and boundary conditions used for the \mf simulation are shown in figure~\ref{fig:ex-gwt-saltlake-bc}. The model grid consists of 135 columns and 57 layers. To accurately capture the number and growth of salt fingers, the model grid has an increased level of resolution beneath the evaporative boundary.  The evaporative boundary is represented in the model by using the Recharge (RCH) Package and specifying a negative recharge rate.  \cite{simmons1999} describe the method for applying the SUTRA code to the salt lake problem. A random numerical perturbation was required to match the formation of the salt fingers observed in the Hele-Shaw experiment. Concentrations along the evaporative boundary were randomly assigned for each node and for each time step.  A similar approach is used here for the \mf simulation, except that the random variations are not reassigned each time step.  The inflow boundary is represented using constant-head cells with a constant inflow concentration.  The \mf model was run for 24,000 seconds (400 minutes) using 60-second transport timesteps. 

% add static parameter table(s)
\input{../tables/ex-gwt-saltlake-01}

% a figure
\begin{StandardFigure}{
                                     Model grid and boundary conditions used for the salt lake problem.  Evaporation occurs from the cells highlighted in red.  Inflow into the model grid occurs through the constant-head cells shown in blue.
                                     }{fig:ex-gwt-saltlake-bc}{../figures/ex-gwt-saltlake-bc.png}
\end{StandardFigure}                                 


% for examples without scenarios
\subsection{Example Results}

The salt lake problem represents a complex system of salt fingers that form, descend, and then coalesce due to the larger-scale flow system. Although the results from \mf are not identical with the results from the Hele-Shaw experiment, \mf seems capable of representing the growth rate and number of salt fingers (fig.~\ref{fig:ex-gwt-saltlake-conc}). In the experiment and model, six or seven salt fingers are initially produced, of which only two persist. The rate of descent is similar for both the experiment and model.  

% a figure
\begin{StandardFigure}{
                                     Color-shaded plots of concentration simulated by \mf for the \cite{simmons1999} problem involving density-driven groundwater flow and solute transport.  Panels can be compared with the panels shown for the Hele-Shaw experient \citep{wooding1997b} and for the SEAWAT-2000 numerical simulation \citep{langevin2003seawat}.
                                     }{fig:ex-gwt-saltlake-conc}{../figures/ex-gwt-saltlake-conc.png}
\end{StandardFigure}                                 

                              


\clearpage
\insection
\section{Rotating Interface Problem}

% Describe source of problem
The rotating interface problem was suggested by \cite{bakker2004} as a  benchmark test for variable-density groundwater flow models. The problem consists of a box filled with three immiscible fluids that are separated by sharp interfaces.  The fluid are initially in an unstable configuration that causes them to rotate.  This salt lake problem was simulated by \cite{langevin2003seawat} using the MODFLOW-based SEAWAT-2000 program.  \cite{bakker2004} also shows simulation results for the SWI Package for MODFLOW and MOCDENS3D.  The approach described by   \cite{langevin2003seawat} is followed here to reproduce the salt lake problem with \mf.

\subsection{Example description}

 The problem consists of a cross-sectional box filled with three fluids of different densities (fig.~\ref{fig:ex-gwt-rotate-config}). The initial boundaries between the fluids are not horizontal, and thus, the fluids rotate. There are two cases of this problem, one for symmetric rotational flow and one for asymmetric rotational flow.  \cite{bakker2004} and \cite{langevin2003seawat} compared simulated velocities at the onset of rotation with velocities obtained using an analytical solution.  In this \mf version, the transient evolution of the rotating surfaces is shown for the case involving symmetric rotational flow.
 
 % a figure
\begin{StandardFigure}{
                                     Configuration and variable definition for the rotating interface problem.  From \cite{bakker2004} and \cite{langevin2003seawat}.  Note that the constant-head boundary is not needed for the \mf simulation.
                                     }{fig:ex-gwt-rotate-config}{../images/ex-gwt-rotate-config.png}
\end{StandardFigure}            

Model parameters used for the \mf simulation of the rotating interface problem are shown in table~\ref{tab:ex-gwt-rotate-01} The model grid and initial conditions used for the \mf simulation are shown in figure~\ref{fig:ex-gwt-rotate-bc}. The model grid consists of 300 columns and 80 layers.  The interfaces between the three fluids are straight and slope down and to the right. The aquifer and fluids are assumed to be incompressible, and the effects of concentration on fluid viscosity are assumed to be negligible. The freshwater hydraulic conductivity is homogeneous and isotropic. Symmetric rotational flow results when the density for the middle fluid (zone 2) is set as the average of the two outer fluids (zone 1 and 3).   Although a constant-head condition can be applied to help with convergence for this problem, it was not required for this \mf simulation.  The \mf model was run for 10,000 days  divided into 1000 timesteps. 

% add static parameter table(s)
\input{../tables/ex-gwt-rotate-01}

% a figure
\begin{StandardFigure}{
                                     Model grid and initial conditions used for the rotating interface problem.  Colors represent three different water types.
                                     }{fig:ex-gwt-rotate-bc}{../figures/ex-gwt-rotate-bc.png}
\end{StandardFigure}                                 


% for examples without scenarios
\subsection{Example Results}

The rotating interface problem represents a complex variable-density flow system that results from an unstable initial density configuration.  During the simulation period, the fluid rotate toward a stable position with lighter water overlying denser water.  Although the problem dictates that the fluids are immiscible, the \mf simulation shows some mixing caused by numerical dispersion.

% a figure
\begin{StandardFigure}{
                                     Color-shaded plots of concentration simulated by \mf for the \cite{bakker2004} problem involving rotating interfaces.
                                     }{fig:ex-gwt-rotate-conc}{../figures/ex-gwt-rotate-conc.png}
\end{StandardFigure}                                 



\clearpage
\insection
\section{Hecht-Mendez 3D Borehole Heat Exchanger Problem}

% Describe source of problem
The models presented in \cite{hechtMendez2010} apply MT3DMS \citep{zheng1999mt3dms} as a heat transport simulator.  Both 2-dimensional and 3-dimensional demonstration problems are presented that explore the use of a ``borehole heat exchanger'' (BHE), a ``closed'' geothermal system that uses a heat pump for cycling water and anti-freeze fluids in pipes for mining heat from an aquifer \citep{diao2004}.  Figure~\ref{fig:hechtMendezBHE} depicts the kind of system modeled in this example.  

Among the examples presented in \cite{hechtMendez2010}, this work recreates the 3D example that simulates heat exchange between a BHE and the aquifer it is placed in.  Among the suite of examples included in the MODFLOW6-examples.git repo, this is the first one demonstrating the suitability of the GWT model within \mf for simulating saturated zone heat transport.  To verify the applicability of \mf6 as a groundwater heat transport simulator, we compare the \mf solution to established analytical solutions.  The analytical solutions are described in more detail below.

% a figure
\begin{StandardFigure}{
                                     Example of a closed ground source heat pump extracting heat beneath a private residence.  Image taken from \cite{hecht2008}.
                                     }{fig:hechtMendezBHE}{../images/hechtMendezBHE.png}
\end{StandardFigure}       

\subsection{Example description}

Through appropriate substitution of heat-related transport terms into the groundwater solute transport equation, \mf (as is the case with MT3DMS and MT3D-USGS) may be used as a heat simulator for the saturated zone by observing that the heat transport equation, 


\clearpage
\insection
\section{Stallman Problem}

% Describe source of problem
\cite{stallman1965steady} presents an analytical solution for transient heat flow in the subsurface in response to s sinusoidally varying temperature boundary imposed at land surface.  The problem includes heat convection in response to downward groundwater flow.  The problem also includes heat conduction through the fully saturated aquifer material.  The analytical solution quantifies the  temperature variation as a function of depth and time for this one-dimensional transient problem.

This section presents the results of a \mf simulation and the corresponding analytical solution presented by \cite{stallman1965steady}.  The \mf simulation includes a GWF Model and a GWT Model.  Although the GWT Model was developed for solute transport, input parameters can be modified so that the GWT Model can approximate heat transport.

\subsection{Example Description}

The example problem presented here consists of a vertical profile from land surface (0 m) to a depth of 60 m.  There are 120 model cells used to represent the profile.  Groundwater flow is simulated using steady-state conditions with the top and bottom cells assigned as constant heads.  The head in the top cell is given a value of 60 m and the head in the bottom cell is set to 59.701801 m.  This results in a constant downward flow velocity that remains constant during the simulation.

The model simulates 10 sinusoidal periods over a total duration of 10 years (wave length = 1 year) with a total of 600 stress periods and 6 time steps per period (time step = 1 day).

 For the GWT Model setup, an ambient temperature of 10 $^o C$ is given as the initial condition. The temperature variation at the surface boundary is 5 $^o C$ and varies with time according to
 $T_{BC} = 10+5sin(2\pi t/T)$, where t is the current time and T is the wave length of one year.
Solute analogs for the heat transport problem were calculated from thermal parameters.  The diffusion coefficient is given as 1.02882E-06 ($m^2/s$), and linear sorption is activated with porosity = 0.35, bulk density = 1709.5 ($kg/m^3$) and the distribution coefficient = 0.000191663.  
Model parameters used for this example are shown in Table~\ref{tab:ex-gwt-stallman-01}.

% add static parameter table(s)
\input{../tables/ex-gwt-stallman-01}

% for examples without scenarios
\subsection{Example Results}

The simulated temperature profile from \mf shows good agreement with the temperature profile from the Stallman analytical solution for a simulation time of 9.02 yr (figure \ref{fig:ex-gwt-stallman-conc}).  

% a figure
\begin{StandardFigure}{
                                     Comparison of the temperature profile simulated with \mf (dashed line) and calculated with the \cite{stallman1965steady} analytical solution (blue circles).
                                     }{fig:ex-gwt-stallman-conc}{../figures/ex-gwt-stallman-conc.png}
\end{StandardFigure}

\clearpage
\insection
\section{Synthetic Valley Problem}

% Describe source of problem
This example is based on a flow and transport problem described in \cite{hughes2023flopy}. The Synthetic Valley examples represents a developed alluvial valley surrounded by low permeability bedrock. The model includes the Blue Lake and Straight River surface water features (figure~\ref{fig:ex-gwt-synthetic-valley-river-discretization}). The upper two layers represent an unconfined aquifer, the third layer represents a confining unit, and the lower three layers represent the lower aquifer unit. The confining unit only exists in the northern part of the model domain as shown in figure~\ref{fig:ex-gwt-synthetic-valley-river-discretization}.

% a figure
\begin{StandardFigure}{
                                     Map showing the Voronoi grid used to discretize the model domain and the location of Blue Lake, Straight River, and the areal extent of the confining unit separating the upper and lower aquifer units for the Synthetic Valley example in \cite{hughes2023flopy}.
                                     }{fig:ex-gwt-synthetic-valley-river-discretization}{../figures/ex-gwt-synthetic-valley-river-discretization.png}
\end{StandardFigure}                                 

\subsection{Example description}

The 6,096 m x 3,810 m model domain is discretized using a Voronoi grid, with 6,343 active cells per layer, and the discretization by vertices (DISV) package (figure~\ref{fig:ex-gwt-synthetic-valley-river-discretization}). The model grid was refined within Blue Lake, around Straight River, and around pumping wells P1, P2, and P3. The parameters used for this problem are listed in table~\ref{tab:ex-gwt-synthetic-valley-01}. 

% add static parameter table(s)
\input{../tables/ex-gwt-synthetic-valley-01}

In this example, both groundwater flow \citep{modflow6gwf} and solute transport \citep{modflow6gwt} are simulated. To better represent solute transport, the lower aquifer has been discretized into three layers. Confining units have to be explicitly simulated in \mf, therefore, a total of six layers are simulated. The bottom of layers 1, 2, 3, and 4 were set to constant values of -1.53, -15.24, -15.55 and -30.48 m, respectively. Model layer 3 represents the confining unit and is relatively thin (0.3 m). The \texttt{IDOMAIN} concept \citep{modflow6gwf} was used to eliminate cells in model layer 3 (by setting \texttt{IDOMAIN=-1}) where the confining unit does not exist. In these areas, the thickness of layer 3 was set to zero and \texttt{IDOMAIN} was set to -1, which marks these cells in layer 3 as ``vertical pass through cells'' and results in cells in layer 2 being directly connected to cells in layer 4.  

The bottom of the model (layer 6) is based on \cite{hill1998controlled} and the bottom of layer 5 was specified to be half the distance between the bottom of layers 4 and 6. The top of the model was constructed from topographic contours developed for the model that was used as the starting point for \cite{hill1998controlled}; the top of the model is shown in Figure~\ref{fig:ex-gwt-synthetic-valley-head}A. The top of the model and the bottom of layer 6 were resampled from the data used in \cite{hill1998controlled}.

The horizontal hydraulic conductivity was discretized into five zones with values of 45.72, 50.29, 60.96, 83.82, and 121.92 m/d; the lowest hydraulic conductivity zone was located south of Blue Lake and the highest hydraulic conductivity zone was located beneath Blue Lake. The vertical hydraulic conductivity in the upper and lower aquifer was specified to be one quarter of the horizontal hydraulic conductivity. The horizontal and vertical hydraulic conductivity in the confining unit was set equal to 9.14$\times10^{-4}$ m/d. The horizontal and vertical hydraulic conductivity were resampled from the data used in \cite{hill1998controlled}.

For the groundwater transport model, the porosity, longitudinal dispersivity, and transverse dispersivity were set to values specified in table~\ref{tab:ex-gwt-synthetic-valley-01}. For the transport model, the Total Variation Diminishing scheme available in the GWT model \citep{modflow6gwt} was used to simulate advection. Molecular diffusion was not represented.

Straight River is simulated using the streamflow routing (SFR) package, and Blue Lake is simulated using the LAK package (figure~\ref{fig:ex-gwt-synthetic-valley-river-discretization}). Straight River was discretized into 108 SFR reaches. The bed thickness and width of each SFR reach were set to values specified in table~\ref{tab:ex-gwt-synthetic-valley-01}. The leakance for each SFR reach was calculated using the bed thickness, reach width, and reach length in each cell and based on a total Straight River conductance of 50,971.72 m$^2$/d. Specified rainfall and potential evaporation rates specified in table~\ref{tab:ex-gwt-synthetic-valley-01} were defined for each Straight River reach.

Blue Lake was simulated as a lake on top of the model grid and only had vertical connections to 1,406 cells in the underlying upper aquifer (model layer 1). A bed leakance of 0.0013 1/d was specified for each cell connected to Blue Lake. Specified rainfall and potential evaporation rates specified in table~\ref{tab:ex-gwt-synthetic-valley-01} were defined for Blue Lake.

Drain (DRN) cells were specified in each cell in model layer 1 that was not connected to Blue Lake to prevent water levels from exceeding the top of the model. The conductance of each DRN cell was based on the horizontal cell area and the drain bed thickness and vertical hydraulic conductivity specified in table~\ref{tab:ex-gwt-synthetic-valley-01}. Linear scaling of the drainage conductance was applied to improve model convergence and ranged from 0 m$^2$/d when groundwater levels were greater than or equal to 0.3048 m below the top of the model to the specified conductance when groundwater water levels were greater than or equal to the top of the model.

Uniform recharge and potential evapotranspiration rates were specified using the recharge (RCH) and evapotranspiration (EVT) packages, respectively, and were equal to the rates specified in the SFR and LAK packages. The EVT surface was specified to be the top of the model and the EVT extinction depth was specified to be 1 m.

Pumping rates for wells P1, P2, and P3 were -7,600, -7,600, and -1,900 m$^3$/d, respectively. All groundwater pumpage was extracted from model layer 6.

Transport was not simulated in the LAK and SFR packages. Instead, a specified concentration condition with a concentration of 1.0 mg/L was specified for Blue Lake. All other stress packages were assumed to have a concentration of 0 mg/L.

An initial head of 11 m was specified for every cell. An initial stage of 3.44 m was specified for Blue Lake. An initial concentration of 0 mg/L was specified for every cell in the transport model.


% for examples without scenarios
\subsection{Example Results}

The groundwater flow model used the Newton-Raphson Formulation with Newton under-relaxation to improve convergence. The groundwater flow and transport models used the Bi-conjugate Stabilized linear accelerator and simple solver settings.
 
The groundwater flow and transport models were run for a total of 30 years. The groundwater flow model used a single steady-state time step and groundwater flow results were used to run the transport model with a total of 60 time steps with a constant length of 182.625 days.
 
Simulated heads and vectors of specific discharge in model layer 1 are shown in figure~\ref{fig:ex-gwt-synthetic-valley-head}B. Specific discharge is greatest on the east side of Blue Lake and in the vicinity of the three pumping wells and Straight River. 

% a figure
\begin{StandardFigure}{
                                     Color shaded plot of (A) topography and (B) simulated steady-state heads and specific discharge rates in model layer 1.
                                     }{fig:ex-gwt-synthetic-valley-head}{../figures/ex-gwt-synthetic-valley-head.png}
\end{StandardFigure}                                 


Simulated concentrations at the end of 30-years in all six model layers are shown in figure~\ref{fig:ex-gwt-synthetic-valley-conc}. Simulated concentrations are highest beneath Blue Lake in model layer 1 and do not vary much in model layers 1 and 2. Simulated concentrations in model layer 3 are limited to the extent of the confining unit because the remaining cells in the layer are defined to be ``vertical pass through cells''. The lateral extent of the solute plume does not vary much south of Blue Lake because of the lack of confinement in these areas.

% a figure
\begin{StandardFigure}{
                                     Contours of concentrations at the end of 30 years in model layer (A) 1, (B) 2, (C) 3, (D) 4, (E) 5, and (F) 6. The extent of the confining unit in model layer 3 is also shown on (C).
                                     }{fig:ex-gwt-synthetic-valley-conc}{../figures/ex-gwt-synthetic-valley-conc.png}
\end{StandardFigure}                                  


\clearpage
\insection
\section{Radial Groundwater Flow Model}
% Describe source of problem
Radial groundwater models simulate and analyze the flow of groundwater in a radial coordinate system. The purpose of radial groundwater models is to understand the behavior of groundwater flow in response to groundwater extraction or to estimate aquifer properties. Radial groundwater models are axisymmetric with the constant aquifer properties defined in circular radial bands and vertical layers. The radial grid’s innermost band is cylinder (single radius) and typically contains a groundwater well. Extending outward, the subsequent radial bands are ring shaped that are bound by an inner and outer radius (hollow cylinder). This example demonstrates the capability of the Unstructured Discretization (DISU) Package to represent a radial groundwater flow model.

This example replicates in MODFLOW 6 using FloPy the ``Pumping Well'' radial model described in \cite{bedekar2019axisym}. The original model used MODFLOW-USG to demonstrate that it is possible to use unstructured grid cells to simulate a banded, radial model. The MODFLOW 6 results are compared with the \cite{neuman1974effect} analytical solution for radial, unconfined flow with a partially penetrating well.


\subsection{Example description}

% spatial discretization 
The example consists of a multi-layer, radial model representing an unconfined, homogeneous, and isotropic aquifer with a partially penetrating well. The radial grid represents the horizontal direction and is composed of 22 radial bands that vary in outer radius from 0.25 $ft$ to 2000 $ft$ 
(fig.~\ref{fig:ex-gwf-rad-disu-grid}\textit{A}). Radial band 1 is a cylinder with a radius 0.25 $ft$ and radial band 22 is a hollow cylinder with an inner and outer radius of 1500 $ft$ and 2000 $ft$, respectively. The vertical direction consists of 25 uniform-layers that are 2 $ft$ thick; with a total aquifer thickness of 50 $ft$. A partially penetrating well is located at radial band 1 
(fig.~\ref{fig:ex-gwf-rad-disu-grid}\textit{B}) and extracts water from the bottom 10 $ft$ (layer 21 to 25) at a rate of 4000 $ft^3/day$. The remaining model properties are summarized in table~\ref{tab:ex-gwf-radial-01}.

\begin{StandardFigure}{
                                     Plan view of the two-dimensional, radial model grid.
                                     \textit{A}, plan view of entire model grid containing 22 radial bands, and 
                                     \textit{B}, plan view of the 5 innermost radial bands and the well location is marked in red.
                                     }{fig:ex-gwf-rad-disu-grid}{../figures/ex-gwf-rad-disu-grid.png}
\end{StandardFigure}                                 


% temporal discretization
The model simulation evaluates drawdown in response to pumping at an observation point. The simulation time frame is composed of 1 stress period of length 10 $days$, subdivided into 24 time steps with a multiplier of 1.24. The aquifer is initially saturated with an initial water level is set to 50 $ft$, which is the reference level for drawdown. Observations are made at a radial distance of 40 $ft$ (radial band 12) at the model’s top (1 $ft$ depth; layer 1), middle (25 $ft$ depth; layer 13), and bottom (49 $ft$ depth; layer 25) for every time step. 

% material properties

% initial conditions

% boundary conditions

% add static parameter table(s)
\input{../tables/ex-gwf-radial-01.tex}

% Extra Section
\subsection{Radial Setup with Unstructured Discretization (DISU)}
To represent a radial grid, the DISU defines cell connectivity based on radial band circumference, circular areas, and the radial distance between bands. All radial bands have the same properties for each layer. That is, each layer’s radial band has the same radius, circumference and area and only shifts the top and bottom elevation downward by 2 $ft$. A maximum of 4 connections is possible for any given node; the connections represent flow towards the inner band, to the outer band, upward flow, and downward flow.

Node numbers are assigned based on the radial band number (R), layer number (L), and the total number of radial bands (nradial) as described in equation~\ref{eq:radial_disu_node}.

\begin{equation}
   \text{node} = \text{nradial} \cdot \left(\text{L}-1\right)+\text{R}
   \label{eq:radial_disu_node}
\end{equation}

% where:
% \begin{conditions}
%  node    & is the DISU node number,
%  nradial & is the total number of radial bands,
%  L       & is the layer number,
%  R       & is the radial band number.
% \end{conditions}


To make the following more compact, radial band and layer numbers are presented as ``(R, L)'' pairs. The first node is (1, 1) and is a cylinder that is connected to the next radial band, (2, 1), and the cylinder beneath it, (1, 2). The nodal connections (JA) for node 1 is then nodes 2 and 23. The second node, (2, 1), is connected to the inner band at (1, 1), an outer band at (3, 1), and the downward band (2, 2). The JA connections for node 2 is then nodes 1, 3, and 24. Similarly node 24 is located at (2, 2) is connected to the inner band (1, 2), outer band (3, 2), upward band (2, 1), and downward band (2, 3). The JA connections for node 24 is then nodes 23, 25, 2, and 46, respectively.

The input requires specifying the plan view, cell surface area. This is calculated using equation~\ref{eq:radial_disu_area}.
\begin{equation}
   \text{AREA}_j = \pi\left(r_j^2-r_{j-1}^2\right)
   \label{eq:radial_disu_area}
\end{equation}
where $r$ is the outer radius for radial band $j$ and $r_0=0$. The surface area for any radial band is the same for all layers.

The connection length (CL12) depends on if the cell connection is in the vertical or radial direction. Vertical connections are half the distance between the top and bottom elevation of the cell. Since all cells have a thickness of 2 $ft$, CL12 is 1 $ft$ for all vertical connections. The radial direction has a CL12 for radial band 1 equal to its outer radius. For the rest of the radial bands CL12 is same for both the inner and outer directions and is half the distance between the inner and outer radius. That is, $\text{CL12}_j = 0.5\left(r_j-r_{j-1}\right)$, where $r$ is the outer radius for radial band $j$ and $r_0={-r}_1$. 

The input HWVA is the plan view area ($ft^2$) for vertical connections and the width ($ft$) perpendicular to flow for horizontal connections. Since all layers have the same radii, the vertical $\text{HWVA}_j$ is equal to $\text{AREA}_j$. The horizontal HWVA width is equal the radial band’s circumference ($2\pi r$) that the flow passes through. For flow towards the inner band, $\text{HWVA}_{j,\ \text{inner}} = 2\pi r_{j-1}$, and towards the outer band, $\text{HWVA}_{j,\ \text{outer}}\ = 2\pi r_j$.

To assist with the FloPy setup a script called get\_disu\_radial\_kwargs.py is provided. This script provides the function \texttt{get\_disu\_radial\_kwargs} that assembles the nodal connections and aforementioned properties (JA, AREA, CL12, HWVA). The function input expects the number of layers, number of radial bands, outer radius for each band, surface elevations, and layer thicknesses. 


% for examples without scenarios
\subsection{Example Results}

Figures \ref{fig:ex-gwf-rad-disu-obs-head} and \ref{fig:ex-gwf-rad-disu-obs-dimensionless} present the simulated head and dimensionless drawdown results of the radial model with an initial head of 50 $ft$ and 4000 $ft^3/d$ pumping for 10 days. In the figures the circles represent the MODFLOW 6 (MF6) solution at the end of the time step and the lines are the analytical solution from Equation 17 in \cite{neuman1974effect}. The analytical solution uses dimensionless time and drawdown, so the results are presented in both head (MF6 native solution; fig.~\ref{fig:ex-gwf-rad-disu-obs-head}) and dimensionless drawdown (analytical native solution; fig.~\ref{fig:ex-gwf-rad-disu-obs-dimensionless}). Dimensionless time with respect to specific yield and dimensionless drawdown are defined as:
%
\begin{equation}
   t_y = \frac{Tt}{S_yr^2}
   \label{eq:dimensionless_time}
\end{equation}
%
\begin{equation}
   s_d = \frac{4\pi Ts}{Q}
   \label{eq:dimensionless_drawdown}
\end{equation}
%
where $t_y$ is dimensionless time with respect to specific yield (-), $T$ is the initial, radial direction transmissivity ($ft^2/d$), $t$ is the simulation time ($d$), $S_y$ is the specific yield (-), $r$ is the radial distance from the well to the observation point ($ft$), $s_d$ is dimensionless drawdown (-), $s$ is the drawdown from the initial water table elevation ($ft$), and $Q$ is the pumping rate ($ft^3/d$).

\begin{StandardFigure}{
                                     Radial model analytical solution \citep{neuman1974effect} and simulated groundwater (MF6) head 40 $ft$ from the well (radial band 12)
                                     at the model’s Top (1 $ft$ depth; layer 1), Middle (25 $ft$ depth; layer 13), and Bottom (49 $ft$ depth; layer 25).
                                     }{fig:ex-gwf-rad-disu-obs-head}{../figures/ex-gwf-rad-disu-obs-head.png}
\end{StandardFigure}                                 


\begin{StandardFigure}{
                                     Radial model analytical dimensionless solution \citep{neuman1974effect} and simulated groundwater (MF6) dimensionless drawdown 40 $ft$ from the well (radial band 12)
                                     at the model’s Top (1 $ft$ depth; layer 1), Middle (25 $ft$ depth; layer 13), and Bottom (49 $ft$ depth; layer 25).
                                     }{fig:ex-gwf-rad-disu-obs-dimensionless}{../figures/ex-gwf-rad-disu-obs-dimensionless.png}
\end{StandardFigure}                                 

In figure~\ref{fig:ex-gwf-rad-disu-obs-head}, the MF6 head compare very well to the analytical solution. In figure~\ref{fig:ex-gwf-rad-disu-obs-dimensionless}, the MF6 dimensionless drawdown deviates from the analytical solution initially and then yields similar results. This deviation is more apparent in the first four circles (from the left) because of the small numbers presented on a log-log plot. These errors occur in the first 50 seconds of the 10-day simulation and are negligible in comparison to the rest of the simulation.


\clearpage
\insection
\section{Curvilinear Groundwater Flow Model}
% Describe source of problem
MODFLOW 6 vertex grids specify a set of vertices that define the edges of each model cell. Two (x, y) vertex pairs are connected in the horizontal plane by a straight line to represent edges of model cells. A set of vertices can represent a curvilinear grid \citep{romero2006grid} by a piecewise linear representation of the curved surfaces. Curvilinear grids can represent a radial model or replicate a curved flow domain with a major change in the dominant flow path. For example, a 90$^{\circ}$ curvilinear grid can change the dominant flow path from the x-direction to the y-direction along the same model ``column''.  This example demonstrates the capability of the Discretization by Vertices (DISV) Package to represent a curvilinear grid.

Using FloPy this example replicates the steady-state, curvilinear model described in \cite{romero2006grid} using the Figure 3d grid. The original model used a modified MODFLOW-88 \citep{modflow88} code to demonstrate the ability of MODFLOW to simulate curvilinear models by distorting the model grid to use trapezoids. The results were validated against the analytical solution presented in equation 5.4 in \cite{crank1975diffusion}.


\subsection{Example Description}

% spatial discretization
The example consists of a steady-state, 90$^{\circ}$ curvilinear model representing a single layer, confined, homogeneous, and isotropic aquifer (fig.~\ref{fig:ex-gwf-curve-90-grid}). The positive x-axis represents the 0$^{\circ}$ angle and the positive y-axis is the 90$^{\circ}$ angle. The curvilinear model is discretized by radial bands and columns. The radial bands are a piecewise linear approximation of a curve and the columns are the discretization within a radial band.

The radial band numbering starts at 1 for the inner most band and the column number starts at 1 for the column closes to the x-axis. For MODFLOW 6, the DISV package identifies cells by the CELL2D number ($icell2d$). For this example, $icell2d$ starts at radial band 1, column 1 and increases sequentially in the column direction, then the radial direction (fig.~\ref{fig:ex-gwf-curve-90-grid}).

All vertices between two radial bands have the same radial distance from the axis origin. The inner- and outer most vertices have a radial distance of 4 $ft$ and 20 $ft$, respectively. The radial distance between band vertices is 1 $ft$ for a total of 16 curvilinear radial bands. The radial bands are discretized in 5$^{\circ}$ increments for a total of 18 columns per band. The single model layer is 10 $ft$ thick with a transmissivity of 0.19 $ft^2/day$. A constant head boundary condition is assigned to the inner and outer most radial bands at 10 $ft$ and 3.334 $ft$, respectively. The remaining model properties are summarized in table~\ref{tab:ex-gwf-curvilinear-90-01}.

\begin{StandardFigure}{
                                     Model grid used for the 90$^{\circ}$ curvilinear vertex model.
                                     Constant-head cells are marked in blue.
                                     The inner constant head is 10 $ft$ and the outer constant head is 3.33 $ft$.
                                     Cell numbers are shown inside each model cell.
                                     Vertices are also numbered and are shown in red.
                                     }{fig:ex-gwf-curve-90-grid}{../figures/ex-gwf-curve-90-grid.png}
\end{StandardFigure}


% add static parameter table(s)
\input{../tables/ex-gwf-curvilinear-90-01.tex}

% for examples without scenarios
\subsection{Example Results}
The curvilinear vertex model is solved using one, steady state, stress period and compares it to the analytical solution (eqn. 5.4 in \cite{crank1975diffusion}). Figure~\ref{fig:ex-gwf-curve-90-head} presents the MODFLOW 6 simulated head and flow lines for all model cells. Figure~\ref{fig:ex-gwf-curve-90-obs-head} presents the head solution along column 9 from all radial bands ($icell2d\colon 9, 27, 45, \ldots, 261, 279$) and compares it to the analytical solution. The MODFLOW 6 head solution occurs at the cell center rather than the edge, so the radial distance of the inner most head solution is 4.5 $ft$ instead of 4 $ft$. The MODFLOW 6 results are nearly identical to the analytical solution with a relative error of 0.05\% and root mean square error of 0.0037 $ft$.

\begin{StandardFigure}{
                                     Steady state head solution and flow rate line arrows from the MODFLOW 6 curvilinear vertex model.
                                     }{fig:ex-gwf-curve-90-head}{../figures/ex-gwf-curve-90-head.png}
\end{StandardFigure}

\begin{StandardFigure}{
                                     Steady state head solution from the MODFLOW 6 curvilinear vertex model (MF6) compared with the analytical solution \citep{crank1975diffusion}.
                                     The MF6 head values are taken from each radial band along column 10.
                                     The radial distance of each MF6 solution is the distance from the vertex axis origin to the cell center.
                                     }{fig:ex-gwf-curve-90-obs-head}{../figures/ex-gwf-curve-90-obs-head.png}
\end{StandardFigure}


\clearpage
\insection
\section{Multipart Curvilinear Groundwater Flow Model}
% Describe source of problem
This example demonstrates how the MODFLOW 6 DISV Package can be used to simulate a multipart curvilinear model. This model extends the MODFLOW 6 ``Curvilinear Groundwater Flow Model'' example to reproduce the model grid presented in Figure 6 of \cite{romero2006grid}. The hypothetical, curvilinear grid represents a meandering, curved flow path that traditional, structured grids cannot simulate. Figure~6 in \cite{romero2006grid} was introduced as an illustration of a curvilinear MODFLOW grid, but did develop it as a actual simulation model. This example illustrates that MODFLOW 6 can simulate this hypothetical grid.


\subsection{Example Description}

% spatial discretization
The hypothetical, curvilinear grid is composed of three distinct model regions that are combined to form the final grid. The first region (Left Grid; fig.~\ref{fig:ex-gwf-curvilin-grid-components}\textit{A}) is a curvilinear grid with 16 radial bands that start at 180$^{\circ}$ and end at 270$^{\circ}$ with a column discretization of 5$^{\circ}$ (18 columns). The Left Grid's inner- and outer-most radius is 4 $ft$ and 20 $ft$, respectively, and the radial direction vertices are 1 $ft$ apart. The second region (Center Grid; fig.~\ref{fig:ex-gwf-curvilin-grid-components}\textit{B}) is a 1 $ft$ rectangular, structured grid with 16 rows and 18 columns. The third region (Right Grid; fig.~\ref{fig:ex-gwf-curvilin-grid-components}\textit{C}) is an identical curvilinear grid as the first, but starts at 90$^{\circ}$ and end at 0$^{\circ}$. The three regions are combined to derive the hypothetical, curvilinear grid presented in figure~\ref{fig:ex-gwf-curvilin-grid}.

The hypothetical, curvilinear grid contains a single, 10 $ft$ thick, model layer with a transmissivity of 0.19 $ft^2/day$. There are two constant head boundary conditions that are placed along the columns of the curvilinear regions (fig.~\ref{fig:ex-gwf-curvilin-grid}). The first constant head boundary is 10 $ft$ and is along the first column of the Left Grid. The second constant head boundary is 3.334 $ft$ and is along the last column of the Right Grid. The remaining model properties are summarized in table~\ref{tab:ex-gwf-curvilinear-01}.

\begin{StandardFigure}{
                                     Three regions that combine to construct the hypothetical, curvilinear grid. 
                                     \textit{A}, Left curvilinear grid from 180$^{\circ}$ to 270$^{\circ}$, 
                                     \textit{B}, Center 16 by 18 rectangular grid , and 
                                     \textit{C}, Right curvilinear grid from 90$^{\circ}$ to 0$^{\circ}$. 
                                     Grid vertices are marked in yellow. Note, the x, y coordinate positions 
                                     are included for relative comparisons and not for the specific spatial location. 
                                     }{fig:ex-gwf-curvilin-grid-components}{../figures/ex-gwf-curvilin-grid-components.png}
\end{StandardFigure}

\begin{StandardFigure}{
                                     Plan view of the hypothetical, curvilinear grid with a meandering, curved flow path. 
                                     Constant-head cells are marked in blue. 
                                     The cell numbers $1, 19, \ldots, 253, 271$ constant head is 10 $ft$. The cell numbers $577, 595, \ldots, 829, 847$ constant head is 3.33 $ft$. 
                                     Grid cell numbers are shown inside each model cell. 
                                     Grid vertices are yellow with vertex numbers in red. 
                                     }{fig:ex-gwf-curvilin-grid}{../figures/ex-gwf-curvilin-grid.png}
\end{StandardFigure}

% add static parameter table(s)
\input{../tables/ex-gwf-curvilinear-01.tex}

% for examples without scenarios
\subsection{Example Results}
The hypothetical, curvilinear grid (fig. ~\ref{fig:ex-gwf-curvilin-grid}.) is solved using one, steady state, stress period. Figure~\ref{fig:ex-gwf-curvilin-head} presents the MODFLOW 6 simulated head and flow lines for all model cells.

\begin{StandardFigure}{
                                     Steady state head solution and specific discharge vectors from the MODFLOW 6 curvilinear grid with a meandering, curved flow path.
                                     }{fig:ex-gwf-curvilin-head}{../figures/ex-gwf-curvilin-head.png}
\end{StandardFigure}




% GWE Model examples
\clearpage
\insection
\section{Radial Heat Transport}

% Describe source of problem
A dedicated heat transport model referred to as the groundwater energy transport (GWE) model was released with version 6.5.0 of \mf. Previously, to simulate groundwater heat transport in \mf, a user could use the groundwater solute transport model, commonly referred to as simply groundwater transport (without specifying ``solute'' transport), to mimic heat transport by using its input parameters as surrogates for heat transport parameters \citep{modflow6gwt, ma2010, langevin2008seawat}. Now, with GWE, users may specify native heat transport parameter values in the appropriate GWE package.

\subsection{Example description}

This example demonstrates use of the GWE model.  This demonstration compares simulated results from a GWE model to an analytical solution that was published in \cite{alKhoury2020}.  Both the groundwater flow (GWF) and GWE models employ a DISV grid type \citep{modflow6software} with the numerical grid setup in a radial manner (fig.~\ref{fig:ex-gwe-radial-grid}). The grid geometry facilitates outward propagation of heat from a borehole heat exchanger (BHE) \citep{hechtMendez2010} located in the center of the radially-symetric model grid (fig.~\ref{fig:ex-gwe-radial-grid}). Groundwater flow moves from left to right. In this way, the model simulates heat flow in a convective and conductive heat transport environment moving past a cylindrical heat source.

% a figure
\begin{StandardFigure}{
    Configuration of the DISV model grid used in the radial transport problem.  Model grid originally published in \cite{alKhoury2020}.  Please refer to figure~\ref{fig:ex-gwe-radial-slow-gridinset} for a zoomed-in view of the model grid in the vicinity of the BHE.}{fig:ex-gwe-radial-grid}{../figures/ex-gwe-radial-slow-grid.png}
\end{StandardFigure}

% inset figure showing a zoomed in portion of the model grid
\begin{StandardFigure}{
    A zoomed-in view showing the refined discretization in close proximity to the BHE.  Cell dimensions are sub-centimeter scale around the perimeter of the BHE.  The radius of the BHE is 7.5 $cm$.}{fig:ex-gwe-radial-slow-gridinset}{../figures/ex-gwe-radial-slow-gridinset.png}
\end{StandardFigure}

Constant heads on the left and right sides of the model domain are specified such that the resulting left-to-right groundwater velocity is $1 \times 10^{-5} \tfrac{m}{s}$ (fig.~\ref{fig:ex-gwe-radial-slow-head}).  The heat source located in the center of the numerical model is represented using the energy source loading (ESL) package with a known rate of energy input [referred to as a Dirichlet boundary condition in \cite{alKhoury2020}].  The initial temperature throughout the model domain is 0.0 $^{\circ}C$. Energy is added to the grid cell in the middle of the model domain at a rate of 100 $\tfrac{W}{m}$.  Parameters used for the \mf simulation of the heat transport problem that uses a radially-symmetric grid are shown in table~\ref{tab:ex-gwe-radial-01}.

% add static parameter table(s)
\input{../tables/ex-gwe-radial-01}

% figure showing groundwater head
\begin{StandardFigure}{
    A head gradient is established in the outer-most ring of grid cells to drive groundwater flow from left to right. The combination of the groundwater head gradient with the hydraulic conductivity (table~\ref{tab:ex-gwe-radial-01}) results in a groundwater velocity of $1 \times 10^{-5} \tfrac{m}{s}$.}{fig:ex-gwe-radial-slow-head}{../figures/ex-gwe-radial-slow-head.png}
\end{StandardFigure}

% for examples without scenarios
\subsection{Example Results}

Results from the GWE model run are compared to a published analytical solution 48 hours after the start of the simulation \citep{alKhoury2020}.  Isotemperature contours at 1, 2, 3, 4, 6, and 8 $^{\circ}C$ demonstrate that GWE results compare well with the analytical solution (fig.~\ref{fig:ex-gwe-radial-temp48}).

% a results figure
\begin{StandardFigure}{
    Simulation results for the GWE model run compared with an analytical solution.}{fig:ex-gwe-radial-temp48}{../figures/ex-gwe-radial-temp48.png}
\end{StandardFigure}


\clearpage
\insection
\section{Interacting Borehole Heat Exchangers in a Geothermal Setting}

% Describe source of problem
Shallow groundwater geothermal investigations often include more than one borehole heat exchanger (BHE) \citep{alKhoury2021}. In such applications, understanding the thermal interaction among multiple BHEs as well as on the flowing groundwater is made easier with a numerical groundwater flow and heat transport model.  In this example, the accuracy of the groundwater energy transport (GWE) model is demonstrated for a convective-conductive porous domain with multiple thermally-interacting BHEs using an analytical solution first published in \cite{alKhoury2021}.  

\subsection{Example description}

For this example nine BHEs are arranged in a 3 $\times$ 3 configuration with a spacing of 5 $m$ from each other. The grid extent is 90 $m \times$ 60 $m$.  Each BHE represents a cyclindrical source of heat with energy being added at a rate of 100 $\tfrac{W}{m}$ using the energy source loading (ESL) package. In order to better simulate the outward propagation of heat from each BHE, the discretization by vertices (DISV) grid type for both the groundwater flow (GWF) and GWE models was employed (fig~\ref{fig:ex-gwe-geotherm-grid}). Grid refinement was added around each BHE (fig~\ref{fig:ex-gwe-geotherm-grid-inset}). Grid discretization is coarsened toward the perimeter of the model grid. 

% a figure
\begin{StandardFigure}{
    Configuration of the DISV model grid used to demonstrate the use of GWE in a geothermal transport problem.  The original numerical model grid was published in \cite{alKhoury2021}.  The red box shows the location of figure~\ref{fig:ex-gwe-geotherm-grid-inset} which provides zoomed-in detail of the model grid in the vicinity of the BHEs.}
    {fig:ex-gwe-geotherm-grid}{../figures/ex-gwe-geotherm-grid.png}
\end{StandardFigure}            

% inset figure showing a zoomed in portion of the model grid
\begin{StandardFigure}{
    A zoomed-in view showing the refined discretization in close proximity to the BHEs.  Cell dimensions are on the scale of centimeters around the perimeter of the BHEs.  The diameter of each BHE is 10.0 $cm$.}
    {fig:ex-gwe-geotherm-grid-inset}{../figures/ex-gwe-geotherm-grid-inset.png}
\end{StandardFigure}            

In order to test the \mf solution against the published analytical solution, the heat transport model simulates a porous media domain with a porosity of 0.20. Within the GWF model, two constant head (CHD) packages were setup on the left and right sides of the model, respectively, to drive groundwater flow from left to right with a velocity of $1 \times 10^{-5} \tfrac{m}{s}$ (fig~\ref{fig:ex-gwe-geotherm-head}).  The initial temperature throughout the model domain is 0.0 $^{\circ}C$. Energy is added to the grid cell in the middle of the model domain at a rate of 100 $\tfrac{W}{m}$.  Parameters used for the \mf simulation of the geothermal heat transport problem are shown in table~\ref{tab:ex-gwe-geotherm-01}.

% add static parameter table(s)
\input{../tables/ex-gwe-geotherm-01}

% figure showing groundwater head
\begin{StandardFigure}{
    The established head gradient drives groundwater flow from left to right with a velocity of $1 \times 10^{-5} \tfrac{m}{s}$.}
    {fig:ex-gwe-geotherm-head}{../figures/ex-gwe-geotherm-head.png}
\end{StandardFigure}            

% for examples without scenarios
\subsection{Example Results}

Results from the geothermal model run are compared to a published analytical solution 50 days after the start of the simulation \citep{alKhoury2021}.  Isotemperature contours at 1, 2, 3, 4, 6, and 8 $^{\circ}C$ provide a visual summary of the match between GWE and the analytical solution (fig.~\ref{fig:ex-gwe-geotherm-temp50days}).  Isotemperature contours match particularly well at the lower temperatures ($\leq 2 ^{\circ}C$).  At temperatures  $>2^{\circ}C$, the simulated temperatures have not advanced as far in the downgradient direction as the analytical solution would suggest. 

% a figure
\begin{StandardFigure}{
    Simulated and analytical isotemperature contours for the geothermal example problem.  The solid lines correspond to the GWE solution while the dashed line represents the analytical solution published by \cite{alKhoury2021}.}
    {fig:ex-gwe-geotherm-temp50days}{../figures/ex-gwe-geotherm-temp50days.png}
\end{StandardFigure}                                 


\clearpage
\insection
\section{Infiltrating Heat Front}

% Describe source of problem
An analytical solution used in chemical engineering for modeling concentration in packed-bed reactors , commonly referred to as a ``Danckwerts'' (or ``third-type'') boundary condition, is adapted here for confirming the accuracy of heat transport as solved by the UZE package in \mf.  The analytical solution has the following characteristics: (1) it solves for total energy flux (advection and conduction) along a 1-dimensional (1D) profile [i.e., there is no conduction (``thermal bleeding'') with the surrounding materials], (2) heat flux can only enter the active domain with the inflow (in this case, the infiltration) and it does not require the temperature at the boundary to be equal to the temperature of the infiltration, and (3) it solves for total heat flux (instead of temperature) throughout the 1D domain.

The Danckwerts analytical solution takes the following form:

\bigskip
$q_{T_z} = q_{T_0} + \dfrac{1}{2} \left( q_{T_{infil}} - q_{T_0} \right) \left( \textit{erfc} \biggl\{ \dfrac{z-\nu t}{2 \sqrt{Dt}} \biggr\}  + \textit{exp} \Bigl\{ \dfrac{\nu z}{D}\Bigr\} \cdot \textit{erfc} \biggl\{ \dfrac{z+\nu t}{2 \sqrt{Dt}} \biggr\} \right)$
\bigskip

\noindent where $q_{T_z}$ is the heat flux and depth $z$, $q_{T_0}$is the infiltrating heat flux at $t=0$, $q_{T_{infil}}$ is the amount of infiltrating heat flux at $t > 0$, $\textit{erfc}$ is the complementary error function, $z$ is the distance from infiltrating heat flux boundary, which, in this example is the depth below land surface, $t$ is time (in days), $\nu$ represents the ``thermal convection velocity'' determined from,

\bigskip
$\nu=q \cdot \dfrac{\rho_w C_{p_w}}{S_{w_z} \theta \rho_w C_{p_w} + \left( 1-\theta \right) \rho_s C_{p_w}}$
\bigskip

\noindent with $q$ equal to the volumetric infiltration rate, $\rho_w$ is the density of water, $C_{p_w}$ is the heat capacity of water, $S_{w_z}$ is the saturation at depth $z$, $\theta$ is the water content, $rho_s$ is the density of a aquifer solids, and $C_{p_w}$ is the heat capacity of the aquifer solids. Additionally, $D$ represents the bulk thermal diffusivity,

\bigskip
$D=\dfrac{k_{T_{bulk}}}{S_{w_z} \theta \rho_w C_{p_w} + \left(1 - \theta \right) \rho_s C_{p_w}}$
\bigskip

\noindent and $k_{T_{bulk}}$ is the bulk thermal conductivity represented by,

\bigskip
$k_{T_{bulk}} = S_{w_z} \theta k_{T_w} + \left(1-\theta \right) k_{T_s}$
\bigskip

\noindent and $k_{T_w}$ and $k_{T_s}$ are the thermal conductivities of the water and aquifer material, respectively.

\subsection{Example description}

A 1D model grid of the unsaturated zone is used to simulate the downward migration of an infiltrating heat front ({fig.~\ref{fig:ex-gwe-danck}}).  Steady flow conditions are simulated with the GWF model.  The UZF package simulates flow through the unsaturated zone with a constant infiltration rate of 0.01 $\frac{m}{d}$. An initial temperature of 10$^{\circ}C$ is specified for the entire model domain.  After steady flow and transport conditions are established with a quasi-steady-state stress period, the temperature of the infiltration is increased to 20$^{\circ}C$ resulting in an energy source loading of 9.68$\frac{J}{s}$ (calculated from $q_{infil} \cdot T_{infil} \cdot \rho_w \cdot C_{p_w}$). Other pertinent parameter values are provided in table~\ref{tab:ex-gwe-danckwerts-01}. 

A constant head boundary is placed at the bottom of the model to remove water that recharges the water table in order to prevent the water table from rising into the unsaturated column ({fig.~\ref{fig:ex-gwe-danck}}).

% figure of model domain
\begin{StandardFigure}{
   View of 1-dimensional model setup. Total thickness of the unsaturated zone is 10 $m$ and is discretized with 100 cells that are each 10 $cm$ thick.}{fig:ex-gwe-danck}{../images/ex-gwe-danck.png}
\end{StandardFigure}

% add static parameter table(s)
\input{../tables/ex-gwe-danckwerts-01}

% for examples without scenarios
\subsection{Example Results}

The heat front that migrates downward through the unsaturated zone as a result of energy loading associated with the infiltration is shown in figure~\ref{fig:ex-gwe-danckwerts-01} at 10, 50, and 100 days.  Figure~\ref{fig:ex-gwe-danckwerts-02} shows the same comparison broken out into 3 subplots, but further parses the downward heat migration into its advective (dark green) and conductive (light green) components.  Where the temperature gradients are steepest, the conductive flux of energy plays a more prominent role in the downward migration of heat.

% a results figure
\begin{StandardFigure}{
   Comparison of simulated migration of infiltrating heat front to an analytical solution}{fig:ex-gwe-danckwerts-01}{../figures/ex-gwe-danckwerts-01.png}
\end{StandardFigure}

% a 2nd results figure
\begin{StandardFigure}{
   The same infiltrating heat fronts as shown in figure~\ref{fig:ex-gwe-danckwerts-01}, but highlights the advective and conductive heat fluxes separately}{fig:ex-gwe-danckwerts-02}{../figures/ex-gwe-danckwerts-02.png}
\end{StandardFigure}


\clearpage
\insection
\section{Viscosity}

% Describe source of problem
In hydrogeologic settings where variations in viscosity significantly impact groundwater flows, the viscosity package can be activated to scale intercell conductance in response to temperature or concentration variations.  Adjustment of intercell conductance may, in turn, have important implications for solute transport or geothermal model applications.  For example, a migrating contaminant plume may be significantly delayed or sped up in settings where the ambient groundwater temperature is significantly different from the reference groundwater temperature – the temperature at which viscosity effects are negligible.

\subsection{Example description}

Application of the viscosity (VSC) package is demonstrated with a simple two-dimensional (2D) test problem.  The 2D demonstration model is adapted from test case 3 in \cite{zheng1999mt3dms} titled ``Two-dimensional transport in a uniform flow field’’.  Specified heads drive groundwater flow through a confined system from left to right.  No flow boundaries are specified along the other model edges.  Water injected into the middle of the model domain enters with a specified concentration of 1,000 $mg/L$.  In the original problem, the simulated development of a contaminant plume was compared to an established analytical solution \citep{wilson1978} to confirm that the advective and dispersive spread of the contaminant was accurately simulated.

For this demonstration, the original groundwater flow (GWF) and groundwater solute transport (GWT) models remain unchanged.  However, an additional 3 models are added to the /mf simulation to explore the importance of viscosity effects on the developing and migrating contaminant plume.  This results in five models in a single /mf simulation (fig.~\ref{fig:vsc-test-setup}).  The three additional models include one each of a GWF, GWT, and groundwater energy transport (GWE) model. The only difference between the original GWF model and the duplicated GWF model is the activation of the VSC package that get the calculated groundwater temperatures from the only GWE model included in the simulation.

% a figure
\begin{StandardFigure}{
    The /mf simulation setup includes two GWF models (blue rectangles), two GWT models (green rectangles), and one GWE model (orange rectangle). Solid arrows highlight the exchange of information among the models included in the simulation.  Dashed arrows show the generation of output used to highlight the effects of viscosity on a solute transport model.}
    {fig:vsc-test-setup}{../images/vsc-test-setup.png}
\end{StandardFigure}

The initial temperature of the GWE model is 4$^{\circ}C$ which also happens to be the temperature of the groundwater entering the model domain through the constant head boundary on the left side of the model.  Inputs to the VSC package, including parameters for the nonlinear viscosity formulation, are listed in table ~\ref{tab:ex-gwe-vsc-01}.  Within /mf, specifically the VSC package (within the GWF model), pointers to the simulated groundwater temperature enable the VSC package to modify intercell conductance calculated by the node-property flow (NPF) package.  Altered intercell conductance in turn affects the flow solution used by the duplicated GWT model (fig.~\ref{fig:vsc-test-setup}) that would otherwise be the same as the original GWT model ultimately resulting in an altered transport solution, leading to the comparison shown next.

% add static parameter table(s)
\input{../tables/ex-gwe-vsc-01}

% brief description of results
\subsection{Example Results}

Output from each GWT model is collected and displayed in figure ~\ref{fig:ex-gwe-vsc-conc-2plts}. Using simple visual inspection differences are apparent between the two plots; however, the magnitude of the difference is more readily observed by differencing the plots, as shown in Figure~\ref{fig:ex-gwe-vsc-diff-02}.  Wherever this difference is negative, meaning the concentrations in the viscosity model run are greater than the no-viscosity model run, the calculated concentrations in the no viscosity model are considered under predicted.  Conversely, higher concentrations in the no viscosity model run compared to the with viscosity model suggest concentrations are over predicted when neglecting viscosity.

It is vital to emphasize the contrived nature of this example.  Given that the calculated temperature field is constant throughout the model domain, the so-called ``effects'' of viscosity shown in Figure~\ref{fig:ex-gwe-vsc-diff-02} would normally be  swamped by the uncertainty of the hydraulic conductivity field.  A more appropriate application of the viscosity package is likely in settings where stark contrasts in groundwater temperature have been observed that may affect the flow field.  In such a setting, activation of the viscosity package may appropriately adjust the intercell conductance values in response to viscosity and reduce compansatory adjustment of the hydraulic conductivity field during a calibration routine, for example.


% a results figure
\begin{StandardFigure}{
   (Top) A concentration plume after 365 days that does not account for the effects of viscosity.  (Bottom) The calculated concentration plume after 365 days that also includes the use of a GWE model and the viscosity package}{fig:ex-gwe-vsc-conc-2plts}{../figures/ex-gwe-vsc-conc-2plts.png}
\end{StandardFigure}

% a 2nd results figure
\begin{StandardFigure}{
   Difference between two concentration fields calculated by two closely related model runs.  In one model setup, the effect of viscosity on flows is ignored, whereas the other model setup does account for viscosity.  The difference is calculated by subtracting the 2D concentration field calculated by the model that includes viscosity from the model run that ignores viscosity.}{fig:ex-gwe-vsc-diff-02}{../figures/ex-gwe-vsc-diff-02.png}
\end{StandardFigure}


\clearpage
\insection
\section{Aquifer Thermal Energy Storage}

% Describe source of problem
Aquifer thermal energy storage (ATES) systems use groundwater wells to store and extract energy from aquifers for reducing energy costs associated with heating, ventilating, and air conditioning (HVAC) systems.  For example, during cold winter months, an ATES system pumps air-chilled water into an aquifer for later recovery (extraction) during hot summer months, thereby reducing energy costs associated with cooling. Furthermore, when the water is pulled out for cooling purposes during hot summer months and is warmed as a result, it can subsequently be stored in a different part of the aquifer and used for heating purposes during cold winter months.  

% add scenario parameter table
\input{../tables/ex-gwe-ates-01}

An obvious requirement for an ATES system is that a suitable aquifer for storing water is present.  The viability of an ATES system also depends on the size of the proposed ATES site. That is, the cold and warm storage areas need to have enough separation to ensure the two storage regions do not interact. Thus, some level of hydrogeologic investigation and accompanying model analysis to determine an aquifer's suitability for hosting an ATES will likely precede the development of an ATES. The groundwater energy transport (GWE) model in \mf offers the functionality for investigating proposed sites for ATES systems.

\subsection{Example description}

A hypothetical ATES is demonstrated with \mf using a two-dimensional (2D) discretization by vertices (DISV) grid with triangular cells.  Water is extracted and re-injected into the left side of the middle layer of an idealized 3-layer aquifer system (fig.~\ref{fig:ex-gwe-ates-prsity}) that otherwise has no-flow boundaries on all other sides.  There is additional grid refinement on the left-side of the model domain where the well is located.  The user-specified aquifer properties in the upper and lower layers are identical; properties in the middle layer are distinct from the upper and lower layers (fig.~\ref{fig:ex-gwe-ates-prsity}).  Table ~ref{tab:ex-gwe-ates-01} lists the flow and transport parameters used in the simulation.  Figure~\ref{fig:ex-gwe-ates-pmprate} shows the extraction and injection pumping intervals.

% a figure
\begin{StandardFigure}{
    The DISV grid used in the groundwater flow (GWF) and groundwater energy transport (GWE) models.  Flow and transport parameter values for the middle layer, referred to as zone 1, are provided in table~\ref{tab:ex-gwe-ates-01}.  Property values for zone 2 comprising both the upper and lower layers also are shown in table~\ref{tab:ex-gwe-ates-01}.}
    {fig:ex-gwe-ates-prsity}{../figures/ex-gwe-ates-prsity.png}
\end{StandardFigure}

% a figure
\begin{StandardFigure}{
    A time series of the extraction and injection pumping rates. Negative and positive values indicate extraction and injection, respectively.  Pumping occurs on the left side of zone 1 (figure~\ref{fig:ex-gwe-ates-prsity}). }
    {fig:ex-gwe-ates-pmprate}{../figures/ex-gwe-ates-pmprate.png}
\end{StandardFigure}

% for examples without scenarios
\subsection{Scenario Results}

Simulated temperatures within the aquifer are shown at (A) 210, (B) 340, (C) 520, and (D) 1,270 days (fig.~\ref{fig:ex-gwe-ates-temp2x2}).  The displayed times correspond to (A) 10 days after the initial injection period, (B) the end of the first 130 day injection period, (C) the end of the second extraction period, and (D) the end of the simulation period after cycling through 4 extraction and 3 injection periods (fig.~\ref{fig:ex-gwe-ates-pmprate}).  Owing to the higher hydraulic conductivity of zone 1, the injected water predominantly flows deeper into zone 1, though some will flow into the upper and lower zone 2 areas.  Importantly, however, the example demonstrates ``thermal bleeding'' into zone 2 that prevents the full amount of injected thermal energy from being re-extracted which is indicative of an ATES systems with efficiencies less than 1.0.  

% a figure
\begin{StandardFigure}{
    Simulated aquifer temperatures at 210, 340, 520, and 1270 days.}
    {fig:ex-gwe-ates-temp2x2}{../figures/ex-gwe-ates-temp2x2.png}
\end{StandardFigure}


\clearpage
\insection
\section{The Barends Problem}

% Describe source of problem
Underground thermal energy storage (UTES) systems, sometimes referred to as aquifer thermal energy storage (ATES) systems, can help bolster sustainable energy supply portfolios. In UTES, energy can be injected into an aquifer at one time to be withdrawn as needed at a later time.  However, ``thermal bleeding,'' or in other words conductive heat exchange with over and underlying geological formations may degrade the efficacy of UTES systems as injected heat diffuses away from the storage site.  To better understand the potential exploitation of UTES at various sites, numerical models are often used to verify field tests and further explore the suitability of a site for UTES \citep{barends2010}.  This demonstration of \mf, hereafter referred to as the ``Barends'' problem, compares \mf GWE output to temperatures predicted by an analytical solution that first appeared in \cite{barends2010}.  The Barends problem provides an excellent test of the Conduction (CND) Package.

\subsection{Example description}

This problem uses a 2D vertical cross-section comprised of two zones - a 100 $m$ thick saturated zone that is overlain by a 100 $m$ thick dry overburden (fig.~\ref{fig:barends-model-setup}). Within the saturated zone, groundwater flow is simulated as a 1-dimensional (1D) flow-field moving from left to right.  Vertical temperature gradients within the groundwater reservoir are ignored by assuming the temperature of the groundwater reservoir is uniform over the entire 100 $m$ thick sequence.  Functionally, this is accomplished using a single layer to represent the groundwater reservoir.  The bottom boundary of the groundwater reservoir is considered sealed for both flow and heat.  However, the upper boundary of the saturated zone - the interface with the overburden - is sealed for flow but not for heat transfer.  

% a figure
\begin{StandardFigure}{
    A simplified depiction of the DISU grid \citep{modflow6gwf} setup that corresponds to the \cite{barends2010} analytical solution. (A) One layer is used to represent the groundwater reservoir, resulting in 1D groundwater flow below the overburden. (B) Multiple 1D vertical columns are aligned next to one another giving the appearance of a 2D grid; however, there are no horizontal connections resulting in vertical heat flow (conduction) only.}
    {fig:barends-model-setup}{../images/barends-model-setup.png}
\end{StandardFigure}

For simulating only conductive heat exchange at the overburden-groundwater reservoir interface, the vertical hydraulic conductivities are set extremely low throughout the entire model domain (Table~\ref{tab:ex-gwe-barends-01}).  The horizontal hydraulic conductivities are set such that the horizontal groundwater velocity is $1.2649 \times 10^{-8} \frac {m}{s}$ within the groundwater reservoir.  Flow is injected in the left-most groundwater reservoir cell and extracted from the right-most cell at the same rate.  Within the overburden, there is no convection (as already mentioned) and thermal conduction is 1-dimensional (1D) in the vertical direction (fig.~\ref{fig:barends-model-setup}).  To accomplish this, inter-cell connections are only established for vertically-aligned neighbors using a DISU grid type \citep{modflow6gwf}.  As a reminder, DISU facilitates explicit specification of which cells are connected; thus, by omitting connections between lateral neighbors there is no horizontal exchange (conduction) of heat within the overburden.  

% add static parameter table(s)
%\input{../tables/ex-gwe-barends-01}

The model domain is initialized with a uniform temperature of $80^{\circ}C$ at the beginning of the simulation ($T_0$).  $30^{\circ}C$ water enters on the left side of the groundwater reservoir at the start of the simulation ($T_0 > 0$).  As cold water flows into the groundwater reservoir, a thermal gradient between the overburden and saturated zone is established that results in the aforementioned thermal bleeding of heat from the overburden into the groundwater reservoir.  Geometric grid refinement is added in the vertical direction just above the overburden-groundwater reservoir interface (fig.~\ref{fig:ex-gwe-barends-gridView})

% a figure
\begin{StandardFigure}{
    A zoomed-in view of the numerical grid showing enhanced vertical refinement above the overburden-groundwater reservoir interface (y=100 $m$).}
    {fig:ex-gwe-barends-gridView}{../figures/ex-gwe-barends-gridView.png}
\end{StandardFigure}

The rate of thermal bleeding at each overburden-groundwater reservoir cell interface changes as the pulse of cold water injected into the groundwater reservoir migrates inward.  Despite the model grid being wired as a series of interconnected 1D cell groupings in both the horizontal (groundwater reservoir) and vertical directions (overburden), its final arrangement has the appearance of a 2D grid, which is how the results are presented.  The analytical solution to which the \mf output is compared is given as Equation 4 in \cite{barends2010},

\begin{equation}
	T' - T_0 = \dfrac{2 \left( T_1 - T_0 \right)}{\sqrt{\pi}} \cdot e^{\left( \displaystyle{ \frac{x v}{2D} } \right)} {\displaystyle \int\limits_{\dfrac{x}{2 \sqrt{Dt}}}^{\infty} e^{-\sigma^2 - \left( \dfrac{xv}{4D\sigma} \right)^2} erfc \Biggl[ \biggl( \dfrac{x^2 h' \sqrt{D'}}{8DH \sigma^2} + \dfrac{z}{2 \sqrt{D'}} \biggr) \biggl( t - \dfrac{x^2}{4D \sigma^2} \biggr)^{-\frac{1}{2}} \Biggr] d\sigma }
	\label{eq:bar4}
\end{equation}

after some helpful manipulations that make it easier to solve Equation~\ref{eq:bar4}, the form of the analytical solution is altered to,

\begin{equation}
  T' - T_0 = \dfrac{2 \left( T_1 - T_0 \right)}{\sqrt{\pi}} {\displaystyle \int\limits_{\dfrac{x}{2 \sqrt{Dt}}}^{\infty} e^{\displaystyle{ \left( \sigma - \frac{x v}{4 D \sigma} \right)^2}} erfc \Biggl[ \biggl( \dfrac{x^2 h' \sqrt{D'}}{8DH \sigma^2} + \dfrac{z}{2 \sqrt{D'}} \biggr) \biggl( t - \dfrac{x^2}{4D \sigma^2} \biggr)^{-\frac{1}{2}} \Biggr] d\sigma },
	\label{eq:bar-alt}
\end{equation}


% brief description of results
\subsection{Example Results}

After 300 years of simulation time, \mf GWE temperatures compare very well to the analytical solution (fig.~\ref{fig:ex-gwe-barends-300yrs}). 

% a figure
\begin{StandardFigure}{
    Temperatures predicted by the \mf GWE model compared to temperatures predicted by an analytical solution after 300 years of simulation time.}
    {fig:ex-gwe-barends-300yrs}{../figures/ex-gwe-barends-300yrs.png}
\end{StandardFigure}





\clearpage
\insection
\section{Thermal Loading of Borehole Heat Exchangers}

% Describe source of problem
Borehole Heat Exchangers (BHE's) are increasingly used to provide a sustainable source of cooling and heating for buildings. The thermal energy demands may vary seasonally, with heat extraction occurring during the winter season and heat injection during the warmer summer months. As a result, the subsurface temperature near the BHE's will show seasonal variability. This is illustrated here using the Groundwater Energy Transport (GWE) model. The results are compared with an analytical solution to verify the accuracy of the \mf simulation. 

\subsection{Example description}

Five BHE's are arbitrarily placed in a confined aquifer with uniform background flow in the x-direction. The BHE's are fully screened across the aquifer thickness and heat injection/extraction is evenly distributed along the borehole length. The resulting temperature field therefore does not vary along the z-axis, effectively rendering the problem two-dimensional. Heat extraction and injection follow a seasonal energy demand typical for residential BHE systems in temperate climates (figure~\ref{fig:ex-gwe-bhe-injection-rates}). During winter, heat is extracted from the aquifer whereas heat is injected (or cold is extracted) during summer. Note that on a yearly basis, more heat is extracted than injected.

\begin{StandardFigure}{
    Heat injection rate per unit aquifer thickness for each BHE.
    }{fig:ex-gwe-bhe-injection-rates}{../figures/ex-gwe-bhe-injection-rates.png}
\end{StandardFigure}                                 

The \mf model consists of a single, confined layer of unit thickness to simulate the 2D system. The domain extent is 80 $m$ $\times$ 80 $m$ and is discretized using rectangular cells of 1 $m$ $\times$ 1 $m$. Flow is steady-state and uniform in the x-direction as implemented by constant-heads at the left and right boundaries. The BHE's are placed in the center of the model using the Energy Source Loading (ESL) package with energy source loading rates varying every two months. This annual energy loading cycle is repeated for three years (figure~\ref{fig:ex-gwe-bhe-injection-rates}). Each loading phase is simulated using 10 time steps. The initial background temperature equals 0 $^{\circ} C$, so the simulated temperature field represents the change in temperature. Model parameters are summarized in table~\ref{tab:ex-gwe-bhe-01}.

% add static parameter table(s)
\input{../tables/ex-gwe-bhe-01}

\subsection{Analytical solution}

An analytical solution for the described 2D problem can be derived based on the POINT2 algorithm provided by \cite{wexler1992} (equation 76) describing 2D solute transport for a continuous point source in an aquifer with uniform background flow:

\begin{equation}
    C(x,y,t) = \frac{C_0Q'}{4n\pi\sqrt{D_xD_y}}exp(\frac{v(x-x_c)}{2D_x})\int_0^t\frac{1}{\tau} exp(-\frac{v^2}{4D_x}\tau - \frac{(x-x_c)^2}{4D_x\tau}-\frac{(y-y_c)^2}{4D_y\tau})d\tau
    \label{eq:pointtwo}
\end{equation}

By dividing $v$, $D_x$, $D_y$ and $Q'$ by the retardation coefficient $R$ $[-]$, linear equilibrium sorption can be included.

Using the analogy between the solute transport equation and the heat transport equation (see e.g. \cite{zheng2010mt3dmsv5.3}), equation~\ref{eq:pointtwo} can be used to simulate 2D heat transport from a continuous point source in an aquifer with uniform background flow by transforming the governing heat transport parameters into the solute transport parameters $R$, $D_m$ and $C_0$:

\begin{align}
    k_0 &= n k_w + (1 - n) k_s \label{eq:bhe-k0}\\
    D_m &= \frac{k_0}{n  \rho_w  C_w}  \label{eq:bhe-Dm}\\
    \rho_b &= (1 - n) \rho_s  \label{eq:bhe-rhob}\\
    K_d &= \frac{C_s}{C_w \rho_w}  \label{eq:bhe-KD}\\
    R &= 1 + \frac{K_d \rho_b}{n}  \label{eq:bhe-R}\\
    Q' &= 1  \label{eq:bhe-Q}\\
    C_0 &= \frac{F_0}{\rho_w C_w}  \label{eq:bhe-c0}
\end{align}

where the heat injection rate $F_0$ per unit aquifer thickness $[ET^{-1}L^{-1}]$ is converted to the injection concentration $C_0$ and the injection rate per unit aquifer thickness $Q'$ is set to unity. Since equation~\ref{eq:pointtwo} is linear, the superposition principle can be applied to allow for multiple BHE's in space as well as time-varying energy loading.

% for examples without scenarios
\subsection{Example Results}

Simulated temperature change contours after 1.5 years show that the groundwater temperature has decreased around the BHE field (figure~\ref{fig:ex-gwe-bhe-contours}). This decrease has stretched in the direction of flow as the cooler groundwater is transported downgradient. A good agreement is found between the analytical solution and the \mf solution. A time series of the simulated temperature change at a location downgradient of the BHE field shows the seasonal variation in groundwater temperature (figure~\ref{fig:ex-gwe-bhe-ts}). Since the annual heating demand is larger than the cooling demand, a thermal imbalance of the subsurface occurs during the early stages of operation. This causes the groundwater system to initially cool down before reaching a thermal dynamic equilibrium. As a result, the effectiveness of the residential heating is reduced in the long term as a larger temperature difference now needs to be met to sufficiently heat the building during winter, which needs to be taken into account when dimensioning the system. The \mf results again show a good agreement with the analytical solution.

% a figure
\begin{StandardFigure}{
    Contours of the simulated temperature change after 1.5 years using the \mf GWE model (dashed red line) and the analytical solution (solid black line). The green cross marks the location of the observation well in figure~\ref{fig:ex-gwe-bhe-ts}. The black dots show the locations of the BHE's.
    }{fig:ex-gwe-bhe-contours}{../figures/ex-gwe-bhe-contours.png}
\end{StandardFigure}                                 

\begin{StandardFigure}{
    Time series of the simulated temperature change at the location shown in figure~\ref{fig:ex-gwe-bhe-contours} using the \mf GWE model (dashed red line) and the analytical solution (solid black line). 
    }{fig:ex-gwe-bhe-ts}{../figures/ex-gwe-bhe-ts.png}
\end{StandardFigure}    


% PRT Model examples
\clearpage
\insection
\section{PRT/MP7 Example 1: Particle Tracking on a Structured Grid with Steady-State Flow}

This example demonstrates particle-tracking by reproducing example problem 1 from the MODPATH 7\citep{pollock2016modpath7} example problems document \citep{modpath7examples}. While only the MODFLOW 6 PRT solution is shown below, PRT and MODPATH 7 results have been evaluated for agreement.

\subsection{Example description}

The example first runs a groundwater flow (GWF) model simulating steady-state flow on a structured grid. The flow system includes an upper and lower aquifer separated by a confining layer with lower conductivity. The grid has 3 layers, 21 rows, and 20 columns, with square cells 500 feet to a side. The system includes two boundary conditions: a well in layer 3, row 11, column 10, and a river in layer 1, column 20~(fig~\ref{fig:ex-prt-mp7-p01-config}). Model parameters for this example are summarized in table~\ref{tab:ex-prt-mp7-p01-01}.

\begin{StandardFigure}{
    Conceptual model. Image reproduced from the MODPATH 7 examples document \citep{modpath7examples}.
    }{fig:ex-prt-mp7-p01-config}{../images/ex-prt-mp7-p01-config.png}
\end{StandardFigure}

\input{../tables/ex-prt-mp7-p01-01.tex}

\subsection{Example Results}

In this example a MODFLOW 6 particle tracking (PRT) model runs in the same simulation as a groundwater flow (GWF) model~(fig~\ref{fig:ex-prt-mp7-p01-head}), which provides it with intercell flows via a GWF-PRT model exchange.

\begin{StandardFigure}{
    Heads simulated by the MODFLOW 6 groundwater flow (GWF) model.
    }{fig:ex-prt-mp7-p01-head}{../figures/mp7-p01-head.png}
\end{StandardFigure}

In subproblem 1A, a line of 21 particles is placed at the water table in layer 1 for column 3, rows 1 through 21. In subproblem 1B, a denser release configuration is used which places a 3 x 3 array of particles on the top face of every cell in layer 1. Both simulations track particles forward to their discharge points.

Subproblem 1A path points on a 1000-day time interval are visualized in 2D~(fig~\ref{fig:ex-prt-mp7-p01-paths-layer}) and 3D~(fig~\ref{fig:ex-prt-mp7-p01-paths-3d}).

\begin{StandardFigure}{
    Particle pathlines and points (1A), 1000-day interval, colored by layer, overhead map view.
    }{fig:ex-prt-mp7-p01-paths-layer}{../figures/mp7-p01-paths-layer.png}
\end{StandardFigure}

\begin{StandardFigure}{
    Particle path points (1A), 1000-day interval, colored by layer, three-dimensional perspective.
    }{fig:ex-prt-mp7-p01-paths-3d}{../figures/mp7-p01-paths-3d.pdf}
\end{StandardFigure}

To illustrate discharge points, pathlines are colored by discharge area (well or river) in fig~\ref{fig:ex-prt-mp7-p01-paths}. To show capture areas, starting locations of all particles are color-coded according to the zone value of the cells in which they terminate in fig~\ref{fig:ex-prt-mp7-p01-rel-destination}. Travel time analysis is also a common use case for particle tracking. Particle release points are colored by total travel time to capture in fig~\ref{fig:ex-prt-mp7-p01-rel-travel-time}.

\begin{StandardFigure}{
    Particle pathlines, colored by destination: particles with red pathlines are captured by the well, particles with blue pathlines are captured by the river.
    }{fig:ex-prt-mp7-p01-paths}{../figures/mp7-p01-paths.png}
\end{StandardFigure}

\begin{StandardFigure}{
    Particle release points, colored by destination.
    }{fig:ex-prt-mp7-p01-rel-destination}{../figures/mp7-p01-rel-destination.png}
\end{StandardFigure}

\begin{StandardFigure}{
    Particle release points, colored by travel time.
    }{fig:ex-prt-mp7-p01-rel-travel-time}{../figures/mp7-p01-rel-travel-time.png}
\end{StandardFigure}

\clearpage
\insection
%\section{PRT/MP7 Example 2: Particle Tracking on an Unstructured Grid with Steady-State Flow}

This example demonstrates a MODFLOW 6 particle tracking (PRT) model by reproducing example problem 2 from the MODPATH 7 \citep{pollock2016modpath7} example problems document \citep{modpath7examples}. An equivalent MODPATH 7 model is constructed for comparison, though only PRT results are shown.

\subsection{Example description}

PRT/MP7 Example 2 modifies the flow system from PRT/MP7 Example 1 with a quad-refined unstructured grid. The region near the well is refined three levels. This example also employs a backwards tracking analysis, in which groundwater flows are reversed with FloPy before providing them to PRT to track particle trajectories in reverse. The system includes the same well and river boundary conditions as in example 1. Model parameters for this example are summarized in table~\ref{tab:ex-prt-mp7-p02-01}.

\input{../tables/ex-prt-mp7-p02-01.tex}

\subsection{Example Results}

In this example a MODFLOW 6 particle tracking (PRT) model runs in a separate simulation from the groundwater flow (GWF) model. Intercell flows are read by the PRT model from the binary budget file written by the GWF model.

Subproblem 2A configures 16 particles for release on the horizontal faces of the cell that contains the well. Subproblem 2B configures 100 particles for release on the horizontal faces of the cell that contains the well, with another 16 particles released from the top of the cell.

Subproblem 2A pathlines and points on a 1000-day time interval are visualized in plan view in fig~\ref{fig:ex-prt-mp7-p02-paths}. Subproblem 2A pathlines and 2B recharge points are shown in 3D in fig~\ref{fig:ex-prt-mp7-p02-paths-3d}. Subproblem 2B recharge points are colored by travel time in fig~\ref{fig:ex-prt-mp7-p02-endpts}.

\begin{StandardFigure}{
	Particle pathlines and 1000-day points for subproblem 2A. Points are colored by layer.
}
	{fig:ex-prt-mp7-p02-paths}
	{../figures/mp7-p02-paths.png}
\end{StandardFigure}

\begin{StandardFigure}{
	Three-dimensional perspective of pathlines and 1000-day points for subproblem 2A, and recharge points for subproblem 2B. Points are colored by layer.
}
	{fig:ex-prt-mp7-p02-paths-3d}
	{../figures/mp7-p02-paths-3d.png}
\end{StandardFigure}

\begin{StandardFigure}{
	Particle recharge points for subproblem 2B, colored by travel time.
}
	{fig:ex-prt-mp7-p02-endpts}
	{../figures/mp7-p02-endpts.png}
\end{StandardFigure}

\clearpage
\insection
\section{Forward Particle Tracking, Structured Grid, Transient Flow}

This example demonstrates a MODFLOW 6 particle tracking (PRT) model by reproducing example problem 3 from the MODPATH 7 \citep{pollock2016modpath7} example problems document \citep{modpath7examples}. An equivalent MODPATH 7 model is constructed for comparison, though only PRT results are shown.

\subsection{Example description}

PRT/MP7 Example 3 modifies the flow system from PRT/MP7 Example 1 with three stress periods: first a steady-state period with a single time step, length 100,000 days, then a transient period with 10 time steps, each with length 36,500 days, and lastly a steady-state period with a single time step lasting 100,000 days.

Boundary conditions are also modified in this example. There is not one but two wells, one in the first layer and one in the third layer. There is also a drain in the first layer, extending from roughly the center of the grid to the river on the grid's right boundary. Both wells are inactive for the first stress period, then begin to pump as the 2nd stress period begins (after 100,000 days), and continue to pump at a constant rate for the rest of the simulation. Model parameters for this example are summarized in table~\ref{tab:ex-prt-mp7-p03-01}.

Particles are released in batches from a 2x2-cell square (4 total cells) in the upper left quadrant of the grid. Ten batches are released in total: the first batch is released at 90,000 days, after which batches are released every 20 days until 200 days have elapsed.

\input{../tables/ex-prt-mp7-p03-01.tex}

\subsection{Example Results}

In this example a MODFLOW 6 particle tracking (PRT) model runs in a separate simulation from the groundwater flow (GWF) model~(fig~\ref{fig:ex-prt-mp7-p03-head}). Intercell flows are read by the PRT model from the binary budget file written by the GWF model.

Path points on a 2000-day interval are visualized in plan view in fig~\ref{fig:ex-prt-mp7-p03-paths-layer} and in 3D in fig~\ref{fig:ex-prt-mp7-p03-paths-3d}. Release and termination points are colored by destination in fig~\ref{fig:ex-prt-mp7-p03-rel-term}.

\begin{StandardFigure}{
    Head simulated by the MODFLOW 6 groundwater flow (GWF) model.
    }{fig:ex-prt-mp7-p03-head}{../figures/ex-prt-mp7-p03-head.png}
\end{StandardFigure}

\begin{StandardFigure}{
    2000-day particle path points. Points are colored by travel time.
    }{fig:ex-prt-mp7-p03-paths-layer}{../figures/ex-prt-mp7-p03-paths-layer.png}
\end{StandardFigure}

\begin{StandardFigure}{
    Three-dimensional perspective of pathlines and 2000-day points. Points are colored by destination.
    }{fig:ex-prt-mp7-p03-paths-3d}{../figures/ex-prt-mp7-p03-paths-3d.png}
\end{StandardFigure}

\begin{StandardFigure}{
    Particle release and termination points, colored by destination.
    }{fig:ex-prt-mp7-p03-rel-term}{../figures/ex-prt-mp7-p03-rel-term.png}
\end{StandardFigure}

\clearpage
\insection
%\section{Backward Particle Tracking, Refined Grid, Lateral Flow Boundaries}

This example demonstrates a MODFLOW 6 particle tracking (PRT) model by reproducing example problem 4 from the MODPATH 7 \citep{pollock2016modpath7} example problems document \citep{modpath7examples}. An equivalent MODPATH 7 model is constructed for comparison, though only PRT results are shown.

\subsection{Example description}

PRT/MP7 Example 4 involves steady-state flow on an unstructured quad-refined grid. The grid has a large, irregular inactive region around its boundary. The left side of the active region is lined with injection wells to represent boundary flows. There is a quad-refined region in the upper left quadrant of the grid, in which two pumping wells are located. Particles are released from two areas: around the horizontal faces of both pumping well cells, and from the left faces of a constant-head boundary along the active region on the right border of the grid. Particles are then tracked backwards to terminating locations along the left border of the grid. Like PRT/MP7 Example 2, groundwater flows are reversed with FloPy before providing them to PRT for backwards tracking. Model parameters for this example are summarized in table~\ref{tab:ex-prt-mp7-p04-01}.

\input{../tables/ex-prt-mp7-p04-01.tex}

\subsection{Example Results}

In this example a MODFLOW 6 particle tracking (PRT) model runs in a separate simulation from the groundwater flow (GWF) model. Intercell flows are read by the PRT model from the binary budget file written by the GWF model. The model grid and boundary conditions are shown in fig~\ref{fig:ex-prt-mp7-p04-grid}.

\begin{StandardFigure}{
    Model grid and boundary conditions, with close-up view of the refined region and the pumping wells.
    }{fig:ex-prt-mp7-p04-grid}{../figures/ex-prt-mp7-p04-grid.png}
\end{StandardFigure}

Heads and pathlines are shown in fig~\ref{fig:ex-prt-mp7-p04-paths}.

\begin{StandardFigure}{
    Heads from the flow model, with particle tracking pathlines superimposed.
    }{fig:ex-prt-mp7-p04-paths}{../figures/ex-prt-mp7-p04-paths.png}
\end{StandardFigure}

% Hybrid examples
\clearpage
\insection
\section{Thermal Profile along a Particle Path}

% Describe source of problem
This example demonstrates the simultaneous use of three different model types in \mf: (1) groundwater flow (GWF), (2) groundwater energy transport (GWE), and (3) particle transport (PRT).  Both GWF and GWE solve for steady flow and temperature conditions, respectively. PRT is used to route two particles through the steady flow and heat transport fields and once informed of the particle locations through time, each x-y location is mapped to a temperature that is then plotted.

\subsection{Example description}

A Voronoi model grid is simulated with the discretization by vertices package (DISV).  The simulation domain is 2,000 $m$ by 1,000 $m$ and is used by all three model types mentioned above.  In the flow model, arbitrarily configured specified head boundaries around the perimeter of the model are as follows: the left-hand boundary is set to 2.0 m, the bottom boundary cells are set to 1.8 m, and the right-hand boundary is set to 1.0 m.  Two extraction wells are located in the model domain with the host cells shaded red ({fig.~\ref{fig:ex-gwe-prt}}).  One is located near the lowerleft of the domain and the second extraction well is located just right of the center of the simulation domain.  One injection well is located near the upper right portion of the active domain with the host cell also shaded red ({fig.~\ref{fig:ex-gwe-prt}}). 

The GWE model uses a standard set of thermal parameters listed in table~\ref{tab:ex-gwe-prt-01}.  A thermal gradient along the left-hand boundary and is setup such water entering the domain in the lower left corner starts at 0.0 $^{\circ}C$ with a linear gradient increasing to 100.0 $^{\circ}C$ in the uppper-left corner of the simulation domain.  Flow enters the bottom boundary at 80.0 $^{\circ}C$.  The injection well, or the furthest red shaded cell in figure~\ref{fig:ex-gwe-prt}, injects water at 80.0 $^{\circ}C$.

PRT routes the two particles from their entry point on the left side of the model domain to where they exit along the right-hand boundary.

% add static parameter table(s)
\input{../tables/ex-gwe-prt-01}

% for examples without scenarios
\subsection{Example Results}

Results from the GWE and PRT model runs are shown in figure~\ref{fig:ex-gwe-prt}. Blue dashed lines show each particles path through the simulation domain, with both particles moving from left to right through the domain.  The total travel distance of each particle is different and as a result the total travel times also are different.  Particle 2 (fig.~\ref{fig:ex-gwe-prt}) reaches a temporary stagnation point to the left of the injection well but eventually moves toward the lower boundary before being swept toward the right boundary where it eventually exits the simulation domain.  The temperature experienced by each respective particle along its path is generally one of warming, both with respect to its x-position (fig.~\ref{fig:ex-gwe-prt}; middle plot) or time (fig.~\ref{fig:ex-gwe-prt}; bottom plot)

% a results figure
\begin{StandardFigure}{
   (Top) Model grid color-filled with the steady-state temperature field and specific discharge arrows.  Dashed blue lines show the paths of particles 1 and 2 and they traverse the simulation domain. (Middle) The x-position of each particle is plotted against its temperature as it traverses the simulation domain. (Bottom) The travel time of each particle is plotted against its temperature as it traverses the simulation domain.}{fig:ex-gwe-prt}{../figures/ex-gwe-prt.png}
\end{StandardFigure}


%MODFLOW API GWT examples


% -------------------------------------------------
\clearpage
\inreferences
\bibliography{./mf6examples}
\bibliographystyle{usgs.bst}



\end{document}
