\section{Viscosity}

% Describe source of problem
In hydrogeologic settings where variations in viscosity significantly impact groundwater flows, the viscosity package can be activated to scale intercell conductance in response to temperature or concentration variations.  Adjustment of intercell conductance may, in turn, have important implications for solute transport or geothermal model applications.  For example, a migrating contaminant plume may be delayed, sped-up, or warped in settings where significant variation in groundwater temperature persists within the study site.

\subsection{Example description}

Application of the viscosity (VSC) package is demonstrated with a simple two-dimensional (2D) test problem.  The 2D demonstration model is adapted from test case 3 in \cite{zheng1999mt3dms} titled ``Two-dimensional transport in a uniform flow field''.  Specified heads drive groundwater flow through a confined system from left to right.  No flow boundaries are specified along the other model edges.  Water injected into the middle of the model domain enters with a specified concentration of 1,000 $mg/L$.  In the original problem, the simulated development of a contaminant plume was compared to an established analytical solution \citep{wilson1978} to confirm that the advective and dispersive spread of the contaminant was accurately simulated.

For this demonstration, the original groundwater flow (GWF) and groundwater solute transport (GWT) models remain unchanged and are referred to as the ``baseline'' GWF and GWT models hereafter (fig.~\ref{fig:vsc-test-setup}).  However, an additional 3 models are added to the \mf simulation to explore the importance of viscosity effects on the contaminant plume.  This results in five models in a single \mf simulation.  The three additional models include one each of a GWF, GWT, and groundwater energy transport (GWE) model. The only difference between the baseline GWF model and the ``viscosity GWF'' model (fig.~\ref{fig:vsc-test-setup}) is the activation of the VSC package that accounts for the groundwater temperatures calculated by the GWE model included in the simulation.

% a figure
\begin{StandardFigure}{
    The \mf simulation setup includes two GWF models (blue rectangles), two GWT models (green rectangles), and one GWE model (orange rectangle). Solid arrows highlight the exchange of information among the models included in the simulation.  Dashed arrows show the generation of output used to highlight the effects of viscosity on a solute transport model.}
    {fig:vsc-test-setup}{../images/vsc-test-setup.png}
\end{StandardFigure}

Moreover, for this demonstration, two \mf simulations are run that vary the magnitude of the temperature gradient between the upper and lower model boundaries.  In the first scenario, a 30$^{\circ}C$ difference is simulated between the upper and lower boundary (table~\ref{tab:ex-gwe-vsc-scenario}).  A 90$^{\circ}C$ gradient exists in the second \mf simulation.  The initial temperatures in either simulation reflect a linear gradient between the upper and lower boundary temperatures (table~\ref{tab:ex-gwe-vsc-scenario}).  Water entering the left side of the model domain enters at a temperature consistent with the initial temperature setup (i.e., varies linearly between the specified temperatures of the upper and lower boundaries).

% add scenario parameter table
\input{../tables/ex-gwe-vsc-scenario}

Inputs to the VSC package, including parameters for the nonlinear viscosity formulation, are listed in table ~\ref{tab:ex-gwe-vsc-01}.  Within \mf, specifically the VSC package (within the GWF model), pointers to the simulated groundwater temperature enable the VSC package to modify intercell conductance calculated by the node-property flow (NPF) package.  Altered intercell conductance in turn affects the flow solution used by the viscosity-affected GWT model (fig.~\ref{fig:vsc-test-setup}) that would otherwise be the same as the baseline GWT model. 


% add static parameter table
\input{../tables/ex-gwe-vsc-01}

% brief description of results
\subsection{Example Results}

For the scenario with a 30$^{\circ}C$ temperature gradient, output from the baseline and viscosity-affected GWT models is collected, differenced, and displayed in figure ~\ref{fig:ex-gwe-vsc-diff-02-30C}. Warm colors show where higher concentrations are simulated by the baseline GWT model.  An alternative way to think about this result is that ignoring the effects of viscosity can lead to an over-prediction of concentrations in portions of the migrating plume.  Conversely, wherever the differenced concentration field is negative, meaning the concentrations in the viscosity-affected model run are greater than in the baseline model run, the calculated concentrations in the no viscosity model may be considered under predicted.  Interestingly, the plume is asymmetric along the direction of flow.  Below y coordinate values of approximately 130 $m$, concentrations in the baseline GWT simulation appear to be over-predicted.  For y coordinate values greater than 130 $m$, there is a mixture of under- and over-prediction of concentrations within the baseline model.  The leading edge of the plume (left side) shows a general over-prediction of the concentration in the baseline GWT model while the down-gradient region of the plume is under-predicted.

In the 90$^{\circ}C$ temperature gradient scenario, results are markedly different relative to the 30$^{\circ}C$ temperature gradient scenario (fig.~\ref{fig:ex-gwe-vsc-diff-02-90C}.  First, the relative position of the under- and over-predicted concentrations is reversed in the 90$^{\circ}C$ temperature gradient scenario compared to the 30$^{\circ}C$ temperature gradient scenario.  That is, the region of under-predicted concentrations is to the left of the over-predicted region in the 90$^{\circ}C$ temperature gradient scenario.  Second, the down-gradient portion of the plume bends to the right relative to the direction of flow.  Third, the extent and magnitude of the spread of the plume is greater in the 90$^{\circ}C$ temperature gradient scenario owing to the higher temperatures that reduce viscosity and lead to higher conductance in the associated GWF model.

In summary, settings with stark contrasts in groundwater temperature may also have significant variations in the flow field as a result of viscosity effects.  Thus, activation of the viscosity package may appropriately adjust the intercell conductance values in response to viscosity and reduce compansatory adjustment of the hydraulic conductivity field during a calibration routine, for example.


% a results figure
\begin{StandardFigure}{
   The difference between a flow-transport model setup that ignores the effects of viscosity (baseline simulation) on a migrating plume and a model setup that simulates a 30$^{\circ}C$ temperature gradient in the groundwater to demonstrate the effect of viscosity on the migrating plume.  Differences are calculated after 365 days of simulation time.  The difference is calculated by subtracting the 2D concentration field calculated by the model that accounts for viscosity from the model run that ignores viscosity.}{fig:ex-gwe-vsc-diff-02-30C}{../figures/ex-gwe-vsc-diff-02-30C.png}
\end{StandardFigure}

% a 2nd results figure
\begin{StandardFigure}{
   Similar to figure ~\ref{fig:ex-gwe-vsc-diff-02-30C}, the difference between two concentration fields calculated by two closely related model runs.  In one model setup, the effect of viscosity is ignored whereas the other model setup simulates a 90$^{\circ}C$ temperature gradient within the model domain.  As with the 30$^{\circ}C$ temperature gradient, differences are calculated after 365 days of simulation time.}{fig:ex-gwe-vsc-diff-02-90C}{../figures/ex-gwe-vsc-diff-02-90C.png}
\end{StandardFigure}
