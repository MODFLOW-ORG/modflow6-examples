\section{Two-Dimensional Transport in a Diagonal Flow Field (MT3DMS Example Problem 4)}

The third demonstrated MT3DMS-\mf transport comparson is for a two-dimensional transport in a diagonal flow field. This problem is similar to the preceding problem with two important changes. First, the flow direction is now oriented at a 45-degree angle relative to rows and columns of the numerical grid. Owing the use of MT3DMS for comparison, the \mf solution uses the traditional DIS package (not DISU or DISV).  The second notable change is that the number of rows and columns has been expanded in order to accomodate a longer simulation period of 1,000 days. Because of the orientation of the flow field relative to the model grid, and the sharpness of the migrating concentration front, this test problem presents a challenging set of conditions to simulate. Three scenarios test alternative advection formulations, as summarized in table~\ref{tab:ex-gwt-mt3dms-p04-scenario}

% add scenario table
\input{../tables/ex-gwt-mt3dms-p04-scenario}

Model parameter values for this problem are provided in table~\ref{tab:ex-gwt-mt3dms-p04-01}.

% add 2nd static parameter value table
\input{../tables/ex-gwt-mt3dms-p04-01}

The same analytical solution used in the previous problem can be used for this problem after applying the necessary updates to select parameters - most noteably the dispersion and porosity terms and that an inter-model comparison is drawn after 1,000 days instead of 365 days. Figure 36 in \cite{zheng1999mt3dms} shows four different solutions for this problem: (1) analytical, (2) Method of Characteristics (MOC), (3) upstream finite difference (FD), and (4) Total Variation Dimishing (TVD) or ``ULTIMATE'' scheme.  Both the MOC and TVD solutions demonstrate a reasonable agreement with the analytical solution. However, the upstream finite difference solution reflects considerably more spread from simulation of too much dispersion - in this case numerical dispersion instead of hydrodynamic dispersion. 

The \mf transport solution is compared to all three numerical solutions (FD, TVD, and MOC) presented in \cite{zheng1999mt3dms}. The first comparison shows complete agreement between MT3DMS and the \mf transport solution when the finite difference approach is applied (figure~\ref{fig:ex-gwt-mt3dms-p04a}). 

% MT3DMS manual figure 36a
\begin{StandardFigure}
	{Comparison of the MT3DMS and \mf numerical solutions for two-dimensional transport in a diagonal flow field. Both models are using their respective finite difference solutions without the TVD option} 
	{fig:ex-gwt-mt3dms-p04a}
	{../figures/ex-gwt-mt3dms-p04a.png}
\end{StandardFigure}

Figure~\ref{fig:ex-gwt-mt3dms-p04b} shows a comparison between the MT3DMS and \mf solution with the respective TVD options for each model activated.  Owing to the fact that MT3DMS uses a third-order TVD scheme while \mf uses a second-order scheme, differences between the two solutions are expected. 

% MT3DMS manual figure 36b
\begin{StandardFigure}
	{Comparison of the MT3DMS and \mf numerical solutions for two-dimensional transport in a diagonal flow field. Both models are using their respective finite difference solutions with the use of the TVD option, which serves as the main difference with results displayed in figure~\ref{fig:ex-gwt-mt3dms-p04b}} 
	{fig:ex-gwt-mt3dms-p04b}
	{../figures/ex-gwt-mt3dms-p04b.png}
\end{StandardFigure}

The third model comparison shows the largest difference between the two solutions (figure~\ref{fig:ex-gwt-mt3dms-p04c}). Because the MOC solution is the closest facsimile of the analytical solution, comparison of \mf with the MT3DMS MOC solution is as close to a comparison with the analytical solution as will be shown for the current set of model runs. 

% MT3DMS manual figure 36c
\begin{StandardFigure}
	{Comparison of the MT3DMS and \mf numerical solutions for two-dimensional transport in a diagonal flow field. Here, MT3DMS is using a MOC technique to find a solution while \mf uses finite difference without TVD activated.} 
	{fig:ex-gwt-mt3dms-p04c}
	{../figures/ex-gwt-mt3dms-p04c.png}
\end{StandardFigure}



