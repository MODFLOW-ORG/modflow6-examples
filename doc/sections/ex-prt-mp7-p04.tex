\section{PRT/MP7 Example 4: Backward Tracking on an Unstructured Quadpatch Grid}

This example demonstrates a MODFLOW 6 particle tracking (PRT) model by reproducing example problem 4 from the MODPATH 7 \citep{pollock2016modpath7} example problems document \citep{modpath7examples}. An equivalent MODPATH 7 model is constructed for comparison, though only PRT results are shown.

\subsection{Example description}

PRT/MP7 Example 4 involves steady-state flow on an unstructured quad-refined grid. The grid has a large, irregular inactive region around its boundary. The left side of the active region is lined with injection wells to represent boundary flows. There is a quad-refined region in the upper left quadrant of the grid, in which two pumping wells are located. Particles are released from two areas: around the horizontal faces of both pumping well cells, and from the left faces of a constant head boundary along the active region on the right border of the grid. Particles are then tracked backwards to terminating locations along the left border of the grid. Model parameters for this example are summarized in table~\ref{tab:ex-prt-mp7-p04-01}.

\input{../tables/ex-prt-mp7-p04-01.tex}

\subsection{Example Results}

In this example a MODFLOW 6 particle tracking (PRT) model runs in a separate simulation from the groundwater flow (GWF) model. Intercell flows are read by the PRT model from the binary budget file written by the GWF model. The model grid and boundary conditions are shown in fig~\ref{fig:ex-prt-mp7-p04-grid}.

\begin{StandardFigure}{
    Model grid and boundary conditions, with close-up view of the refined region and the pumping wells.
    }{fig:ex-prt-mp7-p04-grid}{../figures/mp7-p04-grid.png}
\end{StandardFigure}

Heads and pathlines are shown in fig~\ref{fig:ex-prt-mp7-p04-paths}.

\begin{StandardFigure}{
    Heads from the flow model, with particle tracking pathlines superimposed.
    }{fig:ex-prt-mp7-p04-paths}{../figures/mp7-p04-paths.png}
\end{StandardFigure}