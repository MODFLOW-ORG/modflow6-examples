\section{Two-Dimensional Transport in a Radial Flow Field (MT3DMS Example Problem 5)}

The next example problem tests two-dimensional transport in a radial flow field.  The radial flow field is established by injecting water in the center of the model domain (row 16, column 16) and allowing it to flow outward toward the model perimeter. No regional groundwater flow gradient exists as in some of the previous comparisons with MT3DMS. Constant head cells located around the perimeter of the model drain water and solute from the simulation domain. Solute enters the model domain through the injection well with a unit concentration.  The starting concentration is zero across the entire domain. Flow remains steady and confined throughout the 27 day simulation period. The aquifer is homogenous, isotropic, and boundaries are sufficiently far from the injection well to avoid solute reaching the boundary during the simulation interval. Table~\ref{tab:ex-gwt-mt3dms-p0501} summarizes many of the model inputs:

% add 2nd static parameter value table
\begin{StandardTable}
	{Hydraulic and transport properties used in the two-dimensional transport in a radial flow field problem.  From \cite{zheng1999mt3dms}}
	{tab:ex-gwt-mt3dms-p0501}
	{../tables/ex-gwt-mt3dms-p0501}
\end{StandardTable}

An analytical solution for this problem was originally given in \cite{moench1981}. The MT3DMS solution with the TVD option activated most closely matched the analytical solution.  Therefore the TVD option is activated in both MT3DMS and \mf for verifying the transport solution. Figure \ref{fig:ex-gwt-mt3d-p05-xsec} shows a slight under simulation of the outward spread of solute in the \mf solution compared to MT3DMS. Figure \ref{fig:ex-gwt-mt3d-p05-planView} shows close agreement among the MT3DMS and \mf isoconcentration contours with the TVD advection scheme activated. 

% This figure doesn't exist in the MT3DMS manual
\begin{StandardFigure}
	{Comparison of the MT3DMS and \mf numerical solutions for a point source in a two-dimensional radial flow field simulation.  The thick black line in figure~\ref{fig:ex-gwt-mt3d-p05-planView} shows the location of this profile view of concentrations.  The analytical solution for this problem was originally given in \citep{moench1981} and is not shown here} 
	{fig:ex-gwt-mt3d-p05-xsec}
	{../figures/ex-gwt-mt3d-p05-xsec}
\end{StandardFigure}

% MT3DMS manual figure 37
\begin{StandardFigure}
	{Comparison of the MT3DMS and \mf numerical solutions for two-dimensional transport in a radial flow field.  The thick black line shows the location of the concentration profile shown in figure~\ref{fig:ex-gwt-mt3d-p05-xsec}. Both models are using their respective finite difference solutions with the use of the TVD option.} 
	{fig:ex-gwt-mt3d-p05-planView}
	{../figures/ex-gwt-mt3d-p05-planView}
\end{StandardFigure}
