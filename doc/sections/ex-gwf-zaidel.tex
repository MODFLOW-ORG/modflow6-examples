\section{Zaidel Problem}

% Describe source of problem
One of the most challenging numerical cases for \MF arises from drying-rewetting problems often associated with abrupt changes in the elevations of impervious base of a thin unconfined aquifer. This problem simulates a discontinuous water table configuration over a stairway impervious base and flow between constant-head boundaries at the left and right sides of the model domain. This problem is based on the problems that compared the analytical solution of \cite{zaidel2013discontinuous}~to \mfn~\cite[see][figure~6]{zaidel2013discontinuous}.

\subsection{Example Description}
% spatial discretization  and temporal discretization
Model parameters for the example are summarized in table~\ref{tab:ex-gwf-zaidel-01}. The model consists of a grid of 200 columns, 1 row, and 1 layer and a bottom altitude of ranging from 20 to 0 m (fig.~\ref{fig:ex-gwf-zaidel-01}). The discretization is 5 m in the row direction and 1 m in the column direction for all cells. A single steady-stress period with a total length of 1 day is simulated.

% add static parameter table(s)
\input{../tables/ex-gwf-zaidel-01}

A constant horizontal hydraulic conductivity of 0.0001 $m/d$ was specified in all cells. An initial head of 23 $m$ was specified in all model cells. Constant head boundary cells were specified in column 1 and 200. The constant head value in column 1 is 23 $m$ and was used in all simulations. A constant head value of 1 and 10 $m$ was specified in column 200 based on the values evaluated by \cite{zaidel2013discontinuous}.


\begin{StandardFigure}{
                                     Discontinuous water table configuration over a multistep impervious base.
                                     Simulated results for the case where the constant head in column 200 is equal
                                     to 1 meter is shown. The impervious model base and the location of constant
                                     head boundary cells is also shown.
                                     }{fig:ex-gwf-zaidel-01}{../figures/ex-gwf-zaidel-01.png}
\end{StandardFigure}                                 


% for examples without scenarios
\subsection{Example Results}

Simulated results for the case with the constant head in column 200 equal to 1 $m$ and 10 $m$ are shown in figures~\ref{fig:ex-gwf-zaidel-01} and~\ref{fig:ex-gwf-zaidel-02}, respectively. Simulated results compare well with the results in \cite{zaidel2013discontinuous}.

% a figure
\begin{StandardFigure}{
                                     Discontinuous water table configuration over a multistep impervious base.
                                     Simulated results for the case where the constant head in column 200 is equal
                                     to 10 meters is shown. The impervious model base and the location of constant
                                     head boundary cells is also shown.
                                     }{fig:ex-gwf-zaidel-02}{../figures/ex-gwf-zaidel-02.png}
\end{StandardFigure}                                 

