\section{MT3DMS Problem 6}

In this example problem, concentrations are compared between \mf and MT3D-USGS at an injection/extraction well. The well is fully penetrating in a confined aquifer and injects contaminated water for a period of 2.5 years at a rate of 1 $ft^3/sec$.  At the end of 2.5 years, the injection well is reversed and begins pumping (extracting) contaminated groundwater for a period of 7.5 years, also at the rate of 1 $ft^3/sec$.  \cite{elkadi1988} was the first to develop the test problem which was later used by \cite{zheng1993} to test ongoing method-of-characteristics (MOC) developments.  The model boundary is placed far enough away from the injection/extraction well to ensure no solute exits the model domain during the injection period.  Moreover, steady flow conditions are reached immediately during the injection and extraction stress periods.  Problem specifics are provided in table~\ref{tab:ex-gwt-mt3dms-p06-01}.

% add 2nd static parameter value table
\input{../tables/ex-gwt-mt3dms-p06-01}

An analytical solution for this problem was originally given in \cite{gelhar1971}.  Because this is an advection dominated problem, both numerical solutions invoke their TVD schemes.  The \mf solution shows a quicker rise in concentration at the well site than does the MT3D-USGS solution (figure~\ref{fig:ex-gwt-mt3dms-p06}). 

% MT3DMS manual figure 38
\begin{StandardFigure}
	{Comparison of the MT3D-USGS and \mf numerical solutions at an injection/extraction well. The analytical solution for this problem was originally given in \citep{gelhar1971} and is not shown here} 
	{fig:ex-gwt-mt3dms-p06}
	{../figures/ex-gwt-mt3dms-p06.png}
\end{StandardFigure}

