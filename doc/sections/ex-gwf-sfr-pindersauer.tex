\section{Modified Pinder-Sauer surface water/groundwater exchange problem}

% Describe source of problem
The modified Pinder-Sauer is a surface-water/groundwater exchange problem that is based on the problem of \cite{pinder1971numerical}, which has been used as a benchmark test for coupled surface-water/groundwater models \citep[\textit{e.g.,}][]{hughes2015modflow,swain1996coupled}. \cite{lal2001modification} modified the problem of \cite{pinder1971numerical} to make it easier to set up and to include a sinusoidal inflow hydrograph boundary condition at the upstream end of the surface-water system.

The analytical solution of \cite{lal2001modification} for the discharge at a distance $x$ from the upstream boundary is

\begin{equation} \label{lalanaleqn}
Q_a = 509.70 + 141.58 \exp \left( \frac{\hat{\lambda}_{1} x}{\Lambda} \right) \sin \left( f_{r} t + \frac{\hat{\lambda}_{2} x}{\Lambda} \right),
\end{equation}

\noindent where $f_r$ is the characteristic frequency of the system, $\hat{\lambda}_{1}$ is the amplitude decay constant, $\Lambda$ is the characteristic length related to the wave number of the water-level disturbance, and $\hat{\lambda}_{2}$ is a dimensionless wave number. The terms in \textbf{equation~\ref{lalanaleqn}} are calculated using model parameters such as  friction slope, reach sediment hydraulic conductivity, reach width, \textit{etc.} and are defined in \cite{lal2001modification}. For the case without aquifer exchange, $\hat{\lambda}_{1} = -4.779 \times 10^{-2}$, and $\hat{\lambda}_{2} = -0.3608$. With aquifer exchange, the appropriate values for the variables in \textbf{equation~\ref{lalanaleqn}} are $\hat{\lambda}_{1} = -0.1785$, $\Lambda = 4894.3$ m, $f_{r} = 3.49 \times 10^{-4}$ sec$^{-1}$, $\hat{\lambda}_{2} = -0.3409$, and all other variables are the same as the case without aquifer exchange. 

The model domain represents a flood plane that is 39,624 m long, 427 m across the valley, and underlain by an unconfined aquifer. The flood plane and underlying aquifer are surrounded by impermeable boundaries on all sides. The base of the aquifer is horizontal and specified to be at an elevation of 0.0 m. The flow direction in the model domain is along the long axis of the model from top to bottom.

A total of 65 rows, 15 columns, and 1 layer were used to discretize the model domain. A constant grid spacing of 609.61 m was used for each row. A grid spacing of 28.30 m was used for all columns except the center column (column 8); the grid spacing of column 8 was 30.48 m. The total simulation length was 24 hours and a constant time-step length of 5 minutes was used.

The hydraulic conductivity of the aquifer is 3.048$\times$10$^{-3}$ m/s. The specific yield and specific storage are 0.25, and 1$\times$10$^{-7}$ 1/s, respectively.
 
% add static parameter table(s)
\input{../tables/ex-gwf-sfr-pindersauer-01}

A single river channel is located at the center of the aquifer (column 8) parallel to the long axis of the model domain and is simulated using the kinematic-wave approximation option available in the Streamflow Routing (SFR) package. The channel has a bed slope of 0.001, a width of 30.48 m, and a Manning roughness coefficient of 0.03858 s/m$^{1/3}$. For the case with aquifer exchange, the leakage coefficient is $1.402 \times 10^{-4}$ sec$^{-1}$ and seepage is assumed to occur only from the bottom.

Initially, the saturated thickness of the aquifer is 67.05 and 27.43 m at the upstream and downstream ends of the aquifer, respectively. The initial reach stage in each reach was calculated using \textbf{equation~\ref{lalanaleqn}}, the bed elevation, and Mannings equation using reach parameters.


% add scenario table
\input{../tables/ex-gwf-sfr-pindersauer-scenario}

The sinusoidal flood hydrograph introduced in the upstream SFR reach is,
\nolinebreak
\begin{equation} \label{lalhydrographeqn}
Q = 509.70 + 141.58 \sin \left( \frac{2 \pi t}{T_p} \right),
\end{equation}
\noindent
where $T_p$ is the period of disturbance (sec.) and $t$ is the simulation time (sec.). A $T_p$ value of 5 hours (18,000 sec.) was used.

% for examples without scenarios
\subsection{Example Results}

Simulated relative stage and discharge results 15,240 m downstream of the top end of the model domain are shown in \textbf{figure~\ref{fig:ex-gwf-sfr-ps-obs}}. Relative stage and discharge results were calculated using the initial stage and discharge of 52.12 m and 509.70 m$^3$/s, respectively. For comparison, simulated relative stage and discharge for a simulation without aquifer exchange (leakage coefficient $= 0.00$ sec$^{-1}$) are also shown in  \textbf{figure~\ref{fig:ex-gwf-sfr-ps-obs}}. Analytical results calculated using \textbf{equation~\ref{lalanaleqn}} 15,240 m downstream of the top end of the model domain are also shown in \textbf{figure~\ref{fig:ex-gwf-sfr-ps-obs}}; SFR results are comparable to the analytical solution.


% a figure
\begin{StandardFigure}
	{(A) Simulated relative stage change 15,240 m downstream of the top end of the model domain with and without aquifer exchange (leakage) for the modified Pinder-Sauer problem. (B) Comparison of relative discharge change 15,240 m downstream of the top end of the model domain and the analytical solution of \cite{lal2001modification}  with and without aquifer exchange (leakage).}
	{fig:ex-gwf-sfr-ps-obs}{../figures/ex-gwf-sfr-pindersauer-observations.png}
\end{StandardFigure}



