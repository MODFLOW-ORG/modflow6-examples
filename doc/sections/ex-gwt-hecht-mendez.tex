\section{Hecht-Mendez 3D Borehole Heat Exchanger Problem}

% Describe source of problem
The models presented in \cite{hechtMendez2010} apply MT3DMS \cite{zheng1999mt3dms} as a heat transport simulator.  Both 2-dimensional and 3-dimensional demonstration problems are presented that explore the use of a ``borehole heat exchanger'' (BHE), a ``closed'' geothermal system that uses a heat pump for cycling water and anti-freeze fluids in pipes for mining heat from an aquifer \cite{diao2004}.  Figure~\ref{fig:hechtMendezBHE} depicts the kind of system modeled in this example.  

Among the examples presented in \cite{hechtMendez2010}, this work recreates the 3D example that simulates heat exchange between a BHE and the aquifer.  Among the suite of examples included in the MODFLOW6-examples.git repo, this is the first one demonstrating the suitability of the GWT model within \mf for simulating saturated zone heat transport.  For completeness, we compare the \mf solution to MT3D-USGS \cite{mt3dusgs} as well as to established analytical solutions.

% a figure
\begin{StandardFigure}{
                                     Example of a closed ground source heat pump extracting heat beneath a private residence.  Image taken from \cite{hecht2008}.
                                     }{fig:hechtMendezBHE}{../images/hechtMendezBHE.png}
\end{StandardFigure}       

\subsection{Example description}

Through appropriate substitution of heat-related transport terms into the groundwater solute transport equation, \mf (as is the case with MT3DMS and MT3D-USGS) may be used as a heat simulator for the saturated zone by observing that the heat transport equation, 

\begin{equation*}
\theta \rho_w C_w \frac{\partial T}{\partial t} + \left( 1 - \theta \right) \rho_s C_s \frac{\partial  T_s}{\partial t} = div \left( \left( k_{T_m} + \theta \rho_w C_w \alpha_h \upsilon_a \right) grad T\right) - div \left( \theta \rho_w C_w \upsilon_a T \right) + q_h
\end{equation*}

has a similar form as the groundwater solute transport equation:

\begin{equation*}
\left[ 1 + \frac{\rho_b K_d}{\theta} \right] \theta \frac{\partial  C}{\partial t}
= div \left[ \theta \left( D_m + \alpha \upsilon_a \right) grad C \right] - div \left( \upsilon_a \theta C \right) + q_s C_s - \lambda \theta C
\end{equation*}

The original analysis included three sub-scenarios for the 3D test case.  The first sub-scenario used a Peclet number of 0, indicating no velocity and therefore explored a purely conductive environment.  Owing to the fact that \citep{hechtMendez2010} did not publish results for this particular sub-scenario, owing to the fact a 3D analytical solution for a purely conductive problem was not available, we do not explore it further.  The two remaining sub-scenarios investigate the effect of groundwater velocity on the heat profile down-gradient of the BHE.  To accomplish this goal, Peclet numbers of 1.0 and 10.0 are explored. A Peclet number of 1.0 indicatex an approximate balance between convective (i.e., advective) and conductive heat transport while a Peclet number of 10.0 represents a convection dominated environment.  The parameters used in each of the two scenarios presented are listed in table~\ref{tab:ex-gwt-hecht-mendez-scenario}.  We note that the sub-scenario names listed in table~\ref{tab:ex-gwt-hecht-mendez-scenario} start with ``...-b'' in light of the fact that the original sub-scenario A is not attempted here.

% add static parameter table(s)
\input{../tables/ex-gwt-hecht-mendez-scenario}

The model grid consists of 83 row, 247 columns, and 13 layers.  The flow model solves a steady, confined, uniform flow field moving from left to right, as shown in figure~\ref{fig:hechtMendezSetup}. No flow boundaries exist along the north, south, top, and bottom planes of the model domain.  The velocity of the groundwater varies in each of the two modeled scenarios [recall that scenario A is omitted, but the scenario naming convension is the same as in \cite{hechtMendez2010}] as given in table~\ref{tab:ex-gwt-hecht-mendez-scenario} by varying the fixed head along the east (right) edge of the simulation domain.  Cell lengths and widths vary across the model domain, beginning with 0.5 x 0.5 $m$ grid cells along the perimeter of the model domain with increased resolution in the vicinity of the BHE (grid cells are gradually refined to 0.1 x 0.1 $m$).  All layers are set to an equal thickness of 1 $m$.  Additional flow related parameter values are provided in table~\ref{tab:ex-gwt-hecht-mendez-01}.  

% add static parameter table(s)
\input{../tables/ex-gwt-hecht-mendez-01}

Because the model is 3D, vertical heat transfer among the layers is considered.  The BHE is vertically centered within the model grid, located in layers 6, 7, and 8 and is represented by a constant mass loading boundary condition.  A negative mass (i.e., heat) loading rate of -60 $W/m$ is constant throughout the simulation period and acts as a heat sink in the aquifer.  Justification for the selected heat removal rate is given in \cite{hechtMendez2010}.  Initially, the entire model domain starts out with a constant temperature of 285.15 $^{\circ}K$ (12 $^{\circ}C$).  Additionally, groundwater enters the west boundary of the model at 285.15 $^{\circ}K$ and leaves at the calculated temperature of the groundwater along the east boundary.  Heat exchange does not occur with the no-flow boundaries to facilitate comparison with analytical solutions. 

% a figure
\begin{StandardFigure}{
                                     Model setup.  Image taken from \cite{hechtMendez2010}.
                                     }{fig:hechtMendezSetup}{../images/hechtMendezSetup.png}
\end{StandardFigure}       

For each of the two scenarios simulated here, two points in time are compared to analytical solutions: 10 and 150 days from the start of the simulation representing transient and steady-state conditions.  Although simulated results continue to change after 150 days, the magnitude of the change is relatively minor.  Thus, 150 days was selected as ``close enough'' in order to help keep total simulation times to a minimum for obtaining the steady-state solution.  Owing to numerical dispersion concerns, the total variation diminishing (TVD) schemes are activated in both \mf and MT3D-USGS. 

The analytical solution for a 3D planar source (or sink) is given in \citep{domenico1985} as,

\begin{equation*}
T \left( x, y, z, t \right) = \frac{T_o}{8} \cdot erfc \left[ \frac{Rx - \upsilon_a t}{2 \sqrt{D_x R t}} \right] \cdot \Bigg\{ erf \left[ \frac{ \left( y + \frac{Y}{2} \right)}{2 \sqrt{D_y \frac{x}{v_a}}} \right] -  erf \left[ \frac{ \left( y - \frac{Y}{2} \right)}{2 \sqrt{D_y \frac{x}{v_a}}} \right] \Bigg\} \cdot \Bigg\{ erf \left[ \frac{ \left( z + \frac{Z}{2} \right)}{2 \sqrt{D_z \frac{x}{v_a}}} \right] -  erf \left[ \frac{ \left( z - \frac{Z}{2} \right)}{2 \sqrt{D_z \frac{x}{v_a}}} \right] \Bigg\}
\end{equation*}

Y and Z are the dimensions of the sink; $D_x$, $D_y$, and $D_z$ are the dispersion coefficients for dispersivities in the x, y, and z directions, corresponding to $\alpha_x$, $\alpha_y$, and $\alpha_z$; and $T_o$ is the starting temperature where the BHE is located and is given by,

\begin{equation*}
T_o = \frac{F}{\upsilon_a \theta \rho_w C_w}
\end{equation*}
 
where F is the energy extraction per area of the source ($W/m^2$).  Along the centerline of the plume, depicted in figure~\ref{fig:hechtMendezAnalytical}, the analytical solution under transient conditions reduces to the following,

% a figure
\begin{StandardFigure}{
                                     (A) Depiction of a heat plume spreading down gradient of a planar source (or sink) for which \citep{domenico1985} gives an analytical solution for (B) two spreading directions in both y and z and finally (C) two spreading directions in y and one spreading direction in z.  Image originally published in \cite{hecht2008}.
                                     }{fig:hechtMendezAnalytical}{../images/hechtMendezAnalytical.png}
\end{StandardFigure}       

\begin{equation*}
T \left( x, t \right) = \frac{T_o}{2} \cdot erfc \left[ \frac{Rx - \upsilon_a t}{2 \sqrt{D_x R t}} \right] \cdot erf \left[ \frac{ Y }{4 \sqrt{D_y \frac{x}{v_a}}} \right] \cdot  erf \left[ \frac{ Z }{4 \sqrt{D_z \frac{x}{v_a}}} \right] 
\end{equation*}
 
For steady-state conditions, the complementary error function ($erfc$) term may be neglected while the $\frac {T_o}{2}$ term reduces to $T_o$, leaving,

\begin{equation*}
T \left( x \right) = T_o erf \left[ \frac{ Y }{4 \sqrt{D_y \frac{x}{v_a}}} \right] \cdot  erf \left[ \frac{ Z }{4 \sqrt{D_z \frac{x}{v_a}}} \right] 
\end{equation*}

Model results are compared to these analytical solutions.  

% for examples without scenarios
\subsection{Example Results}

Simulated temperature profiles downstream of the BHE are compared to steady-state and transient (10 days after the start of the simulation) analytical solutions in figures~\ref{fig:ex-gwt-hecht-mendez-B} (Peclet = 1.0) and~\ref{fig:ex-gwt-hecht-mendez-C} (Peclet = 10.0).  In each scenario, simulated results compare well to the expected (i.e., analytical) temperature profiles, with some undersimulation of the temperature down gradient of the BHE shown in figure~\ref{fig:ex-gwt-hecht-mendez-B} (Peclet number equal to 1.0).  Discrepancies between the simulated and analytical results are smaller in the scenario with a Pectlet number of 10.0.  Moreover, under highly convective flow regimes, the transient solution is approximately equal to the steady state solution after 10 days, as shown in figure~\ref{fig:ex-gwt-hecht-mendez-C}.  These results confirm that \mf may be used as a heat transport simulator in the saturated zone.

% a figure
\begin{StandardFigure}{
                                     \mf simulated temperatures are compared to MT3D-USGS as well as analytical solutions in a steady flow model with a Peclet number equal to 1.0. For this scenario, thermal transport is roughly split between convective and conductive processes.  Temperature profiles are calculated along the centerline down gradient of a planar heat sink as shown in figure~\ref{fig:hechtMendezAnalytical}
                                     }{fig:ex-gwt-hecht-mendez-B}{../figures/ex-gwt-hecht-mendez-B.png}
\end{StandardFigure}                                 

% a figure
\begin{StandardFigure}{
                                     \mf simulated temperatures are compared to MT3D-USGS as well as analytical solutions in a steady flow model with a Peclet number equal to 10.0. For this scenario, thermal transport is dominated by convective transport.  Temperature profiles are calculated along the centerline down gradient of a planar heat sink as shown in figure~\ref{fig:hechtMendezAnalytical}
                                     }{fig:ex-gwt-hecht-mendez-C}{../figures/ex-gwt-hecht-mendez-C.png}
\end{StandardFigure}             
                    
