\section{Forward Particle Tracking, Structured Grid, Transient Flow}

This example demonstrates a MODFLOW 6 particle tracking (PRT) model by reproducing example problem 3 from the MODPATH 7 \citep{pollock2016modpath7} example problems document \citep{modpath7examples}. An equivalent MODPATH 7 model is constructed for comparison, though only PRT results are shown.

\subsection{Example description}

PRT/MP7 Example 3 modifies the flow system from PRT/MP7 Example 1 with three stress periods: first a steady-state period with a single time step, length 100,000 days, then a transient period with 10 time steps, each with length 36,500 days, and lastly a steady-state period with a single time step lasting 100,000 days.

Boundary conditions are also modified in this example. There is not one but two wells, one in the first layer and one in the third layer. There is also a drain in the first layer, extending from roughly the center of the grid to the river on the grid's right boundary. Both wells are inactive for the first stress period, then begin to pump as the 2nd stress period begins (after 100,000 days), and continue to pump at a constant rate for the rest of the simulation. Model parameters for this example are summarized in table~\ref{tab:ex-prt-mp7-p03-01}.

Particles are released in batches from a 2x2-cell square (4 total cells) in the upper left quadrant of the grid. Ten batches are released in total: the first batch is released at 90,000 days, after which batches are released every 20 days until 200 days have elapsed.

\input{../tables/ex-prt-mp7-p03-01.tex}

\subsection{Example Results}

In this example a MODFLOW 6 particle tracking (PRT) model runs in a separate simulation from the groundwater flow (GWF) model~(fig~\ref{fig:ex-prt-mp7-p03-head}). Intercell flows are read by the PRT model from the binary budget file written by the GWF model.

Path points on a 2000-day interval are visualized in plan view in fig~\ref{fig:ex-prt-mp7-p03-paths-layer} and in 3D in fig~\ref{fig:ex-prt-mp7-p03-paths-3d}. Release and termination points are colored by destination in fig~\ref{fig:ex-prt-mp7-p03-rel-term}.

\begin{StandardFigure}{
    Head simulated by the MODFLOW 6 groundwater flow (GWF) model.
    }{fig:ex-prt-mp7-p03-head}{../figures/ex-prt-mp7-p03-head.png}
\end{StandardFigure}

\begin{StandardFigure}{
    2000-day particle path points. Points are colored by travel time.
    }{fig:ex-prt-mp7-p03-paths-layer}{../figures/ex-prt-mp7-p03-paths-layer.png}
\end{StandardFigure}

\begin{StandardFigure}{
    Three-dimensional perspective of pathlines and 2000-day points. Points are colored by destination.
    }{fig:ex-prt-mp7-p03-paths-3d}{../figures/ex-prt-mp7-p03-paths-3d.png}
\end{StandardFigure}

\begin{StandardFigure}{
    Particle release and termination points, colored by destination.
    }{fig:ex-prt-mp7-p03-rel-term}{../figures/ex-prt-mp7-p03-rel-term.png}
\end{StandardFigure}