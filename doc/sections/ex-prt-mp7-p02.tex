\section{Backward Particle Tracking, Quad-Refined Grid, Steady-State Flow}

This example demonstrates a MODFLOW 6 particle tracking (PRT) model by reproducing example problem 2 from the MODPATH 7 \citep{pollock2016modpath7} example problems document \citep{modpath7examples}. An equivalent MODPATH 7 model is constructed for comparison, though only PRT results are shown.

\subsection{Example description}

PRT/MP7 Example 2 modifies the flow system from PRT/MP7 Example 1 with a quad-refined unstructured grid. The region near the well is refined three levels. This example also employs a backwards tracking analysis, in which groundwater flows are reversed with FloPy before providing them to PRT to track particle trajectories in reverse. The system includes the same well and river boundary conditions as in example 1. Model parameters for this example are summarized in table~\ref{tab:ex-prt-mp7-p02-01}.

\input{../tables/ex-prt-mp7-p02-01.tex}

\subsection{Example Results}

In this example a MODFLOW 6 particle tracking (PRT) model runs in a separate simulation from the groundwater flow (GWF) model. Intercell flows are read by the PRT model from the binary budget file written by the GWF model.

Subproblem 2A configures 16 particles for release on the horizontal faces of the cell that contains the well. Subproblem 2B configures 100 particles for release on the horizontal faces of the cell that contains the well, with another 16 particles released from the top of the cell.

Subproblem 2A pathlines and points on a 1000-day time interval are visualized in plan view in fig~\ref{fig:ex-prt-mp7-p02-paths}. Subproblem 2A pathlines and 2B recharge points are shown in 3D in fig~\ref{fig:ex-prt-mp7-p02-paths-3d}. Subproblem 2B recharge points are colored by travel time in fig~\ref{fig:ex-prt-mp7-p02-endpts}.

\begin{StandardFigure}{
	Particle pathlines and 1000-day points for subproblem 2A. Points are colored by layer.
}
	{fig:ex-prt-mp7-p02-paths}
	{../figures/mp7-p02-paths.png}
\end{StandardFigure}

\begin{StandardFigure}{
	Three-dimensional perspective of pathlines and 1000-day points for subproblem 2A, and recharge points for subproblem 2B. Points are colored by layer.
}
	{fig:ex-prt-mp7-p02-paths-3d}
	{../figures/mp7-p02-paths-3d.png}
\end{StandardFigure}

\begin{StandardFigure}{
	Particle recharge points for subproblem 2B, colored by travel time.
}
	{fig:ex-prt-mp7-p02-endpts}
	{../figures/mp7-p02-endpts.png}
\end{StandardFigure}