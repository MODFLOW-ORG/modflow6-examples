\section{Capture Fraction Analysis}

All groundwater pumped is balanced by removal of water somewhere, initially from storage in the aquifer and later from capture in the form of increase in recharge and decrease in discharge \citep{leake2010new}. Capture that results in a loss of water in streams, rivers, and wetlands now is a concern in many parts of the United States. Hydrologists commonly use analytical and numerical approaches to study temporal variations in sources of water to wells for select points of interest. Much can be learned about coupled surface/groundwater systems, however, by looking at the spatial distribution of theoretical capture for select times of interest. Development of maps of capture requires (1) a reasonably well-constructed transient or steady state model of an aquifer with head-dependent flow boundaries representing surface water features or evapotranspiration and (2) an automated procedure to run the model repeatedly and extract results, each time with a well in a different cell to evaluate the effect of a flux in a cell on an external boundary condition.

% Describe source of problem
This example presents a streamflow capture analysis of the hypothetical aquifer system of \cite{freyberg1988exercise} and based on the model datasets presented in \cite{hunt2020revisiting}. The original problem of \cite{freyberg1988exercise} documented an exercise where graduate students calibrated a groundwater model and then used it to make forecasts.


\subsection{Example Description}

% spatial discretization  and temporal discretization
A single-layer was used to simulate a shallow, water-table aquifer. The aquifer is surrounded by no-flow boundaries on the bottom and north-east-west sides (fig.~\ref{fig:ex-gwf-capture-01}). There is an outcrop area within the grid, where the water-table aquifer is missing. The aquifer is discretized using 40 rows and 20 columns (250 m on a side). A single steady-state stress period, with a single time step, was simulated. An arbitrary simulation period of 1 day was simulated. Model parameters for the example are summarized in table~\ref{tab:ex-gwf-capture-01}. 

The top of the model was set at an arbitrary elevation of 200 $m$ (table~\ref{tab:ex-gwf-capture-01}). The bottom elevations were not uniform; the aquifer was relatively flat on the east side and sloped gently to the south and west sides \citep[see][fig.~1b]{hunt2020revisiting}. In the western area, and southeastern and southwestern corners of the aquifer, the impermeable bottom elevation was higher making for no-flow outcrop areas within the grid. 


\begin{StandardFigure}{
                                     Diagram showing the model domain and the simulated streamflow capture fraction. The location of river cells, constant head cells, and wells are also shown.
                                     }{fig:ex-gwf-capture-01}{../figures/ex-gwf-capture-01.png}
\end{StandardFigure}                                 


% add static parameter table(s)
\input{../tables/ex-gwf-capture-01}


% material properties
Hydraulic conductivity (K) consisted of six zones  \citep[see][fig.~1c]{hunt2020revisiting} with relatively small changes among them; areas of higher values of K were along the north, east and west boundaries, and adjacent to the western outcrop; K was lower in the south.

% initial conditions and boundary conditions
An initial head of 45 $m$ is specified in each model cell. Flow into the system is from infiltration from precipitation and was represented using the recharge (RCH) package. A constant recharge rate of $1 \times 10^{-9}$ $m/s$ was specified for every active cell in the aquifer. A stream represented by river (RIV) package cells in column 15 in every row in the model and have river stages ranging from 20.1 $m$ in row 1 to 11.25 $m$ in row 40, a conductance of 0.05 $m^2/s$, and bottom elevations ranging from 20 $m$ in row 1 to 10.25 $m$ in row 40 (fig.~\ref{fig:ex-gwf-capture-01}). River cells discharge groundwater from the model in every cells except river cells in row 9 and 10, which are a source of water for the model. Additional discharge of groundwater out of the model is from discharging wells represented by well (WEL) package cells and specified head cells. There are six pumping wells (fig.~\ref{fig:ex-gwf-capture-01}) with a total withdrawal rate of 22.05 $m^3/s$. Heads are specified to be constant on the southern boundary in all active cells in row 40 and range from 16.9 $m$ in column 6 to 11.4 $m$ in column 15.

The Newton-Raphson formulation and Newton-Raphson under-relaxation were used to improve model convergence.


% for examples without scenarios
\subsection{Example Results}

The capture fraction analysis was performed using the \MF Application Program Interface (API). The \mfapi is used because the capture fraction for each cell can be calculated without regenerating the input files and running the model for each cell perturbed with an additional flux term. The \mfapi also allows the capture fraction analysis to be performed without ever running or the model to completion (finalizing the time step), which writes output to the listing file. The capture fraction perturbation flux is added to the model using the API Package, which adds a specified flux directly to the right-hand side of the system of equations for a group of cells. In this example, the streamflow capture fraction for a cell is calculated as

\begin{equation}
	\label{eq:capture-fraction}
	c_{k,i,j} = \frac{Q_{RIV}^{+} - Q_{{RIV}_{\text{base}}}}{|Q^{+}|} ,
\end{equation}

\noindent where $c_{k,i,j}$ is the streamflow capture fraction for a active cell (unitless), $Q_{RIV}^{+}$ is the streamflow the simulation with the perturbed flow in cell $(i,j,k)$ ($L^3/T$), $Q_{{RIV}_{\text{base}}}$ is the streamflow in the base simulation ($L^3/T$), and $Q^{+}$ is the flow that is incrementally added to each active cell  ($L^3/T$).


Use of the \mfapi function requires replacement of the time step loop in \MF with an equivalent time step loop in the programming language used to access in API. In this example, python is used to control \MF. In addition to defining a replacement time step loop, use of the \mfapi requires accessing program data in memory to control other programs being executed or to change \MF data from the values defined in the \MF input files. 

In this example, the right-hand side variable in the API Package is set to the specified perturbation flux rate (table~\ref{tab:ex-gwf-capture-01}) and the node number is modified when evaluating the capture fraction for each active cell. The python function that calculates the capture fraction for a cell is listed below.

\begin{python}
def calculate_capture_fraction(mobj, inode, qbase, q):
  # update the api package with the reduced node number
  tag = mobj.get_var_address("NODELIST", sim_name, "CF-1")
  nodelist = mobj.get_value(tag)
  nodelist[0] = inode + 1  # convert from zero-based to one-based node number
  mobj.set_value(tag, nodelist)

  # solve with the updated extraction location
  solve_current(mobj)

  return (get_streamflow(mobj) - qbase) / abs(q)
\end{python}


\noindent The \texttt{solve\_current()} function is listed below and solves the system of equations, with the updated extraction location, to convergence but does not finalize the time step. 

\begin{python}
def solve_current(mobj):
    max_iter = mobj.get_value(mobj.get_var_address("MXITER", "SLN_1"))

    # convergence loop
    kiter = 0
    mobj.prepare_solve()

    while kiter < max_iter:
        has_converged = mobj.solve()
        kiter += 1
        if has_converged:
            break

    mobj.finalize_solve()
\end{python}

\noindent The \texttt{get\_streamflow()} function gets the calculated flow between \MF and the RIV Package and uses \texttt{numpy} to sum all of the values into a net stream flow ($Q_{RIV}^{+}$). 

\begin{python}
def get_streamflow(mobj):
    tag = mobj.get_var_address("SIMVALS", sim_name, "RIV-1")
    return mobj.get_value(tag).sum()
\end{python}

\noindent All of these functions are tied together in the loop listed below that iterates over all of active cells and adds the calculate capture fraction to the appropriate location in the array used to store the values.

\begin{python}
ireduced_node = -1
for irow in range(nrow):
  for jcol in range(ncol):

    # skip inactive cells
    if idomain[irow, jcol] < 1:
        continue

    # increment reduced node number
    ireduced_node += 1

    # calculate the capture fraction for the cell
    cf = calculate_capture_fraction(mf6, ireduced_node, qbase, cf_q)

    # add the value to the capture array
    capture[irow, jcol] = cf

# finalize the MODFLOW 6 simulation
mf6.finalize()
\end{python} 

\noindent the \texttt{mf6.finalize()} method terminates the \MF simulation. The \texttt{idomain} array is a mapping array that identifies inactive cells that have been removed from the simulation.

The simulated streamflow capture fraction results are shown in figure~\ref{fig:ex-gwf-capture-01}. The stream captures all of the inflow to the model except west of the river in the vicinity close to the constant head boundaries. Cells close to the western-most constant head boundary cells do not contribute any groundwater to the stream.


