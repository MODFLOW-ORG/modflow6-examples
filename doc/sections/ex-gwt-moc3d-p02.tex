\section{MOC3D Problem 2}

% Describe source of problem
This problem corresponds to the second problem presented in the MOC3D report \cite{konikow1996three}, which involves the transport of a dissolved constituent in a steady, three-dimensional flow field.  An analytical solution for this problem is given by \cite{wexler1992}.  This example is simulated with the GWT Model in \mf, which receives flow information from a separate simulation with the GWF Model in \mf.  Results from the GWT Model are compared with the results from the \cite{wexler1992} analytical solution.

\subsection{Example description}
\cite{wexler1992} presents an analytical solution for three dimensional solute transport from a point source in a one-dimensional flow field.  As described by \cite{konikow1996three}, only one quadrant of the three-dimensional domain is represented by the numerical model.  Thus, the solute mass flux specified for the model is one quarter of the solute mass flux used in the analytical solution.  

The parameters used for this problem are listed in table~\ref{tab:ex-gwt-moc3d-p02-01}.  The model grid for this problem consists of 40 layers, 12 rows, and 30 columns.  The top for layer 1 is set to zero, and flat bottoms are assigned to all layers based on a uniform layer thickness of 0.05 $m$.  DELR is set to 3.0 $m$ and DELC is specified with a constant value of 0.5 $m$.  The simulation consists of one stress period that is 400 $d$ in length, and the stress period is divided into 400 equally sized time steps.  Velocity is specified to be 0.1 $m/d$ in the x direction and zero in the y and z directions.  The uniform flow field is represented by specifying a constant inflow rate into all of the cells in column 1 and by specifying a constant head condition to all of the cells in column 30.  A specified solute flux of 10 grams per day is specified to the cell in layer 1, row 12, and column 8.  Any water that leaves through the constant-head cell leaves with the simulated concentration of the water in that last cell.   Advection is solved using the TVD scheme to reduce numerical dispersion.  In addition to the longitudinal dispersion, transverse dispersion is represented with a different value in the horizontal direction than in the vertical direction.  Because the velocity field is perfectly aligned with the model grid, there are no cross-dispersion terms and the problem can be simulated accurately without the need for XT3D.

% add static parameter table(s)
\input{../tables/ex-gwt-moc3d-p02-01}

% for examples without scenarios
\subsection{Example Results}

A comparison of the MODFLOW 6 results with the analytical solution of \cite{wexler1992} is shown for layer 1 in figure~\ref{fig:ex-gwt-moc3d-p02-map}.

% a figure
\begin{StandardFigure}{
                                     Concentrations simulated by the \mf GWT Model and calculated by the analytical solution for three-dimensional flow with transport.  Results are for the end of the simulation (time=400 $d$) and for layer 1.  Black lines represent solute concentration contours from the analytical solution \citep{wexler1992}; blue lines represent solute concentration contours simulated by \mf.  An aspect ratio of 4.0 is specified to enhance the comparison.
                                     }{fig:ex-gwt-moc3d-p02-map}{../figures/ex-gwt-moc3d-p02-map.png}
\end{StandardFigure}            

                
