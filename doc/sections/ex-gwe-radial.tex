\section{Radial Heat Transport}

% Describe source of problem
A dedicated heat transport model referred to as the groundwater energy transport (GWE) model was released with version 6.5.0 of \mf. Previously, to simulate groundwater heat transport in \mf, a user could use the groundwater solute transport model, commonly referred to as simply groundwater transport (without specifying ``solute'' transport), to mimic heat transport by using its input parameters as surrogates for heat transport parameters \citep{modflow6gwt, ma2010, langevin2008seawat}. Now, with GWE, users may specify native heat transport parameter values in the appropriate GWE package.

\subsection{Example description}

This example demonstrates use of the GWE model.  This demonstration compares simulated results from a GWE model to an analytical solution that was published in \cite{alKhoury2020}.  Both the groundwater flow (GWF) and GWE models employ a DISV grid type \citep{modflow6software} with the numerical grid setup in a radial manner (fig.~\ref{fig:ex-gwe-radial-grid}). The grid geometry facilitates outward propagation of heat from a borehole heat exchanger (BHE) \citep{hechtMendez2010} located in the center of the radially-symetric model grid (fig.~\ref{fig:ex-gwe-radial-grid}). Groundwater flow moves from left to right. In this way, the model simulates heat flow in a convective and conductive heat transport environment moving past a cylindrical heat source.

% a figure
\begin{StandardFigure}{
    Configuration of the DISV model grid used in the radial transport problem.  Model grid originally published in \cite{alKhoury2020}.  Please refer to figure~\ref{fig:ex-gwe-radial-slow-gridinset} for a zoomed-in view of the model grid in the vicinity of the BHE.}{fig:ex-gwe-radial-grid}{../figures/ex-gwe-radial-slow-grid.png}
\end{StandardFigure}

% inset figure showing a zoomed in portion of the model grid
\begin{StandardFigure}{
    A zoomed-in view showing the refined discretization in close proximity to the BHE.  Cell dimensions are sub-centimeter scale around the perimeter of the BHE.  The radius of the BHE is 7.5 $cm$.}{fig:ex-gwe-radial-slow-gridinset}{../figures/ex-gwe-radial-slow-gridinset.png}
\end{StandardFigure}

Constant heads on the left and right sides of the model domain are specified such that the resulting left-to-right groundwater velocity is $1 \times 10^{-5} \tfrac{m}{s}$ (fig.~\ref{fig:ex-gwe-radial-slow-head}).  The heat source located in the center of the numerical model is represented using the energy source loading (ESL) package with a known rate of energy input [referred to as a Dirichlet boundary condition in \cite{alKhoury2020}].  The initial temperature throughout the model domain is 0.0 $^{\circ}C$. Energy is added to the grid cell in the middle of the model domain at a rate of 100 $\tfrac{W}{m}$.  Parameters used for the \mf simulation of the heat transport problem that uses a radially-symmetric grid are shown in table~\ref{tab:ex-gwe-radial-01}.

% add static parameter table(s)
\input{../tables/ex-gwe-radial-01}

% figure showing groundwater head
\begin{StandardFigure}{
    A head gradient is established in the outer-most ring of grid cells to drive groundwater flow from left to right. The combination of the groundwater head gradient with the hydraulic conductivity (table~\ref{tab:ex-gwe-radial-01}) results in a groundwater velocity of $1 \times 10^{-5} \tfrac{m}{s}$.}{fig:ex-gwe-radial-slow-head}{../figures/ex-gwe-radial-slow-head.png}
\end{StandardFigure}

% for examples without scenarios
\subsection{Example Results}

Results from the GWE model run are compared to a published analytical solution 48 hours after the start of the simulation \citep{alKhoury2020}.  Isotemperature contours at 1, 2, 3, 4, 6, and 8 $^{\circ}C$ demonstrate that GWE results compare well with the analytical solution (fig.~\ref{fig:ex-gwe-radial-temp48}).

% a results figure
\begin{StandardFigure}{
    Simulation results for the GWE model run compared with an analytical solution.}{fig:ex-gwe-radial-temp48}{../figures/ex-gwe-radial-temp48.png}
\end{StandardFigure}
