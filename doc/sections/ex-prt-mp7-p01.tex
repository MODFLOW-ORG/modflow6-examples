\section{PRT/MP7 Example 1: Particle Tracking on a Structured Grid with Steady-State Flow}

This example is reproduced from the MODPATH 7 example problems document \citep{modpath7examples} to demonstrate particle-tracking. MODFLOW 6 PRT and MODPATH 7 simulations are run and compared side-by-side.

\subsection{Example description}

The example first runs a groundwater flow (GWF) model simulating steady-state flow on a structured grid. The flow system includes an upper and lower aquifer separated by a confining layer with lower conductivity. The grid has 3 layers, 21 rows, and 20 columns, with square cells 500 feet to a side. The system includes two boundary conditions: a well in layer 3, row 11, column 10, and a river in layer 1, column 20~(fig~\ref{fig:ex-prt-mp7-p01-config}). Model parameters for this example are summarized in table~\ref{tab:ex-prt-mp7-p01-01}.

\begin{StandardFigure}{
    Conceptual model. Image reproduced from the MODPATH 7 examples document.
    }{fig:ex-prt-mp7-p01-config}{../images/ex-prt-mp7-p01-config.png}
\end{StandardFigure}

\input{../tables/ex-prt-mp7-p01-01.tex}

\subsection{Example Results}

In this example a MODFLOW 6 particle tracking (PRT) model runs in the same simulation as a groundwater flow (GWF) model~(fig~\ref{fig:ex-prt-mp7-p01-grid-head}), which provides it with intercell flows via a GWF-PRT model exchange.

\begin{StandardFigure}{
    Heads simulated by the MODFLOW 6 groundwater flow (GWF) model.
    }{fig:ex-prt-mp7-p01-grid-head}{../figures/mp7-p01-grid-head.png}
\end{StandardFigure}

In subproblem 1A, a line of 21 particles is placed at the water table in layer 1 for column 3, rows 1 through 21. In subproblem 1B, a denser release configuration is used which places a 3 x 3 array of particles on the top face of every cell in layer 1. Both simulations track particles forward to their discharge points.

MODPATH 7 distinguishes between pathline, endpoint, timeseries, and combined (pathline + timeseries) simulations. In the original document, example 1A is solved by a combined pathline and timeseries simulation, and example 1B is solved in a separate endpoint simulation. Here we will run a single combined pathline and timeseries simulation with two particle groups, one for each subproblem, then extract endpoint data from the MODPATH 7 pathline output file.

MODFLOW 6 PRT makes no distinction between simulation modes, instead offering fine-grained control over tracking events, such as particle release, termination, cell-to-cell transitions, timesteps, and timeseries. In this example we enable all tracking events, and provide a custom timeseries to match the 1000-day timeseries interval used for MODPATH 7.

Subproblem 1A tracks particles at a 1000-day time interval~(fig~\ref{fig:ex-prt-mp7-p01-paths-layer}), and can be visualized in 2D and 3D~(fig~\ref{fig:ex-prt-mp7-p01-paths-3d}).

\begin{StandardFigure}{
    Particle pathlines and points (1A), 1000-day interval, colored by layer, overhead map view.
    }{fig:ex-prt-mp7-p01-paths-layer}{../figures/mp7-p01-paths-layer.png}
\end{StandardFigure}

\begin{StandardFigure}{
    Particle path points (1A), 1000-day interval, colored by layer, three-dimensional perspective.
    }{fig:ex-prt-mp7-p01-paths-3d}{../figures/mp7-p01-paths-3d.pdf}
\end{StandardFigure}

Pathlines can be colored by discharge area (well or river)~(fig~\ref{fig:ex-prt-mp7-p01-paths}).

\begin{StandardFigure}{
    Particle pathlines, colored by destination: particles with red pathlines are captured by the well, particles with blue pathlines are captured by the river.
    }{fig:ex-prt-mp7-p01-paths}{../figures/mp7-p01-paths.png}
\end{StandardFigure}

To show capture areas~(fig~\ref{fig:ex-prt-mp7-p01-rel-destination}), starting locations of all particles can be color-coded according to the zone value of the cells in which they terminate.

\begin{StandardFigure}{
    Particle release points, colored by destination.
    }{fig:ex-prt-mp7-p01-rel-destination}{../figures/mp7-p01-rel-destination.png}
\end{StandardFigure}

Travel time analysis is also a common use case for particle tracking~(fig~\ref{fig:ex-prt-mp7-p01-rel-travel-time}).

\begin{StandardFigure}{
    Particle release points, colored by travel time.
    }{fig:ex-prt-mp7-p01-rel-travel-time}{../figures/mp7-p01-rel-travel-time.png}
\end{StandardFigure}