\section{MODPATH 7 Example 1: Structured Grid, Steady-State Flow}

This example's flow system consists of two aquifers separated by a thin low conductivity confining layer. The system is simulated with a traditional structured model grid consisting of 3 model layers, 21 rows, and 20 columns. Areal grid cells are uniform square cells, 500 feet per side. The well is located in row 11, column 10, and layer 3. The river is located in layer 1, column 20, rows 1 - 21

\begin{StandardFigure}{
    Model grid and boundary conditions.
    }{fig:ex-prt-p01-grid}{../images/mp7-p01-grid.png}
\end{StandardFigure}

In simulation 1A, a line of 21 particles is placed at the water table in layer 1 for column 3, rows 1 through 21. A combined pathline and time-series simulation is run that tracks the particles forward to their discharge locations. The particle stop option is set to extend the final time step to assure that all of the particles reach their discharge locations, regardless of travel time.

\begin{StandardFigure}{
    Model grid and boundary conditions.
    }{fig:ex-prt-p01-paths}{../images/mp7-p01-paths.png}
\end{StandardFigure}

MODPATH often is used to compute recharge capture areas for wells and other hydrologic features. Forward-tracking endpoint simulations are an efficient way to use MODPATH to compute capture areas for wells. In simulation 1B, a capture area for the well is determined by placing a 3 x 3 array of particles on the top face of layer 1 and then tracking the particles forward to their discharge points using a forward endpoint simulation. The endpoint file generated by MODPATH contains information about the starting location and the final location of each particle. The capture area for the well can be displayed for a forward-tracking endpoint simulation by plotting the starting locations of all the particles color-coded according to the zone value of the final cells in which they terminate.

\begin{StandardFigure}{
    Model grid and boundary conditions.
    }{fig:ex-prt-p01-endpts}{../images/mp7-p01-endpts.png}
\end{StandardFigure}

The starting locations file used for simulation 1B has a different, more compact structure than that used for simulation 1A. In simulation 1B, the starting locations are specified using a template and a specified range of cells. The template defines how particles are placed in each cell within the range. That approach allows thousands of particles to be generated automatically with only a few lines of data in the starting locations file. The other data files used in simulation 1A also can be used in simulation 1B. The starting locations of particles that discharge to the cell containing the well are shown in red. Starting locations that discharge to the river are shown with open circles.

\begin{StandardFigure}{
    Model grid and boundary conditions.
    }{fig:ex-prt-p01-paths}{../images/mp7t-p01-rel-capt.png}
\end{StandardFigure}

\begin{StandardFigure}{
    Model grid and boundary conditions.
    }{fig:ex-prt-p01-paths}{../images/mp7-p01-term-capt.png}
\end{StandardFigure}

\begin{StandardFigure}{
    Model grid and boundary conditions.
    }{fig:ex-prt-p01-paths}{../images/mp7-p01-term-tt.png}
\end{StandardFigure}