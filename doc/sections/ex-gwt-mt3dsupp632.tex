\section{MT3DMS Supplemental Guide Problem 6.3.2}

% Describe source of problem
This example is for zero-order production in a dual-domain system.  It is based on example problem 6.3.2 described in \cite{zheng2010mt3dmsv5.3}.  The problem consists of a one-dimensional model grid with inflow into the first cell and outflow through the last cell.  This example is simulated with the GWT Model in \mf, which receives flow information from a separate simulation with the GWF Model in \mf.  This example is designed to test the capabilities of the GWT Model to simulate zero-order production in a dual-domain system with and without sorption.  Results from the GWT Model are compared with the results from a MT3DMS simulation \citep{zheng1990mt3d} that uses flows from a separate MODFLOW-2005 simulation \citep{modflow2005}.  This example was described by \cite{zheng2010mt3dmsv5.3} who showed that the results from MT3DMS were in good agreement with an analytical solution.

\subsection{Example description}

The parameters used for this problem are listed in table~\ref{tab:ex-gwt-mt3dsupp632-01}.  The model grid consists of 401 columns, 1 row, and 1 layer.  The flow problem is confined and steady state with an initial head set to the model top.  The solute transport simulation represents transient conditions, which begin with an initial concentration specified as zero everywhere within the model domain.  A specified flow condition is assigned to the first model cell.  For the source pulse duration, a specified concentration with a value of one is assigned to the first model cell.  Following the source pulse duration the specified concentration in the first cell is zero.  A specified head condition is assigned to the last model cell.  Water exiting the model through the specified head cell leaves with the simulated concentration of that cell.

% add static parameter table(s)
\input{../tables/ex-gwt-mt3dsupp632-01}

% for examples with scenarios
\subsection{Example Scenarios}

This example problem consists of several different scenarios, as listed in table~\ref{tab:ex-gwt-mt3dsupp632-scenario}.  The first two scenarios represent zero-order growth when sorbtion is active.  Sorbtion is not active in the last scenario.  For all three scenarios, there is mass transfer between the mobile domain and the immobile domain.

% add scenario table
\input{../tables/ex-gwt-mt3dsupp632-scenario}


\subsubsection{Scenario Results}

Results from the three scenarios are shown in figure~\ref{fig:ex-gwt-mt3dsupp632}.  The close agreement between the simulated concentrations for the \mf GWT Model and MT3DMS demonstrate the zero-order growth and immobile-domain transfer capabilities for \mf.

% a figure
\begin{StandardFigure}{
                                     Concentrations simulated by the \mf GWT Model and MT3DMS for zero-order growth in a dual-domain system.  Circles are for the GWT Model results; the lines represent simulated concentrations for MT3DMS.
                                     }{fig:ex-gwt-mt3dsupp632}{../figures/ex-gwt-mt3dsupp632.png}
\end{StandardFigure}                                 

